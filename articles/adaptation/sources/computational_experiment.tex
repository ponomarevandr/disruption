%!TEX root = ../main.tex

\section{Вычислительный эксперимент}

\subsection{Параметры модели, краевые условия}

Создана программа, реализующая в рамках разностной схемы \eqref{sch:borders},~\eqref{sch:transition},~\eqref{sch:time_step_min_max} перечисленные ранее алгоритмы адаптации временного шага: по фазовому полю \eqref{sch:time_step_phi}, по энергии \eqref{sch:time_step_energy} и по устойчивости \eqref{sch:time_step_stability}.

Будем использовать параметры модели, отражающие реальный физический эксперимент: см. табл. \ref{tab:parameters}. Часть параметров ($\norm{\nabla \Phi}$, $\epsilon_0$) являются полноценными физическими величинами, часть ($\Gamma$, $m$) могут быть подобраны для~согласования модели с результатами эксперимента. Чертой отделены параметры, которые либо происходят из связанных с диффузной границей допущений ($l$, $\delta$), либо описывают расчетную сетку.

\begin{table}[!t]
\captionsetup{justification=raggedright,singlelinecheck=false}
\caption[]{Параметры модели в расчете}
\centering
\begin{tabular}{|l|c|l|}
	\hline
	Название & Параметр & Значение \\
	\hline
	Электрическое напряжение		& $\norm{\nabla \Phi}$	& $5.625 \cdot 10^6 \; \unitV / \unitm$							\\
	Энергия роста ед. длины канала	& $\Gamma$				& $8.118 \cdot 10^{-10} \; \unitJ / \unitm$						\\
	Диэлектрическая проницаемость	& $\epsilon_0$			& $2.301 \cdot 10^{-11} \; \unitC^2 / (\unitJ \cdot \unitm)$	\\
	Подвижность						& $m$					& $12 \; \unitm^3 / (\unitJ \cdot \units)$						\\
	\hline
	Характерная толщина границы		& $l$ 					& $1.5 \cdot 10^{-6} \; \unitm$									\\
	Регуляризующий параметр 		& $\delta$				& $10^{-3}$														\\
	Размер образца					& $W$					& $3.2 \cdot 10^{-5} \; \unitm$									\\
	Продолжительность опыта			& $T$					& $2 \cdot 10^{-3} \; \units$									\\
	Шаг по пространству				& $h$					& $5 \cdot 10^{-7} \; \unitm$									\\
	Минимальный шаг по времени		& $\taumin$				& $10^{-10} \; \units$											\\
	Максимальный шаг по времени		& $\taumax$				& $\leqslant 6.42 \cdot 10^{-6} \; \units$						\\
	\hline
\end{tabular}
\label{tab:parameters}
\end{table}

При подстановке значений параметров в условие \eqref{cond:stability} устойчивости разностной схемы получаем $\tau \leqslant \min(2.86 \cdot 10^{-10}, 6.42 \cdot 10^{-6}) \, \units$. Неравенство с первым аргументом минимума является ограничением, происходящим из свойств функции $F(\phi)$ (см. формулу~\eqref{cond:stability_first}). Оно может быть ослаблено в зависимости от значений $\phi_j^k$ на текущем временном слое. Неравенство со вторым аргументом носит безусловный характер. Поэтому качестве $\taumax$ взято значение~$6.42 \cdot 10^{-6} \; \units$; $\taumin = 10^{-10} \; \units$ удовлетворяет обоим неравенствам и может рассматриваться как шаг по времени до введения адаптации.

Число узлов сетки по пространству $N_x \equiv M = W / h = 64$, по времени -- $T / \taumax \hm \leqslant N_t \leqslant T / \taumin$, а именно $312 \leqslant N_t \leqslant 2 \cdot 10^7$.

Зададим следующие краевые условия:
\begin{gather}
	\phi(0, t) = \phi(W, t) = 1, \qquad \phi(x, 0) = \phi_0(x),
	\label{exp:boundary} \\
	\phi_0(x) = \begin{cases}
		1 - 0.025 \cdot \left( 1 + \cos \left[ \cfrac{\pi}{0.08} \left( \cfrac{x}{W} - \half \right) \right] \right) \text{ при } \cfrac{x}{W} \in [0.42, 0.58]; \\
		1 \text{ иначе}.
	\end{cases}
	\label{exp:initial}
\end{gather}
Функция $\phi_0(x)$ отлична от $1$ в небольшой области вокруг $x = W / 2$, где <<прогибается>> как один период синусоиды, достигая минимума $\phi = 0.95$.


\subsection{Структура сетки с переменным шагом по времени}

В настоящем подразделе описана техническая особенность расчетной сетки, внедренная авторами при создании программы. Это техническое решение лишь одно из возможных, и его понимание не является необходимым для~восприятия решаемой задачи в целом.

Итак, на каждом временном слое номер $k$ используется своя величина шага~$\tau^k$. Таким образом, расчетная сетка теряет регулярность по времени, и сравнение разных решений по сеточной норме становится нетривиальной задачей. Введем у нерегулярной сетки определенную структуру, в которой сконцентрируем всю сложность вопроса, избегая при этом использования сеточной интерполяции для анализа результатов расчетов.

Пусть $N = N_{t, \text{max}} = T / \taumin \in \Natural$, то есть временной промежуток $[0, T]$ разбит $N + 1$ узлом на $N$ равных отрезков длиной $\taumin$ каждый. Над этим разбиением введем структуру <<типа дерева отрезков>> (см., например, \cite[подраздел~1.2.3.1]{preparata_geometry}). Говоря формально, будем считать допустимыми лишь разбиения вида $D = (0, p_1 \taumin, p_2 \taumin, \dots, p_{n - 1} \taumin,$ \linebreak $N \taumin)$, где $p_k \in \Natural_0$, $k = \overline{0, n}$, $p_k$ строго возрастают, $p_0 = 0$, $p_n = N$; $L_k = p_{k + 1} - p_k = 2^{s_k}$, $s_k \in \Natural_0$, и к тому же $p_k \divby L_k$ (делится нацело). Все возможные интервалы допустимых разбиений образуют подобие двоичного дерева, схематично изображенного на рис. \ref{fig:segment_tree}.

\begin{figure}[!t]
	\centering
	\includesvg[width=0.6\textwidth]{figures/segment_tree.svg}
	\caption{Все возможные интервалы допустимых разбиений для $N = 11$; пример допустимого разбиения выделен цветом}
	\label{fig:segment_tree}
\end{figure}

Описанная структура замечательна тем, что если из любых двух допустимых разбиений $D_1$ и $D_2$ выбрать по интервалу, то либо эти интервалы не пересекаются, либо совпадают, либо один строго вложен в другой. Следовательно, любые два соседних узла объемлющего разбиения $D = D_1 \cap D_2$ (пересечение в смысле множеств) соседствуют также в $D_1$ или в $D_2$ -- в таком смысле $D$ оптимально.

При адаптации шага по времени в выражении \eqref{sch:transition} разностной схемы \eqref{sch:borders},~\eqref{sch:transition},~\eqref{sch:time_step_min_max} на слое $k$ будем использовать максимальное ${\tau'}^k = 2^s \taumin \leqslant \tau^k$, $s \in \Natural_0$, к~тому же допустимое описанным разбиением <<типа дерева отрезков>> временного промежутка $T$ на $N$ отрезков, а именно:
\begin{gather*}
	p_0 = 0, \quad p_{k + 1} = p_k + 2^{s_k} \leqslant N, \; s_k \in \Natural_0; \\
	{\tau'}^k = 2^{s_k} \cdot \taumin \leqslant \tau^k, \quad p_k \divby 2^{s_k}; \\
	s_k \to \max.
\end{gather*}

Для сравнения по равномерной норме $\enorm_{C, h}$ двух сеточных решений~$\phi_1$ и~$\phi_2$ на разбиениях $D_1$ и $D_2$ соответственно ограничим их оба на объемлющем разбиении $D \hm = D_1 \cap D_2$.


\subsection{Результаты расчетов}

На рис. \ref{fig:solution_basic} изображен результат расчета с параметрами из табл. \ref{tab:parameters} и краевыми условиями \eqref{exp:boundary}, \eqref{exp:initial} для разностной схемы \eqref{sch:transition_old}, \eqref{sch:borders} с постоянным шагом по времени $\tau = \taumin$. Видно, как из малого начального возмущения фазового поля $\phi$ постепенно растет канал электрического пробоя. В момент времени $t \approx 1.82 \cdot 10^{-3}$ в~точке $x = W / 2$ фазовое поле~$\phi$  становится мало отличимо от $0$ -- происходит <<пробой насквозь>>. Обратим внимание, что значение фазового поля упало от~$\phi \approx 0.6$ до $\phi \approx 0$ менее чем за время $10^{-5}$, то есть $0.5 \%$ от всей продолжительности эксперимента. Далее канал пробоя растет в~толщину примерно с постоянной скоростью.

\begin{figure}[!t]
	\centering
	\includegraphics[width=\textwidth]{figures/solution_basic.png}
	\vspace{-0.8cm}
	\caption{Решение задачи (расчет без адаптации)}
	\label{fig:solution_basic}
\end{figure}

При исходном наборе параметров разностная задача имеет некоторый <<запас>> численной устойчивости, поэтому удается провести расчеты с постоянным шагом по времени $\tau = 2 \taumin$, $4 \taumin$; дальнейшее увеличение шага без~адаптации оказывается невозможным: устойчивость теряется, и значения~$\phi_j^k$ покидают отрезок $[0, 1]$.

Теперь проведем расчеты с той же конфигурацией системы, но используя схему~\eqref{sch:borders}, \eqref{sch:transition}, \eqref{sch:time_step_min_max} с переменным шагом по времени для каждого из трех методов адаптации: по фазовому полю~\eqref{sch:time_step_phi}, по энергии~\eqref{sch:time_step_energy}, по устойчивости~\eqref{sch:time_step_stability}. Будем подбирать константы~$\tol_i$ таким образом, чтобы добиваться ускорения примерно в $2^s$ раз, $s \in \Natural$, для удобства сравнения с обыкновенным увеличением шага в $2$ раза. Под ускорением понимается отношение числа~$T / \taumin$ временных слоев в исходном расчете к их числу $N$ в расчете с~адаптацией.

Будем сравнивать результаты $\phi_j^k$ расчетов с исходным расчетом ($\tau = \taumin$) по равномерной сеточной норме $\enorm_C$:
\[
	\left\| \phi_j^k - \widetilde{\phi}_j^k \right\|_C = \max\limits_{j, k} \left| \phi_j^k - \widetilde{\phi}_j^k \right|.
\]

Результаты сравнения для трех методов адаптации и для простого увеличения шага по времени изображены на рис. \ref{fig:method_errors}. Даже при двукратном ускорении ошибка расчета оказывается весьма существенной: порядка $10^{-2}$ при $\phi \in [0, 1]$. Однако в типичном решении задачи в большей части расчетной области $\Omega$ фазовое поле $\phi$ близко к $0$ или к~$1$ и лишь в небольшой зоне на~границе канала пробоя принимает промежуточные значения; поэтому будем считать верными решения с ошибкой вплоть до порядка $10^{-1}$ -- положим, до~$0.2$ (верхняя пунктирная линия на графике). Расчет с адаптацией по~энергии не~превышает этого порога при ускорении до $4$ раз, с адаптацией по~фазовому полю и по устойчивости -- при ускорении до $8$ раз.

\begin{figure}[!t]
	\centering
	\includegraphics[width=\textwidth]{figures/method_errors.png}
	\vspace{-0.8cm}
	\caption{Ошибка расчета при ускорении за счет шага по времени}
	\label{fig:method_errors}
\end{figure}

Обратим внимание, что самый крупный шаг по времени, возникающий в~расчетах на рис. \ref{fig:method_errors} при наибольшем ускорении, есть $\tau^k = 2^{13} \cdot \taumin = 8192 \hm \cdot 10^{-10} \approx 8.2 \cdot 10^{-7}$, не достигая ограничения $\taumax = 6.42 \cdot 10^{-6}$.

Зададим теперь $\taumax = 10^{-8}$, чтобы ограничение максимального шага стало существенным -- результат оказывается намного лучше (рис. \ref{fig:method_limited_errors}). Методы адаптации по фазовому полю и по энергии позволяют уверенно ускорить расчет в~$32$ раза; все три метода при ошибке около $0.2$ дают пиковое ускорение почти в~$64$ раза.

\begin{figure}[!t]
	\centering
	\includegraphics[width=\textwidth]{figures/method_limited_errors.png}
	\vspace{-0.8cm}
	\caption{Ошибка расчета при ускорении, $\taumax$ уменьшен}
	\label{fig:method_limited_errors}
\end{figure}

Хотя все три метода адаптации значительно улучшили показатели при уменьшении~$\taumax$, что важно для их практического применения, стоит отметить, что это свидетельствует о некоторой их неоптимальности. На взгляд авторов, в идеале укрупнение шага за счет адаптации должно происходить так, чтобы ошибка решения росла <<равномерно>> на всем протяжении расчета. В таком случае специальное уменьшение $\taumax$ было бы не нужно.

В ходе вычислительного эксперимента (рис. \ref{fig:method_errors}, \ref{fig:method_limited_errors}) лучше всех себя показал первый метод адаптации (он же самый простой) -- по фазовому полю. Адаптация по устойчивости несколько превосходит его в точности при малых величинах ускорения.