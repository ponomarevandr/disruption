%!TEX root = ../main.tex

\section{Математическая модель}

Приведем краткое описание исследуемой математической модели. По\forcehyphenation дробное описание физического смысла уравнений и параметров модели можно найти в работе \cite{ponomarev_stability}.

Рассматривается ограниченная область пространства $\Omega \subset \Real^3$. Распределение фаз вещества в ней задается гладкой функцией $\phi(\vx, t)$, $\phi: \Omega \hm \times [0, +\infty)_t \to [0, 1]$,~-- фазовым полем; вещество может находиться в одной из~двух фаз: $\phi \approx 1$~-- <<неповрежденное>>, $\phi \approx 0$~-- <<полностью разрушенное>> (то есть относящееся к каналу пробоя),~-- а также в промежуточных состояниях в области диффузной границы: $\vx \in \Omega$, $0 + \varepsilon \leqslant \phi(\vx) \leqslant 1 - \varepsilon$, $\varepsilon > 0$ мало.

Диэлектрическая проницаемость среды $\epsilon$ задается следующей формулой:
\[
	\epsilon(\vx, t) = \epsilon[\phi] = \cfrac{\epsilon_0(\vx)}{f(\phi(\vx, t)) + \delta}.
	\label{eq:epsilon}
\]
Здесь $\epsilon_0(\vx)$~-- диэлектрическая проницаемость неповрежденной среды, $f(\phi) \hm = 4\phi^3 - 3\phi^4$~-- интерполирующая функция, $0 < \delta \ll 1$~-- регуляризующий параметр. Запись $\epsilon[\phi]$ означает функциональную зависимость $\epsilon$ от $\phi$.

Помимо фазового поля $\phi$, состояние системы описывает функция $\Phi(\vx, t)$, $\Phi: \Omega \times [0, +\infty)_t \to \Real$,~-- потенциал электрического поля.

Постулируется следующее выражение для свободной энергии $\Pi$ системы:
\begin{gather*}
	\Pi = \int \limits_\Omega \pi d \vx, \\
	\pi = -\half \epsilon[\phi] \cdot \norm{\nabla \Phi}^2 + \Gamma \cfrac{1 - f(\phi)}{l^2} + \cfrac{\Gamma}{4} \norm{\nabla \phi}^2.
\end{gather*}
Здесь $\Gamma > 0$, $l > 0$~-- числовые параметры модели, константы; $\enorm$ обозначает евклидову норму в $\Real^3$.

Принимается, что динамика системы описывается следующими двумя \linebreak уравнениями:
\begin{equation*}
\begin{cases}
	\cfrac{\delta \Pi}{\delta \Phi} = 0; \\[3mm]
	\cfrac{1}{m} \partt{\phi} = -\cfrac{\delta \Pi}{\delta \phi}.
\end{cases}
\end{equation*}
Здесь константа $m > 0$~-- числовой параметр модели, называемый подвижностью. Говоря нестрого, согласно первому уравнению электрический потенциал $\Phi$ распределяется так, чтобы свободная энергия была минимальной при~заданном распределении фазового поля $\phi$; согласно второму~-- фазовое поле~$\phi$ с~определенной скоростью эволюционирует так, чтобы свободная энергия была минимальной при заданном распределении электрического потенциала~$\Phi$.

Отыскав явно вариационные производные в двух уравнениях выше, получим следующую систему уравнений:
\begin{numcases}{}
	\Div(\epsilon[\phi] \nabla \Phi) = 0;
	\label{eq:Phi} \\
	\cfrac{1}{m} \partt{\phi} = \half \epsilon'(\phi) \norm{\nabla \Phi}^2 + \cfrac{\Gamma}{l^2} f'(\phi) + \half \Gamma \Delta \phi.
	\label{eq:phi}
\end{numcases}
Здесь $(\cdot)' \equiv (\cdot)_\phi'$. Система состоит из двух уравнений относительно скалярных полей $\phi$ и $\Phi$ и описывает их совместную эволюцию.

Уравнение \eqref{eq:phi} имеет вид
\[
	\cfrac{1}{m} \partt{\phi} = -F'(\phi; \norm{\nabla \Phi}) + \half \Gamma \Delta \phi,
\]
где
\begin{equation}
	F(\phi; \norm{\nabla \Phi}) = -\half \epsilon[\phi] \cdot \norm{\nabla \Phi}^2 + \Gamma \cfrac{1 - f(\phi)}{l^2}
	\label{eq:allen_cahn_potential}
\end{equation}
есть определенная нелинейная функция от $\phi$, которая к тому же зависит от~$\norm{\nabla \Phi}$ как от параметра. Таким образом, перед нами нелинейное уравнение типа Аллена--Кана \cite{allen_cahn_boundary_theory}.

Из вывода модели очевидна следующая запись формулы для плотности свободной энергии:
\begin{equation}
	\pi = F(\phi; \norm{\nabla \Phi}) + \cfrac{\Gamma}{4} \norm{\nabla \phi}^2.
	\label{eq:energy_density}
\end{equation}

В рассматриваемой задаче функция $F$ имеет различное поведение в зависимости от~значения $\norm{\nabla \Phi}$, как было показано в работе \cite{ponomarev_stability}. Возможны три случая в соответствии со значением параметра
\[
	\xi = \cfrac{\norm{\nabla \Phi}^2 l^2 \epsilon_0}{2 \Gamma},
\]
а именно:
\begin{itemize}
	\item <<слабое напряжение>>, $\xi < \delta^2$: $F(\phi)$ монотонно убывает;
	\item <<среднее напряжение>>, $\delta^2 < \xi < (1 + \delta)^2$: $F(\phi)$ унимодальна, убывание сменяется возрастанием;
	\item <<сильное напряжение>>, $\xi > (1 + \delta)^2$: $F(\phi)$ монотонно возрастает.
\end{itemize}
Наибольший интерес для практики моделирования представляет случай \linebreak <<сильного напряжения>>, так как именно тогда канал пробоя развивается из~сколь угодно малых возмущений неповрежденной среды.