\documentclass[a4paper,12pt]{article}

%%% Поля
\usepackage[
	left=3cm,
	right=1.5cm,
	top=2cm,
	bottom=2cm,
	bindingoffset=0cm
]{geometry}

%%% Работа с русским языком
\usepackage{cmap}						% поиск в PDF
\usepackage{mathtext}					% русские буквы в формулах
\usepackage[T2A]{fontenc}				% кодировка
\usepackage[utf8]{inputenc}				% кодировка исходного текста
\usepackage[english,russian]{babel}		% локализация и переносы
\usepackage{indentfirst}
\frenchspacing

%%% Дополнительная работа с математикой
\usepackage{amsmath,amsfonts,amssymb,amsthm,mathtools}  % AMS

%%% Текст в колонки
\usepackage{multicol}

%%% Списки
\usepackage{enumitem}
\setlist{nosep, leftmargin=*}
\renewcommand{\labelenumi}{\arabic*)}

%%% Системы уравнений
\usepackage{cases}

%%% Таблицы
\usepackage{array}

%%% Рисунки
\usepackage{graphicx}
\usepackage{float}
\usepackage[inkscapepath=build/svg-inkscape]{svg}

%%% Точка в подписях к рисункам
\usepackage[labelsep=period]{caption}

%%% Список литературы
\bibliographystyle{bibliography_style/gost-numeric.bbx}
\usepackage[
	natbib = true,
	style = gost-numeric,
	sorting = none,
	backend = biber,
	language = autobib,
	autolang = other,
	url = false
]{biblatex}
\addbibresource{references.bib}

%%% Исправление символа номера при использовании gost-numeric.bbx
\usepackage{textcomp}
\DefineBibliographyStrings{russian}{number={\textnumero}}

%%% Прямой символ интеграла
\usepackage[integrals]{wasysym}

%%% Гиперссылки
\usepackage[pdftex,unicode]{hyperref}

%%% Перенос знаков в формулах (по Львовскому)
\newcommand*{\hm}[1]{#1\nobreak\discretionary{}{\hbox{$\mathsurround=0pt #1$}}{}}


%%% Свои команды

\newcommand*{\No}{\textnumero}

\newcommand{\vect}[1]{\boldsymbol{#1}}
\newcommand{\vx}{{\vect{x}}}
\newcommand{\vn}{{\vect{n}}}

\newcommand{\taumin}{{\tau_\text{min}}}
\newcommand{\taumax}{{\tau_\text{max}}}
\newcommand{\tol}{{\text{tol}}}

\newcommand{\half}{\cfrac{1}{2}}

\newcommand{\partt}[1]{\cfrac{\partial #1}{\partial t}}
\newcommand{\partx}[1]{\cfrac{\partial #1}{\partial x}}
\newcommand{\partxx}[1]{\cfrac{\partial^2 #1}{\partial x^2}}
\newcommand{\partvn}[1]{\cfrac{\partial #1}{\partial \vn}}

\newcommand{\partflt}[1]{\partial #1 / \partial t}
\newcommand{\partflx}[1]{\partial #1 / \partial x}
\newcommand{\partflxx}[1]{\partial^2 #1 / \partial x^2}
\newcommand{\partflvn}[1]{\partial #1 / \partial \vn}

\newcommand{\diffhh}[1]{\cfrac{{#1}_{j + 1}^k - 2 {#1}_j^k + {#1}_{j - 1}^k}{h^2}}

\newcommand{\Natural}{{\mathbb{N}}}
\newcommand{\Real}{{\mathbb{R}}}
\newcommand{\bigO}{{\mathcal{O}}}
\newcommand{\clOmega}{{\overline{\Omega}}}

\newcommand{\norm}[1]{\| #1 \|}
\newcommand{\enorm}{{\| \cdot \|}}

\newcommand{\unitm}{{\text{м}}}
\newcommand{\units}{{\text{с}}}
\newcommand{\unitJ}{{\text{Дж}}}
\newcommand{\unitC}{{\text{Кл}}}
\newcommand{\unitV}{{\text{В}}}
\newcommand{\unitF}{{\text{Ф}}}

\DeclareRobustCommand{\divby}{%
 	\mathrel{\vbox{\baselineskip.65ex\lineskiplimit0pt\hbox{.}\hbox{.}\hbox{.}}}%
}

\newcommand{\forcehyphenation}{-\linebreak}

\newcommand{\absent}[1]{[...#1...]}


%%% Свои операторы
\DeclareMathOperator{\Div}{{div}}
\DeclareMathOperator{\Int}{{Int}}
\DeclareMathOperator*{\argmin}{{argmin}}


%%% Оформление теорем

\theoremstyle{plain}
\newtheorem{theorem}{Теорема}
\newtheorem{proposition}{Утверждение}

\theoremstyle{remark}
\newtheorem{remark}{Замечание}


%%% Пояснение к меткам
% eq	-- equation
% cond	-- condition
% char	-- characteristic
% sch	-- scheme
% est	-- estimation
% exp	-- experiment
% fig	-- figure
% tab	-- table
% sec	-- section


%%% Описание препринта
% Комментирование конца строки убирает паразитные пробелы
\newcommand{\PreprintTitle}{%
	Адаптация шага по времени в модели типа \\ <<диффузной границы>> на основе уравнения Аллена--Кана%
}
\newcommand{\PreprintTitleEnglish}{%
	Time step adaptation in a diffuse interface model \\ based on Allen--Cahn equation%
}
\newcommand{\PreprintAuthors}{%
	А.~С.~Пономарев\textsuperscript{1}, Е.~В.~Зипунова\textsuperscript{1}, А.~А.~Кулешов\textsuperscript{1}, Е.~Б.~Савенков\textsuperscript{1} \\
	\textsuperscript{1}ИПМ им. М. В. Келдыша РАН%
}
\newcommand{\PreprintAuthorsEnglish}{%
	A.~S.~Ponomarev\textsuperscript{1}, E.~V.~Zipunova\textsuperscript{1}, A.~A.~Kuleshov\textsuperscript{1}, E.~B.~Savenkov\textsuperscript{1} \\
	\textsuperscript{1}Keldysh Institute of Applied Mathematics%
}


%%%%%%%%%%%%%%%%%%%%%%%%%%%%%%%%%%%%%%%%%%%%%%%%%%%%%%%%%%%%%%%%%%%%%%%%%%%%%%%%

\begin{document}

%%%%%%%%%%%%%%%%%%%%%%%%%%%%%%%%%%%%%%

{
	\centering
	{\large \textbf{\PreprintTitle} \par}
	\vspace{2mm}
	\PreprintAuthors \par
	\vspace{2mm}
}
{
	\noindent \small
	Предмет исследования работы -- модель развития канала электрического пробоя типа диффузной границы, основанная на уравнении типа Аллена--Кана. Дано краткое описание модели, приведена используемая авторами явная разностная схема. Временные масштабы процессов в моделируемой системе существенно различаются, поэтому закономерно использование адаптации расчетного шага по времени. \\
	В работе исследованы три различных метода адаптации шага по времени в указанной модели. Для двух методов, предложенных ранее в близких по тематике статьях других авторов, проведен теоретический анализ, выявлена определенная их общность. Третий метод предложен авторами настоящей работы, дано его теоретическое обоснование, приведены детали реализации. Для всех трех алгоритмов адаптации выполнены численные эксперименты в модели с определенными упрощающими допущениями. Получены ошибки решения для различных величин ускорения расчета; выявлен наиболее эффективный из рассмотренных алгоритмов. В результате удалось добиться существенного ускорения расчета при приемлемой величине ошибки, используя метод, простой с точки зрения реализации и с малыми вычислительными затратами. \\
	Исследованные методы адаптации шага по времени универсальны – они могут использоваться и в других моделях типа диффузной границы, основанных на уравнении Аллена–Кана. \\[2mm]
	Ключевые слова: модель типа диффузной границы, уравнение Аллена--Кана, адаптация шага по времени \par
	\vspace{5mm}
}

\begin{otherlanguage}{english}
{
	\centering
	{\large \textbf{\PreprintTitleEnglish} \par}
	\vspace{1mm}
	\PreprintAuthorsEnglish \par
	\vspace{1mm}
}
{
	\noindent \small
	The subject of the present paper is a diffuse interface model describing the development of an electrical breakdown channel. The model is based on the Allen--Cahn equation. A brief description of the model is given; the explicit finite difference scheme used is stated. Time scales of processes in the system vary significantly, thus the use of a calculation time step adaptation is appropriate. \\
	Three different adaptive time-stepping methods for the model are investigated. Two of them are previously proposed in articles on similar themes by other authors. For these methods, theoretical analysis is performed and a certain community is identified. The third method is a new one proposed by the authors, with its theoretical base and implementation details given. Computational experiments are performed for all the three adaptation algorithms, with some simplifying assumptions in the model. Solution errors are obtained for various acceleration values of the calculations. The most effective of the considered methods is exposed. As a result, a significant boost of the calculations with acceptable error values is achieved, using a method that is simple in terms of implementation and with low computational costs. \\
	The analysed adaptation methods can also be applied in other diffuse interface models based on Allen--Cahn equation -- in~this regard, the methods are universal. \\[2mm]
	Key words and phrases: diffuse interface model, Allen--Cahn equation, adaptive time-stepping method \par
	\vspace{2mm}
}
\end{otherlanguage}

%%%%%%%%%%%%%%%%%%%%%%%%%%%%%%%%%%%%%%%

%!TEX root = ../main.tex

\section{Введение}

\begin{frame}{Физическое явление}
\begin{block}{Электрический пробой}
	Явление резкого возрастания тока в диэлектрике при приложении электрического напряжения
	выше критического.
\end{block}
\begin{itemize}
	\item Рассматриваем твердый диэлектрик
	\item Деградация диэлектрических свойств материала
	\item Процесс развивается в ограниченной зоне -- канале пробоя
	\item Сложная физическая природа
\end{itemize}
\end{frame}


\begin{frame}{Математическая модель}
\begin{block}{Модель типа диффузной границы}
	Вещество находится в разных фазах. Состояние вещества описывается гладкой функцией
	$\phi(\vx, t)$ -- фазовым полем.
\end{block}
\begin{itemize}
	\item $\phi = 1$ -- неповрежденная среда
	\item $\phi = 0$ -- полностью разрушенная среда
	\item Зона $\phi \in (0, 1)$ -- диффузная граница
	\item На разрушение среды тратится энергия
\end{itemize}
\begin{figure}
	\includegraphics[width=0.5\textwidth]{figures/diffuse_edge.jpg}
\end{figure}
\end{frame}


\begin{frame}{Математическая модель}
Модель, предложенная в работе \cite{pitike_dielectric_breakdown}:
\begin{itemize}
	\item $\pi = \textcolor{red}{-\half \epsilon[\phi] (\nabla \Phi, \nabla \Phi)} +
	\Gamma \left( \cfrac{1 - f(\phi)}{l^2} + \cfrac{1}{4} (\nabla \phi, \nabla \phi) \right)$
	-- плотность свободной энергии
	\item $\Gamma$ -- энегрия роста канала пробоя на единицу длины
	\item $l$ -- величина <<размытия>> канала
	\item $\epsilon(\vx, t)$ -- диэлектрическая проницаемость среды
	\item $f(\phi)$ -- интерполирующая функция
\end{itemize}
\end{frame}


\begin{frame}{Математическая модель}
\vspace{-0.2cm}
\begin{itemize}
	\item $\epsilon(\vx, t) = \cfrac{\epsilon_0(\vx)}{f(\phi(\vx, t)) +
	\delta}$ -- диэлектрическая проницаемость среды
	\item $f(\phi) = 4 \phi^3 - 3 \phi^4$ -- интерполирующая функция
\end{itemize}
\begin{columns}
\column{0.5\textwidth}
\begin{figure}
	\hspace*{1.4cm}
	\includegraphics[width=0.65\textwidth]{figures/f_form.png}
\end{figure}
\column{0.5\textwidth}
\begin{figure}
	\hspace*{-2cm}
	\includegraphics[width=0.60\textwidth]{figures/eps_form.png}
\end{figure}
\end{columns}
\end{frame}


\begin{frame}{Математическая модель}
\vspace{-0.5cm}
\begin{block}{Уравнения модели}
\begin{itemize}
	\item Уравнение электрического потенциала $\Phi$:
	\begin{equation}
		\Div(\epsilon[\phi] \nabla \Phi) = 0
		\label{equation_potential}
	\end{equation}
	\item Уравнение фазового поля $\phi$:
	\begin{equation}
		\cfrac{1}{m} \partt{\phi} = \half \epsilon'(\phi) \gradscalsq{\Phi} + \cfrac{\Gamma}{l^2} f'(\phi) + \half \Gamma \Delta \phi
		\label{equation_phase}
	\end{equation}
\end{itemize}
\end{block}
Свойства:
\begin{itemize}
	\item связанная система уравнений на $\phi$ и $\Phi$;
	\item уравнение для $\phi$ типа Аллена--Кана, нелинейное.
\end{itemize}
\end{frame}


\begin{frame}{Пример вычислительного эксперимента}
\begin{columns}
\column{0.32\textwidth}
\begin{figure}
	\includegraphics[width=\textwidth]{figures/model_example_1.png}
\end{figure}
\column{0.32\textwidth}
\begin{figure}
	\includegraphics[width=\textwidth]{figures/model_example_2.png}
\end{figure}
\column{0.32\textwidth}
\begin{figure}
	\includegraphics[width=\textwidth]{figures/model_example_3.png}
\end{figure}
\end{columns}
\begin{center}
	Расчет из работы \cite{zipunova_experiment}
\end{center}
\end{frame}


\begin{frame}{Цель работы}
\begin{block}{Цель работы}
	Исследовать качественные характеристики системы уравнений \eqref{equation_potential},
	\eqref{equation_phase} и выполнить ее численный анализ.
\end{block}
Для этого рассмотрим задачу в определенных краевых условиях, упрощающих ее, но позволяющих
установить интересующие свойства.
\end{frame}

%!TEX root = ../main.tex

\section{Математическая модель}

Приведем краткое описание исследуемой математической модели. По\forcehyphenation дробное описание физического смысла уравнений и параметров модели можно найти в работе \cite{ponomarev_stability}.

Рассматривается ограниченная область пространства $\Omega \subset \Real^3$. Распределение фаз вещества в ней задается гладкой функцией $\phi: \Omega \times [0, +\infty)_t \hm \to [0, 1], \; \phi(\vx, t)$~-- фазовым полем; вещество может находиться в одной из двух фаз: $\phi \approx 1$~-- <<неповрежденное>>, $\phi \approx 0$~-- <<полностью разрушенное>> (то есть относящееся к каналу пробоя),~-- а также в промежуточных состояниях в зоне диффузной границы.

Диэлектрическая проницаемость среды $\epsilon$ задается следующей формулой:
\[\epsilon(\vx, t) = \epsilon[\phi] = \cfrac{\epsilon_0(\vx)}{f(\phi(\vx, t)) + \delta}.\]
Здесь $\epsilon_0(\vx)$~-- диэлектрическая проницаемость неповрежденной среды, $f(\phi) \hm = 4\phi^3 - 3\phi^4$~-- интерполирующая функция, $0 < \delta \ll 1$~-- регуляризующий параметр.

Помимо фазового поля $\phi$, состояние системы описывает функция $\Phi: \Omega \hm \times [0, +\infty)_t \to \Real, \; \Phi(\vx, t)$~-- потенциал электрического поля.

Постулируется следующее выражение для свободной энергии системы $\Pi$:
\begin{gather*}
	\Pi = \int \limits_\Omega \pi d \vx, \\
	\pi = -\half \epsilon[\phi] \gradsq{\Phi} + \Gamma \cfrac{1 - f(\phi)}{l^2} + \cfrac{\Gamma}{4} \gradsq{\phi}.
\end{gather*}
Здесь $\Gamma > 0$, $l > 0$~-- числовые параметры модели, константы.

Постулируются два уравнения, определяющие динамику системы:
\begin{equation*}
\begin{cases}
	\cfrac{\delta \Pi}{\delta \Phi} = 0; \\[3mm]
	\cfrac{1}{m} \partt{\phi} = -\cfrac{\delta \Pi}{\delta \phi}.
\end{cases}
\end{equation*}
Здесь константа $m > 0$~-- числовой параметр модели, называемый подвижностью. Говоря нестрого, согласно первому уравнению электрический потенциал $\Phi$ распределяется так, чтобы свободная энергия была минимальной; согласно второму~-- фазовое поле $\phi$ с определенной скоростью стремится к тому, чтобы свободная энергия была минимальной.

Отыскав явно вариационные производные в двух уравнениях выше, получим следующую систему уравнений:
\begin{numcases}{}
	\Div(\epsilon[\phi] \nabla \Phi) = 0;
	\label{eq:Phi} \\
	\cfrac{1}{m} \partt{\phi} = \half \epsilon'(\phi) \gradsq{\Phi} + \cfrac{\Gamma}{l^2} f'(\phi) + \half \Gamma \Delta \phi.
	\label{eq:phi}
\end{numcases}
Здесь $(\cdot)' \equiv (\cdot)_\phi'$. Система состоит из двух уравнений: на $\phi$ и $\Phi$ соответственно; система связная.

Уравнение \eqref{eq:phi} имеет следующий вид:
\[
	\cfrac{1}{m} \partt{\phi} = -F'(\phi; |\nabla \Phi|) + \half \Gamma \Delta \phi,
\]
где $F$ -- определенная нелинейная функция от $\phi$, которая к тому же зависит от $|\nabla \Phi|$ как от параметра. Таким образом, перед нами уравнение типа Аллена--Кана, нелинейное. \absent{Ссылка}

В классической постановке Аллена--Кана $F$ -- двухъямный потенциал. В рассмотриваемой задаче $F$ меняет поведение в зависимости от величины $|\nabla \Phi|$, как было показано в работе \cite{ponomarev_stability}. Возможно три случая в зависимости от величины
\[
	\xi = \cfrac{|\nabla \Phi|^2 l^2 \epsilon_0}{2 \Gamma},
\]
а именно:
\begin{itemize}
	\item \makebox[5.3cm][l]{<<слабое напряжение>>,} \makebox[4.0cm][l]{$\xi < \delta^2$:} $F(\phi)$ монотонно убывает;
	\item \makebox[5.3cm][l]{<<среднее напряжение>>,} \makebox[4.0cm][l]{$\delta^2 < \xi < (1 + \delta)^2$:} $F(\phi)$ выпукла вверх;
	\item \makebox[5.3cm][l]{<<сильное напряжение>>,} \makebox[4.0cm][l]{$\xi > (1 + \delta)^2$:} $F(\phi)$ монотонно возрастает.
\end{itemize}
Наибольший интерес для практики моделирования представляет случай \linebreak <<сильного напряжения>>, так как именно тогда канал пробоя развивается из сколь угодно малых возмущений неповрежденной среды.

\input{sources/difference_scheme}

%!TEX root = ../main.tex

\section{Адаптации по фазовому полю и по энергии}

\subsection{Формулировка методов}

Рассмотрим первые два подхода к адаптации шага по времени, предложенные в статьях \cite{li_time_step} и \cite{zhang_time_step} соответственно. Введем их вместе из-за определенной их общности:
\begin{align}
	\widetilde{\tau}_1^k & = \cfrac{tol_1}{\left\| \left[ \partt{\phi} \right]_h \right\|_C},
	\label{sch:time_step_phi} \\
	\widetilde{\tau}_2^k & = \cfrac{tol_2}{\left| \left[ \cfrac{d \Pi}{dt} \right]_h \right|}.
	\label{sch:time_step_energy}
\end{align}
Здесь $tol_1$ и $tol_2$ -- некоторые числовые константы, подбираемые на практике; символом $[\; \cdot \;]_h$ обозначены разностные производные.

В формуле \eqref{sch:time_step_phi} в качестве $[\partflt{\phi}]_h$ удобно использовать $[\partflt{\phi}]^{k + 1/2}$ из левой части разностного уравнения \eqref{eq:subtractive}. В этом случае
\[
	\left\| \left[ \partt{\phi} \right]_h^{k + 1/2} \right\|_C = \max\limits_{j = 0}^N \left| \left[ \partt{\phi} \right]_j^{k + 1/2} \right|.
\]
Если сделать этого не удается (например, из-за проблем с синхронизацией параллельных вычислений), то можно использовать $[\partflt{\phi}]^{k - 1/2}$, сохраненную с предыдущего шага.

В формуле \eqref{sch:time_step_energy} в знаменателе модуль производной полной энергии $\Pi(t)$. В силу вывода уравнений \eqref{eq:Phi},~\eqref{eq:phi} динамики системы, в адекватном расчете $[\partflt{\Pi}]_h$ либо отрицательна, либо крайне мала по модулю (сеточный эффект колебания системы вблизи минимума $\Pi$), поэтому, вообще говоря, вместо взятия модуля можно написать знак $-$.

Плотность энергии $\pi$ вычисляется из уравнения \eqref{eq:energy_density}, для чего необходима разностная производная $[\partflx{\phi}]_h$. Предлагается использовать следуюшие формулы:
\begin{gather*}
	\pi_j^k = F(\phi_j^k) + \cfrac{\Gamma}{4} \left( \left[ \partt{\phi} \right]_j^k \right)^2, \qquad \left[ \partt{\phi} \right]_j^k = \begin{cases}
		\cfrac{\phi_1^k - \phi_0^k}{\tau} & \text{при } j = 0; \\
		\cfrac{\phi_{j + 1}^k - \phi_{j - 1}^k}{2 \tau} & \text{при } j = \overline{1, N - 1}; \\
		\cfrac{\phi_N^k - \phi_{N - 1}^k}{\tau} & \text{при } j = N;
	\end{cases} \\
	\Pi^k = \cfrac{h \pi_0^k + h \pi_N^k}{2} + \sum\limits_{j = 1}^{N - 1} h \pi_j^k.
\end{gather*}
В формуле \eqref{sch:time_step_energy} будем использовать разностную производную энергии \linebreak с предыдущего шага
\[
	\left[ \cfrac{d \Pi}{dt} \right]_h^{k - 1/2} = \cfrac{\Pi^k - \Pi^{k - 1}}{\tau^{k - 1}}.
\]

Использование формулы \eqref{sch:time_step_phi} для расчета $\widetilde{\tau}^k$ будем условно называть адаптацией по фазовому полю, формулы \eqref{sch:time_step_energy} -- адаптацией по энергии.


\subsection{Связь с нормированием приращения фазового поля}

Общность описанных двух подходов и, возможно, ключ к их интуитивному пониманию заключается в следующем. Для адаптации по фазовому полю рассмотрим норму приращения $[d \phi]_h$:
\[
	\left\| [d \phi]_h^{k + 1/2} \right\|_C = \widetilde{\tau}_1^k \cdot \left\| \left[ \partt{\phi} \right]_h^{k + 1/2} \right\|_C = \cfrac{tol_1}{\left\| \left[ \partt{\phi} \right]_h^{k + 1/2} \right\|_C} \cdot \left\| \left[ \partt{\phi} \right]_h^{k + 1/2} \right\|_C = tol_1.
\]
Выходит нормирование приращения! (С оговоркой на ограничения $\tau_{min}$ и $\tau_{max}$.)

В случае адаптации по энергии можно провести похожее рассуждение. Из~вывода уравнения \eqref{eq:phi} верно следующее равенство:
\[
	\cfrac{d \Pi}{dt} = -\cfrac{1}{m} \int\limits_{\Omega} \left( \partt{\phi} \right)^2 d \vx,
\]
что для сеточных функций дает
\[
	\left| \left[ \cfrac{d \Pi}{dt} \right]_h \right| \approx \cfrac{1}{m} \left\| \left[ \partt{\phi} \right]_h \right\|_2^2.
\]
Таким образом, при адаптации по энергии
\[
	\left\| [d \phi]_h \right\|_2 = \widetilde{\tau}_2^k \cdot \left\| \left[ \partt{\phi} \right]_h^{k + 1/2} \right\|_2 \approx \cfrac{tol_2}{\cfrac{1}{m} \left\| \left[ \partt{\phi} \right]_h^{k - 1/2} \right\|_2^2} \cdot \left\| \left[ \partt{\phi} \right]_h^{k + 1/2} \right\|_2 \approx \cfrac{tol_2 \cdot m}{\left\| \left[ \partt{\phi} \right]_h \right\|_2}.
\]

Авторы не стали отклоняться от предложенного в работе \cite{zhang_time_step} метода адаптации, однако из проделанных рассуждений получается, что из модуля производной энергии в формуле \eqref{sch:time_step_energy} логичнее было бы извлечь квадратный корень, чтобы выполнялось $\left\| [d \phi]_h \right\|_2 \approx tol_2 \cdot \sqrt{m}$.

%!TEX root = ../main.tex

\section{Адаптация по устойчивости}

\subsection{Идея подхода}

В работе \cite{ponomarev_stability} получено следующее условие устойчивости разностной схемы~\eqref{sch:transition_old},~\eqref{sch:borders}:
\[
	\tau \leqslant \cfrac{1}{4m} \min \left( \cfrac{\delta^{5/3}}{|\nabla \Phi|^2 \epsilon_0}, \cfrac{h^2}{\Gamma} \right).
\]
Неравенство с первым аргументом минимума эквивалентно соотношению
\begin{equation}
	m \tau \max\limits_{\phi \in [0, 1]} |F''(\phi)| \leqslant C,
	\label{cond:stability_first}
\end{equation}
где $1 \geqslant C \approx 1.1 / 2$ -- константа, выбранная, во-первых, для создания <<запаса>> в оценке, во-вторых, для удобства формульной записи. Неравенство это, в свою очередь, получено применением для схемы спектрального признака устойчивости. Строго говоря, спектральный признак не дает достаточных условий устойчивости для нелинейных задач, однако на практике ее следует ожидать.

Основная идея подхода к адаптации, предлагаемого авторами в этом разделе, заключается в том, чтобы в неравенстве \eqref{cond:stability_first} заменить формальный максимум по $\phi \in [0, 1]$ на максимум по значениям сеточной функции $\phi_j^k$ и, естественно, взять наибольшее возможное $\tau$. Таким образом получается следующая формула адаптивного шага по времени:
\begin{equation}
	\widetilde{\tau}_3^k = \cfrac{tol_3}{m \cdot \max\limits_{j = 0}^N |F''(\phi_j^k)|}.
	\label{sch:time_step_stability_raw}
\end{equation}
Этот метод будем условно называть методом адаптации по устойчивости.

Идея описанного подхода подразумевает, что для корректной работы схемы должно быть достаточно $tol_3 = 1$, позволяя отказаться от подбора значения. При большей желаемой точности расчета можно провести подбор $tol_3 < 1$.

Однако формула \eqref{sch:time_step_stability_raw} в чистом виде имеет критический недостаток из-за вида функции $F''(\phi)$ и нуждается в доработке, которая будет проделана в следующем подразделе.


\subsection{Доработка метода}

Будем считать, что конфигурация модели относится к случаю, представляющему наибольший практический интерес, -- случаю <<сильного напряжения>> (см. \cite{ponomarev_stability}). Функция $F(\phi)$, заданная формулой \eqref{eq:allen_cahn_potential}, имеет на интервале $(0, 1)$ положительную производную, а значит, строго возрастает. Так как на $(0, 1)$ выполнено $f' > 0$, $\epsilon' < 0$, то $|f'| < |\epsilon'|$. Более того, вблизи точки $0$ верно $\epsilon \approx \epsilon_0 / \delta$, $\delta \ll 1$, то есть $\epsilon(\phi)$ вместе со своими производными много больше $f(\phi)$ с ее производными. Исходя из этого, поведение функции $F(\phi)$ определяется главным образом поведением функции $\epsilon(\phi)$.

В работе \cite{ponomarev_stability} проводится анализ функции $\epsilon(\phi)$ вблизи точки $\phi = 0$. Далее будет приведено определенное обобщение старых результатов, не слишком сложное, но полезное для глубокого понимания вопроса.

Приведем формулы для производных функций $f(\phi)$ и $\epsilon(\phi)$:
\[
	f(\phi) = 4 \phi^3 - 3 \phi^4, \quad f'(\phi) = 12 \phi^2 - 12 \phi^3, \quad f''(\phi) = 24 \phi - 36 \phi^2,
\]
откуда
\begin{gather}
	\epsilon'(\phi) = \epsilon'_f \cdot f' = \cfrac{-\epsilon_0 f'(\phi)}{(f(\phi) + \delta)^2},
	\label{eq:epsilon_phi} \\
	\epsilon''(\phi) = \epsilon''_{ff} \cdot (f')^2 + \epsilon'_f \cdot f'' = \epsilon_0 \cfrac{2 (f'(\phi))^2 - f''(\phi)(f(\phi) + \delta)}{(f(\phi) + \delta)^3}.
	\label{eq:epsilon_phi_phi}
\end{gather}

Рассмотрим замену переменной $\phi = \delta^{1/3} z$, $z \in [0, \delta^{-1/3}]$.

\begin{proposition}
	\label{prop:convergence}
	\begin{gather*}
		\cfrac{\delta \epsilon(\delta^{1/3} z)}{\epsilon_0} \to \cfrac{1}{4 z^3 + 1} = g(z), \qquad \delta^{4/3} \cfrac{\epsilon'(\delta^{1/3} z)}{\epsilon_0} \to \cfrac{-12 z^2}{(4 z^3 + 1)^2} = g'(z), \\
		\cfrac{\delta^{5/3} \epsilon''(\delta^{1/3} z)}{\epsilon_0} \to \cfrac{24z (8 z^3 - 1)}{(4 z^3 + 1)^3} = g''(z)
	\end{gather*}
	поточечно на луче $[0, +\infty)_z$ при $\delta \to +0$.
\end{proposition}

\begin{proposition}
	\label{prop:convergence_uniform}
	В утверждении \ref{prop:convergence} сходимость на отрезке $[0, \delta^{-1/3}]_z$ с подвижной правой границей равномерная, с порядком $\bigO(\delta^{1/3})$. \absent{Так ли это?}
\end{proposition}

Утверждения \ref{prop:convergence} и \ref{prop:convergence_uniform} позволяют записать приближенные представления $\epsilon(\phi) \approx \delta^{-1} g(\delta^{-1/3} \phi)$, $\epsilon'(\phi) \approx \delta^{-4/3} g'(\delta^{-1/3} \phi)$, $\epsilon''(\phi) \approx \delta^{-5/3} g(\delta^{-1/3} \phi)$. Отсюда становится совершенно ясным описанное в работе \cite{ponomarev_stability} убывание корней $\epsilon''$ с порядком $\delta^{1/3}$ и порядок $\delta^{-5/3}$ модулей экстремумов $\epsilon''$.

Функция $\epsilon(\phi)$ на отрезке $[0, 1]$ монотонно убывает; ее производная унимодальна: вначале убывает от $0$, затем возрастает до $0$; вторая производная имеет три промежутка роста: убывает, затем возрастает, затем убывает. $\epsilon''$ вблизи $0$ меняется очень быстро, достигает больших по модулю значений и всецело определяет поведение $F''$. $\epsilon''(\phi)$ имеет ноль $\phi_0 \sim 0.5 \cdot \delta^{1/3}$, а также локальный минимум и локальный максимум в точках
\[
	\phi_\pm \sim \cfrac{1}{\sqrt[3]{32 \pm 12 \sqrt{6}}} \cdot \delta^{1/3}
\]
соответственно, что следует из равномерной сходимости $\epsilon'' \rightrightarrows g''$.

По перечисленным выше причинам функция $F''(\phi)$ в формуле \eqref{sch:time_step_stability_raw} крайне неудобна: вблизи $0$ она достигает больших по модулю значений и к тому же имеет ноль, так что при взятии модуля в зоне больших значений возникает резкий <<провал>> до $0$. Чтобы решить проблему, мажорируем $|F''(\phi)|$ гладкой функцией, не имеющей такого недостатка.
\begin{multline*}
	\cfrac{\delta^{5/3} \epsilon''(\delta^{1/3} t)}{\epsilon_0} = \cfrac{24 (t^2 - \delta^{1/3} t^3) - (24 t - 36 \delta^{1/3} t^2)(4 t^3 - 3 \delta^{1/3} t^4 + 1)}{(4 t^3 - 3 \delta^{1/3} t^4 + 1)^3} = \\
	= \cfrac{12t (16 t^3 - 30 \delta^{1/3} t^4 + 15 \delta^{2/3} t^5 + 3t \delta^{1/3} - 2)}{(4 t^3 - 3 \delta^{1/3} t^4 + 1)^3}.
\end{multline*}
Поведение функции вблизи $0$ определяется слагаемым $-2 \cdot 12t$. Поменяем его знак и получим функцию
\begin{gather*}
	\widetilde{G}(t) = \cfrac{12t (16 t^3 - 30 \delta^{1/3} t^4 + 15 \delta^{2/3} t^5 + 3t \delta^{1/3} + 2)}{(4 t^3 - 3 \delta^{1/3} t^4 + 1)^3}; \\
	G(\phi) = \cfrac{|\nabla \Phi|^2 \epsilon_0}{2} \delta^{-5/3} \widetilde{G}(\delta^{-1/3} \phi) = |\nabla \Phi|^2 \epsilon_0 \cfrac{6 \phi (16 \phi^3 - 30 \phi^4 + 15 \phi^5 + 3 \delta \phi + 2 \delta)}{(4 \phi^3 - 3 \phi^4 + \delta)^3}.
\end{gather*}
$G(\phi) \geqslant 0$ на $[0, 1]$. Имеем $G(x) = |G(x)| \geqslant |F''(\phi)|$.

Исправим формулу \eqref{sch:time_step_stability_raw} в методе адаптации по устойчивости:
\begin{equation}
	\widetilde{\tau}_3^k = \cfrac{tol_3}{m \cdot \max\limits_{j = 0}^N G(\phi_j^k)}.
	\label{sch:time_step_stability}
\end{equation}

%!TEX root = ../main.tex

\section{Вычислительный эксперимент}

\subsection{Параметры модели, краевые условия}

Была создана программа, реализующая в рамках разностной схе\forcehyphenation мы \eqref{sch:borders},~\eqref{sch:transition},~\eqref{sch:time_step_min_max} перечисленные ранее алгоритмы адаптации временного шага: по фазовому полю \eqref{sch:time_step_phi}, по энергии \eqref{sch:time_step_energy} и по устойчивости \eqref{sch:time_step_stability}.

Будем использовать параметры модели, отражающие реальный физический эксперимент: см. табл. \ref{tab:parameters}. Часть параметров ($| \nabla \Phi|$, $\epsilon_0$) являются полноценными физическими величинами, часть ($\Gamma$, $m$) могут быть подобраны для согласования модели с результатами эксперимента. Чертой отделены параметры, которые либо происходят из связанных с диффузной границей допущений ($l$, $\delta$), либо описывают расчетную сетку.

\begin{table}[!t]
\captionsetup{justification=raggedright,singlelinecheck=false}
\caption[]{Параметры модели в расчете}
\centering
\begin{tabular}{|l|c|l|}
	\hline
	Название & Параметр & Значение \\
	\hline
	электрическое напряжение		& $|\nabla \Phi|$	& $5.625 \cdot 10^6 \; \unitV / \unitm$							\\
	энергия роста ед. длины канала	& $\Gamma$			& $8.118 \cdot 10^{-10} \; \unitJ / \unitm$						\\
	диэлектрическая проницаемость	& $\epsilon_0$		& $2.301 \cdot 10^{-11} \; \unitC^2 / (\unitJ \cdot \unitm)$	\\
	подвижность						& $m$				& $12 \; \unitm^3 / (\unitJ \cdot \units)$						\\
	\hline
	характерная толщина границы		& $l$ 				& $1.5 \cdot 10^{-6} \; \unitm$									\\
	регуляризующий параметр 		& $\delta$			& $10^{-3}$														\\
	размер образца					& $W$				& $3.2 \cdot 10^{-5} \; \unitm$									\\
	продолжительность опыта			& $T$				& $2 \cdot 10^{-3} \; \units$									\\
	шаг по пространству				& $h$				& $5 \cdot 10^{-7} \; \unitm$									\\
	минимальный шаг по времени		& $\taumin$		& $10^{-10} \; \units$											\\
	максимальный шаг по времени		& $\taumax$		& $\leqslant 6.42 \cdot 10^{-6} \; \units$						\\
	\hline
\end{tabular}
\label{tab:parameters}
\end{table}

Число узлов сетки по пространству $N_x \equiv M = 64$, по времени -- $312 \hm \leqslant N_t \leqslant 2 \cdot 10^7$.

При подстановке значений параметров в условие \eqref{cond:stability} устойчивости разностной схемы получаем $\tau \leqslant \min(2.86 \cdot 10^{-10}, 6.42 \cdot 10^{-6}) \, \units$. Неравенство с первым аргументом минимума является ограничением, происходящим из свойств функции $F(\phi)$ (см. формулу \eqref{cond:stability_first}). Оно может быть ослаблено в зависимости от значений $\phi_j^k$ на текущем временном слое. Неравенство со вторым аргументом носит безусловный характер. Поэтому качестве $\taumin$ взято значение~$6.42 \cdot 10^{-6} \; \units$; $\taumax = 10^{-10} \; \units$ удовлетворяет обоим неравенствам и может рассматриваться как шаг по времени до введения адаптации.

Зададим следующие краевые условия:
\begin{gather}
	\phi(0, t) = \phi(W, t) = 1, \qquad \phi(x, 0) = \phi_0(x),
	\label{exp:border} \\
	\phi_0(x) = \begin{cases}
		1 - 0.025 \cdot \left( 1 + \cos \left[ \cfrac{\pi}{0.08} \left( \cfrac{x}{W} - \half \right) \right] \right) \text{ при } \cfrac{x}{W} \in [0.42, 0.58]; \\
		1 \text{ иначе}.
	\end{cases}
	\nonumber
\end{gather}
Функция $\phi_0(x)$ отлична от $1$ в небольшой области вокруг $x = W / 2$, где <<прогибается>> как один период синусоиды, достигая минимума $\phi = 0.95$.


\subsection{Структура сетки с переменным шагом по времени}

Итак, на каждом временном слое используется своя величина шага~$\tau^k$. Таким образом, расчетная сетка теряет регулярность по времени, и сравнение разных решений по сеточной норме становится нетривиальной задачей. Введем у нерегулярной сетки определенную структуру, в которой сконцентрируем всю сложность вопроса, избегая при этом использования сеточной интерполяции для результатов расчетов.

Пусть $N = N_{t, max} = T / \taumin \in \Natural$, то есть временной промежуток $[0, T]$ разбит $N + 1$ узлом на $N$ равных отрезков длиной $\taumin$ каждый. Над этим разбиением введем структуру <<типа дерева отрезков>>. Говоря формально, будем считать допустимыми лишь разбиения вида $D = (0, p_1 \taumin, p_2 \taumin, \dots$, $p_{n - 1} \taumin, N \taumin)$, где $p_k \in \Natural_0$, $k = \overline{0, n}$, $p_k$ строго возрастают, $L_k = p_k - p_{k - 1} \hm = 2^{s_k}$, $s_k \in \Natural_0$, и к тому же $p_{k - 1} \divby L_k$.

Описанная структура замечательна тем, что если из любых двух допустимых разбиений $D_1$ и $D_2$ выбрать по интервалу, то либо эти интервалы не пересекаются, либо совпадают, либо один строго вложен в другой. Следовательно, любые два соседних узла объемлющего разбиения $D = D_1 \cap D_2$ (пересечение в смысле множеств) соседствуют также в $D_1$ или в $D_2$ -- в таком ключе $D$ оптимально.

При адаптации шага по времени в разностной схеме \eqref{sch:borders},~\eqref{sch:transition},~\eqref{sch:time_step_min_max} на слое $k$ будем использовать не рассчитываемое $\tau^k$ напрямую, а максимальное ${\tau'}^k = 2^s \taumin \leqslant \tau^k$, $s \in \Natural_0$, к тому же допустимое описанным разбиением <<типа дерева отрезков>> временного промежутка $T$ на $N$ отрезков, а именно:
\begin{gather*}
	p_0 = 0, \quad p_k = p_{k - 1} + 2^{s_k} \leqslant N, \; s_k \in \Natural_0; \\
	{\tau'}^k = 2^{s_k} \cdot \taumin \leqslant \tau^k, \quad p_{k - 1} \divby 2^{s_k}; \\
	s_k \to \max.
\end{gather*}

Для сравнения по равномерной норме $\enorm_{C, h}$ двух сеточных решений~$\phi_1$ и~$\phi_2$ на разбиениях $D_1$ и $D_2$ соответственно ограничим их оба на объемлющем разбиении $D = D_1 \cap D_2$.


\subsection{Результаты расчетов}

На рис. \ref{fig:solution_basic} изображен результат расчета с параметрами из табл. \ref{tab:parameters} и краевыми условиями \eqref{exp:border} для разностной схемы \eqref{sch:transition_old}, \eqref{sch:borders} без адаптации шага по времени. Видно, как из малого начального возмущения фазового поля $\phi$ постепенно растет канал электрического пробоя. В момент времени $t \approx 1.82 \cdot 10^{-3}$ в точке $x = W / 2$ фазовое поле $\phi$  становится мало отличимо от $0$ -- происходит <<пробой насквозь>>. Обратим внимание, что значение фазового поля упало от $\phi \approx 0.6$ до $\phi \approx 0$ менее чем за время $10^{-5}$, то есть $0.5 \%$ от всей продолжительности эксперимента. Далее канал пробоя растет в толщину примерно с постоянной скоростью.

\begin{figure}[!t]
	\centering
	\includegraphics[width=\textwidth]{figures/solution_basic.png}
	\vspace{-0.8cm}
	\caption{Решение задачи (расчет без адаптации)}
	\label{fig:solution_basic}
\end{figure}

Теперь проведем расчеты с той же конфигурацией системы, но используя схему \eqref{sch:borders}, \eqref{sch:transition}, \eqref{sch:time_step_min_max} с переменным шагом по времени для каждого из трех методов адаптации: по фазовому полю~\eqref{sch:time_step_phi}, по энергии~\eqref{sch:time_step_energy}, по устойчивости~\eqref{sch:time_step_stability}.

При слишком больших константах $\tol_1$, $\tol_2$, $\tol_3$ разностная схема закономерно теряет устойчивость и результаты расчетов оказываются неадекватны. В таких случаях либо значения $\phi_j^k$ уходят на бесконечность, либо на графиках явно прослеживаются колебания по узлам сетки.

Для первых двух методов адаптации были опытным путем подобраны минимальные значения констант, при которых описанный вычислительный эксперимент завершается успешно: $\tol_1 = 5 \cdot 10^{-4}$, $\tol_2 = 2 \cdot 10^{-7}$. Адаптация по устойчивости дает адекватный расчет сразу, при $\tol_3 = 1$, что соответствует идее метода.

Провести сравнение решений по стандартной сеточной равномерной норме~$\enorm_{C, h}$ не удается. При введении адаптивного временного шага решение разностной задачи начинает <<отставать>> от исходного -- так проявляет себя ошибка аппроксимации по времени. Так как канал пробоя развивается стремительно, то даже небольшое <<отставание>> приводит к тому, что норма разности решений становится порядка $1$ и не несет значимой информации.

Для сравнения решений будем использовать следующую величину:
\begin{gather*}
	\rho(\phi, \psi) = \max\limits_{k = 0}^n \rho(\phi, \psi; k), \\
	\rho(\phi, \psi; k) = \min\limits_{s = 0}^n \| \phi^k - \psi^s \|_{C, x} = \min\limits_{s = 0}^n \max\limits_{j = 0}^M |\phi_j^k - \psi_j^s|.
\end{gather*}
Формула означает, что каждому моменту времени $t_{1, k}$ первого расчета сопоставляется момент времени $t_{2, s}$ второго расчета, в который сеточное решение~$\psi_j^s$ наиболее близко к $\phi_j^k$ по пространственной равномерной норме.

В описанном выше смысле <<отставание>> $\phi_j^k$ от $\psi_j^s$ есть
\begin{gather*}
	\zeta(\phi, \psi) = \max\limits_{k = 0}^n \zeta(\phi, \psi; k), \\
	\zeta(\phi, \psi; k) = t_2 \left( \argmin\limits_{s = 0}^n \| \phi^k - \psi^s \|_{C, x} \right) - t_{1, k}.
\end{gather*}
Относительным отставанием будем называть величину $\zeta(\phi, \psi) / T$, где $T$ -- длительность эксперимента.

На рис. \ref{fig:peak_error} и \ref{fig:peak_lag} показаны графики отклонения $\rho(\phi, \psi; k)$ и отстава\forcehyphenation ния $\zeta(\phi, \psi; k)$ в зависимости от $t_k$ для расчетов с максимальным наблюдаемым ускорением для каждого из трех методов адаптации. Ускорением считается уменьшение числа временных шагов после введения адаптивного шага; $\phi$ обозначает расчет с адаптией, $\psi$ -- исходный, эталонный, расчет. При построении графика $\rho(\phi, \psi; k)$ был применен фильтр оконного минимума для избавления от сеточных артефактов, связанных, по-видимому, с используемым экономичным сохранением результатов расчетов. Такой прием уместен, поскольку порядок ошибки решения куда более важен, чем ее точное значение.

\begin{figure}[!tp]
	\centering
	\includegraphics[width=\textwidth]{figures/adaptation_peak_error.png}
	\vspace{-0.8cm}
	\caption{Отклонение $\rho(\phi, \psi; k)$ решения с адаптацией от исходного решения}
	\label{fig:peak_error}
	\vspace{1cm}

	\includegraphics[width=\textwidth]{figures/adaptation_peak_lag.png}
	\vspace{-0.8cm}
	\caption{Отставание $\zeta(\phi, \psi; k)$ решения с адаптацией от исходного решения}
	\label{fig:peak_lag}
\end{figure}

В табл. \ref{tab:results_max} перечислены значения $\rho(\phi, \psi)$ и $\zeta(\phi, \psi)$ для указанных выше расчетов. При вычислении максимума в формуле отклонения использовались значения $\rho(\phi, \psi; k)$ с упомянутым оконным фильтром.

В табл. \ref{tab:results_100} перечислены те же характеристики, но для расчетов с ускорением примерно в $100$ раз.

\begin{table}[!t]
\captionsetup{justification=raggedright,singlelinecheck=false}
\caption[]{Результаты расчетов с максимальным ускорением}
\centering
\begin{tabular}{|l|c|c|c|c|}
	\hline
	Тип адаптации & Ускорение (раз) & $\| \phi - \psi \|_{C, h}$ & $\rho(\phi, \psi)$ & $\zeta(\phi, \psi) / T$ \\
	\hline
	по фазовому полю	& $800$		& $0.65$	& $3.64 \cdot 10^{-4}$	& $0.29\%$	\\
	по энергии			& $107$		& $0.68$	& $5.38 \cdot 10^{-4}$	& $0.36\%$	\\
	по устойчивости		& $1474$	& $0.77$	& $1.51 \cdot 10^{-2}$	& $0.71\%$	\\
	\hline
\end{tabular}
\label{tab:results_max}
\end{table}

\begin{table}[!t]
\captionsetup{justification=raggedright,singlelinecheck=false}
\caption[]{Результаты расчетов с ускорением примерно в 100 раз}
\centering
\begin{tabular}{|l|c|c|c|c|}
	\hline
	Тип адаптации & Ускорение (раз) & $\| \phi - \psi \|_{C, h}$ & $\rho(\phi, \psi)$ & $\zeta(\phi, \psi) / T$ \\
	\hline
	по фазовому полю	& $101$	& $0.21$	& $1.23 \cdot 10^{-5}$	& $0.0043\%$	\\
	по энергии			& $101$	& $0.60$	& $3.25 \cdot 10^{-4}$	& $0.19\%$		\\
	по устойчивости		& $100$	& $0.24$	& $2.23 \cdot 10^{-5}$	& $0.0049\%$	\\
	\hline
\end{tabular}
\label{tab:results_100}
\end{table}

Согласно проведенному сравнению, лучше всех себя показал первый метод (он же самый простой) -- адаптация временного шага по фазовому полю. Она показывает высокое ускорение и наименьшую ошибку решения. У адаптации по энергии обе эти характеристики хуже. Адаптация по устойчивости уступает первой в точности, однако имеет определенные особые преимущества: наибольшее пиковое ускорение и возможность использования без подбора коэффициента $\tol_3$ (пусть и при низкой точности решения).

%!TEX root = ../main.tex

\section{Conclusions}

In this paper we study stability properties of the phase-field model
for electrical breakdown channel evolution.
The central result is a classification of the
equilibrium solutions of the model and their stability.
From practical point of view, these results allows to
make meaningful conclusions regarding qualitative and quantitative
properties of the model. Particularly it was shown under which
conditions small perturbations of the equilibrium solutions
develop into channel-like structure typical for of electrical breakdown
process.

Besides this, a simple explicit finite-difference scheme
for solution of the model in spatially one-dimensional setting is considered.
The main question addressed here are stability conditions which guaranties
correctness of the simulations. Deep connections between
stability conditions of the model and the one of the
finite-difference scheme are shown.
The presented results of the numerical simulations confirms
predictions of the theoretical analysis of the model.

% EOF
\endinput

\clearpage
\printbibliography[
	heading=bibintoc
]

\clearpage
\tableofcontents

\end{document}

%%%%%%%%%%%%%%%%%%%%%%%%%%%%%%%%%%%%%%%%%%%%%%%%%%%%%%%%%%%%%%%%%%%%%%%%%%%%%%%%