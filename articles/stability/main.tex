\documentclass[a4paper,12pt]{article}

%%% Поля
\usepackage[
	left=3cm,
	right=1.5cm,
	top=2cm,
	bottom=2cm,
	bindingoffset=0cm
]{geometry}

%%% Работа с русским языком
\usepackage{cmap}						% поиск в PDF
\usepackage{mathtext}					% русские буквы в формулах
\usepackage[T2A]{fontenc}				% кодировка
\usepackage[utf8]{inputenc}				% кодировка исходного текста
\usepackage[english,russian]{babel}		% локализация и переносы
\usepackage{indentfirst}
\frenchspacing

%%% Дополнительная работа с математикой
\usepackage{amsmath,amsfonts,amssymb,amsthm,mathtools}  % AMS

%%% Текст в колонки
\usepackage{multicol}

%%% Списки
\usepackage{enumitem}
\setlist{nosep, leftmargin=*}
\renewcommand{\labelenumi}{\arabic*)}

%%% Системы уравнений
\usepackage{cases}

%%% Рисунки
\usepackage{graphicx}
\usepackage{float}

%%% Точка в подписях к рисункам
\usepackage[labelsep=period]{caption}

%%% Список литературы
\bibliographystyle{bibliography_style/gost-numeric.bbx}
\usepackage[
	natbib = true,
	style = gost-numeric,
	sorting = none,
	backend = biber,
	language = autobib,
	autolang = other
]{biblatex}
\addbibresource{references.bib}

%%% Исправление символа номера при использовании gost-numeric.bbx
\usepackage{textcomp}
\DefineBibliographyStrings{russian}{number={\textnumero}}

%%% Гиперссылки
\usepackage[pdftex,unicode]{hyperref}

%%% Перенос знаков в формулах (по Львовскому)
\newcommand*{\hm}[1]{#1\nobreak\discretionary{}{\hbox{$\mathsurround=0pt #1$}}{}}

%%% Для настроек аннотации
\usepackage{abstract}


%%% Свои команды

\newcommand*{\No}{\textnumero}

\newcommand{\vect}[1]{\boldsymbol{#1}}
\newcommand{\vx}{{\vect{x}}}
\newcommand{\vn}{{\vect{n}}}

\newcommand{\half}{\cfrac{1}{2}}

\newcommand{\partt}[1]{\cfrac{\partial #1}{\partial t}}
\newcommand{\partx}[1]{\cfrac{\partial #1}{\partial x}}
\newcommand{\partxx}[1]{\cfrac{\partial^2 #1}{\partial x^2}}
\newcommand{\partvn}[1]{\cfrac{\partial #1}{\partial \vn}}

\newcommand{\partflt}[1]{\partial #1 / \partial t}
\newcommand{\partflx}[1]{\partial #1 / \partial x}
\newcommand{\partflxx}[1]{\partial^2 #1 / \partial x^2}
\newcommand{\partflvn}[1]{\partial #1 / \partial \vn}

\newcommand{\difftau}[1]{\cfrac{{#1}_j^{k + 1} - {#1}_j^k}{\tau}}
\newcommand{\diffhh}[1]{\cfrac{{#1}_{j + 1}^k - 2 {#1}_j^k + {#1}_{j - 1}^k}{h^2}}

\newcommand{\scalsq}[1]{\left( \nabla #1, \nabla #1 \right)}

\newcommand{\Natural}{{\mathbb{N}}}
\newcommand{\Real}{{\mathbb{R}}}
\newcommand{\bigO}{{\mathcal{O}}}
\newcommand{\clOmega}{{\overline{\Omega}}}

\newcommand{\norm}[1]{\| \, #1 \, \|}
\newcommand{\enorm}{{\| \cdot \|}}

\newcommand{\tpoint}{{\text{.}}}
\newcommand{\tcomma}{{\text{,}}}
\newcommand{\tsemicolon}{{\text{;}}}

\newcommand{\forcehyphenation}{-\linebreak}


%%% Свои операторы
\DeclareMathOperator{\Div}{{div}}


%%% Оформление теорем

\theoremstyle{plain}
\newtheorem{theorem}{Теорема}
\newtheorem{proposition}{Утверждение}

\theoremstyle{remark}
\newtheorem{remark}{Замечание}


%%% Пояснение к меткам
% eq	-- equation
% cond	-- condition
% char	-- characteristic
% sch	-- scheme
% est	-- estimation
% exp	-- experiment
% fig	-- figure
% tab	-- table
% sec	-- section


%%% Описание статьи
\newcommand{\ArticleTitle}{
	Устойчивость стационарных решений в модели развития канала электрического пробоя типа <<диффузной границы>>
}
\newcommand{\ArticleTitleEnglish}{
	Stability of stationary solutions of a diffuse interface model for the electrical breakdown process
}
\newcommand{\ArticleAuthors}{
	А.~С.~Пономарев\textsuperscript{1}, Е.~В.~Зипунова\textsuperscript{1}, Е.~Б.~Савенков\textsuperscript{1}
}
\newcommand{\ArticleAuthorsEnglish}{
	A.~S.~Ponomarev\textsuperscript{1}, E.~V.~Zipunova\textsuperscript{1}, E.~B.~Savenkov\textsuperscript{1}
}
\newcommand{\ArticleInstitutes}{
	\textsuperscript{1}Институт прикладной математики им. М. В. Келдыша РАН, Москва, Россия
}
\newcommand{\ArticleInstitutesEnglish}{
	\textsuperscript{1}Keldysh Institute of Applied Mathematics, Moscow, Russia
}

\title{
	\ArticleTitle
}
\author{
	\ArticleAuthors \\[3mm]
	{\normalsize \ArticleInstitutes}
}
\date{\vspace{-0.8cm}}


%%%%%%%%%%%%%%%%%%%%%%%%%%%%%%%%%%%%%%%%%%%%%%%%%%%%%%%%%%%%%%%%%%%%%%%%%%%%%%%%

\begin{document}

\maketitle

\renewcommand{\abstractname}{}
\renewcommand{\absnamepos}{empty}
\begin{abstract}
{
	\noindent Цель настоящей работы -- исследование качественных характеристик и численный анализ модели типа диффузной границы, описывающей развитие канала электрического пробоя в твердом диэлектрике. Проведен анализ устойчивости положений равновесия системы; установлены условия развития канала пробоя из малых возмущений неповрежденной среды. Построена и изучена разностная схема для задачи, дана содержательная оценка ее устойчивости. Полученные теоретические результаты подтверждены моделированием на компьютере. \\[3mm]
	Ключевые слова: модель типа диффузной границы, фазовое поле, устойчивость, электрический пробой. \\[5mm]
}
\begin{otherlanguage}{english}
	\ArticleTitleEnglish \\[3mm]
	\ArticleAuthorsEnglish \\[3mm]
	\ArticleInstitutesEnglish \\[3mm]
	The aim of the present work is to study qualitative characteristics and to perform a numerical analysis of a diffuse interface model describing the development of an electrical breakdown channel in a solid dielectric. Stability of the system equilibrium positions is analysed. Conditions for the breakdown channel development from small perturbations of the intact medium are found. A differential scheme for the problem is constructed and investigated, an informative estimate of its stability is given. The obtained theoretical results are validated by a computer simulation. \\[3mm]
	Key words and phrases: diffuse interface model, phase field, stability, electrical breakdown.
\end{otherlanguage}
\end{abstract}


%!TEX root = ../main.tex

\section{Введение}

\begin{frame}{Физическое явление}
\begin{block}{Электрический пробой}
	Явление резкого возрастания тока в диэлектрике при приложении электрического напряжения
	выше критического.
\end{block}
\begin{itemize}
	\item Рассматриваем твердый диэлектрик
	\item Деградация диэлектрических свойств материала
	\item Процесс развивается в ограниченной зоне -- канале пробоя
	\item Сложная физическая природа
\end{itemize}
\end{frame}


\begin{frame}{Математическая модель}
\begin{block}{Модель типа диффузной границы}
	Вещество находится в разных фазах. Состояние вещества описывается гладкой функцией
	$\phi(\vx, t)$ -- фазовым полем.
\end{block}
\begin{itemize}
	\item $\phi = 1$ -- неповрежденная среда
	\item $\phi = 0$ -- полностью разрушенная среда
	\item Зона $\phi \in (0, 1)$ -- диффузная граница
	\item На разрушение среды тратится энергия
\end{itemize}
\begin{figure}
	\includegraphics[width=0.5\textwidth]{figures/diffuse_edge.jpg}
\end{figure}
\end{frame}


\begin{frame}{Математическая модель}
Модель, предложенная в работе \cite{pitike_dielectric_breakdown}:
\begin{itemize}
	\item $\pi = \textcolor{red}{-\half \epsilon[\phi] (\nabla \Phi, \nabla \Phi)} +
	\Gamma \left( \cfrac{1 - f(\phi)}{l^2} + \cfrac{1}{4} (\nabla \phi, \nabla \phi) \right)$
	-- плотность свободной энергии
	\item $\Gamma$ -- энегрия роста канала пробоя на единицу длины
	\item $l$ -- величина <<размытия>> канала
	\item $\epsilon(\vx, t)$ -- диэлектрическая проницаемость среды
	\item $f(\phi)$ -- интерполирующая функция
\end{itemize}
\end{frame}


\begin{frame}{Математическая модель}
\vspace{-0.2cm}
\begin{itemize}
	\item $\epsilon(\vx, t) = \cfrac{\epsilon_0(\vx)}{f(\phi(\vx, t)) +
	\delta}$ -- диэлектрическая проницаемость среды
	\item $f(\phi) = 4 \phi^3 - 3 \phi^4$ -- интерполирующая функция
\end{itemize}
\begin{columns}
\column{0.5\textwidth}
\begin{figure}
	\hspace*{1.4cm}
	\includegraphics[width=0.65\textwidth]{figures/f_form.png}
\end{figure}
\column{0.5\textwidth}
\begin{figure}
	\hspace*{-2cm}
	\includegraphics[width=0.60\textwidth]{figures/eps_form.png}
\end{figure}
\end{columns}
\end{frame}


\begin{frame}{Математическая модель}
\vspace{-0.5cm}
\begin{block}{Уравнения модели}
\begin{itemize}
	\item Уравнение электрического потенциала $\Phi$:
	\begin{equation}
		\Div(\epsilon[\phi] \nabla \Phi) = 0
		\label{equation_potential}
	\end{equation}
	\item Уравнение фазового поля $\phi$:
	\begin{equation}
		\cfrac{1}{m} \partt{\phi} = \half \epsilon'(\phi) \gradscalsq{\Phi} + \cfrac{\Gamma}{l^2} f'(\phi) + \half \Gamma \Delta \phi
		\label{equation_phase}
	\end{equation}
\end{itemize}
\end{block}
Свойства:
\begin{itemize}
	\item связанная система уравнений на $\phi$ и $\Phi$;
	\item уравнение для $\phi$ типа Аллена--Кана, нелинейное.
\end{itemize}
\end{frame}


\begin{frame}{Пример вычислительного эксперимента}
\begin{columns}
\column{0.32\textwidth}
\begin{figure}
	\includegraphics[width=\textwidth]{figures/model_example_1.png}
\end{figure}
\column{0.32\textwidth}
\begin{figure}
	\includegraphics[width=\textwidth]{figures/model_example_2.png}
\end{figure}
\column{0.32\textwidth}
\begin{figure}
	\includegraphics[width=\textwidth]{figures/model_example_3.png}
\end{figure}
\end{columns}
\begin{center}
	Расчет из работы \cite{zipunova_experiment}
\end{center}
\end{frame}


\begin{frame}{Цель работы}
\begin{block}{Цель работы}
	Исследовать качественные характеристики системы уравнений \eqref{equation_potential},
	\eqref{equation_phase} и выполнить ее численный анализ.
\end{block}
Для этого рассмотрим задачу в определенных краевых условиях, упрощающих ее, но позволяющих
установить интересующие свойства.
\end{frame}

%!TEX root = main.tex

\section{Постановка задачи и модель}

\subsection{Математическая модель}

Приведем описание математической модели, предложенной в работе \cite{pitike_dielectric_breakdown}.

Итак, рассматривается ограниченная область пространства $\Omega \subset \Real^3$. Распределение фаз вещества в ней задается гладкой функцией $\phi: \Omega \times [0, +\infty)_t \to [0, 1], \; \phi(\vx, t)$ -- фазовым полем; вещество может находиться в одной из двух фаз: $\phi \approx 1$ -- <<неповрежденное>>, $\phi \approx 0$ -- <<полностью разрушенное>> (то есть относящееся к каналу пробоя), -- а также в промежуточных состояниях в зоне диффузной границы.

Диэлектрическую проницаемость среды $\epsilon$ предлагается описать следующей формулой:
\begin{equation}
    \epsilon(\vx, t) = \epsilon[\phi] = \cfrac{\epsilon_0(\vx)}{f(\phi(\vx, t)) + \delta} \tpoint
    \label{eq:epsilon}
\end{equation}
Здесь $\epsilon_0(\vx)$ -- диэлектрическая проницаемость неповрежденной среды; $f(\phi) = 4\phi^3 - 3\phi^4$ -- интерполирующая функция, гладко соединяющая значения $0$ и $1$ ($f(0) = 0, \; f(1) = 1, \; f'(0) = f'(1) = 0$); $0 < \delta \ll 1$ -- регуляризующий параметр. Обратим внимание, что при $\phi = 1 \;\; \epsilon(\vx, t) \approx \epsilon_0(\vx)$, что соответствует диэлектрику; при $\phi = 0 \;\; \epsilon(\vx, t) = \epsilon_0(\vx) / \delta$ (в $\delta^{-1} \gg 1$ раз больше), что соответствует проводнику.

Помимо фазового поля $\phi$, состояние системы описывает также функция $\Phi: \Omega \times [0, +\infty)_t \to \Real, \; \Phi(\vx, t)$ -- потенциал электрического поля.

Постулируется следующее выражение для свободной энергии:
\begin{equation}
    \Pi = \int \limits_\Omega \pi d \vx \tcomma
    \label{eq:free_energy}
\end{equation}
\begin{equation}
    \pi = -\cfrac{1}{2} \epsilon[\phi] (\nabla \Phi, \nabla \Phi) + \Gamma \cfrac{1 - f(\phi)}{l^2} + \cfrac{\Gamma}{4} (\nabla \phi, \nabla \phi) \tpoint
    \label{eq:free_energy_density}
\end{equation}
Здесь $\Gamma > 0, \; l > 0$ -- числовые параметры модели, константы.

Постулируются два уравнения, определяющие динамику системы:
\begin{equation*}
\begin{cases}
    \cfrac{\delta \pi}{\delta \Phi} = 0 \tsemicolon \\
    \cfrac{1}{m} \partt{\phi} = -\cfrac{\delta \pi}{\delta \phi} \tpoint
\end{cases}
\end{equation*}
Здесь константа $m > 0$ -- числовой параметр модели, называемый подвижностью: она имеет смысл скорости изменения $\phi$ под действием единичной <<приложенной силы>>. Говоря нестрого, согласно первому уравнению электрический потенциал $\Phi$ распределяется так, чтобы свободная энергия системы была минимальной; согласно второму -- фазовое поле $\phi$ с определенной скоростью стремится к тому, чтобы свободная энергия была минимальной.

Отыскав явно вариационные производные в двух уравнениях выше, получим следующую систему уравнений:
\begin{numcases}{}
    \Div(\epsilon[\phi] \nabla \Phi) = 0 \tsemicolon
    \label{eq:Phi} \\
    \cfrac{1}{m} \partt{\phi} = \half \epsilon'(\phi) (\nabla \Phi, \nabla \Phi) + \cfrac{\Gamma}{l^2} f'(\phi) + \cfrac{1}{2} \Gamma \triangle \phi \tpoint
    \label{eq:phi}
\end{numcases}
Здесь $(\cdot)' \equiv (\cdot)_\phi'$. Система состоит из двух уравнений: на $\phi$ и $\Phi$ соответственно; система связная, второе уравнение нелинейное, является уравнением типа Аллена--Кана.

В таблице \ref{tab:quantities}, перечислены названия и размерности величин, встречающихся в описанной модели.

\begin{table}[!t]
\captionsetup{justification=raggedright,singlelinecheck=false}
\caption[]{Величины, относящиеся к модели.}
\centering
\begin{tabular}{|c|c|m{11cm}|}
    \hline
    Величина & Размерность & Название либо физический смысл \\
    \hline \hline
    $\phi(\vx, t)$ & $1$ & фазовое поле \\
    \hline
    \rule{0mm}{\tabletopspace}
    $\Phi(\vx, t)$ & $\cfrac{\unitJ}{\unitC}$ & потенциал электрического поля \\[\tablebottomspace]
    \hline
    $\Pi(t)$ & $\unitJ$ & свободная энергия \\
    \hline
    \rule{0mm}{\tabletopspace}
    $\pi(\vx, t)$ & $\cfrac{\unitJ}{\unitm^3}$ & плотность свободной энергии \\[\tablebottomspace]
    \hline
    \rule{0mm}{\tabletopspace}
    $\epsilon(\vx, t)$ & $\cfrac{\unitF}{\unitm} = \cfrac{\unitC^2}{\unitJ \cdot \unitm}$ & диэлектрическая проницаемость среды \\[\tablebottomspace]
    \hline
    \rule{0mm}{\tabletopspace}
    $\epsilon_0(\vx)$ & $\cfrac{\unitF}{\unitm} = \cfrac{\unitC^2}{\unitJ \cdot \unitm}$ & диэлектрическая проницаемость неповрежденной среды \\[\tablebottomspace]
    \hline
    $\delta$ & $1$ & определяет диэлектрическую проницаемость полностью разрушенной среды: она равна $\epsilon_0(\vx)/\delta$ \\
    \hline
    $l$ & $\unitm$ & характерная толщина диффузной границы \\
    \hline
    \rule{0mm}{\tabletopspace}
    $\Gamma$ & $\cfrac{\unitJ}{\unitm}$ & характерная энергия образования единицы длины канала пробоя \\[\tablebottomspace]
    \hline
    \rule{0mm}{\tabletopspace}
    $m$ & $\cfrac{\units \cdot \unitJ}{\unitm^3}$ & подвижность фазового поля $\phi$ \\[\tablebottomspace]
    \hline
\end{tabular}
\label{tab:quantities}
\end{table}


\subsection{Одномерная задача}

Рассмотрим систему со следующими ограничениями. Пусть $\Omega = [0, w]_x \times [0, h]_y \times I_z$, где $w, h > 0, \; I$ -- некоторый отрезок; $\phi(\vx, 0) = \phi_0(\vx) = \phi_0(x),$ $\epsilon_0(\vx) = \epsilon_0(x)$, то есть начальное распределение фаз и диэлектрическая проницаемость неповрежденной среды зависят только от $x$. На границе $\Omega$ будем считать заданным граничное условие $\Phi|_{y = 0} = \Phi^- \in \Real, \; \Phi|_{y = h} = \Phi^+ \in \Real, \; \Phi^- \leqslant \Phi^+$. Такую систему можно представить себе как двумерный (тривиально растянутый по третьему измерению) прямоугольный конденсатор, у которого сверху и снизу обкладки с постоянным электрическим потенциалом, между ними -- диэлектрик, меняющий свойства только по горизонтали.

Попробуем искать решение системы уравнений \eqref{eq:Phi}, \eqref{eq:phi}, имеющее $\phi(\vx, t) = \phi(x, t)$, то есть полагая, что $\phi$ не будет зависеть от $y$ и $z$.

Преобразуем уравнение \eqref{eq:Phi}:
\begin{equation}
    0 = \Div(\epsilon[\phi] \nabla \Phi) = (\nabla \epsilon, \nabla \Phi) + \epsilon \triangle \Phi = \partx{\epsilon} \partx{\Phi} + \epsilon \triangle \Phi \tpoint
    \label{eq:Phi_one_dim}
\end{equation}
Заметим, что независимо от конкретных $\phi$ и $\epsilon_0$ решением является $\Phi(\vx, t) = \Phi^- + (y/h)(\Phi^+ - \Phi^-)$. В этом случае $\partial \Phi / \partial x \equiv 0, \; \triangle \Phi \equiv 0$ и уравнение \eqref{eq:Phi_one_dim} становится тождеством.

Преобразуем уравнение \eqref{eq:phi}:
\begin{equation}
    \cfrac{1}{m} \partt{\phi} = \half \epsilon'(\phi) \left( \cfrac{\Phi^+ - \Phi^-}{h} \right)^2 + \cfrac{\Gamma}{l^2} f'(\phi) + \half \Gamma \partxx{\phi} \tcomma
    \label{eq:one_dim}
\end{equation}
$\phi_0(\vx) = \phi_0(x)$. Решение этого уравнения с начальным условием $\phi_0$ будет зависеть только от $x$ и времени $t$.

Уравнение $\eqref{eq:one_dim}$ можно рассматривать как дифференциальное уравнение в частных производных на функцию $\phi(x, t)$ одной пространственной переменной и решать его на отрезке $[0, w]_x$ числовой прямой.

Для удобства введем $K_\Phi = \|\nabla \Phi\| = (\Phi^+ - \Phi^-)/h$, тогда уравнение \eqref{eq:one_dim} примет вид
\begin{equation}
    \cfrac{1}{m} \partt{\phi} = \half K_\Phi^2 \epsilon'(\phi) + \cfrac{\Gamma}{l^2} f'(\phi) + \half \Gamma \partxx{\phi} \tpoint
    \label{eq:one_dim_simpler}
\end{equation}
$\Phi^+, \Phi^-$ и $h$ перестали входить в уравнение явно -- так мы убрали последнее упоминание о втором (по $y$) измерении пространства.

Для простоты анализа везде далее будем считать $\epsilon_0$ константой.

Итак, пара из решения уравнения \eqref{eq:one_dim_simpler} и $\Phi = \Phi^- + (y/h)(\Phi^+ - \Phi^-)$ является решением исходной системы уравнений \eqref{eq:Phi}, \eqref{eq:phi}, в том случае если краевые условия имеют описанный вид.

%!TEX root = ../main.tex

\section{Stability analysis of equilibrium solutions}
\label{sec:theoretical_analysis}

Under certain conditions, the electrical breakdown can develop
form small perturbations of the undamaged medium properties.
To clarify these conditions in this section we study
stability of constant solutions~$\phi(x, t) \equiv C, \; C \in [0, 1]$,
of the equation~\eqref{eq:one_dim}.

First, one has to find stationary constant solutions of~\eqref{eq:one_dim}.
From the definition~\eqref{eq:epsilon} follows the expressions for the derivatives
of~$\epsilon(\phi)$:
\begin{equation}
	\epsilon'(\phi) = \cfrac{-\epsilon_0 f'(\phi)}{(f(\phi) + \delta)^2} \tsemicolon \qquad \epsilon''(\phi) = \epsilon_0 \cfrac{2 (f'(\phi))^2 - f''(\phi)(f(\phi) + \delta)}{(f(\phi) + \delta)^3} \tpoint
	\label{eq:epsilon_derivatives}
\end{equation}

Substituting~$\phi(x, t) \equiv C$ into~\eqref{eq:one_dim} and
taking~\eqref{eq:epsilon_derivatives} into account, one has:
\begin{equation}
	f'(C) \left( \cfrac{\Gamma}{l^2} - \half K_\Phi^2 \cfrac{\epsilon_0}{(f(C) + \delta)^2} \right) = 0 \tpoint
	\label{eq:equilibrium}
\end{equation}
First, consider the case~$f'(C) = 12C^2 (1 - C) = 0$,
which leads to~$C = 0,1$. Hence, $\phi \equiv 0$ and $\phi \equiv 1$ are
equilibrium solutions.

Second, let~$C \neq 0, 1$. Then
\[
\cfrac{\Gamma}{l^2} = \cfrac{K_\Phi^2 \epsilon_0}{2 (f(C) + \delta)^2}
\quad \text{and} \quad
f(C) + \delta = K_\Phi l \sqrt{\cfrac{\epsilon_0}{2 \Gamma}}.
\]

Note that~$f(C) \in [0, 1]$ and, moreover, $f(\phi)$ is monotonically
increasing. Therefore, in case $K_\Phi l \sqrt{\epsilon_0 / (2
  \Gamma)} \in (\delta, 1 + \delta)$ the
equation~\eqref{eq:equilibrium} has a solution $C_3\ne 0,1$ given by
\begin{equation}
	C_3 = f^{-1} \left( K_\Phi l \sqrt{\cfrac{\epsilon_0}{2 \Gamma}} - \delta \right) \tpoint
	\label{eq:equilibrium_third}
\end{equation}
Otherwise the equation~\eqref{eq:equilibrium} has only two solutions.

So, the number of constant equilibrium solutions depends on the following condition
being satisfied:
\begin{equation}
	\delta^2 < \cfrac{K_\Phi^2 l^2 \epsilon_0}{2 \Gamma} < (1 + \delta)^2 \tpoint
	\label{cond:equilibriums_number}
\end{equation}
It will be shown later how the
condition~\eqref{cond:equilibriums_number} is connected with the
stability properties of the equilibrium solutions and the
equation~\eqref{eq:one_dim} itself.

Let us now procees to the stability analysis of the equilibrium solutions.

Let~$\phi(x, t)$ be a solution of~\eqref{eq:one_dim}, $\delta \phi(x,
t)$ be its perturbation.
Writing down the equation~\eqref{eq:one_dim} for the perturbed solution~$\phi
+ \delta \phi$,
after linearizing we obtain the following equation for~$\delta\phi$:
\begin{equation}
  \cfrac{1}{m} \partt{(\delta \phi)} = \left(\half K_\Phi^2 \epsilon''(\phi) + \cfrac{\Gamma}{l^2} f''(\phi) \right) \delta \phi + \half \Gamma \partxx{(\delta \phi)} \tpoint
  \label{eq:variation}
\end{equation}
For further analysis it is convenient to write~\eqref{eq:variation} as:
%
\begin{equation}
  \partt{(\delta \phi)} = A \delta \phi + B \partxx{(\delta \phi)} \tcomma
  \label{eq:variation_common}
\end{equation}
where~$A$ and~$B  > 0$ are the respective parameters.

Choosing~$\delta \phi = e^{\alpha t} \sin(\omega x)$, one obtains from~\eqref{eq:variation_common}
the following relation for the parameters of the perturbation:
$$\alpha e^{\alpha t} \sin(\omega x) = A e^{\alpha t} \sin(\omega x) - B \omega^2 e^{\alpha t} \sin(\omega x) \tcomma$$
from where follows:
\begin{equation}
  \alpha = A - B \omega^2 \tpoint
  \label{eq:exponent_coefficient}
\end{equation}

% Summing up, let us combine now the three parts of the reasoning.
% Consider equilibrium solution~$\phi \equiv C$ of the
% equation~\eqref{eq:one_dim} perturbed by~$\delta \phi$.

% ; применим к $\delta \phi$ уравнение \eqref{eq:variation}, в
% $\epsilon''$ и $f''$ подставим $\phi = C$. Полученное уравнение
% имеет вид~\eqref{eq:variation_common}.

Now it is easy to see that, depending on the value of the coefficient
$$A = \half K_\Phi^2 \epsilon''(C) + \cfrac{\Gamma}{l^2} f''(C) \tcomma$$
three cases arise:
%
\begin{enumerate}[label=\arabic*.]
\item $A > 0$. In this case, from~$\omega^2 \in [0, A / B)$ it follows
  that~$\alpha > 0$, i.e., there exists a perturbation~$\delta \phi$
  growing in time. Hence, the equilibrium solution~$\phi \equiv C$ is
  unstable.
  
\item $A < 0$. Then, for an arbitrary~$\omega$ remains~$\alpha
  \leqslant A < 0$. Next, any perturbation~$\delta \phi$ in the
  interval~$[0, W]_x$ 
  can be represented as Fourier integral over harmonics decreasing
  at least as the harmonic for~$\omega = 0$.
  Hence, the equilibrium solution~$\phi \equiv C$ is stable.
  
\item $A = 0$. Repeating the same reasoning as in the case~$A < 0$,
  one can observe that there exist arbitrarily slowly decreasing
  harmonics (i.e., harmonics with arbitrarily small values of~$\alpha$).
  This case corresponds to neutral stability of the equilibrium
  solution, and linear analysis does not provide complete
  information.
  This case will be considered in more details later.
\end{enumerate}

We now proceed to the discussion of the particular equilibrium states.

Consider the equilibrium solution~$\phi \equiv 0$.
One has~$f''(0) = 0$, $\epsilon''(0) = 0$
(see~\eqref{eq:epsilon_derivatives}),
which leads to~$A = 0$.
As it was noted before, this case requires an elaborate analysis, which will be
performed later.

Consider the equilibrium solution~$\phi \equiv 1$.
In this case~$f''(0) = -12$, $\epsilon''(0) = 12 \epsilon_0 / (1 +
\delta)^2$ (see~\eqref{eq:epsilon_derivatives}).
As a result, we obtain:
$$A = \half K_\Phi^2 \epsilon''(C) + \cfrac{\Gamma}{l^2} f''(C) = \cfrac{6 K_\Phi^2 \epsilon_0}{(1 + \delta)^2} - \cfrac{12 \Gamma}{l^2} \tpoint$$
The equilibrium state is stable if~$A < 0$, i.e., as:
\begin{equation}
  \cfrac{K_\Phi^2 l^2 \epsilon_0}{2 \Gamma} < (1 + \delta)^2 \tpoint
  \label{cond:equilibrium_1_stable}
\end{equation}      
For this case of an unstable equilibrium, let us find~$\omega_0$ such that
increasing harmonics are replaced by the decreasing ones.
To do this, consider~\eqref{eq:exponent_coefficient} with~$\alpha =
0$, $B = \Gamma/2$ and~$A$ given above to obtain: 
$$0 = \cfrac{6 K_\Phi^2 \epsilon_0}{(1 + \delta)^2} - \cfrac{12 \Gamma}{l^2} - \cfrac{\Gamma}{2} \omega_0^2 \tcomma$$
from where follows:
$$\omega_0 = 2 \sqrt{\cfrac{3 K_\Phi^2 \epsilon_0}{\Gamma (1 + \delta)^2} - \cfrac{6}{l^2}} \tpoint$$

Note that the condition~\eqref{cond:equilibrium_1_stable} is exactly the
right-hand side of the inequality~\eqref{cond:equilibriums_number}.
To explain this and to form a complete picture of what is happening,
let us look at the equilibrium solutions from a slightly different
perspective.

Solving the equation~\eqref{eq:equilibrium}, we were finding the zeros of
the function
\begin{equation}
	\chi(\phi) = \half K_\Phi^2 \epsilon'(\phi) + \cfrac{\Gamma}{l^2} f'(\phi) \tpoint
	\label{eq:equilibruim_characteristic}
\end{equation}
Hence, each equilibrium solution $\phi \equiv C$ uniquely
corresponds to a zero~$C$ of the function~$\chi(\phi)$.
From the derivation of the equation~\eqref{eq:variation} for the perturbation
it follows that in its right-hand side the coefficient at~$\delta
\phi$ is~$\chi'(\phi)$.
Later, analyzing the equation~\eqref{eq:variation_common} for an equilibrium
solution~$\phi \equiv C$, we considered several cases depending on the
sign of the coefficient~$A$, which turns out to be exactly~$\chi'(C)$.


Summing up the results, one can state the following.
The function~$\chi(\phi)$ defined by~\eqref{eq:equilibruim_characteristic}
is smooth on~$[0, 1]$ and always has zeros~$C_1=0$ and~$C_2=1$.
The third zero~$C=C_3\in (0, 1)$ exists under the
condition~\eqref{cond:equilibriums_number}.
Each equilibrium solution~$\phi \equiv C$ uniquely corresponds to a zero
of the function~$\chi(\phi)$. Their stability properties are
described in terms of the sign of~$\chi'(\phi)$ at the zeros: positive
values of~$\chi'$
correspond to the unstable solution and negative ones~---
to the stable one.

It is also clear that in the case of vanishing~$\chi'$
(as for~$\phi = 0$) the linear analysis is not enough~---
it is necessary to analyse the sign of the first higher-order
non-vanishing
derivative of~$\chi$~--- the equilibrium solution is stable if
this derivative is negative and unstable if it is positive.


Finally we show that~$\chi(\phi)$ has a non-vanishing derivative at its
zero~$C=C_3 \in (0, 1)$ (if the latter exists).
Indeed, one has:
$$\chi(C_3) = f'(C_3) \left( \cfrac{\Gamma}{l^2} - \cfrac{K_\Phi^2 \epsilon_0}{2 (f(C_3) + \delta)^2} \right) = 0 \tpoint$$
Taking into account that~$f'(C_3) \ne 0$, one obtains:
$$\cfrac{\Gamma}{l^2} - \cfrac{K_\Phi^2 \epsilon_0}{2 (f(C_3) + \delta)^2} = 0 \tpoint$$
Then:
$$\chi'(\phi)|_{C_3} = f'(C_3) \left( \cfrac{\Gamma}{l^2} - \cfrac{K_\Phi^2 \epsilon_0}{2 (f(\phi) + \delta)^2} \right) ' \bigg|_{C_3} = (f'(C_3))^2 \cfrac{K_\Phi^2 \epsilon_0}{(f(C_3) + \delta)^3} \ne 0 \tpoint$$

Now it is possible to provide a comprehensive analysis of the behavior
of~$\chi(\phi)$ at its zeros. As it can be seen from the conditions~\eqref{cond:equilibriums_number} and~\eqref{cond:equilibrium_1_stable},
its behavior is governed by the value of the parameter
\begin{equation}
  \xi = \cfrac{K_\Phi^2 l^2 \epsilon_0}{2 \Gamma} \tpoint
  \label{char:equilibriums}
\end{equation}

First, consider the case~$0 \leqslant \xi < \delta^2$. The zeros
of~$\chi(\phi)$ are~$0$ and~$1$; $\chi'(0) = 0$, $\chi'(1) < 0$.
The qualitative behavior of~$\chi(\phi)$ is shown schematically on
Fig.~\ref{fig:equilibriums_case_1}.  It can be seen that the equilibrium
solution~$\phi \equiv 0$ is unstable and~$\phi \equiv 1$ is stable.
Such case can be conventionally called the case of ``weak electric field''.
This means that with all the parameters except the electric field being fixed,
the latter is so small that even an almost completely damaged
medium with~$\phi \approx 0$ is ``healed'' over time and evolves to the
completely undamaged state~$\phi \approx 1$.

Second, consider the case~$\delta^2 < \xi < (1 + \delta)^2$.
The zeros of~$\chi(\phi)$ are $C=0$, $C=C_3$
(see~\eqref{eq:equilibrium_third}) and $C=1$; $\chi'(0) = 0, \;
\chi'(1) < 0; \; \chi'(C_3) > 0$
(since~$\chi$ is smooth).
The behavior of~$\chi(\phi)$ in this case is shown on
Fig.~\ref{fig:equilibriums_case_2}.
The equilibrium solutions are: $\phi \equiv 0$~--- stable one, $\phi
\equiv C_3$~--- unstable one, and~$\phi \equiv 1$~--- also stable.
Such case can be conventionally called the case of the ``medium
electric field''.
This means that as the values of~$\phi$ are sufficiently close to~$0$,
the damage increases, i.e.,~$\phi$ tends to zero;
as the values of~$\phi$ are sufficiently close to~$1$,
the damage decreases, i.e.,~$\phi$ tends to one;
at certain intermediate values the equilibrium is unstable.

Finally, consider the case~ $(1 + \delta)^2 < \xi$.
The zeros of~$\chi(\phi)$ are~$C=0$ and~$C=1$; $\chi'(0) = 0$, $\chi'(1)>0$.
The qualitative behavior of~$\chi(\phi)$ is schematically shown on
Fig.~\ref{fig:equilibriums_case_3}.
The equilibrium solutions are: $\phi \equiv 0$~--- the stable one, $\phi
\equiv 1$~--- the unstable one.
This case can be conventionally called the case of ``strong
electric field''.
This means that the electric field is sufficiently strong and
any state arbitrarily close to the completely undamaged one
(i.e., any state close to~$\phi \approx 1$) evolves towards
the completely damaged state~$\phi= 0$.
Essentially this is the case where the completely damaged state develops
from arbitrarily small perturbations of the completely undamaged
equilibrium solution.

In all the three cases stability of the equilibrium solution~$\phi \equiv
0$ is defined by the higher order derivatives of~$\chi(\phi)$.

\begin{figure}[!tp]
  \centering
  \includegraphics[width=0.84\textwidth]{figures/equilibriums_case_1.png}
  \vspace{-0.3cm}
  \caption{Characteristic behavior of~$\chi(\phi)$,
    ``weak electric field'' case.}
  \label{fig:equilibriums_case_1}
  \vspace{0.7cm}
  
  \includegraphics[width=0.84\textwidth]{figures/equilibriums_case_2.png}
  \vspace{-0.3cm}
  \caption{Characteristic behavior of~$\chi(\phi)$,
    ``medium electric field'' case.}
  \label{fig:equilibriums_case_2}
  \vspace{0.7cm}
  
  \includegraphics[width=0.84\textwidth]{figures/equilibriums_case_3.png}
  \vspace{-0.3cm}
  \caption{Characteristic behavior of~$\chi(\phi)$,
    ``strong electric field'' case.}
  \label{fig:equilibriums_case_3}
\end{figure}

% EOF

%!TEX root = ../main.tex

\section{The finite-difference scheme}
\label{sec:differential_scheme}

In this section, we present a finite-difference scheme for solving
the equation~\eqref{eq:one_dim} in the domain~$[0, W]_x \times [0,
+\infty)_t$. The equation is
subjected to initial conditions~\eqref{eq:one_dim_initial} and boundary conditions~\eqref{eq:one_dim_marginal}.

Consider a regular mesh with a time step~$\tau$ and
spatial step~$h$. Let~$W = Nh$ with $N$ being the number of
nodes. The nodes of the spatiotemporal grid are given by~$(jh, k \tau)$,
$j = \overline{0, N}$, $k \in \Natural_0$. Define by~$\phi_j^k$
the value of a mesh function~$\phi$ at the node~$(jh, k \tau)$.
Then the finite-difference approximations read
\begin{equation}
  \cfrac{1}{m} \difftau{\phi} = \half K_\phi^2 \epsilon'(\phi_j^k) + \cfrac{\Gamma}{l^2} f'(\phi_j^k) + \cfrac{\Gamma}{2} \diffhh{\phi} \tpoint
  \label{eq:subtractive}
\end{equation}
or, in the explicit form,
\begin{gather}
  \begin{aligned}
    \phi_j^{k + 1} = \phi_j^k + m \tau \left( \half K_\Phi^2 \epsilon'(\phi_j^k) + \cfrac{\Gamma}{l^2} f'(\phi_j^k) + \cfrac{\Gamma}{2} \diffhh{\phi} \right), \\ j = \overline{1, N - 1}, \quad k \in \Natural_0 \tsemicolon
  \end{aligned}
  \label{sch:transition} \\
  \phi_j^0 = \phi_0(jh); \quad \phi_0^k = \phi_l(k \tau); \quad \phi_N^k = \phi_r(k \tau) \tpoint
  \label{sch:borders}
\end{gather}

It is easy to see that the scheme has the first order of approximation in
time and the second order of approximation in spatial terms.

To study properties of the scheme~\eqref{sch:transition}, \eqref{sch:borders},
the linear theory can be used (see, e.g.,
\cite[Chapter~10]{bahvalov_computational_methods}
or~\cite[Chapter~IX]{kalitkin_computational_methods}).
The central result of the theory states, in a somewhat simplified
form, that if a finite-difference scheme is stable and approximates a
continuous problem then the solution of the finite-dimensional problem
converges to the solution of the continuous one with order
not lower then the order of approximation.

To apply this result for the nonlinear setting~\eqref{sch:transition}, \eqref{sch:borders},
we proceed as follows:
(i) linearize the equation~\eqref{eq:subtractive}
for a fixed~$\phi$ and then (ii) apply the spectral stability
argument~\cite{bahvalov_computational_methods} to the
derived linearized equation. As the stability criteria are
satisfied for the linearized equation, stability should be expected for the
complete, nonlinear, problem. In this case, convergence of the
approximate solution should be expected as well~--- since
the finite-difference problem is stable and approximates the continuous
one.
The results of such non-rigorous analysis will be further confirmed by
numerical computations in the fully nonlinear setting.


\subsection{Stability estimate}

In this section we derive a stability condition for the
finite-difference scheme~\eqref{sch:transition}, \eqref{sch:borders}
using the so-called principal of ``frozen coefficients''
(see, e.g.,~\cite{bahvalov_computational_methods}).
Let~$\phi_j^k$ and~$\phi_j^k + \delta_j^k$ be solutions of the
finite-difference equation~\eqref{eq:subtractive}.
Substitute~$\phi_j^k + \delta_j^k$ into~\eqref{eq:subtractive} to obtain:
\begin{multline*}
  \cfrac{1}{m} \cfrac{(\phi_j^{k + 1} + \delta_j^{k + 1}) - (\phi_j^k + \delta_j^k)}{\tau} = \half K_\Phi^2 [\epsilon'(\phi_j^k) + \epsilon''(\phi_j^k) \delta_j^k + o(\delta_j^k)] + \\ + \cfrac{\Gamma}{l^2} [f'(\phi_j^k) + f''(\phi_j^k) \delta_j^k + o(\delta_j^k)] + \cfrac{\Gamma}{2} \cfrac{(\phi_{j + 1}^k + \delta_{j + 1}^k) - 2 (\phi_j^k + \delta_j^k) + (\phi_{j - 1}^k + \delta_{j - 1}^k)}{h^2} \tpoint
\end{multline*}
Linearizing this equation around~ $\phi_j^k = P$, assuming that
perturbations~$\delta_j^k$ are small and taking into account
that~$\phi_j^k$ is a solution of the finite-difference problem, we obtain:
\begin{equation}
  \delta_j^{k + 1} = \delta_j^k + m \tau \left( \half K_\Phi^2 \epsilon''(P) \delta_j^k + \cfrac{\Gamma}{l^2} f''(P) \delta_j^k + \cfrac{\Gamma}{2} \diffhh{\delta} \right) \tpoint
  \label{eq:scheme_variation}
\end{equation} 

We now apply spectral stability analysis to the derived equation for
perturbations.
Let~$\delta_j^k = \lambda(\theta)^k \cdot \exp(\imath j \theta)$, $\imath^2 = -1$.
Substituting this representation into~\eqref{eq:scheme_variation} one obtains:
$$\lambda(\theta) = 1 + m \tau \left( \half K_\Phi^2 \epsilon''(P) + \cfrac{\Gamma}{l^2} f''(P) + \cfrac{\Gamma}{2} \cfrac{\exp(\imath \theta) - 2 + \exp(-\imath \theta)}{h^2} \right) \tcomma$$
or
\begin{equation}
  \lambda(\theta) = 1 + m \tau \left( \half K_\Phi^2 \epsilon''(P) + \cfrac{\Gamma}{l^2} f''(P) - \cfrac{2 \Gamma}{h^2} \sin^2 \cfrac{\theta}{2} \right) \tpoint
  \label{eq:spectral}
\end{equation}

According to the spectral stability argument, a time
step~$\tau = \tau(h)$ provides stability of the scheme in the
domain~$[0, W]_x \times [0, T]_t$ with~$T<+\infty$ as~$\tau, h \to 0$ if there
exists~$C > 0$ such that for an arbitrary~$\theta$ it
holds~$|\lambda(\theta)| \leqslant \exp(C\tau)$. Note that here it is
also possible to use more strict condition~$|\lambda(\theta)| \leqslant 1 + C\tau$.
If for an arbitrary~$\theta$ it holds~$|\lambda(\theta)| \leqslant 1$,
then stability will be provided also for an unbounded time interval, i.e.,
for~$[0, W]_x \times [0, +\infty)_t$.
Strictly speaking, the spectral argument does not provide a sufficient
stability condition; however, stability should be expected in practice.

First, consider the expression~\eqref{eq:spectral} for~$P=0$.
We have~$f''(0) = 0$, $\epsilon''(0) = 0$, and the equation~\eqref{eq:spectral}
takes the form of
$$\lambda(\theta) = 1 - \cfrac{2 \tau m \Gamma}{h^2} \sin^2 \cfrac{\theta}{2} \tpoint$$
Hence, for an arbitrary~$\theta$ it holds~$|\lambda(\theta)| \leqslant 1$
if and only if
\begin{equation}
  \tau \leqslant \cfrac{h^2}{m \Gamma} \tpoint
  \label{cond:spectral_0}
\end{equation}
As the condition~\eqref{cond:spectral_0} is satisfied, one can expect stability of the scheme
when the solution describes an almost completely damaged state~$\phi\approx0$
in the domain~$[0, W]_x \times [0, +\infty)_t$.

Note that under the condition~\eqref{cond:spectral_0} one also can expect stable computations
for~$[0, W]_x \times [0, T]_t$ for an arbitrary value~$P \in [0, 1]$.
In this case the following is true:
$$
|\lambda(\theta)| \leqslant \left| 1 - \cfrac{2 \tau m \Gamma}{h^2} \sin^2 \cfrac{\theta}{2} \right| + m \tau \left| \half K_\Phi^2 \epsilon''(P) + \cfrac{\Gamma}{l^2} f''(P) \right| \leqslant 1 + m \tau \left| \half K_\Phi^2 \epsilon''(P) + \cfrac{\Gamma}{l^2} f''(P) \right| \tpoint
$$
Hence, there exists~$C$ such that
$|\lambda(\theta)| \leqslant 1 + C \tau$ holds,~--- since~$\epsilon''(\phi)$ and~$f''(\phi)$
are continuous on~$[0, 1]$.
It should be noted that, despite such versatility, the estimate~\eqref{cond:spectral_0}
is poorly applicable in practice and requires clarification, which will be done later.

We now consider the expression~\eqref{eq:spectral} at the value~$P=1$.
Note that~$f''(1) < 0$, $\epsilon''(1) > 0$.
We see that for~$(K_\Phi^2 / 2) \epsilon''(1) + (\Gamma / l^2) f''(1) \leqslant 0$
it is possible to achieve~$|\lambda(\theta)| \leqslant 1$ with demanded sufficiently small
values of~$\tau$ and the condition $\tau \leqslant h^2 / (2m \Gamma)$,
similar to the one for~\eqref{cond:spectral_0}.
Substituting~$f''(1) = -12, \; \epsilon''(1) = 12 \epsilon_0 / (1 + \delta)^2$
(see~\eqref{eq:epsilon_derivatives}),
we obtain
\begin{equation}
  \cfrac{K_\Phi^2 l^2 \epsilon_0}{2 \Gamma (1 + \delta)^2} \leqslant 1 \tpoint
  \label{cond:spectral_possible_1}
\end{equation}

So, under the condition~\eqref{cond:spectral_possible_1}, it is expected that
there exist such values of~$\tau$ и $h$
that the difference scheme is stable for~$\phi \approx 1$
and~$T=+\infty$.
Naturally the condition~\eqref{cond:spectral_possible_1}
is equivalent to the stability condition~\eqref{cond:equilibrium_1_stable}
for the equilibrium state~$\phi \equiv 1$ of the equation~\eqref{eq:one_dim}.

\subsection{Improved stability estimate}

In the previous section form the analysis of equation~\eqref{eq:spectral}
it was derived stability condition~\eqref{cond:spectral_0}
for finite-difference scheme~\eqref{sch:transition} and~\eqref{sch:borders} for~$\phi \approx 0$.
The assumption of its usefulness is based on the fact that typical ``'natural'' solution of the model
will has a form of the transition process from the undamaged state~$\phi=1$ to the completely
damaged state~$\phi=0$ occurring in a finite time interval and then infinitely long staying in the
damaged state~$\phi \approx 0$.


However the performed analysis of the equation~\eqref{eq:spectral}
is not sufficient at~$\phi = 0$. Indeed, it was used that at~$\phi=0$,
$\epsilon''(0) = 0$ (see expression~\eqref{eq:epsilon_derivatives}),~---
but it was not accounted that~$\epsilon''(\phi)$ growth fast and reaches
large values  for small values of~$\delta\approx 0$,
see Fig.~\ref{fig:eps_phi_phi}.
This means that the equations of the model are stable at~$\phi=0$,
but can be unstable in the small neighbourhood of~$\phi=0$.
Such situation is not satisfactory and we now try to improve
the obtained stability estimates.
%
\begin{figure}[!t]
	\centering
	\includegraphics[width=\textwidth]{figures/eps_phi_phi.png}
	\vspace{-0.7cm}
	\caption{Typical behavior of~$\epsilon''(\phi)$ in the vicinity of~$0$.}
	\label{fig:eps_phi_phi}
\end{figure}

To proceed let us estimate extremums of~$\epsilon''(\phi)$ in the neighbourhood of~$0$.
First, find zeros of~$\epsilon'''(\phi)$. We have
\begin{equation}
	\epsilon''' = \epsilon_0 \cfrac{-6 (f')^3 + 6 (f + \delta) f' f'' - (f + \delta)^2 f'''}{(f + \delta)^4},
	\label{eq:epsilon_phi_phi_phi}
\end{equation}
form where:
$$\epsilon'''(\phi) = -6 (f')^3 + 6 (f + \delta) f' f'' - (f + \delta)^2 f''' = 0 \tcomma$$
or, taking~\eqref{eq:epsilon} into account:
$$-3 \cdot 12^2 (1 - \phi)^3 + 36 \left(4 - 3\phi + \cfrac{\delta}{\phi^3} \right)(1 - \phi)(2 - 3\phi) - \left(4 - 3 \phi + \cfrac{\delta}{\phi^3} \right)^2 (1 - 3 \phi) = 0 \tpoint$$

Let~$\delta_n \to +0$ and~$\phi_n \to +0$ such that~$\delta_n / \phi_n^3$ is bounded.
Then:
\begin{gather*}
	-3 \cdot 12^2 \cdot 1^3 + 36 \left(4 + \cfrac{\delta_n}{\phi_n^3} \right) \cdot 1 \cdot 2 - \left(4 + \cfrac{\delta_n}{\phi_n^3} \right)^2 \cdot 1 \to 0 \tcomma \\
	\left(4 + \cfrac{\delta_n}{\phi_n^3} \right)^2 - 72 \left(4 + \cfrac{\delta_n}{\phi_n^3} \right) + 3 \cdot 12^2 \to 0 \tpoint
\end{gather*}
Hence, a sequence~$4 + \delta_n / \phi_n^3$ has not more than two partial limits~$\xi_+$ and~$\xi_-$~---
which are zeros of the equation~$\xi^2 - 72 \xi + 432 = 0$.
To the first zero~$\xi_+ = 36 + 12 \sqrt{6}$ it corresponds
$$\phi_+ = \cfrac{1}{\sqrt[3]{32 + 12 \sqrt{6}}} \sqrt[3]{\delta_n} \approx \cfrac{1}{3.945} \sqrt[3]{\delta_n} \tsemicolon$$
to the second zero~$\xi_- = 36 - 12 \sqrt{6}$ it corresponds
$$\phi_- = \cfrac{1}{\sqrt[3]{32 - 12 \sqrt{6}}} \sqrt[3]{\delta_n} \approx \cfrac{1}{1.376} \sqrt[3]{\delta_n} \tpoint$$

From here it can be seen that for~$\delta \to +0$ the function~$\epsilon'''(\phi)$ has two zeros in the neighbourhood of~$0$:
\begin{equation}
  \phi_{\pm} = \cfrac{1}{\sqrt[3]{32 \pm 12 \sqrt{6}}} \sqrt[3]{\delta} [1 + o(1)] \tpoint
  \label{eq:epsilon_phi_phi_phi_roots}
\end{equation}

We now estimate~$\epsilon''(\phi)$ at~$\phi_{\pm}$  for~$\delta \to +0$. Let~$\phi = (1 / c) \sqrt[3]{\delta}$, $c \in \Real$.
Then:
$$\epsilon'' = \epsilon_0 \cfrac{24 c^5 (8 - c^3)}{(4 + c^3)^3} \delta^{-5 / 3} [1 + o(1)],$$
and:
\begin{equation}
  \epsilon''(\phi_+) \approx -4.378 \epsilon_0 \delta^{-5 / 3}; \quad \epsilon''(\phi_-) \approx 2.216 \epsilon_0 \delta^{-5 / 3} \tpoint
  \label{est:epsilon_phi_phi_bounds}
\end{equation}
The derived estimates are shown as black dashed lines on Fig.~\ref{fig:eps_phi_phi_multiplied}.

\begin{figure}[!t]
	\centering
	\includegraphics[width=\textwidth]{figures/eps_phi_phi_multiplied.png}
	\caption{Qualitative behavior of~$\delta^{5 / 3} \epsilon''(\phi)$ for small values of~$\delta$.}
	\label{fig:eps_phi_phi_multiplied}
\end{figure}

Now, to derive new stability estimate we consider equation~\eqref{eq:spectral} at~$\phi = \phi_+$.
Note that~$\epsilon''(\phi_+) \approx -4.4 \epsilon_0 \delta^{-5 / 3}$.
The term inside braces in~\eqref{eq:spectral} is negative since
$\delta$ is small and~$\epsilon''(\phi_+)$ is negative and large in its absolute value.
Therefore~$f''(\phi_+) > 0$ can be estimated as~$0$~--- such estimate makes inequality stronger.
Then from inequality~\eqref{eq:spectral} it follows that 
$$\lambda(\theta) = 1 + m \tau \left( -\cfrac{2.2 K_\Phi^2 \epsilon_0}{\delta^{5 / 3}} - \cfrac{2 \Gamma}{h^2} \sin^2 \cfrac{\theta}{2} \right) \tpoint$$
Condition~$|\lambda(\theta)| \leqslant 1$ is satisfied for an arbitrary~$\theta$, if and only if
\begin{equation}
  \tau \leqslant \cfrac{1}{m} \left( \cfrac{1.1 K_\Phi^2 \epsilon_0}{\delta^{5 / 3}} + \cfrac{\Gamma}{h^2} \right)^{-1} \tpoint
  \label{cond:spectral_better_theoretical}
\end{equation}

Numerical experiments described in the next sections indicates that
more strong version of the estimate~\eqref{cond:spectral_better_theoretical} is also valid
(note the doubled denominator):
%
\begin{equation}
  \tau \leqslant \cfrac{1}{2m} \left( \cfrac{K_\Phi^2 \epsilon_0}{\delta^{5 / 3}} + \cfrac{\Gamma}{h^2} \right)^{-1} \tpoint
  \label{cond:spectral_better}
\end{equation}

Finally, more simple estimate not weaker then~\eqref{cond:spectral_better} is:
\begin{equation}
  \tau \leqslant \cfrac{1}{4m} \min \left(\cfrac{\delta^{5 / 3}}{K_\Phi^2 \epsilon_0}, \; \cfrac{h^2}{\Gamma} \right) \tpoint
  \label{cond:spectral_better_simpler}
\end{equation}

Note that the derived stability estimate~\eqref{cond:spectral_better}
for finite-difference scheme~\eqref{sch:transition},\eqref{sch:borders}
includes all the parameters of the equation~\eqref{eq:one_dim}, except~$l$.
Notably, this is the only parameter of the model which has somehow artificial nature and can not be
related directly to the underlying physics.

\endinput
% EOF

%!TEX root = ../main.tex

\section{Численное исследование}

Была написана программа, реализующая разностную схему \eqref{sch:transition}, \eqref{sch:borders}. С помощью моделирования проверим полученные ранее теоретические результаты, а именно: устойчивость и сходимость схемы при выполнении условия устойчивости \eqref{cond:spectral_better}, а также свойства положений равновесия системы, описанные в разделе \ref{sec:theoretical_analysis}. Помимо этого проверим поведение полной свободной энергии системы при моделировании.


\subsection{Вычислительный эксперимент: устойчивость}

Зафиксируем параметры уравнения \eqref{eq:one_dim}:
\begin{equation}
	\epsilon_0 = 0.2, \; \delta = 0.04, \; l = 1.0, \; \Gamma = 1.0, \; m = 0.5, \; K_\Phi = 4.8 \tpoint
	\label{exp:parameters}
\end{equation}
Перед нами случай <<сильного напряжения>> (см. выражение~\eqref{char:equilibriums}).

Моделируем решение в области 
\begin{equation}
	\clOmega = [0, W]_x \times [0, T]_t, \; W = 5, \; T = 1 \tpoint
	\label{exp:set}
\end{equation}

Зададим следующие краевые условия:
\begin{equation}
\begin{gathered}
	\phi(0, t) = 1, \; \phi(W, t) = 1 \tcomma \\
	\phi(x, 0) = \phi_0(x) = \begin{cases}
		1, \; \text{если} \; x \leqslant 2.25 \; \text{или} \; x \geqslant 2.75 \tsemicolon \\
		1 - 0.025 \cdot [1 + \cos(4 \pi x)], \; \text{если} \; 2.25 < x < 2.75 \tpoint
	\end{cases}
\end{gathered} \label{exp:borders}
\end{equation}
Обратим внимание, что $\phi_0(x)$ дважды дифференцируема всюду, кроме конечного числа точек, с ограниченной второй производной.

Обозначим $N_x$ число отрезков разбиения $[0, W]_x$ (узлов, соответственно, $N_x + 1$); $N_t$ -- число отрезков разбиения $[0, T]_t$. $h = W / N_x, \; \tau = T / N_t$.

\begin{figure}[!tp]
	\centering
	\includegraphics[width=\textwidth]{figures/typical_solution.png}
	\vspace{-0.8cm}
	\caption{Типичное решение задачи, $N_x = 10^3, \; N_t = 10^5$.}
	\label{fig:typical_solution}
\end{figure}

Для начала посмотрим на типичное решение исследуемой задачи (рис.~\ref{fig:typical_solution}). Видно постепенное развитие канала электрического пробоя (разрушение среды) из небольшого начального возмущения фазового поля $\phi$ неповрежденной среды. Примерно в момент времени $t = 0.55$ канал пробоя <<прорастает насквозь>>, а именно, $\phi$ вблизи точки $x = 2.5$ приближается к нулевому значению. Обратим внимание, что в период времени $t \in (0.3, \; 0.55)$ канал пробоя (область, где $\phi$ существенно отличается от $1$) практически не растет в ширину, а при $t > 0.55$, напротив, растет в ширину почти с постоянной скоростью.

Проверим полученную в предыдущем разделе оценку \eqref{cond:spectral_better} устойчивости разностной схемы. Будем считать, что в вычислительном эксперименте схема неустойчива, если программа завершилась с ошибкой: произошло деление на~$0$ (в формуле \eqref{eq:epsilon} функции $\epsilon(\phi)$ при $f(\phi) = -\delta$) или значения $\phi$ ушли на бесконечность (переполнился тип double). Будем перебирать $N_x$ и $N_t$, запоминая пары соседних точек, в одной из которых устойчивость есть, а в другой нет. Так получим опытную оценку устойчивости схемы. Отобразим ее на графике вместе с оценкой \eqref{cond:spectral_better} (рис. \ref{fig:stability_bounds}).

Эксперимент показывает, что оценка \eqref{cond:spectral_better} удачна: она примерно повторяет контур опытной оценки, к тому же ее график лежит выше, то есть она имеет некоторый <<запас>> до момента, когда в программе возникает ошибка. Именно ради этого <<запаса>> знаменатель исходной оценки  \eqref{cond:spectral_better_theoretical} был удвоен.

\begin{figure}[!tp]
	\centering
	\includegraphics[width=\textwidth]{figures/stability_bounds.png}
	\vspace{-0.7cm}
	\caption{Теоретическая и опытная оценки устойчивости разностной схемы.}
	\label{fig:stability_bounds}
\end{figure}


\subsection{Вычислительный эксперимент: сходимость}

Аппроксимация разностной схемой \eqref{sch:transition}, \eqref{sch:borders} дифференциальной задачи \eqref{eq:one_dim}, \eqref{eq:one_dim_initial}, \eqref{eq:one_dim_marginal} очевидна; для устойчивости схемы в ходе нестрогого анализа получено условие \eqref{cond:spectral_better}, применимое на практике. Теперь экспериментально проверим сходимость.

На множестве $C_2(\clOmega)$ дважды непрерывно дифференцируемых функций в замкнутой области $\clOmega = [0, W]_x \times [0, T]_t$ рассмотрим следующие нормы: непрерывную $\enorm_C$ и $L_2$-норму $\enorm_2$.
$$\norm{f}_C = \max \limits_{(x, t) \in \clOmega} |f(x, t)|; \qquad \norm{f}_2 = \sqrt{\int \limits_{\clOmega} f^2(x, t) dx dt} \tpoint$$

Теперь рассмотрим регулярную сетку $\Omega_{h, \tau} \subset \clOmega$ с некоторой зависимостью $\tau = \tau(h)$. Ограничивая функции из $C_2(\Omega)$ на сетке $\Omega_h = \Omega_{h, \tau(h)}$, получаем множество $C_2(\clOmega)_h$ сеточных функций.

На множестве $C_2(\clOmega)_h$ сеточных функций введем нормы, согласованные с $\enorm_C$ и $\enorm_2$:
$$\norm{f_j^k}_C = \max \limits_{(j, k) \in \Omega_h} |f_j^k|; \qquad \norm{f_j^k}_2 = \sqrt{h \tau \sum \limits_{(j, k) \in \Omega_h} (f_j^k)^2} \tpoint$$

Перейдем к вычислительному эксперименту. Сходимость будем проверять по описанным выше нормам $\enorm_C$ и $\enorm_2$ на множестве сеточных функций. Так как аналитическое решение дифференциальной задачи не известно, будем сравнивать ряд результатов на все более мелких сетках по норме с лучшим результатом в ряду. При сравнении функцию на более мелкой сетке ограничиваем на более крупной, игнорируя часть узлов.

Зафиксируем ранее использовавшиеся параметры уравнения \eqref{exp:parameters}, \eqref{exp:set}; зададим краевые условия \eqref{exp:borders}. Положим $N_x = W / h$ -- число отрезков разбиения по $x$, $N_t = T / \tau$ -- по $t$.

Во всех описанных далее вариантах расчетов соблюдается условие устойчивости~\eqref{cond:spectral_better}.

Для начала зафиксируем $N_x = 200$ и будем перебирать $N_t$, каждый раз увеличивая его вдвое. Сравнение по нормам с результатом при $N_t = 204800$ изображено на рис. \ref{fig:convergence_fixed_nx}. Разностная схема имеет первый порядок аппроксимации по $t$; опыт показывает первый порядок сходимости $\bigO (\tau)$.

Зафиксируем $N_t = 204800$ и будем перебирать $N_x$, каждый раз увеличивая его вдвое. Сравнение по нормам с результатом при $N_x = 1600$ изображено на рис. \ref{fig:convergence_fixed_nt}. Разностная схема имеет второй порядок аппроксимации по $x$; опыт показывает второй порядок сходимости $\bigO (h^2)$.

\begin{figure}[!tp]
	\centering
	\includegraphics[width=0.72\textwidth]{figures/convergence_fixed_nx.png}
	\vspace{-0.2cm}
	\caption{Ошибка решения по норме при фиксированном $N_x = 200$.}
	\label{fig:convergence_fixed_nx}
	\vspace{0.6cm}
	
	\includegraphics[width=0.72\textwidth]{figures/convergence_fixed_nt.png}
	\vspace{-0.2cm}
	\caption{Ошибка решения по норме при фиксированном $N_t = 204800$.}
	\label{fig:convergence_fixed_nt}
	\vspace{0.6cm}
	
	\includegraphics[width=0.72\textwidth]{figures/convergence_connected.png}
	\vspace{-0.2cm}
	\caption{Ошибка решения по норме при $N_t = 0.08 \cdot N_x^2$.}
	\label{fig:convergence_connected}
\end{figure}

Теперь свяжем $N_x$ и $N_t$ уравнением, так чтобы при $h, \tau \to 0$ выполнялось условие устойчивости \eqref{cond:spectral_better}. При выбранных параметрах модели подойдет $N_t = 0.08 \cdot N_x^2$. Аналогично проведем сравнение ряда измерений по норме с лучшим (рис. \ref{fig:convergence_connected}). Как и ожидалось, измерения показывают сходимость $\bigO (\tau + h^2) = \bigO (\tau)$ первого порядка по времени при выбранном уравнении связи.

В первых двух опытах, без стремления обоих шагов сетки к $0$, последовательности сеточных функций имели неясный предел. В третьем же, если принять предположение об устойчивости разностной схемы, сеточные функции сходятся к решению дифференциальной задачи \eqref{eq:one_dim}, \eqref{eq:one_dim_initial}, \eqref{eq:one_dim_marginal}.


\subsection{Вычислительный эксперимент: \\ положения равновесия}

Ранее были исследованы положения равновесия уравнения \eqref{eq:one_dim} вида $\phi \equiv C$. Их количество и устойчивость определяется значением выражения \eqref{char:equilibriums} (обозначено $\xi$). Проверим этот результат экспериментально.

Зададим модели параметры \eqref{exp:parameters}, \eqref{exp:set}, $K_\Phi$ определим позже. В качестве начального условия берем возмущенное положение равновесия: $\phi(x, 0) = C + A \cos(\omega x); \; \phi(0, t) = \phi(0, 0), \; \phi(W, t) = \phi(W, 0)$. Амплитуда $A$ мала, порядка~$0.01$. $N_x = 800, \; N_t = 51200$.

Если положение равновесия устойчиво, то при любом $\omega$ возмущение угасает; если неустойчиво, то существует некоторое $\omega_0$, такое что при $\omega < \omega_0$ возмущение растет.

Положим $K_{\Phi, 1} = 0, \; K_{\Phi, 2} = 1.1, \; K_{\Phi, 3} = 4.8$. Было задано $\delta = 0.04$. В таком случае $\xi_1 = 0 < \delta^2, \; \xi_2 = 0.121 \in (\delta^2, (1 + \delta)^2), \; \xi_3 = 2.304 > (1 + \delta)^2$.

Вначале рассмотрим $K_{\Phi, 1} = 0, \; \xi_1 < \delta^2$ -- случай <<слабого напряжения>>. Система имеет два положения равновесия: $\phi \equiv 0$ неустойчивое, $\phi \equiv 1$ устойчивое. На рис. \ref{fig:equilibrium_1_0}, \ref{fig:equilibrium_1_1} видно теоретически предсказанное поведение возмущенной среды: при $C = 0$ возмущение растет, при $C = 1$ -- затухает. В точке $C = 0$ производная функции $\chi(\phi)$ (см. выражение \eqref{eq:equilibruim_characteristic}) равна $0$, поэтому, чтобы увидеть рост возмущения, приходится брать небольшое $\omega$, обеспечивая небольшое значение $\partflxx{\phi}$. В эксперименте с $\phi \equiv 1$ взято $C = 1 - A$, чтобы значения $\phi$ не превосходили $1$.

\begin{figure}[!t]
	\centering
	\includegraphics[width=0.9\textwidth]{figures/equilibrium_1_0.png}
	\vspace{-0.3cm}
	\caption{Случай <<слабого напряжения>>: возмущенное положение равновесия $\phi \equiv 0$, неустойчивое.}
	\label{fig:equilibrium_1_0}
	\vspace{0.5cm}
	
	\includegraphics[width=0.9\textwidth]{figures/equilibrium_1_1.png}
	\vspace{-0.3cm}
	\caption{Случай <<слабого напряжения>>: возмущенное положение равновесия $\phi \equiv 1$, устойчивое.}
	\label{fig:equilibrium_1_1}
\end{figure}

Теперь рассмотрим $K_{\Phi, 2} = 1.1, \; \xi_2 \in (\delta^2, (1 + \delta)^2)$ -- случай <<среднего напряжения>>. Система имеет три положения равновесия: $\phi \equiv 0$ устойчивое, $\phi \equiv C_3 \approx 0.5$ неустойчивое ($C_3$ -- корень функции $\chi(\phi)$ в интервале $(0, 1)$), $\phi \equiv 1$ устойчивое. Поведение возмущенной среды изображено на рис. \ref{fig:equilibrium_2_0}, \ref{fig:equilibrium_2_05}, \ref{fig:equilibrium_2_1}, оно соответствует теоретическим результатам.

\begin{figure}[!tp]
	\centering
	\includegraphics[width=0.9\textwidth]{figures/equilibrium_2_0.png}
	\vspace{-0.3cm}
	\caption{Случай <<среднего напряжения>>: возмущенное положение равновесия $\phi \equiv 0$, устойчивое.}
	\label{fig:equilibrium_2_0}
	\vspace{0.5cm}

	\includegraphics[width=0.9\textwidth]{figures/equilibrium_2_05.png}
	\vspace{-0.3cm}
	\caption{Случай <<среднего напряжения>>: возмущенное положение равновесия $\phi \equiv C_3 \approx 0.5$, неустойчивое.}
	\label{fig:equilibrium_2_05}
	\vspace{0.5cm}
	
	\includegraphics[width=0.9\textwidth]{figures/equilibrium_2_1.png}
	\vspace{-0.3cm}
	\caption{Случай <<среднего напряжения>>: возмущенное положение равновесия $\phi \equiv 1$, устойчивое.}
	\label{fig:equilibrium_2_1}
\end{figure}

Наконец, рассмотрим $K_{\Phi, 3} = 4.8, \; \xi_3 > (1 + \delta)^2$ -- случай <<сильного напряжения>>. Система имеет два положения равновесия: $\phi \equiv 0$ устойчивое, $\phi \equiv 1$ неустойчивое. Поведение возмущенной среды изображено на рис. \ref{fig:equilibrium_3_0}, \ref{fig:equilibrium_3_1}, оно также соответствует теории.

\begin{figure}[!t]
	\centering
	\includegraphics[width=0.9\textwidth]{figures/equilibrium_3_0.png}
	\vspace{-0.3cm}
	\caption{Случай <<сильного напряжения>>: возмущенное положение равновесия $\phi \equiv 0$, устойчивое.}
	\label{fig:equilibrium_3_0}
	\vspace{0.5cm}
	
	\includegraphics[width=0.9\textwidth]{figures/equilibrium_3_1.png}
	\vspace{-0.3cm}
	\caption{Случай <<сильного напряжения>>: возмущенное положение равновесия $\phi \equiv 1$, неустойчивое.}
	\label{fig:equilibrium_3_1}
\end{figure}


\subsection{Вычислительный эксперимент: свободная энергия}

В рассматриваемой модели введена функция свободной энергии \eqref{eq:energy}, заданная интегралом плотности свободной энергии \eqref{eq:energy_density} по пространству.

Построим график полной свободной энергии системы от времени. Зададим модели параметры \eqref{exp:parameters}, \eqref{exp:set} и краевые условия \eqref{exp:borders}. Ранее мы уже проводили расчеты в этой конфигурации -- графики $\phi$ изображены на рис. \ref{fig:typical_solution}.

Единственное существенное дополнение, требующееся использованной ранее схеме \eqref{sch:transition}, \eqref{sch:borders}, -- вычисление разностной производной $\partial_h \phi_i^j / \partial_h x$. Ограничимся простейшей разностной производной первого порядка, так как искомая величина не влияет на пересчет состояния системы.

Результат вычисления полной свободной энергии $\Pi$ в зависимости от времени изображен на рис. \ref{fig:energy}. Для сравнения с рис. \ref{fig:typical_solution} пунктирными линиями соответствующих цветов отмечены моменты времени. Интересно, что до момента $t = 0.5$ энергия $\Pi$ убывает очень медленно; после, при $t \approx 0.55$, резко падает; затем падение замедляется, и после $t = 0.6$ убывание выравнивается, становясь близким к линейному; так продолжается практически до полного разрушения среды.

\begin{figure}
	\centering
	\includegraphics[width=\textwidth]{figures/energy_total.png}
	\vspace{-0.6cm}
	\caption{Поведение полной свободной энергии системы.}
	\label{fig:energy}
\end{figure}

Отметим, что вывод системы уравнений \eqref{eq:Phi}, \eqref{eq:phi} динамики системы из выражений \eqref{eq:energy}, \eqref{eq:energy_density} для свободной энергии означает, что система в ходе эволюции стремится в положение с как можно меньшей свободной энергией. Поэтому для адекватного моделирования системы крайне важно, чтобы полная свободная энергия $\Pi$ не возрастала. Мы не дали этому теоретического обоснования для используемой разностной схемы, однако проверили на опыте. Другими словами, это означает, что при используемом виде краевых условий и параметрах расчета предложенная схема является градиентно-устойчивой.

%!TEX root = ../main.tex

\section{Conclusions}

In this paper we study stability properties of the phase-field model
for electrical breakdown channel evolution.
The central result is a classification of the
equilibrium solutions of the model and their stability.
From practical point of view, these results allows to
make meaningful conclusions regarding qualitative and quantitative
properties of the model. Particularly it was shown under which
conditions small perturbations of the equilibrium solutions
develop into channel-like structure typical for of electrical breakdown
process.

Besides this, a simple explicit finite-difference scheme
for solution of the model in spatially one-dimensional setting is considered.
The main question addressed here are stability conditions which guaranties
correctness of the simulations. Deep connections between
stability conditions of the model and the one of the
finite-difference scheme are shown.
The presented results of the numerical simulations confirms
predictions of the theoretical analysis of the model.

% EOF
\endinput

\printbibliography

\end{document}

%%%%%%%%%%%%%%%%%%%%%%%%%%%%%%%%%%%%%%%%%%%%%%%%%%%%%%%%%%%%%%%%%%%%%%%%%%%%%%%%