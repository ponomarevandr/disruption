%!TEX root = ../main.tex

\section{Разностная схема}

Будем численно решать уравнение \eqref{eq:one_dim} в области $[0, W]_x \times [0, +\infty)_t$ с краевыми условиями \eqref{eq:one_dim_initial}, \eqref{eq:one_dim_marginal}.

Используем регулярную сетку с временным шагом~$\tau$ и пространственным~$h$. Пусть $N = W / h$ -- целое число. Пусть $(jh, k \tau)$ -- узлы сетки, $j = \overline{0, N}, \; k \in \Natural_0$. Обозначим $\phi_j^k$ значение сеточной функции $\phi$ в узле $(jh, k \tau)$. Перейдем к следующей разностной задаче:
\begin{equation}
	\cfrac{1}{m} \difftau{\phi} = \half K_\phi^2 \epsilon'(\phi_j^k) + \cfrac{\Gamma}{l^2} f'(\phi_j^k) + \cfrac{\Gamma}{2} \diffhh{\phi} \tpoint
	\label{eq:subtractive}
\end{equation}
Имеем четырехточечную явную разностную схему:
\begin{gather}
	\begin{aligned}
		\phi_j^{k + 1} = \phi_j^k + m \tau \left( \half K_\Phi^2 \epsilon'(\phi_j^k) + \cfrac{\Gamma}{l^2} f'(\phi_j^k) + \cfrac{\Gamma}{2} \diffhh{\phi} \right), \\ j = \overline{1, N - 1}, \quad k \in \Natural_0 \tsemicolon
	\end{aligned}
	\label{sch:transition} \\
	\phi_j^0 = \phi_0(jh); \quad \phi_0^k = \phi_l(k \tau); \quad \phi_N^k = \phi_r(k \tau) \tpoint
	\label{sch:borders}
\end{gather}

Легко видеть, что схема имеет первый порядок аппроксимации по времени и второй порядок аппроксимации по пространственной переменной $x$.

Для исследования схемы \eqref{sch:transition}, \eqref{sch:borders} применим элементы теории сходимости линейных разностных схем, приближающих линейные дифференциальные задачи (см., например, \cite[глава 10]{bahvalov_computational_methods} или \cite[глава IX]{kalitkin_computational_methods}). Центральное утверждение упомянутой теории в несколько упрощенном виде звучит так: если разностная задача является устойчивой и аппроксимирует дифференциальную задачу, то решение разностной задачи сходится к решению дифференциальной, причем с порядком не меньше, чем порядок аппроксимации.

Итак, разностную схему \eqref{sch:transition}, \eqref{sch:borders} будем исследовать следующим образом: линеаризуем разностное уравнение \eqref{eq:subtractive} при фиксированном аргументе $\phi$, а затем применим к нему так называемый спектральный признак устойчивости \cite{bahvalov_computational_methods}. При выполнении признака у схемы следует ожидать устойчивость (в <<разумных>> нормах, например, непрерывной и $L_2$-норме), а значит, с учетом аппроксимации следует ожидать и сходимость. Результат подобного нестрогого анализа в дальнейшем будет подтвержден моделированием полного, нелинейного случая.


\subsection{Оценка устойчивости схемы}

Получим условие устойчивости по принципу <<замороженных коэффициентов>> (см., например, \cite{bahvalov_computational_methods}). Пусть $\phi_j^k$ и $\phi_j^k + \delta_j^k$ -- решения разностного уравнения~\eqref{eq:subtractive}. Подставим в него $\phi_j^k + \delta_j^k$, получим:
\begin{multline*}
	\cfrac{1}{m} \cfrac{(\phi_j^{k + 1} + \delta_j^{k + 1}) - (\phi_j^k + \delta_j^k)}{\tau} = \half K_\Phi^2 [\epsilon'(\phi_j^k) + \epsilon''(\phi_j^k) \delta_j^k + o(\delta_j^k)] + \\ + \cfrac{\Gamma}{l^2} [f'(\phi_j^k) + f''(\phi_j^k) \delta_j^k + o(\delta_j^k)] + \cfrac{\Gamma}{2} \cfrac{(\phi_{j + 1}^k + \delta_{j + 1}^k) - 2 (\phi_j^k + \delta_j^k) + (\phi_{j - 1}^k + \delta_{j - 1}^k)}{h^2} \tpoint
\end{multline*}
Линеаризуем уравнение по возмущению $\delta_j^k$ в точке $\phi_j^k = P$ и сократим слагаемые, учитывая, что $\phi_j^k$ есть решение разностной задачи:
\begin{equation}
	\delta_j^{k + 1} = \delta_j^k + m \tau \left( \half K_\Phi^2 \epsilon''(P) \delta_j^k + \cfrac{\Gamma}{l^2} f''(P) \delta_j^k + \cfrac{\Gamma}{2} \diffhh{\delta} \right) \tpoint
	\label{eq:scheme_variation}
\end{equation} 

Применим спектральный признак устойчивости. Пусть $\delta_j^k = \lambda(\theta)^k \cdot \exp(\imath j \theta)$, $\imath^2 = -1$; подставим в уравнение \eqref{eq:scheme_variation} выражение для $\delta_j^k$ и, сократив на $\lambda(\theta)^k \exp(\imath j \theta)$, получим:
$$\lambda(\theta) = 1 + m \tau \left( \half K_\Phi^2 \epsilon''(P) + \cfrac{\Gamma}{l^2} f''(P) + \cfrac{\Gamma}{2} \cfrac{\exp(\imath \theta) - 2 + \exp(-\imath \theta)}{h^2} \right) \tcomma$$
или
\begin{equation}
	\lambda(\theta) = 1 + m \tau \left( \half K_\Phi^2 \epsilon''(P) + \cfrac{\Gamma}{l^2} f''(P) - \cfrac{2 \Gamma}{h^2} \sin^2 \cfrac{\theta}{2} \right) \tpoint
	\label{eq:spectral}
\end{equation}

Согласно спектральному признаку связь $\tau = \tau(h)$ дает устойчивые вычисления в области $[0, W]_x \times [0, T]_t$ при $\tau, h \to 0$, если существует $C > 0$, такое что для любого~$\theta$ выполнено $|\lambda(\theta)| \leqslant e^{C\tau}$. Заметим, что можно использовать условие $|\lambda(\theta)| \leqslant 1 + C\tau$, как более сильное. Если же для любого $\theta$ выполнено $|\lambda(\theta)| \leqslant 1$, то устойчивыми будут вычисления в области $[0, W]_x \times [0, +\infty)_t$ с бесконечным временным интервалом. Строго говоря, условие спектрального признака не является достаточным для устойчивости разностной схемы, однако на практике устойчивость следует ожидать.

Для начала рассмотрим выражение \eqref{eq:spectral} в точке $P = 0$. Имеем $f''(0) = 0, \; \epsilon''(0) = 0$. Уравнение \eqref{eq:spectral} принимает вид
$$\lambda(\theta) = 1 - \cfrac{2 \tau m \Gamma}{h^2} \sin^2 \cfrac{\theta}{2} \tpoint$$
Значит, для любого $\theta$ выполнено $|\lambda(\theta)| \leqslant 1$, если и только если
\begin{equation}
	 \tau \leqslant \cfrac{h^2}{m \Gamma} \tpoint
	 \label{cond:spectral_0}
\end{equation}
При выполнении условия \eqref{cond:spectral_0} следует ожидать устойчивый расчет при полностью разрушенной или близкой к таковой среде ($\phi \approx 0$) в области $[0, W]_x \times [0, +\infty)_t$ с бесконечным временным интервалом.

Заметим, что при условии \eqref{cond:spectral_0} к тому же ожидается устойчивый расчет на множестве $[0, W]_x \times [0, T]_t$ в любой точке $P \in [0, 1]$. В этом случае справедливо неравенство
$$
	|\lambda(\theta)| \leqslant \left| 1 - \cfrac{2 \tau m \Gamma}{h^2} \sin^2 \cfrac{\theta}{2} \right| + m \tau \left| \half K_\Phi^2 \epsilon''(P) + \cfrac{\Gamma}{l^2} f''(P) \right| \leqslant 1 + m \tau \left| \half K_\Phi^2 \epsilon''(P) + \cfrac{\Gamma}{l^2} f''(P) \right| \tpoint
$$
Значит, для некоторого $C$ верно $|\lambda(\theta)| \leqslant 1 + C \tau$, так как $\epsilon''(\phi)$ и $f''(\phi)$ -- непрерывные на отрезке $[0, 1]$ функции. Следует отметить, что, несмотря на подобную универсальность, оценка \eqref{cond:spectral_0} плохо применима на практике и нуждается в уточнении, которое будет сделано позже.

Теперь рассмотрим выражение \eqref{eq:spectral} в точке $P = 1$. Имеем $f''(1) < 0, \; \epsilon''(1) > 0$. Заметим, что при $(K_\Phi^2 / 2) \epsilon''(1) + (\Gamma / l^2) f''(1) \leqslant 0$ можно добиться $|\lambda(\theta)| \leqslant 1$, потребовав, подобно условию \eqref{cond:spectral_0}, $\tau \leqslant h^2 / (2m \Gamma)$ и притом достаточно малое $\tau$. Если подставить в упомянутое неравенство значения $f''(1) = -12, \; \epsilon''(1) = 12 \epsilon_0 / (1 + \delta)^2$ (см. выражение~\eqref{eq:epsilon_derivatives}), то оно преобразуется в
\begin{equation}
	\cfrac{K_\Phi^2 l^2 \epsilon_0}{2 \Gamma (1 + \delta)^2} \leqslant 1 \tpoint
	\label{cond:spectral_possible_1}
\end{equation}

Итак, при условии \eqref{cond:spectral_possible_1} ожидается существование таких $\tau$ и $h$, что расчет устойчив при $\phi \approx 1$ на множестве с бесконечным временным интервалом. Закономерно, что условие \eqref{cond:spectral_possible_1} эквивалентно (с точностью до строгости) условию \eqref{cond:equilibrium_1_stable} устойчивости положения равновесия $\phi \equiv 1$ уравнения \eqref{eq:one_dim}.


\subsection{Улучшенная оценка устойчивости схемы}

В предыдущем разделе из анализа уравнения \eqref{eq:spectral} было получено условие~\eqref{cond:spectral_0} устойчивости разностной схемы \eqref{sch:transition}, \eqref{sch:borders} при $\phi \approx 0$. Предположение о его полезности основано на том, что типичным поведением модели будет некоторый процесс перехода $\phi$ от $1$ к $0$ (<<разрушение>>) за конечное время, а затем бесконечно долгое пребывание в состоянии $\phi \approx 0$.

Однако проделанного анализа уравнения \eqref{eq:spectral} в точке $\phi = 0$ недостаточно. В самом деле, было использовано, что $\epsilon''(0) = 0$ (см. выражение \eqref{eq:epsilon_derivatives}), но не учтено, что $\epsilon''(\phi)$ при малых $\delta$ вблизи $0$ растет очень быстро и достигает больших значений (рис.~\ref{fig:eps_phi_phi}). Получается, что модель, устойчивая в точке $0$, может работать неадекватно в малой ее окрестности. Это нас, конечно, не устраивает -- улучшим оценку устойчивости.

\begin{figure}[!t]
	\centering
	\includegraphics[width=\textwidth]{figures/eps_phi_phi.png}
	\vspace{-0.7cm}
	\caption{Поведение функции $\epsilon''(\phi)$ около $0$.}
	\label{fig:eps_phi_phi}
\end{figure}

Нужно оценить экстремумы функции $\epsilon''(\phi)$ вблизи $0$. Для начала найдем нули $\epsilon'''(\phi)$.
\begin{equation}
	\epsilon''' = \epsilon_0 \cfrac{-6 (f')^3 + 6 (f + \delta) f' f'' - (f + \delta)^2 f'''}{(f + \delta)^4} \tpoint
	\label{eq:epsilon_phi_phi_phi}
\end{equation}
Имеем:
$$\epsilon'''(\phi) = -6 (f')^3 + 6 (f + \delta) f' f'' - (f + \delta)^2 f''' = 0 \tcomma$$
или, с учетом выражения \eqref{eq:epsilon} для $f(\phi)$:
$$-3 \cdot 12^2 (1 - \phi)^3 + 36 \left(4 - 3\phi + \cfrac{\delta}{\phi^3} \right)(1 - \phi)(2 - 3\phi) - \left(4 - 3 \phi + \cfrac{\delta}{\phi^3} \right)^2 (1 - 3 \phi) = 0 \tpoint$$
Пусть $\delta_n \to +0$ и корень $\phi_n \to +0$, причем $\delta_n / \phi_n^3$ ограничено. Тогда:
\begin{gather*}
	-3 \cdot 12^2 \cdot 1^3 + 36 \left(4 + \cfrac{\delta_n}{\phi_n^3} \right) \cdot 1 \cdot 2 - \left(4 + \cfrac{\delta_n}{\phi_n^3} \right)^2 \cdot 1 \to 0 \tcomma \\
	\left(4 + \cfrac{\delta_n}{\phi_n^3} \right)^2 - 72 \left(4 + \cfrac{\delta_n}{\phi_n^3} \right) + 3 \cdot 12^2 \to 0 \tpoint
\end{gather*}
Значит, последовательность $4 + \delta_n / \phi_n^3$ имеет не более двух частичных пределов $\xi_+$ и $\xi_-$ -- корней уравнения $\xi^2 - 72 \xi + 432 = 0$. Первому корню $\xi_+ = 36 + 12 \sqrt{6}$ соответствует
$$\phi_+ = \cfrac{1}{\sqrt[3]{32 + 12 \sqrt{6}}} \sqrt[3]{\delta_n} \approx \cfrac{1}{3.945} \sqrt[3]{\delta_n} \tsemicolon$$
второму корню $\xi_- = 36 - 12 \sqrt{6}$ соответствует
$$\phi_- = \cfrac{1}{\sqrt[3]{32 - 12 \sqrt{6}}} \sqrt[3]{\delta_n} \approx \cfrac{1}{1.376} \sqrt[3]{\delta_n} \tpoint$$

Из проделанного рассуждения следует, что при $\delta \to +0$ функция $\epsilon'''(\phi)$ имеет в окрестности $0$ два корня
\begin{equation}
	\phi_{\pm} = \cfrac{1}{\sqrt[3]{32 \pm 12 \sqrt{6}}} \sqrt[3]{\delta} [1 + o(1)] \tpoint
	\label{eq:epsilon_phi_phi_phi_roots}
\end{equation}

Оценим $\epsilon''(\phi)$ в точках $\phi_{\pm}$ при $\delta \to +0$. Пусть $\phi = (1 / c) \sqrt[3]{\delta}, \; c \in \Real$. Тогда:
$$\epsilon'' = \epsilon_0 \cfrac{24 c^5 (8 - c^3)}{(4 + c^3)^3} \delta^{-5 / 3} [1 + o(1)] \tpoint$$
Отсюда:
\begin{equation}
	\epsilon''(\phi_+) \approx -4.378 \epsilon_0 \delta^{-5 / 3}; \quad \epsilon''(\phi_-) \approx 2.216 \epsilon_0 \delta^{-5 / 3} \tpoint
	\label{est:epsilon_phi_phi_bounds}
\end{equation}
Оценки экстремумов $\epsilon''(\phi)$ вблизи $0$ показаны на рис. \ref{fig:eps_phi_phi_multiplied}.

\begin{figure}[!t]
	\centering
	\includegraphics[width=\textwidth]{figures/eps_phi_phi_multiplied.png}
	\caption{Сравнение функций $\delta^{5 / 3} \epsilon''(\phi)$ при различных значениях $\delta$.}
	\label{fig:eps_phi_phi_multiplied}
\end{figure}

Получим новую оценку устойчивости, рассмотрев уравнение \eqref{eq:spectral} в точке $\phi = \phi_+$. $\epsilon''(\phi_+) \approx -4.4 \epsilon_0 \delta^{-5 / 3}$. Сумма в скобках отрицательна ($\delta$ мало, $\epsilon''(\phi_+)$ велико по модулю и отрицательно), поэтому $f''(\phi_+) > 0$ можно считать равным $0$: оценку это только усилит. Преобразовав уравнение \eqref{eq:spectral}, получим:
$$\lambda(\theta) = 1 + m \tau \left( -\cfrac{2.2 K_\Phi^2 \epsilon_0}{\delta^{5 / 3}} - \cfrac{2 \Gamma}{h^2} \sin^2 \cfrac{\theta}{2} \right) \tpoint$$
Условие $|\lambda(\theta)| \leqslant 1$ справедливо для любого $\theta$, если и только если
\begin{equation}
	\tau \leqslant \cfrac{1}{m} \left( \cfrac{1.1 K_\Phi^2 \epsilon_0}{\delta^{5 / 3}} + \cfrac{\Gamma}{h^2} \right)^{-1} \tpoint
	\label{cond:spectral_better_theoretical}
\end{equation}

Для применения на практике оценку \eqref{cond:spectral_better_theoretical} нужно брать <<с запасом>> (экспериментальное обоснование будет дано позже). Сделаем оценку строже, примерно удвоив знаменатель:
\begin{equation}
	\tau \leqslant \cfrac{1}{2m} \left( \cfrac{K_\Phi^2 \epsilon_0}{\delta^{5 / 3}} + \cfrac{\Gamma}{h^2} \right)^{-1} \tpoint
	\label{cond:spectral_better}
\end{equation}

Более простая оценка не слабее оценки \eqref{cond:spectral_better} выглядит следующим образом:
\begin{equation}
	\tau \leqslant \cfrac{1}{4m} \min \left(\cfrac{\delta^{5 / 3}}{K_\Phi^2 \epsilon_0}, \; \cfrac{h^2}{\Gamma} \right) \tpoint
	\label{cond:spectral_better_simpler}
\end{equation}

Полученная оценка \eqref{cond:spectral_better} устойчивости разностной схемы \eqref{sch:transition}, \eqref{sch:borders} содержит все параметры уравнения \eqref{eq:one_dim}, кроме $l$.