%!TEX root = ../main.tex

\section{The mathematical model and problem statement}
\label{sec:problem_and_model}

\subsection{The mathematical model}
Let us describe briefly the phase-field mathematical model describing
the electrical breakdown channel propagation
according to~\cite{pitike_dielectric_breakdown}.
The model assumes that the electrical breakdown evolution
is described as phase transition from the initial, undamaged, state of
the medium to its damaged state. The spatial domain occupied by the
damaged phase is considered as a breakdown channel.
The breakdown channel development is described as a process of
formation of the domains occupied by the damaged phase.

We assume that the medium occupies a bounded spatial domain~$\Omega \subset
\Real^3$. The state of the medium is given by two scalar-valued fields~
$\phi = \phi(\vx,t)$, $\phi: \Omega \times [0, +\infty)_t \to [0,
1]$~--- the phase field
and~$\Phi: \Omega \times [0, +\infty)_t \to \Real$~--- the electric potential.

It is assumed that~$\phi$ is continuous and suffucuently smooth
with values~$\phi\in[0,1]$. The value~$\phi=1$
corresponds to the initial, undamaged state of the medium,
and the value~$\phi=0$~--- to the damaged phase, which is
related to the state of the medium in the breakdown channel.

The only property of the medium is its electric permittivity~$\epsilon$.
Its value depends on the state of the medium and is
given by:
%
\begin{equation}
  \epsilon(\vx, t) = \epsilon[\phi] = \cfrac{\epsilon_0(\vx)}{f(\phi(\vx, t)) + \delta} \tcomma \quad f(\phi) = 4\phi^3 - 3\phi^4 \tpoint
  \label{eq:epsilon}
\end{equation}
Here~$\epsilon_0(\vx)$ is the permittivity of the undamaged medium;
$f(\phi)$ is the so-called interpolation function smoothly connects
the~$0$ and~$1$ values. It is assumed
that~$f(0) = 0$, $f(1) = 1$, $f'(0) = f'(1) = 0$;
$0 < \delta \ll 1$ is a small regularizing parameter.
Note that at~$\phi = 1$ we have~$\epsilon(\vx, t) \approx
\epsilon_0(\vx)$ (which corresponds to the undamaged phase)
and at~$\phi = 0$ we have $\epsilon(\vx, t) = \epsilon_0(\vx) /\delta \gg 1$
(which corresponds to the damaged phase, assumed to be an ideal conductor).

The following expression for the Helmholtz free energy~$\Pi$ of the system
is postulated:
\begin{gather}
  \Pi = \int \limits_\Omega \pi d \vx \tcomma
  \label{eq:energy} \\
  \pi = -\half \epsilon[\phi] \scalsq{\Phi} + \Gamma \cfrac{1 - f(\phi)}{l^2} + \cfrac{\Gamma}{4} \scalsq{\phi} \tpoint
  \label{eq:energy_density}
\end{gather}
Here~$\Gamma > 0$ and~$l > 0$ are parameters;
$\Gamma$ can be interpreted as the energy needed to create a breakdown
channel of unit length and~$l$ is the characteristic width of the
diffuse interface.

Evolution of fields~$\phi$ and $\Phi$ is governed by the following
system of equation:
%
\begin{equation}\label{eq:var}
\cfrac{\delta \Pi}{\delta \Phi} = 0 \tsemicolon \qquad \cfrac{1}{m}
\partt{\phi} = -\cfrac{\delta \Pi}{\delta \phi} \tpoint.
\end{equation}
Here~$m > 0$ is the so-called mobility. It has the meaning of the change speed
of~$\phi$ under a unit ``force'' applied.
According to~\eqref{eq:var}, the fields evolve in a way to minimize~\eqref{eq:energy}.

The explicit form of~\eqref{eq:var} with~$\Pi$ given
by~\eqref{eq:energy} is:
\begin{gather}
  \Div(\epsilon[\phi] \nabla \Phi) = 0 \tsemicolon
  \label{eq:phi:a} \\ % \eqref{eq:Phi} \\
  \cfrac{1}{m} \partt{\phi} = \half \epsilon'(\phi) \scalsq{\Phi} + \cfrac{\Gamma}{l^2} f'(\phi) + \half \Gamma \Delta \phi,
  \label{eq:phi:b}
\end{gather}
%
where~$(\cdot)' \equiv (\cdot)_\phi'$ and $(\vect{a},\vect{b})$ is the
dot product.
The first equation,~\eqref{eq:phi:a}, in this system is linear in~$\Phi$.
The second one,~\eqref{eq:phi:b}, is a nonlinear Allen-Cahn type equation.

\subsection{One-dimensional problem}
Consider a one-dimensional form of~\eqref{eq:phi:a}
and~\eqref{eq:phi:b}.
Let the closed domain~$\clOmega$ be~$\clOmega = [0, W]_x \times [0, H]_y
\times I_z$,
with~$W,H > 0$ and~$I$ being a closed interval.
Let~$\epsilon_0(\vx) = \epsilon_0(x)$
and also~$\phi(\vx, 0) = \phi_0(x)$,~--- i.e., the distribution of
electric permittivity and the initial condition for~$\phi$ depend only on the
$x$-coordinate.
We also assume that the following boundary conditions are defined
at~$\partial\Omega$: $\phi|_{x = 0} = \phi_l(t)$, $\phi|_{x = W} = \phi_r(t)$
and also~$\partflvn{\phi}= 0$ at the faces of~$\clOmega$ perpendicular to the~$Oy$ and~$Oz$
axes. For~$\Phi$ we set: $\Phi|_{y = 0} = \Phi^-$, $\Phi|_{y = H} = \Phi^+$, where~$\Phi^-, \Phi^+ \in \Real$,
and also~$\partflvn{\Phi} = 0$ at the faces of~$\clOmega$ perpendicular to the~$Ox$ and~$Oz$ axes.
Here~$\partflvn{}$ denotes the directional derivative along the outward unit normal~$\vect{n}$ to~$\partial\Omega$.

Taking  the formulated assumptions into account,
the solution of~\eqref{eq:phi:a},\eqref{eq:phi:b} has the form
$\phi(\vx, t) = \phi(x, t)$, $\Phi(\vx, t) = \Phi(y, t)$,~---
i.e.,  $\phi$ does not depend on~$y$ and~$z$, $\Phi$~--- on~$x$ and~$z$.

Therefore, \eqref{eq:phi:a} can be reduced to:
\begin{equation}
  0 = \Div(\epsilon[\phi] \nabla \Phi) = (\nabla \epsilon, \nabla \Phi) + \epsilon \Delta \Phi \equiv  \epsilon \triangle \Phi,
  \label{eq:Phi_one_dim}
\end{equation}
since~$(\nabla \epsilon, \nabla \Phi) = 0$ as~$\epsilon$ does not depend on~$y$ and~$z$.
The solution of~\eqref{eq:Phi_one_dim} satisfying the boundary conditions is~$\Phi(\vx, t) = \Phi^- + (y / H)(\Phi^+ - \Phi^-)$.

Substituting the obtained solution for~$\Phi$ into~\eqref{eq:phi:b} we obtain:
\begin{equation}
  \cfrac{1}{m} \partt{\phi} = \half \epsilon'(\phi) \left( \cfrac{\Phi^+ - \Phi^-}{H} \right)^2 + \cfrac{\Gamma}{l^2} f'(\phi) + \half \Gamma \partxx{\phi} \tpoint
  \label{eq:one_dim_preparation}
\end{equation}

The solution~$\phi(x, t)$ of~\eqref{eq:one_dim_preparation} is defined in
the spatially one-dimensional domain~$[0, W]_x \times [0, +\infty)_t$.
For further convenience we write~\eqref{eq:one_dim_preparation} as:
\begin{equation}
  \cfrac{1}{m} \partt{\phi} = \half K_\Phi^2 \epsilon'(\phi) + \cfrac{\Gamma}{l^2} f'(\phi) + \half \Gamma \partxx{\phi},\quad
  K_\Phi = \norm{\nabla \Phi} = (\Phi^+ - \Phi^-) / H. 
  \label{eq:one_dim}
\end{equation}
Note that now the parameters~$\Phi^+$, $\Phi^-$ and~$H$ do not enter the considered equation explicitly.

Equation~\eqref{eq:one_dim} is supplemented by the initial condition
\begin{equation}
	\phi(x, 0) = \phi_0(x)
	\label{eq:one_dim_initial}
\end{equation}
and also by the boundary conditions
\begin{equation}
  \phi(0, t) = \phi_l(t), \quad \phi(W, t) = \phi_r(t) \tpoint
  \label{eq:one_dim_marginal}
\end{equation}

For simplicity of further analysis we also assumed that~$\epsilon_0(x) = \text{const}$.

So, the pair of functions~$\phi$ and~$\Phi$,
where~$\phi$ is the solution of the problem~\eqref{eq:one_dim},\eqref{eq:one_dim_initial},\eqref{eq:one_dim_marginal}
and~$\Phi$ is given by~$\Phi = \Phi^- + (y / H)(\Phi^+ - \Phi^-)$,
satisfies~\eqref{eq:phi:a}, \eqref{eq:phi:b} under the provided assumptions.

% EOF
\endinput
