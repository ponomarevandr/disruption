%!TEX root = ../main.tex

\section{Mathematical model and problem statement}
\label{sec:problem_and_model}

\subsection{Mathematical model}
Let us describe briefly a phase-field mathematical model for
description of electrical breakdown channel propagation
according to~\cite{pitike_dielectric_breakdown}.
The model assumes that the electrical breakdown evolution
is described as a phase transition from the initial, undamaged state of
the medium, to its damaged state. Spatial domain occupied by the
damaged phase is considered as a breakdown channel.
The breakdown channel development is described as a process of
formation of the domains occupied by the damaged phase.

Assume that the medium occupies bounded spatial domain~$\Omega \subset
\Real^3$. The state of the medium is given by two scalar-valued fields~
$\phi = \phi(\vx,t)$, $\phi: \Omega \times [0, +\infty)_t \to [0,
1]$~--- the phase field
and~$\Phi: \Omega \times [0, +\infty)_t \to \Real$~--- the electric potential.

It is assumed that~$\phi$ is continuous and suffucuently smooth
with values~$\phi\in[0,1]$. The value~$\phi=1$
corresponds to the initial, undamaged state of the medium,
and the value~$\phi=0$ correspond to the damaged phase, which is
related to the state of medium in the breakdown channel.

The only property of the medium is its electric permittivity~$\epsilon$.
Its value depends on the state of the medium and is
given by:
%
\begin{equation}
  \epsilon(\vx, t) = \epsilon[\phi] = \cfrac{\epsilon_0(\vx)}{f(\phi(\vx, t)) + \delta} \tcomma \quad f(\phi) = 4\phi^3 - 3\phi^4 \tpoint
  \label{eq:epsilon}
\end{equation}
Here~$\epsilon_0(\vx)$ is permittivity of the undamaged medium;
$f(\phi)$ is the so called interpolation function which smoothly
interpolates betwee~$0$ and~$1$ values. It is assumed
that~$f(0) = 0$, $f(1) = 1$, $f'(0) = f'(1) = 0$;
$0 < \delta \ll 1$ is a small regularizing parameter.
Note that at~$\phi = 1$ one have~$\epsilon(\vx, t) \approx
\epsilon_0(\vx)$ (which correspond to the undamaged phase)
and at~$\phi = 0$ one have $\epsilon(\vx, t) = \epsilon_0(\vx) /\delta \gg 1$
(which corresponds to the damaged phase, assumed to be an ideal conductor).

The following expression for Helmholtz free energy~$\Pi$ of the system
is postulated:
\begin{gather}
  \Pi = \int \limits_\Omega \pi d \vx \tcomma
  \label{eq:energy} \\
  \pi = -\half \epsilon[\phi] \scalsq{\Phi} + \Gamma \cfrac{1 - f(\phi)}{l^2} + \cfrac{\Gamma}{4} \scalsq{\phi} \tpoint
  \label{eq:energy_density}
\end{gather}
Here~$\Gamma > 0$ and~$l > 0$ are parameters;
$\Gamma$ can be interpreted as energy needed to create breakdown
channel of the unit length and~$l$ is characteristic width of the
diffuse interface.

Evolution of fields~$\phi$ and $\Phi$ are governed by the following
system of equation:
%
\begin{equation}\label{eq:var}
\cfrac{\delta \Pi}{\delta \Phi} = 0 \tsemicolon \qquad \cfrac{1}{m}
\partt{\phi} = -\cfrac{\delta \Pi}{\delta \phi} \tpoint.
\end{equation}
Here~$m > 0$ is the so called mobility. It has a meaning of the speed
of change in~$\phi$ under applied unit ``force''.
According to~\eqref{eq:var}, the fields are evolved in a way to minimize~\eqref{eq:energy}.

An explicit form of~\eqref{eq:var} with~$\Pi$ given
by~\eqref{eq:energy} reads:
\begin{gather}
  \Div(\epsilon[\phi] \nabla \Phi) = 0 \tsemicolon
  \label{eq:phi:a} \\ % \eqref{eq:Phi} \\
  \cfrac{1}{m} \partt{\phi} = \half \epsilon'(\phi) \scalsq{\Phi} + \cfrac{\Gamma}{l^2} f'(\phi) + \half \Gamma \Delta \phi,
  \label{eq:phi:b}
\end{gather}
%
where~$(\cdot)' \equiv (\cdot)_\phi'$ and $(\vect{a},\vect{b})$ is a
dot product.
The first equation,~\eqref{eq:phi:a}, in this system is linear in~$\Phi$.
The second one,~\eqref{eq:phi:b}, is nonlinear Allen-Cahn type equation.

\subsection{One-dimensional problem}
Consider a one-dimensional form of~\eqref{eq:phi:a}
and~\eqref{eq:phi:b}.
Let the closed domain~$\Bar\Omega$ be~$\clOmega = [0, W]_x \times [0, H]_y
\times I_z$,
with~$W,H > 0$ and~$I$ being a closed interval.
Let~$\epsilon_0(\vx) = \epsilon_0(x)$
and also~$\phi(\vx, 0) = \phi_0(x)$,~--- i.e., distribution of
electric permittivity and initial condition for~$\phi$ depend only on
$x$-coordinate.
We also assume that the following boundary conditions are defined
at~ $\partial\clOmega$: $\phi|_{x = 0} = \phi_l(t)$, $\phi|_{x = W} = \phi_r(t)$,
and also~$\partflvn{\phi}= 0$ at the faces of~$\clOmega$ which are perpendicular to~$Oy$ and~$Oz$
axes. For~$\Phi$ we set: $\Phi|_{y = 0} = \Phi^-$, $\Phi|_{y = H} = \Phi^+$, where~$\Phi^-, \Phi^+ \in \Real$,
and also~$\partflvn{\Phi} = 0$ at the faces of~$\clOmega$ which are perpendicular to~$Ox$ and~$Oz$ axes.
Here~$\partflvn{}$ defines derivative in the direction of the outward unit normal~$\vect{n}$ to the~$\partial\Bar\Omega$.

Taking into account the formulated assumptions,
the solution of~\eqref{eq:phi:a},\eqref{eq:phi:b} will be of the form
$\phi(\vx, t) = \phi(x, t)$, $\Phi(\vx, t) = \Phi(y, t)$,~---
i.e.,  $\phi$ does not depend~$y$ and~$z$, $\Phi$~--- does not depend on~$x$ and~$z$.

In this case~\eqref{eq:phi:a} can be reduced to:
\begin{equation}
  0 = \Div(\epsilon[\phi] \nabla \Phi) = (\nabla \epsilon, \nabla \Phi) + \epsilon \Delta \Phi \equiv  \epsilon \triangle \Phi,
  \label{eq:Phi_one_dim}
\end{equation}
since~$(\nabla \epsilon, \nabla \Phi) = 0$ as~$\epsilon$ does not depend on~$y$ and~$z$.
Solution of~\eqref{eq:Phi_one_dim} which satisfies boundary conditions is~$\Phi(\vx, t) = \Phi^- + (y / H)(\Phi^+ - \Phi^-)$.

Substituting the obtained solution for~$\Phi$ into~\eqref{eq:phi:b} one arrives to:
\begin{equation}
  \cfrac{1}{m} \partt{\phi} = \half \epsilon'(\phi) \left( \cfrac{\Phi^+ - \Phi^-}{H} \right)^2 + \cfrac{\Gamma}{l^2} f'(\phi) + \half \Gamma \partxx{\phi} \tpoint
  \label{eq:one_dim_preparation}
\end{equation}

Solution~$\phi(x, t)$ of~\eqref{eq:one_dim_preparation} is defined in
the spatially one-dimensional domain~$[0, W]_x \times [0, +\infty)_t$.
For further convenience we write~\eqref{eq:one_dim_preparation} as
\begin{equation}
  \cfrac{1}{m} \partt{\phi} = \half K_\Phi^2 \epsilon'(\phi) + \cfrac{\Gamma}{l^2} f'(\phi) + \half \Gamma \partxx{\phi},\quad
  K_\Phi = \norm{\nabla \Phi} = (\Phi^+ - \Phi^-) / H. 
  \label{eq:one_dim}
\end{equation}
Note that now parameters~$\Phi^+$, $\Phi^-$ and~$H$ does not enter the latter equation explicitly.

Equation~\eqref{eq:one_dim} has to be supplemented by the initial conditions
\begin{equation}
	\phi(x, 0) = \phi_0(x);
	\label{eq:one_dim_initial}
\end{equation}
and also by the boundary conditions of the form:
\begin{equation}
  \phi(0, t) = \phi_l(t), \quad \phi(W, t) = \phi_r(t) \tpoint
  \label{eq:one_dim_marginal}
\end{equation}

For simplicity in what further it also assumed that~$\epsilon_0(x) = \text{const}$.

So, a pair of functions~$\phi$ and~$\Phi$,
where~$\phi$ is a solution of the problem~\eqref{eq:one_dim},\eqref{eq:one_dim_initial},\eqref{eq:one_dim_marginal}
and~$\Phi$ given by~$\Phi = \Phi^- + (y / H)(\Phi^+ - \Phi^-)$,
satisfies~\eqref{eq:phi:a}, \eqref{eq:phi:b} under provided assumptions.

% EOF
\endinput
