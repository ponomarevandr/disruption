%!TEX root = ../main.tex

\section{Introduction}

Electric breakdown of solid dielectric is a rapid process which
involves a variety of mutually interrelated physical
mechanisms~\cite{vorobiev_dielectric_physics, dissado_1992}. Their
complete identification and characterization at the level acceptable
for the predictive first-principal modelling and simulation is
practically impossible at present. Due to this reason, the promising
approach to the electric breakdown modelling is utilization of the
phenomenological models of the reasonable complexity, suitable to
analyse practical settings.

Among a variety of the approaches suggested to describe electric breakdown the particular place is occupied by
the phase field (or diffuse interface) model originally introduced in~\cite{pitike_dielectric_breakdown}.
According to the phase field modelling framework, the essentially one dimensional
breakdown channel is described using scalar phase field, i.e., a
smooth
function~$\phi = \phi(\vect{x},t)$ with values within the~$[0, 1]$
interval.
It is assumed that the channel occupies
a spatial domain where $\phi = 0$, while in completely undamaged medium~$\phi = 1$. The
spatial domain with~$0 < \phi < 1$ is considered to be a ``diffuse'' boundary of the finite
width which separate damaged and undamaged medium. The process of the breakdown
channel development (``growth'') is described as evolution of~$\phi$ in time~$t$, governed by
the physically motivated evolutionary~PDE. Damaged and undamaged states of the
medium are treated as two different phases. The process itself is described as a phase
transition between undamaged and damaged phases which occurs under specified
conditions.
Diffuse interface models are mainly phenomenological ones. In other
words they are mostly based, by assumption, on certain macroscopic laws and
fundamental assumptions and doesn't rely on any first-principal
physics or elementary mechanisms related to the
particular macroscopic problem under consideration.

Up to now diffuse interface models provides a solid and widely used
framework to describe multi-phase phenomena in
hydrodynamics~\cite{lamorgese_flow_modeling, kim_fluid_flows, xu_hydrodynamics},
solid mechanics and mechanics of fracture~\cite{ambati_fracture},
material science~\cite{provatas_materials}, solidification and phase
transition problems,~\cite{boettinger_solidification,
  cartalade_phase_separation, gransaly_solidification},
phase field crystals~\cite{emmerich_crystal, asadi_crystal,
  provatas_crystal}.

The model suggested
in~\cite{pitike_dielectric_breakdown}
can be considered as further generalization of widely known diffuse
fracture models in solid mechanics.
Investigation and further generalizations of the model suggested
in~\cite{pitike_dielectric_breakdown},
are considered in~\cite{zipunova_higher_codimension, zipunova_conservative,
zipunova_thermomechanical}.

The main topic of this paper is an analysis of the stability of the
stationary equilibrium solutions of the phase field model for
electric breakdown development.
In fact, it is shown that in the considered model,
developing of the electric breakdown channel is a process
closely related to the loss of the stability of the equilibrium
solutions.

The model is considered in its simplest form, as its presented
in~\cite{pitike_dielectric_breakdown}.
Moreover, spatially one-dimensional setting is used to perform
theoretical analysis. As a final result stability conditions are
formulated in terms of the model parameters, under which small perturbations of the constant
equilibrium solution loss stability and solution evolves into
breakdown-channel-like structure.

To confirm obtained theoretical results we present a simple explicit
finite-difference scheme which allows to analyse breakdown process
numerically. Since the goal of our computations is to analyse
a~process
of development of instabilities in the initial phase field
distribution,
a careful and comprehensive analysis if the stability of the scheme is
performed in order to clearly distinguish between numerical artifacts and
development of the ``physical'' instabilities of the solution.
The performed simulations completely confirms the obtained theoretical
results.

The structure of the paper is as follows.
In section~2 we present mathematical models which are considered in
the rest of the paper. Section~3 devoted to the stability analysis of
the equilibrium solutions. Finally, in section~4 we present numerical
results which confirms theoretical findings.
In conclusion we outline and summarize the main results of the paper.

\endinput

% EOF