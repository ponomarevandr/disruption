%!TEX root = ../main.tex

\section{Introduction}

Electrical breakdown of a solid dielectric is a rapid process, which
involves a variety of mutually interrelated physical
mechanisms~\cite{vorobiev_dielectric_physics, dissado_1992}. At present, it is
practically impossible to identify and characterize them at the level acceptable
for predictive first-principal modelling and simulation. Therefore, a promising
approach to the electrical breakdown modelling is to utalize reasonably complex
phenomenological models suitable for practical settings analysis.

Among the variety of approaches suggested to describe electrical breakdown,
a particular place is occupied by the phase field (or diffuse interface) model
originally introduced in~\cite{pitike_dielectric_breakdown}.
According to the phase field modelling framework, the essentially one dimensional
breakdown channel is described using a scalar phase field, i.e., a smooth
function~$\phi = \phi(\vect{x},t)$ with values within the~$[0, 1]$
interval. It is assumed that the channel occupies
a spatial domain where $\phi = 0$, while in completely undamaged medium~$\phi = 1$. The
spatial domain with~$0 < \phi < 1$ separating damaged and undamaged medium
is considered to be a ``diffuse'' boundary of finite width. The process of the breakdown
channel development (``growth'') is described as evolution of~$\phi$ over time~$t$, governed by
a physically motivated evolutionary~PDE. The damaged and undamaged states of the
medium are treated as two different phases. The process itself is described as phase
transition between them. The transition occurs under specified conditions.
Diffuse interface models are mainly phenomenological ones. In other
words, they are mostly based on certain macroscopic laws and
fundamental assumptions and do not rely on first-principal
physics and elementary mechanisms related to the
particular macroscopic problem under consideration.

Today diffuse interface models provide a solid and widely used
framework to describe multi-phase phenomena in
hydrodynamics~\cite{lamorgese_flow_modeling, kim_fluid_flows, xu_hydrodynamics},
solid and fracture mechanics~\cite{ambati_fracture},
material science~\cite{provatas_materials}, solidification and phase
transition problems,~\cite{boettinger_solidification,
  cartalade_phase_separation, gransaly_solidification},
phase field crystals~\cite{emmerich_crystal, asadi_crystal,
  provatas_crystal}.

The model suggested
in~\cite{pitike_dielectric_breakdown}
can be considered as a generalization of widely known diffuse
interface models in solid mechanics.
The model is investigated and further generalized in~\cite{zipunova_higher_codimension, zipunova_conservative,
zipunova_thermomechanical}.

The main aim of this paper is to analyse stability of
stationary equilibrium solutions of the phase-field model for the
electrical breakdown development.
In fact, it is shown that the development of the electrical breakdown channel
in the considered model is closely related to the stability loss of
the equilibrium solutions.

The model is considered in its simplest form, as it is presented
in~\cite{pitike_dielectric_breakdown}.
Moreover, spatially one-dimensional setting is used to perform
theoretical analysis. As a final result, a stability condition is
formulated in terms of the model parameters. Whenever the condition is
violated, the intact medium loses stability and small perturbations
cause it to evolve into a breakdown-channel-like structure.

To confirm the obtained theoretical results we present a simple explicit
finite-difference scheme allowing to analyse the breakdown process
numerically. Since the goal of our computations is to analyse the instability
development of in the initial phase field distribution,
a careful and comprehensive analysis of stability for the scheme is
performed. The analysis is essential to clearly distinguish between numerical
artifacts and the development of ``physical'' instabilities in the solution.
The performed simulation completely confirms the obtained theoretical
results.

The structure of the paper is as follows.
In section~\ref{sec:problem_and_model} we present the mathematical model
considered in the rest of the paper. Section~\ref{sec:theoretical_analysis} is
devoted to the stability analysis of equilibrium solutions. In
section~\ref{sec:differential_scheme} we analyse the finite-difference scheme. Finally, in
section~\ref{sec:computational_analysis} we present numerical results confirming the
theoretical findings. In conclusion, we outline and summarize the main results of the paper.

\endinput

% EOF