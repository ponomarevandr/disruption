%!TEX root = ../main.tex

\section{Параллели с механикой}

\begin{frame}{Связь с моделированием трещин}
\begin{itemize}
	\item Изучение статьи \cite{sargado_high_accuracy}
\end{itemize}
\vspace{0.5cm}
\centering
\begin{tabular}{|>{\centering\arraybackslash}m{7cm}|>{\centering\arraybackslash}m{7cm}|}
	\hline
	& \\[-3mm]
	смещение $\vect{u}$ 										& электрический потенциал $\Phi$ \\[1mm]
	\hline
	& \\[-3mm]
	деформация $\hat{\epsilon}$									& напряженность электрического поля $\vect{E}$ \\[1mm]
	\hline
	& \\[-3mm]
	упругость (жёсткость) $\hat{C}$								& диэлектрическая проницаемость $\epsilon$ \\[1mm]
	\hline
	& \\[-3mm]
	напряжение $\hat{\sigma} = \hat{C} : \hat{\epsilon}$	& электрическая индукция $\vect{D} = \epsilon \vect{E}$ \\[1mm]
	\hline
\end{tabular}
\end{frame}


\begin{frame}{Связь с моделированием трещин}
\begin{itemize}
	\item Рост канала электрического пробоя соответствует не разрушению, а отвердеванию:
	\[
		\epsilon(\phi = 0) \gg \epsilon(\phi = 1), \qquad \hat{C}(\phi = 0) \ll \hat{C}(\phi = 1).
	\]
	\item Задача разрушения: граничные условия на смещение / электрический потенциал,\\
	задача отвердевания: граничные условия на напряжение / электрическую индукцию.
\end{itemize}
\end{frame}


\begin{frame}{Модель с измененными граничными условиями}
	\[
		\Div(\epsilon[\phi] \nabla \Phi) = 0
	\]
	\[
		\cfrac{1}{m} \partt{\phi} = \half \epsilon'[\phi] \cdot \| \nabla \Phi \|^2 + \cfrac{\Gamma}{l^2} f'(\phi) + \cfrac{\Gamma}{2} \Delta \phi
	\]
	\[
		\Phi|_{y = 0} = -\Phi|_{y = H} = \Phi^+(t)
	\]
	\[
		q^+ = \int\limits_{y = 0} \epsilon \cfrac{\partial \Phi}{\partial y} dS = \int\limits_{y = H} \epsilon \cfrac{\partial \Phi}{\partial y} dS
	\]
	\begin{itemize}
		\item Потенциал на обкладках $\Phi^+(t) = -\Phi^-(t)$ есть функция времени
		\item $q^+ = -q^-$ -- постоянный поверхностный заряд на обкладках
	\end{itemize}
\end{frame}