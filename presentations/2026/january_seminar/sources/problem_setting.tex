%!TEX root = ../main.tex

\section{Постановка задачи}

\begin{frame}{Математическая модель}
\begin{block}{Модель типа диффузной границы}
	Вещество находится в разных фазах. Состояние вещества описывается гладкой функцией $\phi(\vx, t)$ -- фазовым полем.
\end{block}
\begin{itemize}
	\item $\phi = 1$ -- неповрежденная среда
	\item $\phi = 0$ -- полностью разрушенная среда
	\item Зона $\phi \in (0, 1)$ -- диффузная граница
	\item На разрушение среды тратится энергия
\end{itemize}
\begin{figure}
	\includegraphics[width=0.5\textwidth]{figures/diffuse_edge.jpg}
\end{figure}
\end{frame}


\begin{frame}{Математическая модель}
\vspace{-0.3cm}
\begin{block}{Уравнения динамики системы}
	\begin{itemize}
		\item Уравнение электрического потенциала $\Phi$:
		\[
			\Div(\epsilon[\phi] \nabla \Phi) = 0
		\]
		\item Уравнение фазового поля $\phi$ (типа Аллена--Кана):
		\[
			\cfrac{1}{m} \partt{\phi} = -F'(\phi; \| \nabla \Phi \|) + \cfrac{\Gamma}{2} \Delta \phi
		\]
	\end{itemize}
\end{block}
\begin{itemize}
	\item Плотность свободной энергии
	\vspace{-0.2cm}
	\[
		\pi = F(\phi; \| \nabla \Phi \|) + \cfrac{\Gamma}{4} \gradsq{\phi}
	\]
	\item $m$, $\Gamma$ -- параметры модели, константы
\end{itemize}
\end{frame}


\begin{frame}{Математическая модель}
\[
	F(\phi; \| \nabla \Phi \|_2) = -\half \epsilon[\phi] \cdot \| \nabla \Phi \|_2^2 + \Gamma \cfrac{1 - f(\phi)}{l^2}
\]
\[
	\epsilon(\vx, t) = \cfrac{\epsilon_0(\vx)}{f(\phi(\vx, t)) + \delta}
\]
\[
	f(\phi) = 4 \phi^3 - 3 \phi^4
\]
\begin{itemize}
	\item $l$, $\delta$ -- параметры модели
\end{itemize}
\end{frame}


\begin{frame}{Разностная схема}
\vspace{-0.9cm}
\[
	\cfrac{1}{m} \partt{\phi} = -F'(\phi; \| \nabla \Phi \|) + \cfrac{\Gamma}{2} \partxx{\phi}
\]
\vspace{-0.4cm}
\begin{itemize}
	\item $\| \nabla \Phi \|$ -- параметр
\end{itemize}
\begin{block}{Разностная задача}
	\[
		\cfrac{1}{m} \cfrac{\phi_i^{j + 1} - \phi_i^j}{\tau} = -F'(\phi_i^j; \| \nabla \Phi \|) + \cfrac{\Gamma}{2} \cfrac{\phi_{i + 1}^j - 2 \phi_i^j + \phi_{i - 1}^j}{h^2}
	\]
	\[\phi_i^0 = \phi_0(ih); \quad \phi_0^j = \phi_l(j \tau); \quad \phi_n^j = \phi_r(j \tau)\]
	Сетка регулярная; $\tau$ -- шаг по времени, $h$ -- шаг по пространству.
\end{block}
Явная разностная схема первого порядка по времени, второго по пространству.
\end{frame}


\begin{frame}{Открытые вопросы}
\large
\begin{itemize}
	\item Снижение вычислительных затрат на расчет
	\item Уточнение определяющих соотношений модели
	\item Воспроизведение в модели физического эксперимента
\end{itemize}
\begin{center}
\begin{minipage}{0.95\textwidth}
\begin{block}{Цель работы}
	Предложить ответы на назревшие вопросы
\end{block}
\end{minipage}
\end{center}
\end{frame}


\begin{frame}{Решения}
\Large
\begin{itemize}
	\item Интегрирование по времени с адаптивным шагом
	\item Настройка <<большой>> программы для моделирования
	\item Использование результатов из теории трещин
\end{itemize}
\end{frame}