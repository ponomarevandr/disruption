\documentclass[aspectratio=169]{beamer}

%%% Работа с русским языком
\usepackage{cmap}					% поиск в PDF
\usepackage{mathtext} 				% русские буквы в формулах
\usepackage[T2A]{fontenc}			% кодировка
\usepackage[utf8]{inputenc}			% кодировка исходного текста
\usepackage[english,russian]{babel}	% локализация и переносы
\usepackage{indentfirst}
\frenchspacing

%%% Дополнительная работа с математикой
\usepackage{amsmath,amsfonts,amssymb,amsthm,mathtools}  % AMS

%%% Текст в колонки
\usepackage{multicol}

%%% Системы уравнений
\usepackage{cases}

%%% Таблицы
\usepackage{array}

%%% Картинки
\usepackage{graphicx}
\usepackage{float}

%%% Список литературы
\usepackage[sorting=none]{biblatex}
\renewbibmacro{in:}{\ifentrytype{article}
    {}
    {\bibstring{in}\printunit{\intitlepunct}}
}
\addbibresource{references.bib}

%%% Гиперссылки
\usepackage{hyperref}

%%% Перенос знаков в формулах (по Львовскому)
\newcommand*{\hm}[1]{#1\nobreak\discretionary{}
{\hbox{$\mathsurround=0pt #1$}}{}}


%%% Свои команды

\newcommand*{\No}{\textnumero}

\newcommand{\vect}[1]{\textbf{\textit{#1}}}
\newcommand{\vx}{{\vect{x}}}

\newcommand{\avphi}{{\widetilde{\phi}}}

\newcommand{\half}{\cfrac{1}{2}}

\newcommand{\partt}[1]{\cfrac{\partial #1}{\partial t}}
\newcommand{\partx}[1]{\cfrac{\partial #1}{\partial x}}
\newcommand{\partxx}[1]{\cfrac{\partial^2 #1}{\partial x^2}}
\newcommand{\partr}[1]{\cfrac{\partial #1}{\partial r}}

\newcommand{\gradscalsq}[1]{\left( \nabla #1, \nabla #1 \right)}

\newcommand{\Natural}{{\mathbb{N}}}
\newcommand{\Real}{{\mathbb{R}}}
\newcommand{\bigO}{{\mathcal{O}}}
\newcommand{\clOmega}{{\overline{\Omega}}}

\newcommand{\norm}[1]{\| #1 \|}
\newcommand{\enorm}{{\| \cdot \|}}

\newcommand{\plapl}[2]{\Div(\norm{\nabla #1}_2^{#2} \nabla #1)}
\newcommand{\bilapl}[1]{\Delta^2 #1}

\newcommand{\gridfunc}[1]{\left[ #1 \right]}


%%% Свои операторы
\DeclareMathOperator{\Div}{{div}}
\DeclareMathOperator{\Const}{{const}}

%%% Цвета
\definecolor{green}{RGB}{0,200,0}
\definecolor{red}{RGB}{200,0,0}

%%% Тема оформления
\usetheme{Madrid}


%%% Титульный лист
\title[Электрический пробой]{Моделирование канала электрического пробоя \\ методом диффузной границы}
\author[]{
	Пономарев Андрей Сергеевич\textsuperscript{1,2}
}
\institute[]{
	\textsuperscript{1}МФТИ (НИУ) \\
	\textsuperscript{2}ИПМ им. М. В. Келдыша РАН
}
\date[]{
	Вычислительная классическая и многофазная \\ гидродинамика и термомеханика сплошной среды \\
	\vspace{3mm}
	4--8 ноября 2024 года
}
\logo{\includegraphics[height=0.8cm]{../figures/labels.jpg}}


\begin{document}

\AtBeginSection[]{
	\begin{frame}{Содержание}
	\Large
	\tableofcontents[currentsection]
	\end{frame}
}

\begin{frame}
\titlepage
\end{frame}

\begin{frame}{Содержание}
\Large
\tableofcontents
\end{frame}

%!TEX root = ../main.tex

\section{Введение}

\begin{frame}{Физическое явление}
\begin{block}{Электрический пробой}
	Явление резкого возрастания тока в диэлектрике при приложении электрического напряжения
	выше критического.
\end{block}
\begin{itemize}
	\item Рассматриваем твердый диэлектрик
	\item Деградация диэлектрических свойств материала
	\item Процесс развивается в ограниченной зоне -- канале пробоя
	\item Сложная физическая природа
\end{itemize}
\end{frame}


\begin{frame}{Математическая модель}
\begin{block}{Модель типа диффузной границы}
	Вещество находится в разных фазах. Состояние вещества описывается гладкой функцией
	$\phi(\vx, t)$ -- фазовым полем.
\end{block}
\begin{itemize}
	\item $\phi = 1$ -- неповрежденная среда
	\item $\phi = 0$ -- полностью разрушенная среда
	\item Зона $\phi \in (0, 1)$ -- диффузная граница
	\item На разрушение среды тратится энергия
\end{itemize}
\begin{figure}
	\includegraphics[width=0.5\textwidth]{figures/diffuse_edge.jpg}
\end{figure}
\end{frame}


\begin{frame}{Математическая модель}
Модель, предложенная в работе \cite{pitike_dielectric_breakdown}:
\begin{itemize}
	\item $\pi = \textcolor{red}{-\half \epsilon[\phi] (\nabla \Phi, \nabla \Phi)} +
	\Gamma \left( \cfrac{1 - f(\phi)}{l^2} + \cfrac{1}{4} (\nabla \phi, \nabla \phi) \right)$
	-- плотность свободной энергии
	\item $\Gamma$ -- энегрия роста канала пробоя на единицу длины
	\item $l$ -- величина <<размытия>> канала
	\item $\epsilon(\vx, t)$ -- диэлектрическая проницаемость среды
	\item $f(\phi)$ -- интерполирующая функция
\end{itemize}
\end{frame}


\begin{frame}{Математическая модель}
\vspace{-0.2cm}
\begin{itemize}
	\item $\epsilon(\vx, t) = \cfrac{\epsilon_0(\vx)}{f(\phi(\vx, t)) +
	\delta}$ -- диэлектрическая проницаемость среды
	\item $f(\phi) = 4 \phi^3 - 3 \phi^4$ -- интерполирующая функция
\end{itemize}
\begin{columns}
\column{0.5\textwidth}
\begin{figure}
	\hspace*{1.4cm}
	\includegraphics[width=0.65\textwidth]{figures/f_form.png}
\end{figure}
\column{0.5\textwidth}
\begin{figure}
	\hspace*{-2cm}
	\includegraphics[width=0.60\textwidth]{figures/eps_form.png}
\end{figure}
\end{columns}
\end{frame}


\begin{frame}{Математическая модель}
\vspace{-0.5cm}
\begin{block}{Уравнения модели}
\begin{itemize}
	\item Уравнение электрического потенциала $\Phi$:
	\begin{equation}
		\Div(\epsilon[\phi] \nabla \Phi) = 0
		\label{equation_potential}
	\end{equation}
	\item Уравнение фазового поля $\phi$:
	\begin{equation}
		\cfrac{1}{m} \partt{\phi} = \half \epsilon'(\phi) \gradscalsq{\Phi} + \cfrac{\Gamma}{l^2} f'(\phi) + \half \Gamma \Delta \phi
		\label{equation_phase}
	\end{equation}
\end{itemize}
\end{block}
Свойства:
\begin{itemize}
	\item связанная система уравнений на $\phi$ и $\Phi$;
	\item уравнение для $\phi$ типа Аллена--Кана, нелинейное.
\end{itemize}
\end{frame}


\begin{frame}{Пример вычислительного эксперимента}
\begin{columns}
\column{0.32\textwidth}
\begin{figure}
	\includegraphics[width=\textwidth]{figures/model_example_1.png}
\end{figure}
\column{0.32\textwidth}
\begin{figure}
	\includegraphics[width=\textwidth]{figures/model_example_2.png}
\end{figure}
\column{0.32\textwidth}
\begin{figure}
	\includegraphics[width=\textwidth]{figures/model_example_3.png}
\end{figure}
\end{columns}
\begin{center}
	Расчет из работы \cite{zipunova_experiment}
\end{center}
\end{frame}


\begin{frame}{Цель работы}
\begin{block}{Цель работы}
	Исследовать качественные характеристики системы уравнений \eqref{equation_potential},
	\eqref{equation_phase} и выполнить ее численный анализ.
\end{block}
Для этого рассмотрим задачу в определенных краевых условиях, упрощающих ее, но позволяющих
установить интересующие свойства.
\end{frame}

%!TEX root = ../main.tex

\section{Постановка задачи}

\begin{frame}{Одномерная задача}
\vspace{-0.3cm}
\begin{itemize}
	\item Область $\clOmega = [0, W]_x \times [0, H]_y \times I_z$ в форме параллелепипеда
	\item $\phi(\vx, 0) = \phi_0(\vx) = \phi_0(x)$, $\epsilon_0(\vx) = \epsilon_0(x)$ не зависят от $y$ и $z$
	\item $\Phi|_{y = 0} = \Phi^- \in \mathbb{R}$, $\Phi|_{y = h} = \Phi^+ \in \mathbb{R}$
\end{itemize}
Подробнее в работе \cite{ponomarev_stability}. \\[0.3cm]
Решением является функция электрического потенциала
\begin{columns}
\column{0.65\textwidth}
	\vspace{-1cm}
	$$\Phi(\vx, t) = \Phi^- \hm + \cfrac{y}{h}(\Phi^+ - \Phi^-).$$
\column{0.35\textwidth}
	\begin{figure}
		\vspace*{-2.3cm}
		\hspace*{0.5cm}
		\includegraphics[width=0.85\textwidth]{figures/one_dim_problem.jpg}
	\end{figure}
\end{columns}
\vspace{-0.4cm}
Тогда уравнение на $\phi$ принимает вид
\begin{block}{}
	$$\cfrac{1}{m} \partt{\phi} = \half K_\Phi^2 \epsilon'(\phi) + \cfrac{\Gamma}{l^2} f'(\phi) + \half \Gamma \partxx{\phi}$$
\end{block}
$K_\Phi = \cfrac{\Phi^+ - \Phi^-}{h} = \norm{\nabla \Phi}$. Будем считать $\epsilon_0 = \Const$.
\end{frame}


\section{Теоретический анализ}

\begin{frame}{Анализ положений равновесия}
\begin{itemize}
	\item Канал пробоя может развиваться из малых возмущений свойств неповрежденной среды. Выясним условия развития.
	\item Рассмотрим положения равновесия вида $\phi(x, t) \equiv C$. Положению равновесия соответствует ноль $C$ функции
\end{itemize}
$$\chi(\phi) = \half K_\Phi^2 \epsilon'(\phi) + \cfrac{\Gamma}{l^2} f'(\phi).$$
\vspace{-0.5cm}
\begin{itemize}
	\item Исследуем положения равновесия на устойчивость спектральным методом: к $\phi \equiv C$ прибавим возмущение $\delta \phi = e^{\alpha t} \cos(\omega x)$, линеаризуем уравнение на $\delta \phi$.
	\item $\chi(\phi)$ возрастает в $C \Longrightarrow$ равновесие неустойчиво; $\chi(\phi)$ убывает в $C \Longrightarrow$ равновесие устойчиво.
\end{itemize}
\end{frame}


\begin{frame}{Анализ положений равновесия}
\vspace{-0.9cm}
\begin{columns}
\column{0.3\textwidth}
	\begin{center}
		<<Слабое>> напряжение
	\end{center}
	\vspace{-0.2cm}
	$$0 \leqslant \cfrac{K_\Phi^2 l^2 \epsilon_0}{2 \Gamma} < \delta^2$$
	\vspace{-0.7cm}
	\begin{figure}
		\includegraphics[width=\textwidth]{figures/equilibriums_case_1.png}
	\end{figure}
\column{0.3\textwidth}
	\begin{center}
		<<Среднее>> напряжение
	\end{center}
	\vspace{-0.2cm}
	$$\delta^2 < \cfrac{K_\Phi^2 l^2 \epsilon_0}{2 \Gamma} < (1 + \delta)^2$$
	\vspace{-0.7cm}
	\begin{figure}
		\includegraphics[width=\textwidth]{figures/equilibriums_case_2.png}
	\end{figure}
\column{0.3\textwidth}
	\begin{center}
		<<Сильное>> напряжение
	\end{center}
	\vspace{-0.2cm}
	$$(1 + \delta)^2 < \cfrac{K_\Phi^2 l^2 \epsilon_0}{2 \Gamma}$$
	\vspace{-0.7cm}
	\begin{figure}
		\includegraphics[width=\textwidth]{figures/equilibriums_case_3.png}
	\end{figure}
\end{columns}
\begin{columns}
\column{0.3\textwidth}
	\hspace{0.5cm}
	$\phi \equiv 0$ неустойчивое \\
	\hspace{0.5cm}
	$\phi \equiv 1$ устойчивое
\column{0.3\textwidth}
	\hspace{0.5cm}
	$\phi \equiv 0$ устойчивое \\
	\hspace{0.5cm}
	$\phi \equiv С_3$ неустойчивое \\
	\hspace{0.5cm}
	$\phi \equiv 1$ устойчивое
\column{0.3\textwidth}
	\hspace{0.5cm}
	$\phi \equiv 0$ устойчивое \\
	\hspace{0.5cm}
	$\phi \equiv 1$ неустойчивое
\end{columns}
\end{frame}


\section{Численный анализ}

\begin{frame}{Разностная схема}
\begin{block}{Разностная задача}
	$$\cfrac{1}{m} \cfrac{\phi_i^{j + 1} - \phi_i^j}{\tau} = \half K_\phi^2 \epsilon'(\phi_i^j) + \cfrac{\Gamma}{l^2} f'(\phi_i^j) + \cfrac{\Gamma}{2} \cfrac{\phi_{i + 1}^j - 2 \phi_i^j + \phi_{i - 1}^j}{h^2}$$
	$$\phi_i^0 = \phi_0(ih); \quad \phi_0^j = \phi_l(j \tau); \quad \phi_n^j = \phi_r(j \tau)$$
	Сетка регулярная; $\tau$ -- шаг по времени, $h$ -- шаг по пространству.
\end{block}
Явная разностная схема первого порядка по времени, второго -- по пространству.
\end{frame}


\begin{frame}{Оценка устойчивости}
\begin{itemize}
	\item Рассмотрим возмущенное решение $\phi_i^j + \delta_i^j$. Линеаризуем уравнение на возмущение $\delta_i^j$ в точке $\phi_i^j = P$:
\end{itemize}
$$\delta_i^{j + 1} = \delta_i^j + m \tau \left( \half K_\Phi^2 \epsilon''(P) \delta_i^j + \cfrac{\Gamma}{l^2} f''(P) \delta_i^j + \cfrac{\Gamma}{2} \cfrac{\delta_{i + 1}^j - 2 \delta_i^j + \delta_{i - 1}^j}{h^2} \right).$$
\begin{itemize}
	\item Применим спектральный признак устойчивости:
\end{itemize}
$$1 > | \lambda(\theta) | = \left| 1 + m \tau \left( \half K_\Phi^2 \epsilon''(P) + \cfrac{\Gamma}{l^2} f''(P) - \cfrac{2 \Gamma}{h^2} \sin^2 \cfrac{\theta}{2} \right) \right|.$$
\begin{itemize}
	\item Исследуем вблизи $P = 0$.
\end{itemize}
\end{frame}


\begin{frame}{Оценка устойчивости}
\begin{block}{Условие устойчивости}
	$$\tau \leqslant \cfrac{1}{2 m} \left( \cfrac{K_\Phi^2 \epsilon_0}{\delta^{5/3}} + \cfrac{\Gamma}{h^2} \right)^{-1}$$
\end{block}
\begin{block}{Упрощенное условие устойчивости}
	$$\tau \leqslant \cfrac{1}{4 m} \min \left(\cfrac{\delta^{5/3}}{K_\Phi^2 \epsilon_0}, \; \cfrac{h^2}{\Gamma} \right)$$
\end{block}
\end{frame}


\begin{frame}{Вычисления: типичное решение}
\vspace{-0.4cm}
\begin{columns}
\column{0.88\textwidth}
\begin{figure}
	\includegraphics[width=\textwidth]{figures/typical_solution.png}
\end{figure}
\column{0.12\textwidth}
\hfill \\
\vspace{3.5cm}
\hspace{-2.5cm}
Узлов по измерениям: \\
\hspace{-2.5cm}
$N_x = 10^3$, $N_t = 10^5$
\end{columns}
\end{frame}


\begin{frame}{Вычисления: проверка устойчивости}
\vspace{-0.4cm}
\begin{columns}
\column{0.7\textwidth}
\begin{figure}
	\includegraphics[width=\textwidth]{figures/stability_bounds.png}
\end{figure}
\column{0.3\textwidth}
Условие устойчивости:
$$\tau \leqslant \cfrac{1}{2m} \left( \cfrac{K_\Phi^2 \epsilon_0}{\delta^{5/3}} +
\cfrac{\Gamma}{h^2} \right)^{-1}$$
\end{columns}
\end{frame}


\begin{frame}{Вычисления: проверка сходимости}
\vspace{-0.3cm}
\begin{figure}
	\includegraphics[width=0.58\textwidth]{figures/convergence_connected.png}
\end{figure}
\vspace{-0.6cm}
\begin{center}
	Здесь, согласно оценке устойчивости, $\tau = \cfrac{h^2}{4m \Gamma}$
\end{center}
\end{frame}


\begin{frame}{Вычисления: положения равновесия}
\vspace{-0.4cm}
\begin{center}
	$(1 + \delta)^2 < \cfrac{K_\Phi^2 l^2 \epsilon_0}{2 \Gamma}$ -- <<сильное>> напряжение
\end{center}
\vspace{-0.4cm}
\begin{columns}
\column{0.5\textwidth}
\begin{figure}
	\includegraphics[width=\textwidth]{figures/equilibrium_3_0.png}
\end{figure}
\vspace{-0.8cm}
\begin{center}
	$\phi \equiv 0$, устойчивое
\end{center}
\column{0.5\textwidth}
\begin{figure}
	\includegraphics[width=\textwidth]{figures/equilibrium_3_1.png}
\end{figure}
\vspace{-0.8cm}
\begin{center}
	$\phi \equiv 1$, неустойчивое
\end{center}
\end{columns}
\end{frame}


\begin{frame}{Свободная энергия}
\vspace{-0.5cm}
$$\Pi(t) = \int\limits_\Omega \pi(x, t) dx$$
$$\pi = -\half \epsilon[\phi] \gradscalsq{\Phi} + \Gamma \left( \cfrac{1 - f(\phi)}{l^2} + \cfrac{1}{4} \gradscalsq{\phi} \right)$$
\begin{itemize}
	\item Уравнения \eqref{equation_potential}, \eqref{equation_phase} выведены так, что система в ходе эволюции стремится в состояние с как можно меньшей полной свободной энергией $\Pi$.
	\item Необходимо, чтобы указанное свойство выполнялось при моделировании.
\end{itemize}
\end{frame}


\begin{frame}{Вычисления: свободная энергия}
\vspace{-0.6cm}
\begin{figure}
	\includegraphics[width=0.88\textwidth]{figures/energy_total.png}
\end{figure}
\end{frame}

%!TEX root = ../main.tex

\section{Исследование обобщения модели}

\begin{frame}{Постановка задачи}
Исследуем распределение фазового поля вокруг проводников ($\phi = 0$) различного вида. Пусть $\Phi \equiv 0$. Рассмотрим следующие краевые задачи:
\begin{enumerate}
	\item $\clOmega = [0, +\infty)_x \times I_y \times J_z$, $\phi|_{x = 0} = 0$, $\phi \to 1$ при $r = x \to +\infty$ -- плоский случай;
	\item $\clOmega = \mathbb{R}_x \times \mathbb{R}_y \times J_z$, $\phi|_{x, y = 0} = 0$, $\phi \to 1$ при $r = \sqrt{x^2 + y^2} \to +\infty$ -- \\ цилиндрический случай;
	\item $\clOmega = \mathbb{R}_x \times \mathbb{R}_y \times \mathbb{R}_z$, $\phi|_{x, y, z = 0} = 0$, $\phi \to 1$ при $r = \sqrt{x^2 + y^2 + z^2} \to +\infty$ -- \\ сферический случай.
\end{enumerate}
Ищем стационарное решение $\phi = \phi(r)$.
\end{frame}


\begin{frame}{Суть проблемы}
\begin{columns}
\column{0.5\textwidth}
	\centering
	Плоский случай
\column{0.5\textwidth}
	\centering
	Цилиндрический случай
\end{columns}
\vspace{0.5cm}
\begin{columns}
\column{0.49\textwidth}
	Задача Коши:
	\vspace{-0.3cm}
	$$\phi(0) = 0; \qquad \partr{\phi} = \cfrac{2}{l} \sqrt{1 - f(\phi)}$$
	\vspace{-0.3cm}
	\begin{figure}
		\includegraphics[width=\textwidth]{figures/result_volumes.png}
	\end{figure}
\column{0.01\textwidth}
	\rule{0.4pt}{0.7\textheight}
\column{0.49\textwidth}
	\centering
	Задача поставлена некорректно \\ и решения не имеет \cite{zipunova_higher_codimension}. \\[0.2cm]
	Вместе с тем канал пробоя -- одномерный объект в трехмерном пространстве!
\end{columns}
\end{frame}


\begin{frame}{Обобщение модели}
\vspace{-0.3cm}
\begin{block}{Обобщение модели, предложенное в работе \cite{zipunova_higher_codimension}}
\begin{itemize}
	\item Уравнение электрического потенциала $\Phi$:
	$$\Div(\epsilon[\phi] \nabla \Phi) = 0$$
	\item Уравнение фазового поля $\phi$:
	$$\cfrac{1}{m} \partt{\phi} = \half \epsilon'(\phi) \gradscalsq{\Phi} + \cfrac{\Gamma}{l^2} f'(\phi) + \half \Gamma \Delta \phi - \alpha \cfrac{\Gamma l^2}{4} \bilapl{\phi} + \beta \Gamma l^{p - 2} \plapl{\phi}{p - 2}$$
\end{itemize}
\end{block}
\begin{itemize}
	\item $\bilapl{\phi} = \Delta(\Delta \phi)$ -- билапласиан
	\item $\plapl{\phi}{p - 2}$ -- $p$-лапласиан
\end{itemize}
Везде далее $p = 4$.
\end{frame}


\begin{frame}{Разностная схема}
На границе $r = 0$ области моделирования у решения $\phi$ ожидается особенность. \\
Идея подхода:
\begin{itemize}
	\item используем метод конечных объемов: к ячейке $\Omega_i$ отнесено среднее $\avphi_i$ функции $\phi$;
	\item в первой и второй ячейках приближаем $\phi$ ЛК базисных функций, одна из которых имеет тот же вид особенности, что решение $\phi$;
	\item как и в классическом МКО, уравнения на $\avphi_i$ являются следствием балансовых соотношений.
\end{itemize}
Преимущества подхода:
\begin{itemize}
	\item точно учитываются граничные условия;
	\item точно учитывается асимптотика решения $\phi$ при $r \to +0$.
\end{itemize}
\end{frame}


\begin{frame}{Разностная схема}
$$\cfrac{1}{m} (\avphi_i^{j + 1} - \avphi_i^j) = \tau \cfrac{\Gamma}{l^2} f'(\avphi_i^j) + \cfrac{\tau}{dV_i} \Gamma \textcolor{red}{(\rho_{i + 1/2}^j S_{i + 1/2} - \rho_{i - 1/2}^j S_{i - 1/2})};$$
$$dV_i = r_{i + 1/2}^{k + 1} - r_{i - 1/2}^{k + 1}; \qquad S_{i \pm 1/2} = (k + 1) r_{i \pm 1/2}^k;$$
$$\rho_{i \pm 1/2}^j = \half \left[ \cfrac{\partial \phi}{\partial r} \right]_{i \pm 1/2}^j - \alpha \cfrac{l^2}{4} \left[ \cfrac{\partial (\Delta \phi)}{\partial r} \right]_{i \pm 1/2}^j + \beta l^2 \left( \left[ \cfrac{\partial \phi}{\partial r} \right]_{i \pm 1/2}^j \right)^3;$$
$$\widetilde{\Delta \phi}_i^j = \cfrac{1}{dV_i} \left( \left[ \cfrac{\partial \phi}{\partial r} \right]_{i + 1/2}^j S_{i + 1/2} - \left[ \cfrac{\partial \phi}{\partial r} \right]_{i - 1/2}^j S_{i - 1/2} \right)$$
Подробнее в работе \cite{ponomarev_finite_volumes}.
\end{frame}


\begin{frame}{Полученные результаты}
\vspace{-0.3cm}
\begin{center}
	Предполагаемые виды особенности решения $\phi$ в точке $r = 0$
\end{center}
\begin{tabular}{|m{3cm}||m{3.5cm}|m{3.5cm}|m{3.5cm}|}
	\hline
	\vspace*{2mm} \hfill \vspace*{2mm} &\centering $\alpha = 0$, $\beta = 0$ &
	\centering $\alpha = 0$, $\beta \neq 0$ & \centering \arraybackslash $\alpha \neq 0$ \\
	\hline
	\hline
	\vspace{2mm} Плоский \linebreak случай \vspace{2mm} &
	\textcolor{green}{Без особенности} & \textcolor{green}{Без особенности} & \textcolor{green}{Без особенности} \\
	\hline
	\vspace{2mm} Цилиндрический \linebreak случай \vspace{2mm} &
	\textcolor{red}{Не имеет решения} & $r^{2/3}$ & $r^2 \ln r$ \\
	\hline
	\vspace{2mm} Сферический \linebreak случай \vspace{2mm} &
	\textcolor{red}{Предположительно не имеет решения} & $r^{1/3}$ & \textcolor{red}{Предположительно не имеет решения} \\
	\hline
\end{tabular}
\begin{itemize}
	\item Базисная функция $g$ подбирается так, чтобы
	\vspace{-0.2cm}
	$$\rho[\phi = g] \cdot S \to C > 0 \text{ при } r \to +0.$$
\end{itemize}
\end{frame}


\begin{frame}{Полученные результаты}
\vspace{-0.5cm}
\begin{figure}
	\includegraphics[width=0.81\textwidth]{figures/result_volumes_cyl_bi.png}
\end{figure}
\vspace{-0.7cm}
\begin{center}
	Цилиндрический случай, $\alpha = 1$: особенность вида $r^2 \ln r$
\end{center}
\end{frame}


\begin{frame}{Полученные результаты}
\vspace{-0.5cm}
\begin{figure}
	\includegraphics[width=0.81\textwidth]{figures/result_volumes_sph_p.png}
\end{figure}
\vspace{-0.7cm}
\begin{center}
	Сферический случай, $\alpha = 0$, $\beta = 1$: особенность вида $r^{1/3}$
\end{center}
\end{frame}

%!TEX root = ../main.tex

\section{Conclusions}

In this paper we study stability properties of the phase-field model
for electrical breakdown channel evolution.
The central result is a classification of the
equilibrium solutions of the model and their stability.
From practical point of view, these results allows to
make meaningful conclusions regarding qualitative and quantitative
properties of the model. Particularly it was shown under which
conditions small perturbations of the equilibrium solutions
develop into channel-like structure typical for of electrical breakdown
process.

Besides this, a simple explicit finite-difference scheme
for solution of the model in spatially one-dimensional setting is considered.
The main question addressed here are stability conditions which guaranties
correctness of the simulations. Deep connections between
stability conditions of the model and the one of the
finite-difference scheme are shown.
The presented results of the numerical simulations confirms
predictions of the theoretical analysis of the model.

% EOF
\endinput


\begin{frame}{Литература}
\printbibliography
\end{frame}

\begin{frame}{}
\begin{center}
	\Large
	Спасибо за внимание!
\end{center}
\end{frame}

\end{document}