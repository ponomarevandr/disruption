%!TEX root = ../main.tex

\section{Постановка задачи}

\begin{frame}{Одномерная задача}
\vspace{-0.3cm}
\begin{itemize}
	\item Область $\clOmega = [0, w]_x \times [0, h]_y \times I_z$ в форме параллелепипеда
	\item $\phi(\vx, 0) = \phi_0(\vx) = \phi_0(x), \; \epsilon_0(\vx) =
	\epsilon_0(x)$ не зависят от $y$ и $z$
	\item $\Phi|_{y = 0} = \Phi^- \in \mathbb{R}, \; \Phi|_{y = h} = \Phi^+ \in \mathbb{R}$
\end{itemize}
\vspace{0.3cm}
Решением является функция электрического потенциала
\begin{columns}
\column{0.65\textwidth}
\vspace{-1cm}
$$\Phi(\vx, t) = \Phi^- \hm + \cfrac{y}{h}(\Phi^+ - \Phi^-)$$
\column{0.35\textwidth}
\begin{figure}
	\vspace*{-2cm}
	\hspace*{0.5cm}
	\includegraphics[width=0.85\textwidth]{figures/one_dim_problem.jpg}
\end{figure}
\end{columns}
\vspace{-0.4cm}
Тогда уравнение на $\phi$ принимает вид
\begin{block}{}
	$$\cfrac{1}{m} \cfrac{\partial \phi}{\partial t} = \cfrac{1}{2} K_\Phi^2 \epsilon'(\phi) +
	\cfrac{\Gamma}{l^2} f'(\phi) + \cfrac{1}{2} \Gamma \cfrac{\partial^2 \phi}{\partial x^2}$$
\end{block}
$K_\Phi = \cfrac{\Phi^+ - \Phi^-}{h} = \norm{\nabla \Phi}$. Будем считать $\epsilon_0 = \Const$.
\end{frame}


\section{Теоретический анализ}

\begin{frame}{Анализ положений равновесия}
\begin{itemize}
	\item Пробой может развиваться из малых возмущений свойств неповрежденной среды. Выясним
	условия развития.
	\item Рассмотрим положения равновесия вида $\phi(x, t) \equiv C$. Положению равновесия
	соответствует ноль $C$ функции
\end{itemize}
$$\chi(\phi) = \cfrac{1}{2} K_\Phi^2 \epsilon'(\phi) + \cfrac{\Gamma}{l^2} f'(\phi)$$
\vspace{-0.5cm}
\begin{itemize}
	\item Исследуем положения равновесия на устойчивость спектральным методом: к $\phi \equiv C$
	прибавим возмущение $\delta \phi = e^{\alpha t}\cos(\omega x)$, линеаризуем уравнение на
	$\delta \phi$.
	\item $\chi(\phi)$ возрастает в $C \Longrightarrow$ равновесие неустойчиво; $\chi(\phi)$
	убывает в $C \Longrightarrow$ равновесие устойчиво.
\end{itemize}
\end{frame}


\begin{frame}{Анализ положений равновесия}
\vspace{-0.9cm}
\begin{columns}
\column{0.3\textwidth}
\begin{center}
	<<Слабое>> напряжение
\end{center}
\vspace{-0.2cm}
$$0 \leqslant \cfrac{K_\Phi^2 l^2 \epsilon_0}{2 \Gamma} < \delta^2$$
\vspace{-0.7cm}
\begin{figure}
	\includegraphics[width=\textwidth]{figures/equilibriums_case_1.png}
\end{figure}
\column{0.3\textwidth}
\begin{center}
	<<Среднее>> напряжение
\end{center}
\vspace{-0.2cm}
$$\delta^2 < \cfrac{K_\Phi^2 l^2 \epsilon_0}{2 \Gamma} < (1 + \delta)^2$$
\vspace{-0.7cm}
\begin{figure}
	\includegraphics[width=\textwidth]{figures/equilibriums_case_2.png}
\end{figure}
\column{0.3\textwidth}
\begin{center}
	<<Сильное>> напряжение
\end{center}
\vspace{-0.2cm}
$$(1 + \delta)^2 < \cfrac{K_\Phi^2 l^2 \epsilon_0}{2 \Gamma}$$
\vspace{-0.7cm}
\begin{figure}
	\includegraphics[width=\textwidth]{figures/equilibriums_case_3.png}
\end{figure}
\end{columns}
\begin{columns}
\column{0.3\textwidth}
\hspace{0.5cm}
$\phi \equiv 0$ неустойчивое \\
\hspace{0.5cm}
$\phi \equiv 1$ устойчивое
\column{0.3\textwidth}
\hspace{0.5cm}
$\phi \equiv 0$ устойчивое \\
\hspace{0.5cm}
$\phi \equiv С_3$ неустойчивое \\
\hspace{0.5cm}
$\phi \equiv 1$ устойчивое
\column{0.3\textwidth}
\hspace{0.5cm}
$\phi \equiv 0$ устойчивое \\
\hspace{0.5cm}
$\phi \equiv 1$ неустойчивое
\end{columns}
\end{frame}


\section{Численный анализ}

\begin{frame}{Разностная схема}
\begin{block}{Разностная задача}
	$$\cfrac{1}{m} \cfrac{\phi_a^{b + 1} - \phi_a^b}{\tau} = \cfrac{1}{2} K_\phi^2
	\epsilon'(\phi_a^b) + \cfrac{\Gamma}{l^2} f'(\phi_a^b) + \cfrac{\Gamma}{2}
	\cfrac{\phi_{a + 1}^b - 2 \phi_a^b + \phi_{a - 1}^b}{h^2}$$
	$$\phi_a^0 = \phi_0(ah); \quad \phi_0^b = \phi_l(b \tau); \quad \phi_n^b = \phi_r(b \tau)$$
	Сетка регулярная; $\tau$ -- шаг по времени, $h$ -- шаг по пространству.
\end{block}
Явная разностная схема первого порядка по времени, второго -- по пространству.
\end{frame}


\begin{frame}{Оценка устойчивости}
\begin{itemize}
\item Рассмотрим возмущенное решение $\phi_a^b + \delta_a^b$. Линеаризуем уравнение на возмущение
$\delta_a^b$ в точке $\phi_a^b = P$:
\end{itemize}
$$\delta_a^{b + 1} = \delta_a^b + m \tau \left( \cfrac{1}{2} K_\Phi^2 \epsilon''(P) \delta_a^b +
\cfrac{\Gamma}{l^2} f''(P) \delta_a^b + \cfrac{\Gamma}{2} \cfrac{\delta_{a + 1}^b - 2 \delta_a^b +
\delta_{a - 1}^b}{h^2} \right)$$
\begin{itemize}
\item Применим спектральный признак устойчивости:
\end{itemize}
$$1 > \lambda(\theta) = 1 + m \tau \left( \cfrac{1}{2} K_\Phi^2 \epsilon''(P) +
\cfrac{\Gamma}{l^2} f''(P) - \cfrac{2 \Gamma}{h^2} \sin^2 \cfrac{\theta}{2} \right)$$
\begin{itemize}
\item Исследуем вблизи $P = 0$.
\end{itemize}
\end{frame}


\begin{frame}{Оценка устойчивости}
\begin{block}{Условие устойчивости}
	$$\tau \leqslant \cfrac{1}{2m} \left( \cfrac{K_\Phi^2 \epsilon_0}{\delta^{5/3}} +
	\cfrac{\Gamma}{h^2} \right)^{-1}$$
\end{block}
\begin{block}{Упрощенное условие устойчивости}
	$$\tau \leqslant \cfrac{1}{4m} \min \left(\cfrac{\delta^{5/3}}{K_\Phi^2 \epsilon_0}, \;
	\cfrac{h^2}{\Gamma} \right)$$
\end{block}
\end{frame}


\begin{frame}{Вычисления: типичное решение}
\vspace{-0.4cm}
\begin{columns}
\column{0.88\textwidth}
\begin{figure}
	\includegraphics[width=\textwidth]{figures/typical_solution.png}
\end{figure}
\column{0.12\textwidth}
\hfill \\
\vspace{3.5cm}
\hspace{-2.5cm}
Узлов по измерениям: \\
\hspace{-2.5cm}
$n_x = 10^3, \; n_t = 10^5$
\end{columns}
\end{frame}


\begin{frame}{Вычисления: проверка устойчивости}
\vspace{-0.4cm}
\begin{columns}
\column{0.7\textwidth}
\begin{figure}
	\includegraphics[width=\textwidth]{figures/stability_bounds.png}
\end{figure}
\column{0.3\textwidth}
Условие устойчивости:
$$\tau \leqslant \cfrac{1}{2m} \left( \cfrac{K_\Phi^2 \epsilon_0}{\delta^{5/3}} +
\cfrac{\Gamma}{h^2} \right)^{-1}$$
\end{columns}
\end{frame}


\begin{frame}{Вычисления: проверка сходимости}
\vspace{-0.3cm}
\begin{figure}
	\includegraphics[width=0.58\textwidth]{figures/convergence_connected.png}
\end{figure}
\vspace{-0.6cm}
\begin{center}
	Здесь, согласно оценке устойчивости, $\tau = \cfrac{h^2}{4m \Gamma}$
\end{center}
\end{frame}


\begin{frame}{Вычисления: положения равновесия}
\vspace{-0.4cm}
\begin{center}
	$(1 + \delta)^2 < \cfrac{K_\Phi^2 l^2 \epsilon_0}{2 \Gamma}$ -- <<сильное>> напряжение
\end{center}
\vspace{-0.4cm}
\begin{columns}
\column{0.5\textwidth}
\begin{figure}
	\includegraphics[width=\textwidth]{figures/equilibrium_3_0.png}
\end{figure}
\vspace{-0.6cm}
\begin{center}
	$\phi \equiv 0$ \\
	устойчивое
\end{center}
\column{0.5\textwidth}
\begin{figure}
	\includegraphics[width=\textwidth]{figures/equilibrium_3_1.png}
\end{figure}
\vspace{-0.6cm}
\begin{center}
	$\phi \equiv 1$ \\
	неустойчивое
\end{center}
\end{columns}
\end{frame}


\begin{frame}{Свободная энергия}
\vspace{-0.7cm}
$$\Pi(t) = \int\limits_\Omega \pi(x, t) dx$$
$$\pi(x, t) = \pi_1(x, t) + \pi_2(x, t) + \pi_3(x, t)$$
\vspace{-0.3cm}
\begin{itemize}
	\item $\pi_1(x, t) = -\cfrac{K_\Phi^2}{2} \, \epsilon(\phi)$ -- плотность энергии
	электрического поля;
	\item $\pi_2(x, t) = \Gamma \cfrac{1 - f(\phi)}{l^2}$ -- плотность энергии, отнесенной
	к веществу внутри канала;
	\item $\pi_3(x, t) = \cfrac{\Gamma}{4} \left( \cfrac{\partial \phi}{\partial x} \right)^2$ --
	плотность энергии, отнесенной к граничной зоне канала
\end{itemize}
\end{frame}


\begin{frame}{Вычисления: свободная энергия}
\vspace{-0.8cm}
\begin{columns}
\column{0.5\textwidth}
\begin{figure}
	\includegraphics[width=\textwidth]{figures/density_electrical.png}
\end{figure}
\column{0.5\textwidth}
\begin{figure}
	\includegraphics[width=\textwidth]{figures/density_depth.png}
\end{figure}
\end{columns}
\end{frame}


\begin{frame}{Вычисления: свободная энергия}
\vspace{-0.6cm}
\begin{figure}
	\includegraphics[width=0.86\textwidth]{figures/density_surface.png}
\end{figure}
\end{frame}


\begin{frame}{Вычисления: свободная энергия}
\vspace{-0.6cm}
\begin{figure}
	\includegraphics[width=0.88\textwidth]{figures/energy_total.png}
\end{figure}
\end{frame}