\documentclass[aspectratio=169]{beamer}

%%% Работа с русским языком
\usepackage{cmap}					% поиск в PDF
\usepackage{mathtext} 				% русские буквы в формулах
\usepackage[T2A]{fontenc}			% кодировка
\usepackage[utf8]{inputenc}			% кодировка исходного текста
\usepackage[english,russian]{babel}	% локализация и переносы
\usepackage{indentfirst}
\frenchspacing

%%% Дополнительная работа с математикой
\usepackage{amsmath,amsfonts,amssymb,amsthm,mathtools}  % AMS

%%% Текст в колонки
\usepackage{multicol}

%%% Системы уравнений
\usepackage{cases}

%%% Таблицы
\usepackage{array}

%%% Картинки
\usepackage{graphicx}
\usepackage{float}

%%% Список литературы
\usepackage[sorting=none]{biblatex}
\renewbibmacro{in:}{\ifentrytype{article}
    {}
    {\bibstring{in}\printunit{\intitlepunct}}
}
\addbibresource{references.bib}

%%% Гиперссылки
\usepackage{hyperref}

%%% Перенос знаков в формулах (по Львовскому)
\newcommand*{\hm}[1]{#1\nobreak\discretionary{}
{\hbox{$\mathsurround=0pt #1$}}{}}


%%% Свои команды

\newcommand{\vect}[1]{\textbf{#1}}
\newcommand{\vx}{{\vect{x}}}
\newcommand{\vn}{{\vect{n}}}

\newcommand{\half}{\cfrac{1}{2}}

\newcommand{\partt}[1]{\cfrac{\partial #1}{\partial t}}
\newcommand{\partx}[1]{\cfrac{\partial #1}{\partial x}}
\newcommand{\partxx}[1]{\cfrac{\partial^2 #1}{\partial x^2}}
\newcommand{\partvn}[1]{\cfrac{\partial #1}{\partial \vn}}

\newcommand{\partflt}[1]{\partial #1 / \partial t}
\newcommand{\partflx}[1]{\partial #1 / \partial x}
\newcommand{\partflxx}[1]{\partial^2 #1 / \partial x^2}
\newcommand{\partflvn}[1]{\partial #1 / \partial \vn}

\newcommand{\gradsq}[1]{(\nabla #1, \nabla #1)}

\newcommand{\difftau}[1]{\cfrac{{#1}_j^{k + 1} - {#1}_j^k}{\tau}}
\newcommand{\diffhh}[1]{\cfrac{{#1}_{j + 1}^k - 2 {#1}_j^k + {#1}_{j - 1}^k}{h^2}}

\newcommand{\Natural}{{\mathbb{N}}}
\newcommand{\Real}{{\mathbb{R}}}
\newcommand{\bigO}{{\mathcal{O}}}
\newcommand{\clOmega}{{\overline{\Omega}}}

\newcommand{\norm}[1]{\| \, #1 \, \|}
\newcommand{\enorm}{{\| \cdot \|}}

\newcommand{\unitm}{{\text{м}}}
\newcommand{\units}{{\text{с}}}
\newcommand{\unitJ}{{\text{Дж}}}
\newcommand{\unitC}{{\text{Кл}}}
\newcommand{\unitV}{{\text{В}}}
\newcommand{\unitF}{{\text{Ф}}}

\newcommand{\nologo}{\setbeamertemplate{logo}{}}


%%% Свои операторы
\DeclareMathOperator{\Div}{{div}}
\DeclareMathOperator{\Const}{{const}}
\DeclareMathOperator{\Tol}{{tol}}

%%% Цвета
\definecolor{green}{RGB}{0,200,0}
\definecolor{red}{RGB}{200,0,0}

%%% Тема оформления
\usetheme{Madrid}


%%% Титульный лист
\title[Вопросы моделирования канала пробоя]{Теоретические и численные вопросы \\ моделирования развития канала электрического пробоя \\ методом диффузной границы}
\author[]{
	\raggedright
	\vspace*{-0.4cm}
	\hfill \break
	\hspace*{8cm}
	\textbf{Студент:} \linebreak
	\hspace*{8cm}
	\vspace{0.2cm}
	Пономарев Андрей Сергеевич \linebreak
	\hspace*{8cm}
	\textbf{Научный руководитель:} \linebreak
	\hspace*{8cm}
	\vspace{0.2cm}
	Савенков Евгений Борисович
	\hspace*{8cm}
	\textbf{Консультант:} \linebreak
	\hspace*{8cm}
	Зипунова Елизавета Вячеславовна
}
\date[]{
	\vspace*{-0.5cm}
	\break
	14.11.2025
}
\logo{\includegraphics[height=0.8cm]{../figures/labels.jpg}}


\begin{document}

\AtBeginSection[]{
	\begin{frame}{Содержание}
	\Large
	\tableofcontents[currentsection]
	\end{frame}
}

\begin{frame}
\titlepage
\end{frame}


%!TEX root = ../main.tex

\section{Постановка задачи}

\begin{frame}{Математическая модель}
\vspace{-0.3cm}
\begin{block}{Уравнения динамики системы}
	\begin{itemize}
		\item Уравнение электрического потенциала $\Phi$:
		\[
			\Div(\epsilon[\phi] \nabla \Phi) = 0
		\]
		\item Уравнение фазового поля $\phi$ (типа Аллена--Кана):
		\[
			\cfrac{1}{m} \partt{\phi} = \half \epsilon'(\phi) \gradsq{\Phi} + \cfrac{\Gamma}{l^2} f'(\phi) + \half \Gamma \Delta \phi
		\]
	\end{itemize}
\end{block}
\begin{itemize}
	\item Плотность свободной энергии
	\vspace{-0.2cm}
	\[
		\pi = -\half \epsilon[\phi] \gradsq{\Phi} + \Gamma \cfrac{1 - f(\phi)}{l^2} + \cfrac{\Gamma}{4} \gradsq{\phi}
	\]
\end{itemize}
\vspace{-1.9cm}
{\raggedleft Подробнее: \cite{ponomarev_stability}, \cite{zipunova_higher_codimension} \par}
\vspace{0.9cm}
\begin{columns}
\column{0.33\textwidth}
	\vspace{0.35cm}
	\[
		f(\phi) = 4 \phi^3 - 3 \phi^4
	\]
\column{0.33\textwidth}
	\[
		\epsilon(\vx, t) = \cfrac{\epsilon_0(\vx)}{f(\phi(\vx, t)) + \delta}
	\]
\column{0.33\textwidth}
\end{columns}
\end{frame}


\begin{frame}{Математическая модель}
\vspace{-0.3cm}
\begin{block}{Уравнения динамики системы}
	\begin{itemize}
		\item Уравнение электрического потенциала $\Phi$:
		\[
			\Div(\epsilon[\phi] \nabla \Phi) = 0
		\]
		\item Уравнение фазового поля $\phi$ (типа Аллена--Кана):
		\[
			\cfrac{1}{m} \partt{\phi} = -F'(\phi; |\nabla \Phi|) + \half \Gamma \Delta \phi
		\]
	\end{itemize}
\end{block}
\begin{itemize}
	\item Плотность свободной энергии
	\vspace{-0.2cm}
	\[
		\pi = F(\phi; |\nabla \Phi|) + \cfrac{\Gamma}{4} \gradsq{\phi}
	\]
	\item $m$, $\Gamma$ -- параметры модели, константы
\end{itemize}
\end{frame}


\begin{frame}{Разностная схема}
\vspace{-0.9cm}
\[
	\cfrac{1}{m} \partt{\phi} = -F'(\phi; |\nabla \Phi|) + \half \Gamma \partxx{\phi}
\]
\vspace{-0.4cm}
\begin{itemize}
	\item $|\nabla \Phi|$ -- параметр
\end{itemize}
\begin{block}{Разностная задача}
	\[
		\cfrac{1}{m} \cfrac{\phi_i^{j + 1} - \phi_i^j}{\tau} = \half K_\phi^2 \epsilon'(\phi_i^j) + \cfrac{\Gamma}{l^2} f'(\phi_i^j) + \cfrac{\Gamma}{2} \cfrac{\phi_{i + 1}^j - 2 \phi_i^j + \phi_{i - 1}^j}{h^2}
	\]
	\[\phi_i^0 = \phi_0(ih); \quad \phi_0^j = \phi_l(j \tau); \quad \phi_n^j = \phi_r(j \tau)\]
	Сетка регулярная; $\tau$ -- шаг по времени, $h$ -- шаг по пространству.
\end{block}
Явная разностная схема первого порядка по времени, второго -- по пространству.
\end{frame}


\begin{frame}{Типичное решение}
\vspace{-0.4cm}
\begin{columns}
\column{0.88\textwidth}
\begin{figure}
	\includegraphics[width=\textwidth]{figures/typical_solution.png}
\end{figure}
\column{0.12\textwidth}
\hfill \\
\vspace{3.5cm}
\hspace{-2.5cm}
Из работы \cite{ponomarev_stability}. \\
\hspace{-2.5cm}
Узлов по измерениям: \\
\hspace{-2.5cm}
$N_x = 10^3$, $N_t = 10^5$
\end{columns}
\end{frame}


\begin{frame}{Цель работы}
\begin{itemize}
	\item Типичное поведение модели: долгий период медленных изменений, \\ затем стремительное развитие пробоя.
\end{itemize}
\begin{block}{Цель работы}
	Исследовать несколько подходов к адаптации расчетного шага по времени.
\end{block}
\end{frame}
{}
\input{sources/time_step_adaptation}

%!TEX root = ../main.tex

\section{Параллели с механикой}

\begin{frame}{Связь с моделированием трещин}
\begin{itemize}
	\item Анализ статьи \cite{sargado_high_accuracy}
\end{itemize}
\vspace{0.5cm}
\centering
\begin{tabular}{|>{\centering\arraybackslash}m{7cm}|>{\centering\arraybackslash}m{7cm}|}
	\hline
	& \\[-3mm]
	смещение $\vect{u}$ & электрический потенциал $\Phi$ \\[1mm]
	\hline
	& \\[-3mm]
	деформация $\hat{\epsilon}$			& напряженность электрического поля $\vect{E}$ \\[1mm]
	\hline
	& \\[-3mm]
	упругость (жёсткость) $\hat{C}$		& диэлектрическая проницаемость $\epsilon$ \\[1mm]
	\hline
\end{tabular}
\end{frame}


\begin{frame}{Литература}
\printbibliography
\end{frame}

\begin{frame}{}
\begin{center}
	\Large
	Спасибо за внимание!
\end{center}
\end{frame}

\end{document}