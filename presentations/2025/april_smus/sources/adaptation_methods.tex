%!TEX root = ../main.tex

\section{Методы адаптации}

\begin{frame}{Общий вид схемы с адаптивным шагом}
\begin{itemize}
	\item Вводится переменный шаг $\tau^k$:
	\[
		\phi_j^{k + 1} = \phi_j^k + m \tau^k \left( -F'(\phi_j^k) + \cfrac{\Gamma}{2} \diffhh{\phi} \right)
	\]
	\item Значение $\tau^k$ ограничено заранее выбранными $\tau_{min}$ и $\tau_{max}$:
	\[
		\tau^k = \max \left[ \tau_{min}, \min( \tau_{max}, \widetilde{\tau}^k) \right]
	\]
\end{itemize}
\end{frame}


\begin{frame}{Методы адаптации}
\vspace{-0.7cm}
\begin{block}{Методы адаптации}
	\begin{itemize}
		\item По фазовому полю:
		\[
			\widetilde{\tau}_1^k = \cfrac{tol_1}{\left\|  \cfrac{\partial_h \phi}{\partial t} \right\|_{C, h}}
		\]
		\item По полной энергии:
		\[
			\widetilde{\tau}_2^k = \cfrac{tol_2}{\left| \cfrac{d_h \Pi}{dt} \right|}
		\]
	\end{itemize}
\end{block}
Предложены в статьях \cite{li_time_step} и \cite{zhang_time_step}.
\end{frame}

{\nologo
\begin{frame}{Методы адаптации}
\vspace{-0.6cm}
\begin{itemize}
	\item Условие устойчивости схемы \cite{ponomarev_stability}:
	\vspace{-0.3cm}

	\[
		\tau \leqslant \cfrac{1}{4m} \min \left( \cfrac{\delta^{5/3}}{|\nabla \Phi|^2 \epsilon_0}, \cfrac{h^2}{\Gamma} \right)
	\]
	\vspace{-0.3cm}
	\item Неравенство с первым аргументом $\min$ можно переписать в виде
	\[
		m \tau \max\limits_{\phi \in [0, 1]} |F''(\phi)| \leqslant 1
	\]
\end{itemize}
\vspace{-0.3cm}
\begin{block}{Адаптация по устойчивости}
	\[
		\widetilde{\tau}_3^k = \cfrac{tol_3}{m \cdot \max\limits_{j = 0}^N G(\phi_j^k)},
	\]
	где $G(\phi)$ мажорирует $|F''(\phi)|$
\end{block}
\end{frame}
}