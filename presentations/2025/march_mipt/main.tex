\documentclass[aspectratio=169]{beamer}

%%% Работа с русским языком
\usepackage{cmap}					% поиск в PDF
\usepackage{mathtext} 				% русские буквы в формулах
\usepackage[T2A]{fontenc}			% кодировка
\usepackage[utf8]{inputenc}			% кодировка исходного текста
\usepackage[english,russian]{babel}	% локализация и переносы
\usepackage{indentfirst}
\frenchspacing

%%% Дополнительная работа с математикой
\usepackage{amsmath,amsfonts,amssymb,amsthm,mathtools}  % AMS

%%% Текст в колонки
\usepackage{multicol}

%%% Системы уравнений
\usepackage{cases}

%%% Таблицы
\usepackage{array}

%%% Картинки
\usepackage{graphicx}
\usepackage{float}

%%% Список литературы
\usepackage[sorting=none]{biblatex}
\renewbibmacro{in:}{\ifentrytype{article}
    {}
    {\bibstring{in}\printunit{\intitlepunct}}
}
\addbibresource{references.bib}

%%% Гиперссылки
\usepackage{hyperref}

%%% Перенос знаков в формулах (по Львовскому)
\newcommand*{\hm}[1]{#1\nobreak\discretionary{}
{\hbox{$\mathsurround=0pt #1$}}{}}


%%% Свои команды

\newcommand{\vect}[1]{\boldsymbol{#1}}
\newcommand{\vx}{{\vect{x}}}
\newcommand{\vn}{{\vect{n}}}

\newcommand{\half}{\cfrac{1}{2}}

\newcommand{\partt}[1]{\cfrac{\partial #1}{\partial t}}
\newcommand{\partx}[1]{\cfrac{\partial #1}{\partial x}}
\newcommand{\partxx}[1]{\cfrac{\partial^2 #1}{\partial x^2}}
\newcommand{\partvn}[1]{\cfrac{\partial #1}{\partial \vn}}

\newcommand{\partflt}[1]{\partial #1 / \partial t}
\newcommand{\partflx}[1]{\partial #1 / \partial x}
\newcommand{\partflxx}[1]{\partial^2 #1 / \partial x^2}
\newcommand{\partflvn}[1]{\partial #1 / \partial \vn}

\newcommand{\gradsq}[1]{(\nabla #1, \nabla #1)}

\newcommand{\difftau}[1]{\cfrac{{#1}_j^{k + 1} - {#1}_j^k}{\tau}}
\newcommand{\diffhh}[1]{\cfrac{{#1}_{j + 1}^k - 2 {#1}_j^k + {#1}_{j - 1}^k}{h^2}}

\newcommand{\Natural}{{\mathbb{N}}}
\newcommand{\Real}{{\mathbb{R}}}
\newcommand{\bigO}{{\mathcal{O}}}
\newcommand{\clOmega}{{\overline{\Omega}}}

\newcommand{\norm}[1]{\| \, #1 \, \|}
\newcommand{\enorm}{{\| \cdot \|}}

\newcommand{\unitm}{{\text{м}}}
\newcommand{\units}{{\text{с}}}
\newcommand{\unitJ}{{\text{Дж}}}
\newcommand{\unitC}{{\text{Кл}}}
\newcommand{\unitV}{{\text{В}}}
\newcommand{\unitF}{{\text{Ф}}}

\newcommand{\nologo}{\setbeamertemplate{logo}{}}


%%% Свои операторы
\DeclareMathOperator{\Div}{{div}}
\DeclareMathOperator{\Const}{{const}}

%%% Цвета
\definecolor{green}{RGB}{0,200,0}
\definecolor{red}{RGB}{200,0,0}

%%% Тема оформления
\usetheme{Madrid}


%%% Титульный лист
\title[Адаптация шага по времени]{Адаптация шага по времени в модели типа диффузной границы, \\ содержащей уравнение Аллена–Кана}
\author[]{
	Пономарев А. С.\textsuperscript{1,2}, Савенков Е. Б.\textsuperscript{2}, Зипунова Е. В.\textsuperscript{2}
}
\institute[МФТИ, ИПМ]{
	\textsuperscript{1}МФТИ (НИУ) \\
	\textsuperscript{2}ИПМ им. М. В. Келдыша РАН
}
\date[67-я Конференция МФТИ]{
	67-я Всероссийская научная конференция МФТИ \\[1mm]
	31.03.2025
}
\logo{\includegraphics[height=0.8cm]{../figures/labels.jpg}}


\begin{document}

\AtBeginSection[]{
	\begin{frame}{Содержание}
	\Large
	\tableofcontents[currentsection]
	\end{frame}
}

\begin{frame}
\titlepage
\end{frame}


%!TEX root = ../main.tex

\section{Введение}

\begin{frame}{Физическое явление}
\begin{block}{Электрический пробой}
	Явление резкого возрастания тока в диэлектрике при приложении электрического напряжения
	выше критического.
\end{block}
\begin{itemize}
	\item Рассматриваем твердый диэлектрик
	\item Деградация диэлектрических свойств материала
	\item Процесс развивается в ограниченной зоне -- канале пробоя
	\item Сложная физическая природа
\end{itemize}
\end{frame}


\begin{frame}{Математическая модель}
\begin{block}{Модель типа диффузной границы}
	Вещество находится в разных фазах. Состояние вещества описывается гладкой функцией
	$\phi(\vx, t)$ -- фазовым полем.
\end{block}
\begin{itemize}
	\item $\phi = 1$ -- неповрежденная среда
	\item $\phi = 0$ -- полностью разрушенная среда
	\item Зона $\phi \in (0, 1)$ -- диффузная граница
	\item На разрушение среды тратится энергия
\end{itemize}
\begin{figure}
	\includegraphics[width=0.5\textwidth]{figures/diffuse_edge.jpg}
\end{figure}
\end{frame}


\begin{frame}{Математическая модель}
Модель, предложенная в работе \cite{pitike_dielectric_breakdown}:
\begin{itemize}
	\item $\pi = \textcolor{red}{-\half \epsilon[\phi] (\nabla \Phi, \nabla \Phi)} +
	\Gamma \left( \cfrac{1 - f(\phi)}{l^2} + \cfrac{1}{4} (\nabla \phi, \nabla \phi) \right)$
	-- плотность свободной энергии
	\item $\Gamma$ -- энегрия роста канала пробоя на единицу длины
	\item $l$ -- величина <<размытия>> канала
	\item $\epsilon(\vx, t)$ -- диэлектрическая проницаемость среды
	\item $f(\phi)$ -- интерполирующая функция
\end{itemize}
\end{frame}


\begin{frame}{Математическая модель}
\vspace{-0.2cm}
\begin{itemize}
	\item $\epsilon(\vx, t) = \cfrac{\epsilon_0(\vx)}{f(\phi(\vx, t)) +
	\delta}$ -- диэлектрическая проницаемость среды
	\item $f(\phi) = 4 \phi^3 - 3 \phi^4$ -- интерполирующая функция
\end{itemize}
\begin{columns}
\column{0.5\textwidth}
\begin{figure}
	\hspace*{1.4cm}
	\includegraphics[width=0.65\textwidth]{figures/f_form.png}
\end{figure}
\column{0.5\textwidth}
\begin{figure}
	\hspace*{-2cm}
	\includegraphics[width=0.60\textwidth]{figures/eps_form.png}
\end{figure}
\end{columns}
\end{frame}


\begin{frame}{Математическая модель}
\vspace{-0.5cm}
\begin{block}{Уравнения модели}
\begin{itemize}
	\item Уравнение электрического потенциала $\Phi$:
	\begin{equation}
		\Div(\epsilon[\phi] \nabla \Phi) = 0
		\label{equation_potential}
	\end{equation}
	\item Уравнение фазового поля $\phi$:
	\begin{equation}
		\cfrac{1}{m} \partt{\phi} = \half \epsilon'(\phi) \gradscalsq{\Phi} + \cfrac{\Gamma}{l^2} f'(\phi) + \half \Gamma \Delta \phi
		\label{equation_phase}
	\end{equation}
\end{itemize}
\end{block}
Свойства:
\begin{itemize}
	\item связанная система уравнений на $\phi$ и $\Phi$;
	\item уравнение для $\phi$ типа Аллена--Кана, нелинейное.
\end{itemize}
\end{frame}


\begin{frame}{Пример вычислительного эксперимента}
\begin{columns}
\column{0.32\textwidth}
\begin{figure}
	\includegraphics[width=\textwidth]{figures/model_example_1.png}
\end{figure}
\column{0.32\textwidth}
\begin{figure}
	\includegraphics[width=\textwidth]{figures/model_example_2.png}
\end{figure}
\column{0.32\textwidth}
\begin{figure}
	\includegraphics[width=\textwidth]{figures/model_example_3.png}
\end{figure}
\end{columns}
\begin{center}
	Расчет из работы \cite{zipunova_experiment}
\end{center}
\end{frame}


\begin{frame}{Цель работы}
\begin{block}{Цель работы}
	Исследовать качественные характеристики системы уравнений \eqref{equation_potential},
	\eqref{equation_phase} и выполнить ее численный анализ.
\end{block}
Для этого рассмотрим задачу в определенных краевых условиях, упрощающих ее, но позволяющих
установить интересующие свойства.
\end{frame}

%!TEX root = ../main.tex

\section{Постановка задачи}

\begin{frame}{Математическая модель}
\vspace{-0.3cm}
\begin{block}{Уравнения динамики системы}
	\begin{itemize}
		\item Уравнение электрического потенциала $\Phi$:
		\[
			\Div(\epsilon[\phi] \nabla \Phi) = 0
		\]
		\item Уравнение фазового поля $\phi$ (типа Аллена--Кана):
		\[
			\cfrac{1}{m} \partt{\phi} = \half \epsilon'(\phi) \gradsq{\Phi} + \cfrac{\Gamma}{l^2} f'(\phi) + \half \Gamma \Delta \phi
		\]
	\end{itemize}
\end{block}
\begin{itemize}
	\item Плотность свободной энергии
	\vspace{-0.2cm}
	\[
		\pi = -\half \epsilon[\phi] \gradsq{\Phi} + \Gamma \cfrac{1 - f(\phi)}{l^2} + \cfrac{\Gamma}{4} \gradsq{\phi}
	\]
\end{itemize}
\vspace{-1.9cm}
{\raggedleft Подробнее: \cite{ponomarev_stability}, \cite{zipunova_higher_codimension} \par}
\vspace{0.9cm}
\begin{columns}
\column{0.33\textwidth}
	\vspace{0.35cm}
	\[
		f(\phi) = 4 \phi^3 - 3 \phi^4
	\]
\column{0.33\textwidth}
	\[
		\epsilon(\vx, t) = \cfrac{\epsilon_0(\vx)}{f(\phi(\vx, t)) + \delta}
	\]
\column{0.33\textwidth}
\end{columns}
\end{frame}


\begin{frame}{Математическая модель}
\vspace{-0.3cm}
\begin{block}{Уравнения динамики системы}
	\begin{itemize}
		\item Уравнение электрического потенциала $\Phi$:
		\[
			\Div(\epsilon[\phi] \nabla \Phi) = 0
		\]
		\item Уравнение фазового поля $\phi$ (типа Аллена--Кана):
		\[
			\cfrac{1}{m} \partt{\phi} = -F'(\phi; |\nabla \Phi|) + \half \Gamma \Delta \phi
		\]
	\end{itemize}
\end{block}
\begin{itemize}
	\item Плотность свободной энергии
	\vspace{-0.2cm}
	\[
		\pi = F(\phi; |\nabla \Phi|) + \cfrac{\Gamma}{4} \gradsq{\phi}
	\]
	\item $m$, $\Gamma$ -- параметры модели, константы
\end{itemize}
\end{frame}


\begin{frame}{Разностная схема}
\vspace{-0.9cm}
\[
	\cfrac{1}{m} \partt{\phi} = -F'(\phi; |\nabla \Phi|) + \half \Gamma \partxx{\phi}
\]
\vspace{-0.4cm}
\begin{itemize}
	\item $|\nabla \Phi|$ -- параметр
\end{itemize}
\begin{block}{Разностная задача}
	\[
		\cfrac{1}{m} \cfrac{\phi_i^{j + 1} - \phi_i^j}{\tau} = \half K_\phi^2 \epsilon'(\phi_i^j) + \cfrac{\Gamma}{l^2} f'(\phi_i^j) + \cfrac{\Gamma}{2} \cfrac{\phi_{i + 1}^j - 2 \phi_i^j + \phi_{i - 1}^j}{h^2}
	\]
	\[\phi_i^0 = \phi_0(ih); \quad \phi_0^j = \phi_l(j \tau); \quad \phi_n^j = \phi_r(j \tau)\]
	Сетка регулярная; $\tau$ -- шаг по времени, $h$ -- шаг по пространству.
\end{block}
Явная разностная схема первого порядка по времени, второго -- по пространству.
\end{frame}


\begin{frame}{Типичное решение}
\vspace{-0.4cm}
\begin{columns}
\column{0.88\textwidth}
\begin{figure}
	\includegraphics[width=\textwidth]{figures/typical_solution.png}
\end{figure}
\column{0.12\textwidth}
\hfill \\
\vspace{3.5cm}
\hspace{-2.5cm}
Из работы \cite{ponomarev_stability}. \\
\hspace{-2.5cm}
Узлов по измерениям: \\
\hspace{-2.5cm}
$N_x = 10^3$, $N_t = 10^5$
\end{columns}
\end{frame}


\begin{frame}{Цель работы}
\begin{itemize}
	\item Типичное поведение модели: долгий период медленных изменений, \\ затем стремительное развитие пробоя.
\end{itemize}
\begin{block}{Цель работы}
	Исследовать несколько подходов к адаптации расчетного шага по времени.
\end{block}
\end{frame}

%!TEX root = ../main.tex

\section{Методы адаптации}

\begin{frame}{Общий вид схемы с адаптивным шагом}
\begin{itemize}
	\item Вводится переменный шаг $\tau^k$:
	\[
		\phi_j^{k + 1} = \phi_j^k + m \tau^k \left( -F'(\phi_j^k) + \cfrac{\Gamma}{2} \diffhh{\phi} \right)
	\]
	\item Значение $\tau^k$ ограничено заранее выбранными $\tau_{min}$ и $\tau_{max}$:
	\[
		\tau^k = \max \left[ \tau_{min}, \min( \tau_{max}, \widetilde{\tau}^k) \right]
	\]
\end{itemize}
\end{frame}


\begin{frame}{Методы адаптации}
\vspace{-0.7cm}
\begin{block}{Методы адаптации}
	\begin{itemize}
		\item По фазовому полю:
		\[
			\widetilde{\tau}_1^k = \cfrac{tol_1}{\left\| \left[ \partt{\phi} \right]_h \right\|_C}
		\]
		\item По полной энергии:
		\[
			\widetilde{\tau}_2^k = \cfrac{tol_2}{\left| \left[ \cfrac{d \Pi}{dt} \right]_h \right|}
		\]
	\end{itemize}
\end{block}
Предложены в статьях \cite{li_time_step} и \cite{zhang_time_step}.
\end{frame}

{\nologo
\begin{frame}{Методы адаптации}
\vspace{-0.6cm}
\begin{itemize}
	\item Условие устойчивости схемы \cite{ponomarev_stability}:
	\vspace{-0.3cm}

	\[
		\tau \leqslant \cfrac{1}{4m} \min \left( \cfrac{\delta^{5/3}}{|\nabla \Phi|^2 \epsilon_0}, \cfrac{h^2}{\Gamma} \right)
	\]
	\vspace{-0.3cm}
	\item Неравенство с первым аргументом $\min$ можно переписать в виде
	\[
		m \tau \max\limits_{\phi \in [0, 1]} |F''(\phi)| \leqslant 1
	\]
\end{itemize}
\vspace{-0.3cm}
\begin{block}{Адаптация по устойчивости}
	\[
		\widetilde{\tau}_3^k = \cfrac{tol_3}{m \cdot \max\limits_{j = 0}^N G(\phi_j^k)},
	\]
	где $G(\phi)$ мажорирует $|F''(\phi)|$
\end{block}
\end{frame}
}

%!TEX root = ../main.tex

\section{Вычислительный эксперимент}

\begin{frame}{Параметры модели}
\vspace{-1cm}
\begin{itemize}
	\item Параметры, отражающие реальный физический эксперимент:
\end{itemize}
\centering
\begin{tabular}{|l|c|l|}
	\hline
	Название & Переменная & Значение \\
	\hline
	электрическое напряжение		& $|\nabla \Phi|$	& $5.625 \cdot 10^6 \; \unitV / \unitm$							\\
	энергия роста ед. длины канала	& $\Gamma$			& $8.118 \cdot 10^{-10} \; \unitJ / \unitm$						\\
	диэлектрическая проницаемость	& $\epsilon_0$		& $2.301 \cdot 10^{-11} \; \unitC^2 / (\unitJ \cdot \unitm)$	\\
	подвижность						& $m$				& $12 \; \unitm^3 / (\unitJ \cdot \units)$						\\
	\hline
	характерная толщина границы		& $l$ 				& $1.5 \cdot 10^{-6} \; \unitm$									\\
	регуляризующий параметр 		& $\delta$			& $10^{-3}$														\\
	размер образца					& $L$				& $3.2 \cdot 10^{-5} \; \unitm$									\\
	продолжительность				& $T$				& $2 \cdot 10^{-3} \; \units$									\\
	шаг по пространству				& $h$				& $5 \cdot 10^{-7} \; \unitm$									\\
	минимальный шаг по времени		& $\tau_{min}$		& $10^{-10} \; \units$											\\
	максимальный шаг по времени		& $\tau_{max}$		& $\leqslant 6.42 \cdot 10^{-6} \; \units$						\\
	\hline
\end{tabular}
\end{frame}


\begin{frame}{Поведение системы}
\vspace{-0.3cm}
\begin{figure}
	\includegraphics[width=\textwidth]{figures/system_behaviour.png}
\end{figure}
\end{frame}


\begin{frame}{Результаты расчетов}
\centering
\begin{tabular}{|l|c|c|c|}
	\hline
	Тип адаптации & Ускорение (раз) & Отклонение по $\phi$ & Запаздывание \\
	\hline \hline
	по фазовому полю	& 800	& $3.64 \cdot 10^{-4}$	& $0.3\%$		\\
	по энергии			& 107	& $5.38 \cdot 10^{-4}$	& $0.36\%$		\\
	по устойчивости		& 1474	& $1.54 \cdot 10^{-2}$	& $0.71\%$		\\
	\hline \hline
	по фазовому полю	& 101	& $1.23 \cdot 10^{-5}$	& $0.004\%$		\\
	по энергии			& 101	& $3.27 \cdot 10^{-4}$	& $0.19\%$		\\
	по устойчивости		& 100	& $2.23 \cdot 10^{-5}$	& $0.0046\%$	\\
	\hline
\end{tabular}
\end{frame}

%!TEX root = ../main.tex

\section{Conclusions}

In this paper we study stability properties of the phase-field model
for electrical breakdown channel evolution.
The central result is a classification of the
equilibrium solutions of the model and their stability.
From practical point of view, these results allows to
make meaningful conclusions regarding qualitative and quantitative
properties of the model. Particularly it was shown under which
conditions small perturbations of the equilibrium solutions
develop into channel-like structure typical for of electrical breakdown
process.

Besides this, a simple explicit finite-difference scheme
for solution of the model in spatially one-dimensional setting is considered.
The main question addressed here are stability conditions which guaranties
correctness of the simulations. Deep connections between
stability conditions of the model and the one of the
finite-difference scheme are shown.
The presented results of the numerical simulations confirms
predictions of the theoretical analysis of the model.

% EOF
\endinput


\begin{frame}{Литература}
\printbibliography
\end{frame}

\begin{frame}{}
\begin{center}
	\Large
	Спасибо за внимание!
\end{center}
\end{frame}

\end{document}