%!TEX root = main.tex

\section{Постановка задачи и модель}

\subsection{Математическая модель}

Приведем описание математической модели, предложенной в работе \cite{pitike_dielectric_breakdown}.

Итак, рассматривается ограниченная область пространства $\Omega \subset \Real^3$. Распределение фаз вещества в ней задается гладкой функцией $\phi: \Omega \times [0, +\infty)_t \to [0, 1], \; \phi(\vx, t)$ -- фазовым полем; вещество может находиться в одной из двух фаз: $\phi \approx 1$ -- <<неповрежденное>>, $\phi \approx 0$ -- <<полностью разрушенное>> (то есть относящееся к каналу пробоя), -- а также в промежуточных состояниях в зоне диффузной границы.

Диэлектрическую проницаемость среды $\epsilon$ предлагается описать следующей формулой:
\begin{equation}
	\epsilon(\vx, t) = \epsilon[\phi] = \cfrac{\epsilon_0(\vx)}{f(\phi(\vx, t)) + \delta} \tpoint
	\label{eq:epsilon}
\end{equation}
Здесь $\epsilon_0(\vx)$ -- диэлектрическая проницаемость неповрежденной среды; $f(\phi) = 4\phi^3 - 3\phi^4$ -- интерполирующая функция, гладко соединяющая значения $0$ и $1$ ($f(0) = 0, \; f(1) = 1, \; f'(0) = f'(1) = 0$); $0 < \delta \ll 1$ -- регуляризующий параметр. Обратим внимание, что при $\phi = 1 \;\; \epsilon(\vx, t) \approx \epsilon_0(\vx)$, что соответствует диэлектрику; при $\phi = 0 \;\; \epsilon(\vx, t) = \epsilon_0(\vx) / \delta$ (в $\delta^{-1} \gg 1$ раз больше), что соответствует проводнику.

Помимо фазового поля $\phi$, состояние системы описывает также функция $\Phi: \Omega \times [0, +\infty)_t \to \Real, \; \Phi(\vx, t)$ -- потенциал электрического поля.

Постулируется следующее выражение для свободной энергии:
\begin{equation}
	\Pi = \int \limits_\Omega \pi d \vx \tcomma
	\label{eq:free_energy}
\end{equation}
\begin{equation}
	\pi = -\cfrac{1}{2} \epsilon[\phi] (\nabla \Phi, \nabla \Phi) + \Gamma \cfrac{1 - f(\phi)}{l^2} + \cfrac{\Gamma}{4} (\nabla \phi, \nabla \phi) \tpoint
	\label{eq:free_energy_density}
\end{equation}
Здесь $\Gamma > 0, \; l > 0$ -- числовые параметры модели, константы.

Постулируются два уравнения, определяющие динамику системы:
\begin{equation*}
\begin{cases}
	\cfrac{\delta \pi}{\delta \Phi} = 0 \tsemicolon \\
	\cfrac{1}{m} \partt{\phi} = -\cfrac{\delta \pi}{\delta \phi} \tpoint
\end{cases}
\end{equation*}
Здесь константа $m > 0$ -- числовой параметр модели, называемый подвижностью: она имеет смысл скорости изменения $\phi$ под действием единичной <<приложенной силы>>. Говоря нестрого, согласно первому уравнению электрический потенциал $\Phi$ распределяется так, чтобы свободная энергия системы была минимальной; согласно второму -- фазовое поле $\phi$ с определенной скоростью стремится к тому, чтобы свободная энергия была минимальной.

Отыскав явно вариационные производные в двух уравнениях выше, получим следующую систему уравнений:
\begin{numcases}{}
	\Div(\epsilon[\phi] \nabla \Phi) = 0 \tsemicolon
	\label{eq:Phi} \\
	\cfrac{1}{m} \partt{\phi} = \half \epsilon'(\phi) (\nabla \Phi, \nabla \Phi) + \cfrac{\Gamma}{l^2} f'(\phi) + \cfrac{1}{2} \Gamma \triangle \phi \tpoint
	\label{eq:phi}
\end{numcases}
Здесь $(\cdot)' \equiv (\cdot)_\phi'$. Система состоит из двух уравнений: на $\phi$ и $\Phi$ соответственно; система связная, второе уравнение нелинейное, является уравнением типа Аллена--Кана.

В таблице \ref{tab:quantities}, перечислены названия и размерности величин, встречающихся в описанной модели.

\begin{table}[!t]
\captionsetup{justification=raggedright,singlelinecheck=false}
\caption[]{Величины, относящиеся к модели.}
\centering
\begin{tabular}{|c|c|m{11cm}|}
	\hline
	Величина & Размерность & Название либо физический смысл \\
	\hline \hline
	$\phi(\vx, t)$ & $1$ & фазовое поле \\
	\hline
	\rule{0mm}{\tabletopspace}
	$\Phi(\vx, t)$ & $\cfrac{\unitJ}{\unitC}$ & потенциал электрического поля \\[\tablebottomspace]
	\hline
	$\Pi(t)$ & $\unitJ$ & свободная энергия \\
	\hline
	\rule{0mm}{\tabletopspace}
	$\pi(\vx, t)$ & $\cfrac{\unitJ}{\unitm^3}$ & плотность свободной энергии \\[\tablebottomspace]
	\hline
	\rule{0mm}{\tabletopspace}
	$\epsilon(\vx, t)$ & $\cfrac{\unitF}{\unitm} = \cfrac{\unitC^2}{\unitJ \cdot \unitm}$ & диэлектрическая проницаемость среды \\[\tablebottomspace]
	\hline
	\rule{0mm}{\tabletopspace}
	$\epsilon_0(\vx)$ & $\cfrac{\unitF}{\unitm} = \cfrac{\unitC^2}{\unitJ \cdot \unitm}$ & диэлектрическая проницаемость неповрежденной среды \\[\tablebottomspace]
	\hline
	$\delta$ & $1$ & определяет диэлектрическую проницаемость полностью разрушенной среды: она равна $\epsilon_0(\vx)/\delta$ \\
	\hline
	$l$ & $\unitm$ & характерная толщина диффузной границы \\
	\hline
	\rule{0mm}{\tabletopspace}
	$\Gamma$ & $\cfrac{\unitJ}{\unitm}$ & характерная энергия образования единицы длины канала пробоя \\[\tablebottomspace]
	\hline
	\rule{0mm}{\tabletopspace}
	$m$ & $\cfrac{\units \cdot \unitJ}{\unitm^3}$ & подвижность фазового поля $\phi$ \\[\tablebottomspace]
	\hline
\end{tabular}
\label{tab:quantities}
\end{table}


\subsection{Одномерная задача}

Рассмотрим систему со следующими ограничениями. Пусть $\Omega = [0, w]_x \times [0, h]_y \times I_z$, где $w, h > 0, \; I$ -- некоторый отрезок; $\phi(\vx, 0) = \phi_0(\vx) = \phi_0(x),$ $\epsilon_0(\vx) = \epsilon_0(x)$, то есть начальное распределение фаз и диэлектрическая проницаемость неповрежденной среды зависят только от $x$. На границе $\Omega$ будем считать заданным граничное условие $\Phi|_{y = 0} = \Phi^- \in \Real, \; \Phi|_{y = h} = \Phi^+ \in \Real, \; \Phi^- \leqslant \Phi^+$. Такую систему можно представить себе как двумерный (тривиально растянутый по третьему измерению) прямоугольный конденсатор, у которого сверху и снизу обкладки с постоянным электрическим потенциалом, между ними -- диэлектрик, меняющий свойства только по горизонтали.

Попробуем искать решение системы уравнений \eqref{eq:Phi}, \eqref{eq:phi}, имеющее $\phi(\vx, t) = \phi(x, t)$, то есть полагая, что $\phi$ не будет зависеть от $y$ и $z$.

Преобразуем уравнение \eqref{eq:Phi}:
\begin{equation}
	0 = \Div(\epsilon[\phi] \nabla \Phi) = (\nabla \epsilon, \nabla \Phi) + \epsilon \triangle \Phi = \partx{\epsilon} \partx{\Phi} + \epsilon \triangle \Phi \tpoint
	\label{eq:Phi_one_dim}
\end{equation}
Заметим, что независимо от конкретных $\phi$ и $\epsilon_0$ решением является $\Phi(\vx, t) = \Phi^- + (y/h)(\Phi^+ - \Phi^-)$. В этом случае $\partial \Phi / \partial x \equiv 0, \; \triangle \Phi \equiv 0$ и уравнение \eqref{eq:Phi_one_dim} становится тождеством.

Преобразуем уравнение \eqref{eq:phi}:
\begin{equation}
	\cfrac{1}{m} \partt{\phi} = \half \epsilon'(\phi) \left( \cfrac{\Phi^+ - \Phi^-}{h} \right)^2 + \cfrac{\Gamma}{l^2} f'(\phi) + \half \Gamma \partxx{\phi} \tcomma
	\label{eq:one_dim}
\end{equation}
$\phi_0(\vx) = \phi_0(x)$. Решение этого уравнения с начальным условием $\phi_0$ будет зависеть только от $x$ и времени $t$.

Уравнение $\eqref{eq:one_dim}$ можно рассматривать как дифференциальное уравнение в частных производных на функцию $\phi(x, t)$ одной пространственной переменной и решать его на отрезке $[0, w]_x$ числовой прямой.

Для удобства введем $K_\Phi = \|\nabla \Phi\| = (\Phi^+ - \Phi^-)/h$, тогда уравнение \eqref{eq:one_dim} примет вид
\begin{equation}
	\cfrac{1}{m} \partt{\phi} = \half K_\Phi^2 \epsilon'(\phi) + \cfrac{\Gamma}{l^2} f'(\phi) + \half \Gamma \partxx{\phi} \tpoint
	\label{eq:one_dim_simpler}
\end{equation}
$\Phi^+, \Phi^-$ и $h$ перестали входить в уравнение явно -- так мы убрали последнее упоминание о втором (по $y$) измерении пространства.

Для простоты анализа везде далее будем считать $\epsilon_0$ константой.

Итак, пара из решения уравнения \eqref{eq:one_dim_simpler} и $\Phi = \Phi^- + (y/h)(\Phi^+ - \Phi^-)$ является решением исходной системы уравнений \eqref{eq:Phi}, \eqref{eq:phi}, в том случае если краевые условия имеют описанный вид.