%!TEX root = ../main.tex

\section{Введение}

Электрический пробой -- это явление резкого возрастания тока в диэлектрике при приложении электрического напряжения выше некоторого критического значения. Механизм разрушения диэлектрика под действием электрического поля сложен и многообразен: оно может иметь различные причины, характер развития, сопутствующие физические процессы \cite{vorobiev_dielectric_physics}. В настоящей работе рассматривается исключительно случай твердого диэлектрика.

Вопрос о пробое твердых диэлектриков имеет важное практическое значение, так как таковые используются в любой электроизоляционной конструкции. В области техники это явление практически всегда вредно: оно приводит к нарушению работы электрической цепи и разрушению изоляции.

Деградация диэлектрических свойств материала, как правило, происходит лишь в ограниченной зоне, называемой каналом пробоя. Сложность моделирования развития канала пробоя имеет две составляющие. Во-первых, сложно само физическое явление. Во-вторых, канал пробоя -- это эффективно одномерный объект, развивающийся в трехмерной среде, то есть объект коразмерности 2.

Среди многообразия математических моделей, созданных для описания развития канала электрического пробоя, выделим предложенную в работе \cite{pitike_dielectric_breakdown} модель типа диффузной границы.

В настоящее время модели типа диффузной границы составляют целый класс подходов для решения прикладных задач гидродинамики \cite{lamorgese_flow_modeling}, \cite{kim_fluid_flows}, \cite{xu_hydrodynamics}, механики деформируемого тела и теории трещин \cite{ambati_fracture}, материаловедения \cite{provatas_materials}, солидификации и теории фазовых переходов \cite{boettinger_solidification}, \cite{cartalade_phase_separation}, \cite{gransaly_solidification}, описания кристаллических структур \cite{emmerich_crystal}, \cite{asadi_crystal}, \cite{provatas_crystal}. Предложенная в работе \cite{pitike_dielectric_breakdown} модель развития канала электрического пробоя построена как формальное обобщение известных моделей типа диффузной границы для распространения трещин в упругой среде.

Исследование и дальнейшее развитие модели, предложенной в работе \cite{pitike_dielectric_breakdown}, можно найти в статьях \cite{zipunova_higher_codimension}, \cite{zipunova_conservative}, \cite{zipunova_thermomechanical}.

Перечислим основные положения метода диффузной границы в применении к описанию развития канала электрического пробоя. Сразу отметим, что рассматриваемая модель феноменологическая: без рассмотрения причин образования пробоя на микроуровне строится уравнение состояния среды, согласующееся с базовыми законами физики. Вещество в системе находится в разных состояниях (фазах), то есть не однородно, причем части вещества в одном и том же состоянии образуют некоторые однородные области. С точки зрения физики <<раздел>> фаз имеет пренебрежимо малую толщину, его можно считать плоской поверхностью. Однако в моделях типа диффузной границы принимается, что распределение фаз в пространстве задано гладкой функцией~$\phi$ (фазовым полем): внутри каждой области однородности значение $\phi$ близко к постоянному, а на разделяющем слое (<<диффузной границе>>) меняется пусть и быстро, но непрерывно. В рассматриваемой модели вещество имеет две различные фазы: <<неповрежденное>> -- $\phi \approx 1$ -- и <<полностью разрушенное>> (то есть относящееся к каналу пробоя) -- $\phi \approx 0$. На разрушение среды, то есть на постепенный переход вещества от состояния $\phi \approx 1$ к состоянию $\phi \approx 0$ тратится энергия электрического поля.

Настоящая работа посвящена исследованию модели развития канала электрического пробоя, предложенной в работе \cite{pitike_dielectric_breakdown}. Цель работы -- исследование качественных характеристик модели и ее численный анализ, представляющие интерес ввиду сложности модели.

Для анализа качественных характеристик модели исходная дифференциальная задача была рассмотрена в определенных краевых условиях, упрощающих ее. Проведен теоретический анализ устойчивости положений равновесия системы, посредством чего установлено, при каких условиях происходит развитие канала электрического пробоя из малых возмущений неповрежденной среды.

Построена явная разностная схема первого порядка на регулярной сетке. Дана содержательная оценка устойчивости разностной схемы. С помощью моделирования на компьютере подтверждены полученные теоретические результаты: проверена оценка устойчивости схемы, сходимость сеточного решения при выполнении условия устойчивости, свойства положений равновесия, невозрастание свободной энергии системы.