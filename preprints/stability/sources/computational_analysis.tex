%!TEX root = ../main.tex

\section{Численное исследование}

\subsection{Вычислительный эксперимент: устойчивость}

Была написана программа, реализующая разностную схему \eqref{sch:transition}, \eqref{sch:borders}. Проверим на практике устойчивость и сходимость.

Зафиксируем параметры уравнения \eqref{eq:one_dim}:
\begin{equation}
    \epsilon_0 = 0.2, \; \delta = 0.04, \; l = 1.0, \; \Gamma = 1.0, \; m = 0.5, \; K_\Phi = 4.8 \tpoint
    \label{exp:parameters}
\end{equation}
Отметим, что перед нами случай <<сильного напряжения>> (см. выражение \eqref{char:equilibriums}).

Моделируем решение в области 
\begin{equation}
    \Omega = [0, W]_x \times [0, T]_t, \; W = 5, \; T = 1 \tpoint
    \label{exp:set}
\end{equation}

Зададим следующие краевые условия:
\begin{equation}
\begin{gathered}
    \phi(0, t) = 1, \; \phi(W, t) = 1 \tcomma \\
    \phi(x, 0) \equiv \phi_0(x) = \begin{cases}
        1, \; \text{если} \; x \leqslant 2.25 \; \text{или} \; x \geqslant 2.75 \tsemicolon \\
        1 - 0.025 \cdot [1 + \cos(4 \pi x)], \; \text{если} \; 2.25 < x < 2.75 \tpoint
    \end{cases}
\end{gathered} \label{exp:borders}
\end{equation}
Обратим внимание, что $\phi_0(x)$ дважды дифференцируема всюду, кроме конечного числа точек, с ограниченной второй производной.

Обозначим $n_x$ количество отрезков разбиения $[0, W]_x$ (узлов, соответственно, $n_x + 1$); $n_t$ -- количество отрезков разбиения $[0, T]_t$. $h = W / n_x, \; \tau = T / n_t$.

\begin{figure}[!tp]
    \centering
    \includegraphics[width=\textwidth]{figures/typical_solution.png}
    \vspace{-0.8cm}
    \caption{Типичное решение задачи, $n_x = 10^3, \; n_t = 10^5$.}
    \label{fig:typical_solution}
\end{figure}

Для начала посмотрим на типичное решение исследуемой задачи (рис. \ref{fig:typical_solution}). Видно постепенное развитие канала электрического пробоя (разрушение среды) из небольшого начального возмущения фазового поля $\phi$ неповрежденной среды. Примерно в момент времени $t = 0.55$ канал пробоя <<прорастает насквозь>>, а именно, $\phi$ вблизи точки $x = 2.5$ приближается к нулевому значению. Обратим внимание, что в период времени $t \in (0.3, \; 0.55)$ канал пробоя (область, где $\phi$ существенно отличается от $1$) практически не растет в ширину, а при $t > 0.55$, напротив, растет в ширину почти с постоянной скоростью.

Проверим полученную в предыдущем разделе оценку \eqref{cond:spectral_better} устойчивости разностной схемы. Будем считать, что в вычислительном эксперименте схема неустойчива, если программа завершилась с ошибкой: произошло деление на $0$ (в формуле \eqref{eq:epsilon} функции $\epsilon(\phi)$ при $f(\phi) = -\delta$) или значения $\phi$ ушли на бесконечность (переполнился тип double). Будем перебирать $n_x$ и $n_t$, запоминая пары соседних точек, в одной из которых устойчивость есть, а в другой нет. Так получим опытную оценку устойчивости схемы. Отобразим ее на графике вместе с оценкой \eqref{cond:spectral_better} (рис. \ref{fig:stability_bounds}).

Эксперимент показывает, что оценка \eqref{cond:spectral_better} удачна: она примерно повторяет контур опытной оценки, к тому же ее график лежит выше, то есть она имеет некоторый <<запас>> до момента, когда в программе возникает ошибка. Именно ради этого <<запаса>> знаменатель исходной оценки  \eqref{cond:spectral_better_theoretical} был удвоен.

\begin{figure}[!tp]
    \centering
    \includegraphics[width=\textwidth]{figures/stability_bounds.png}
    \vspace{-0.7cm}
    \caption{Теоретическая и опытная оценки устойчивости разностной схемы.}
    \label{fig:stability_bounds}
\end{figure}


\subsection{Вычислительный эксперимент: сходимость}

Аппроксимация схемой \eqref{sch:transition}, \eqref{sch:borders} задачи Коши \eqref{eq:one_dim}, \eqref{eq:cauchy_borders} очевидна; для устойчивости схемы получено условие \eqref{cond:spectral_better}, строго говоря, являющееся лишь необходимым, но применимое на практике. Теперь экспериментально проверим сходимость.

Поясним связь формальных конструкций предыдущего раздела с решаемой задачей Коши.

На множестве $C_2(\Omega)$ дважды непрерывно дифференцируемых функций в замкнутой области $\Omega = [0, W]_x \times [0, T]_t$ рассмотрим следующие нормы: непрерывную $\enorm_C$ и $L_2$-норму $\enorm_2$.
$$\norm{f}_C = \max \limits_{(x, t) \in \Omega} f(x, t); \qquad \norm{f}_2 = \sqrt{\int \limits_{\Omega} f^2(x, t) dx dt} \tpoint$$
Эти же нормы введем на множествах $C(\partial \Omega)$ и $C(\Int \Omega)$ непрерывных функций на границе и внутренности $\Omega$ соответственно.

Теперь рассмотрим регулярную сетку $\Omega_{h, \tau}$; введем некоторую зависимость $\tau = \tau(h)$. Ограничивая функции из $C_2(\Omega)$, $C(\Int \Omega)$ и $C(\partial \Omega)$ на множестве $\Omega_h = \Omega_{h, \tau(h)}$, получаем множества $C_2(\Omega)_h$, $C(\Int \Omega)_h$ и $C(\partial \Omega)_h$ сеточных функций.

На перечисленных множествах сеточных функций введем нормы, согласованные с $\enorm_C$ и $\enorm_2$:
$$\norm{f_a^b}_C = \max \limits_{(a, b) \in \Omega_h} f_a^b; \qquad \norm{f_a^b}_2 = \sqrt{\cfrac{1}{h \tau}\sum \limits_{(a, b) \in \Omega_h} (f_a^b)^2} \tpoint$$

Задачу Коши \eqref{eq:one_dim}, \eqref{eq:cauchy_borders} легко привести к виду \eqref{eq:formal_differential}, где $R: C_2(\Omega) \longrightarrow C(\Int \Omega)$, $r: C_2(\Omega) \longrightarrow C(\partial \Omega)$; разностную задачу \eqref{sch:transition}, \eqref{sch:borders} -- к виду \eqref{eq:formal_subtractive}, где $R_h: C_2(\Omega)_h \longrightarrow C(\Int \Omega)_h$, $r_h: C_2(\Omega)_h \longrightarrow C(\partial \Omega)_h$.

Перейдем к вычислительному эксперименту. Сходимость будем проверять по описанным выше нормам $\enorm_C$ и $\enorm_2$ на множестве сеточных функций. Так как аналитическое решение задачи Коши не известно, будем сравнивать ряд результатов на все более мелких сетках по норме с лучшим результатом в ряду. При сравнении функцию на более мелкой сетке ограничиваем на более крупной, игнорируя часть узлов.

Зафиксируем ранее использовавшиеся параметры уравнения \eqref{exp:parameters}, \eqref{exp:set}; зададим краевые условия \eqref{exp:borders}. Положим $n_x = W / h$ -- число отрезков разбиения по $x$, $n_t = T / \tau$ -- по $t$.

Во всех описанных далее вариантах расчетов соблюдается условие устойчивости~\eqref{cond:spectral_better}.

Для начала зафиксируем $n_x = 200$ и будем перебирать $n_t$, каждый раз увеличивая его вдвое. Сравнение по нормам с результатом при $n_t = 204800$ изображено на рис. \ref{fig:convergence_fixed_nx}. Разностная схема имеет первый порядок аппроксимации по $t$; опыт показывает первый порядок сходимости ошибки $O(\tau)$.

Зафиксируем $n_t = 204800$ и будем перебирать $n_x$, каждый раз увеличивая его вдвое. Сравнение по нормам с результатом при $n_x = 1600$ изображено на рис. \ref{fig:convergence_fixed_nt}. Разностная схема имеет второй порядок аппроксимации по $x$; опыт показывает второй порядок сходимости ошибки $O(h^2)$.

\begin{figure}[!tp]
    \centering
    \includegraphics[width=0.72\textwidth]{figures/convergence_fixed_nx.png}
    \vspace{-0.2cm}
    \caption{Ошибка решения по норме при фиксированном $n_x = 200$.}
    \label{fig:convergence_fixed_nx}
    \vspace{0.6cm}
    
    \includegraphics[width=0.72\textwidth]{figures/convergence_fixed_nt.png}
    \vspace{-0.2cm}
    \caption{Ошибка решения по норме при фиксированном $n_t = 204800$.}
    \label{fig:convergence_fixed_nt}
    \vspace{0.6cm}
    
    \includegraphics[width=0.72\textwidth]{figures/convergence_connected.png}
    \vspace{-0.2cm}
    \caption{Ошибка решения по норме при $n_t = 0.08 \cdot n_x^2$.}
    \label{fig:convergence_connected}
\end{figure}

Теперь свяжем $n_x$ и $n_t$ уравнением, так чтобы при $h, \tau \to 0$ выполнялось условие устойчивости \eqref{cond:spectral_better}. При выбранных параметрах модели подойдет $n_t = 0.08 \cdot n_x^2$. Аналогично проведем сравнение ряда измерений по норме с лучшим (рис. \ref{fig:convergence_connected}). Как и ожидалось, измерения показывают сходимость $O(\tau + h^2) = O(\tau)$ первого порядка по времени при выбранном уравнении связи.

В первых двух опытах, без стремления обоих шагов сетки к $0$, последовательности сеточных функций имели неясный предел. В третьем же, если принять предположение об устойчивости разностной схемы, сеточные функции сходятся к решению задачи Коши \eqref{eq:one_dim}, \eqref{eq:cauchy_borders}.


\subsection{Вычислительный эксперимент: положения равновесия}

Ранее были исследованы положения равновесия уравнения \eqref{eq:one_dim} вида $\phi \equiv C$. Их количество и устойчивость определяется значением выражения \eqref{char:equilibriums} (обозначено $\xi$). Проверим этот результат экспериментально.

Зададим модели параметры \eqref{exp:parameters}, \eqref{exp:set}. В качестве начального условия берем возмущенное положение равновесия: $\phi(x, 0) = C + A \cos(\omega x); \; \phi(0, t) \equiv \phi(0, 0), \; \phi(W, t) \equiv \phi(W, 0)$. Амплитуда $A$ мала, порядка $0.01$. $n_x = 800, \; n_t = 51200$.

Если положение равновесия устойчиво, то при любом $\omega$ возмущение угасает; если неустойчиво, то существует некоторое $\omega_0$, такое что при $\omega < \omega_0$ возмущение растет.

Положим $K_{\Phi, 1} = 0, \; K_{\Phi, 2} = 1.1, \; K_{\Phi, 3} = 4.8$. Как было сказано выше, $\delta = 0.04$. В таком случае $\xi_1 = 0 < \delta^2, \; \xi_2 = 0.121 \in (\delta^2, (1 + \delta)^2), \; \xi_3 = 2.304 > (1 + \delta)^2$.

Вначале рассмотрим $K_{\Phi, 1} = 0, \; \xi_1 < \delta^2$ -- случай слабого электрического напряжения. Система имеет два положения равновесия: $\phi \equiv 0$ неустойчивое, $\phi \equiv 1$ устойчивое. На рис. \ref{fig:equilibrium_1_0}, \ref{fig:equilibrium_1_1} видно теоретически предсказанное поведение возмущенной среды: при $C = 0$ возмущение растет, при $C = 1$ -- затухает. В точке $C = 0$ производная функции $\chi(\phi)$ (см. выражение \eqref{eq:equilibruim_characteristic}) равна $0$, поэтому, чтобы увидеть рост возмущения, приходится брать небольшое $\omega$, обеспечивая небольшое значение $\partial^2 \phi / \partial x^2$. В эксперименте с $\phi \equiv 1$ взято $C = 1 - A$, чтобы значения $\phi$ не превосходили $1$.

\begin{figure}[!t]
    \centering
    \includegraphics[width=0.9\textwidth]{figures/equilibrium_1_0.png}
    \vspace{-0.3cm}
    \caption{Случай <<слабого напряжения>>: возмущенное положение равновесия $\phi \equiv 0$, неустойчивое.}
    \label{fig:equilibrium_1_0}
    \vspace{0.5cm}
    
    \includegraphics[width=0.9\textwidth]{figures/equilibrium_1_1.png}
    \vspace{-0.3cm}
    \caption{Случай <<слабого напряжения>>: возмущенное положение равновесия $\phi \equiv 1$, устойчивое.}
    \label{fig:equilibrium_1_1}
\end{figure}

Теперь рассмотрим $K_{\Phi, 2} = 1.1, \; \xi_2 \in (\delta^2, (1 + \delta)^2)$ -- случай среднего напряжения. Система имеет три положения равновесия: $\phi \equiv 0$ устойчивое, $\phi \equiv C_3 \approx 0.5$ неустойчивое ($C_3$ -- корень функции $\chi(\phi)$ в интервале $(0, 1)$), $\phi \equiv 1$ устойчивое. Поведение возмущенной среды изображено на рис. \ref{fig:equilibrium_2_0}, \ref{fig:equilibrium_2_05}, \ref{fig:equilibrium_2_1}, оно соответствует теоретическим результатам.

\begin{figure}[!tp]
    \centering
    \includegraphics[width=0.9\textwidth]{figures/equilibrium_2_0.png}
    \vspace{-0.3cm}
    \caption{Случай <<среднего напряжения>>: возмущенное положение равновесия $\phi \equiv 0$, устойчивое.}
    \label{fig:equilibrium_2_0}
    \vspace{0.5cm}

    \includegraphics[width=0.9\textwidth]{figures/equilibrium_2_05.png}
    \vspace{-0.3cm}
    \caption{Случай <<среднего напряжения>>: возмущенное положение равновесия $\phi \equiv C_3 \approx 0.5$, неустойчивое.}
    \label{fig:equilibrium_2_05}
    \vspace{0.5cm}
    
    \includegraphics[width=0.9\textwidth]{figures/equilibrium_2_1.png}
    \vspace{-0.3cm}
    \caption{Случай <<среднего напряжения>>: возмущенное положение равновесия $\phi \equiv 1$, устойчивое.}
    \label{fig:equilibrium_2_1}
\end{figure}

Наконец, рассмотрим $K_{\Phi, 3} = 4.8, \; \xi_3 > (1 + \delta)^2$ -- случай сильного напряжения. Система имеет два положения равновесия: $\phi \equiv 0$ устойчивое, $\phi \equiv 1$ неустойчивое. Поведение возмущенной среды изображено на рис. \ref{fig:equilibrium_3_0}, \ref{fig:equilibrium_3_1}, оно также соответствует теории.

\begin{figure}[!t]
    \centering
    \includegraphics[width=0.9\textwidth]{figures/equilibrium_3_0.png}
    \vspace{-0.3cm}
    \caption{Случай <<сильного напряжения>>: возмущенное положение равновесия $\phi \equiv 0$, устойчивое.}
    \label{fig:equilibrium_3_0}
    \vspace{0.5cm}
    
    \includegraphics[width=0.9\textwidth]{figures/equilibrium_3_1.png}
    \vspace{-0.3cm}
    \caption{Случай <<сильного напряжения>>: возмущенное положение равновесия $\phi \equiv 1$, неустойчивое.}
    \label{fig:equilibrium_3_1}
\end{figure}