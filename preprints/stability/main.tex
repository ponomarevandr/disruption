\documentclass[a4paper,12pt]{article}

%%% Размер шрифта
\usepackage[14pt]{extsizes}

%%% Поля
\usepackage[
	left=2cm,
	right=2cm,
	top=2cm,
	bottom=3cm,
	bindingoffset=0cm
]{geometry}

%%% Работа с русским языком
\usepackage{cmap}						% поиск в PDF
\usepackage{mathtext}					% русские буквы в формулах
\usepackage[T2A]{fontenc}				% кодировка
\usepackage[utf8]{inputenc}				% кодировка исходного текста
\usepackage[english,russian]{babel}		% локализация и переносы
\usepackage{indentfirst}
\frenchspacing

%%% Дополнительная работа с математикой
\usepackage{amsmath,amsfonts,amssymb,amsthm,mathtools}  % AMS

%%% Текст в колонки
\usepackage{multicol}

%%% Списки
\usepackage{enumitem}
\setlist{nosep, leftmargin=*}
\renewcommand{\labelenumi}{\arabic*)}

%%% Системы уравнений
\usepackage{cases}

%%% Таблицы
\usepackage{array}

%%% Рисунки
\usepackage{graphicx}
\usepackage{float}

%%% Точка в подписях к рисункам
\usepackage[labelsep=period]{caption}

%%% Список литературы
\bibliographystyle{bibliography_style/gost-numeric.bbx}
\usepackage[
	natbib = true,
	style = gost-numeric,
	sorting = none,
	backend = biber,
	language = autobib,
	autolang = other
]{biblatex}
\addbibresource{references.bib}

%%% Исправление символа номера при использовании gost-numeric.bbx
\usepackage{textcomp}
\DefineBibliographyStrings{russian}{number={\textnumero}}

%%% Гиперссылки
\usepackage[pdftex,unicode]{hyperref}

%%% Перенос знаков в формулах (по Львовскому)
\newcommand*{\hm}[1]{#1\nobreak\discretionary{}{\hbox{$\mathsurround=0pt #1$}}{}}


%%% Свои команды

\newcommand*{\No}{\textnumero}

\newcommand{\vect}[1]{\boldsymbol{#1}}
\newcommand{\vx}{{\vect{x}}}

\newcommand{\half}{\cfrac{1}{2}}
\newcommand{\partt}[1]{\cfrac{\partial #1}{\partial t}}
\newcommand{\partx}[1]{\cfrac{\partial #1}{\partial x}}
\newcommand{\partxx}[1]{\cfrac{\partial^2 #1}{\partial x^2}}

\newcommand{\Real}{{\mathbb{R}}}

\newcommand{\tabletopspace}{9mm}
\newcommand{\tablebottomspace}{3mm}

\newcommand{\tpoint}{{\text{.}}}
\newcommand{\tcomma}{{\text{,}}}
\newcommand{\tsemicolon}{{\text{;}}}

\newcommand{\unitm}{{\text{м}}}
\newcommand{\units}{{\text{с}}}
\newcommand{\unitJ}{{\text{Дж}}}
\newcommand{\unitC}{{\text{Кл}}}
\newcommand{\unitF}{{\text{Ф}}}


%%% Свои операторы
\DeclareMathOperator{\Div}{div}


%%% Оформление теорем

\theoremstyle{plain}
\newtheorem{proposition}{Утверждение}

\theoremstyle{remark}
\newtheorem{remark}{Замечание}


%%% Описание препринта
\newcommand{\PreprintTitle}{
	Неизотермическая консервативная модель динамики развития канала электрического пробоя типа <<диффузной границы>>
}
\newcommand{\PreprintTitleEnglish}{
	Neizotermicheskaya konservativnaya modelie dinamiki razvitiya kanala oelektricheskogo proboya tipa <<diffouznoy granicea>>
}
\newcommand{\PreprintAuthors}{
	А.С.~Пономарев, Е.Б.~Савенков, Е.В.~Зипунова
}
\newcommand{\PreprintAuthorsEnglish}{
	A.S.~Ponomarev, E.B.~Savenkov, E.V.~Zipunova
}


%%%%%%%%%%%%%%%%%%%%%%%%%%%%%%%%%%%%%%%%%%%%%%%%%%%%%%%%%%%%%%%%%%%%%%%%%%%%%%%%

\begin{document}

%%%%%%%%%%%%%%%%%%%%%%%%%%%%%%%%%%%%%%
\begin{titlepage}

\begin{center}
	РОССИЙСКАЯ АКАДЕМИЯ НАУК \\
	ОРДЕНА ЛЕНИНА \\
	ИНСТИТУТ ПРИКЛАДНОЙ МАТЕМАТИКИ \\
	имени М. В. КЕЛДЫША \\

	\vspace*{60mm}
	\Large{\PreprintAuthors} \\
	\vspace*{20mm}
	\textbf{\large \PreprintTitle} \\
	\vspace*{110mm}
	\Large{Москва, 2024}
	\vspace*{-50mm}
\end{center}

\end{titlepage}
%%%%%%%%%%%%%%%%%%%%%%%%%%%%%%%%%%%%%%%

\setcounter{page}{2}

\thispagestyle{empty}

\noindent \emph{\PreprintAuthors}, 
\PreprintTitle
\vspace*{3mm}

\noindent \textbf{Аннотация} \\
{
	\small
	Цель настоящей работы -- исследование качественных характеристик и численный анализ модели типа диффузной границы, описывающей развитие канала электрического пробоя в твердом диэлектрике. Проведен анализ устойчивости положений равновесия системы; установлены условия развития канала пробоя из малых возмущений неповрежденной среды. Построена и изучена разностная схема для задачи, дана содержательная оценка ее устойчивости. Полученные теоретические результаты подтверждены моделированием на компьютере. \\[3mm]
	\textbf{Ключевые слова:} модель типа диффузной границы, фазовое поле, устойчивость,	электрический пробой. \\[5mm]
}
\emph{\PreprintAuthorsEnglish},
\PreprintTitleEnglish \\[3mm]
\textbf{Abstract} \\
{
	\small
	[Annotaciya rabotea na angliyskom yazeake.] \\[3mm]
	\textbf{Key words and phrases:} diffuse interface models, phase field, stability, electrical breakdown. \\[5mm]
}

\newpage

%!TEX root = ../main.tex

\section{Введение}

\begin{frame}{Физическое явление}
\begin{block}{Электрический пробой}
	Явление резкого возрастания тока в диэлектрике при приложении электрического напряжения
	выше критического.
\end{block}
\begin{itemize}
	\item Рассматриваем твердый диэлектрик
	\item Деградация диэлектрических свойств материала
	\item Процесс развивается в ограниченной зоне -- канале пробоя
	\item Сложная физическая природа
\end{itemize}
\end{frame}


\begin{frame}{Математическая модель}
\begin{block}{Модель типа диффузной границы}
	Вещество находится в разных фазах. Состояние вещества описывается гладкой функцией
	$\phi(\vx, t)$ -- фазовым полем.
\end{block}
\begin{itemize}
	\item $\phi = 1$ -- неповрежденная среда
	\item $\phi = 0$ -- полностью разрушенная среда
	\item Зона $\phi \in (0, 1)$ -- диффузная граница
	\item На разрушение среды тратится энергия
\end{itemize}
\begin{figure}
	\includegraphics[width=0.5\textwidth]{figures/diffuse_edge.jpg}
\end{figure}
\end{frame}


\begin{frame}{Математическая модель}
Модель, предложенная в работе \cite{pitike_dielectric_breakdown}:
\begin{itemize}
	\item $\pi = \textcolor{red}{-\half \epsilon[\phi] (\nabla \Phi, \nabla \Phi)} +
	\Gamma \left( \cfrac{1 - f(\phi)}{l^2} + \cfrac{1}{4} (\nabla \phi, \nabla \phi) \right)$
	-- плотность свободной энергии
	\item $\Gamma$ -- энегрия роста канала пробоя на единицу длины
	\item $l$ -- величина <<размытия>> канала
	\item $\epsilon(\vx, t)$ -- диэлектрическая проницаемость среды
	\item $f(\phi)$ -- интерполирующая функция
\end{itemize}
\end{frame}


\begin{frame}{Математическая модель}
\vspace{-0.2cm}
\begin{itemize}
	\item $\epsilon(\vx, t) = \cfrac{\epsilon_0(\vx)}{f(\phi(\vx, t)) +
	\delta}$ -- диэлектрическая проницаемость среды
	\item $f(\phi) = 4 \phi^3 - 3 \phi^4$ -- интерполирующая функция
\end{itemize}
\begin{columns}
\column{0.5\textwidth}
\begin{figure}
	\hspace*{1.4cm}
	\includegraphics[width=0.65\textwidth]{figures/f_form.png}
\end{figure}
\column{0.5\textwidth}
\begin{figure}
	\hspace*{-2cm}
	\includegraphics[width=0.60\textwidth]{figures/eps_form.png}
\end{figure}
\end{columns}
\end{frame}


\begin{frame}{Математическая модель}
\vspace{-0.5cm}
\begin{block}{Уравнения модели}
\begin{itemize}
	\item Уравнение электрического потенциала $\Phi$:
	\begin{equation}
		\Div(\epsilon[\phi] \nabla \Phi) = 0
		\label{equation_potential}
	\end{equation}
	\item Уравнение фазового поля $\phi$:
	\begin{equation}
		\cfrac{1}{m} \partt{\phi} = \half \epsilon'(\phi) \gradscalsq{\Phi} + \cfrac{\Gamma}{l^2} f'(\phi) + \half \Gamma \Delta \phi
		\label{equation_phase}
	\end{equation}
\end{itemize}
\end{block}
Свойства:
\begin{itemize}
	\item связанная система уравнений на $\phi$ и $\Phi$;
	\item уравнение для $\phi$ типа Аллена--Кана, нелинейное.
\end{itemize}
\end{frame}


\begin{frame}{Пример вычислительного эксперимента}
\begin{columns}
\column{0.32\textwidth}
\begin{figure}
	\includegraphics[width=\textwidth]{figures/model_example_1.png}
\end{figure}
\column{0.32\textwidth}
\begin{figure}
	\includegraphics[width=\textwidth]{figures/model_example_2.png}
\end{figure}
\column{0.32\textwidth}
\begin{figure}
	\includegraphics[width=\textwidth]{figures/model_example_3.png}
\end{figure}
\end{columns}
\begin{center}
	Расчет из работы \cite{zipunova_experiment}
\end{center}
\end{frame}


\begin{frame}{Цель работы}
\begin{block}{Цель работы}
	Исследовать качественные характеристики системы уравнений \eqref{equation_potential},
	\eqref{equation_phase} и выполнить ее численный анализ.
\end{block}
Для этого рассмотрим задачу в определенных краевых условиях, упрощающих ее, но позволяющих
установить интересующие свойства.
\end{frame}

%!TEX root = main.tex

\section{Постановка задачи и модель}

\subsection{Математическая модель}

Приведем описание математической модели, предложенной в работе \cite{pitike_dielectric_breakdown}.

Итак, рассматривается ограниченная область пространства $\Omega \subset \Real^3$. Распределение фаз вещества в ней задается гладкой функцией $\phi: \Omega \times [0, +\infty)_t \to [0, 1], \; \phi(\vx, t)$ -- фазовым полем; вещество может находиться в одной из двух фаз: $\phi \approx 1$ -- <<неповрежденное>>, $\phi \approx 0$ -- <<полностью разрушенное>> (то есть относящееся к каналу пробоя), -- а также в промежуточных состояниях в зоне диффузной границы.

Диэлектрическую проницаемость среды $\epsilon$ предлагается описать следующей формулой:
\begin{equation}
    \epsilon(\vx, t) = \epsilon[\phi] = \cfrac{\epsilon_0(\vx)}{f(\phi(\vx, t)) + \delta} \tpoint
    \label{eq:epsilon}
\end{equation}
Здесь $\epsilon_0(\vx)$ -- диэлектрическая проницаемость неповрежденной среды; $f(\phi) = 4\phi^3 - 3\phi^4$ -- интерполирующая функция, гладко соединяющая значения $0$ и $1$ ($f(0) = 0, \; f(1) = 1, \; f'(0) = f'(1) = 0$); $0 < \delta \ll 1$ -- регуляризующий параметр. Обратим внимание, что при $\phi = 1 \;\; \epsilon(\vx, t) \approx \epsilon_0(\vx)$, что соответствует диэлектрику; при $\phi = 0 \;\; \epsilon(\vx, t) = \epsilon_0(\vx) / \delta$ (в $\delta^{-1} \gg 1$ раз больше), что соответствует проводнику.

Помимо фазового поля $\phi$, состояние системы описывает также функция $\Phi: \Omega \times [0, +\infty)_t \to \Real, \; \Phi(\vx, t)$ -- потенциал электрического поля.

Постулируется следующее выражение для свободной энергии:
\begin{equation}
    \Pi = \int \limits_\Omega \pi d \vx \tcomma
    \label{eq:free_energy}
\end{equation}
\begin{equation}
    \pi = -\cfrac{1}{2} \epsilon[\phi] (\nabla \Phi, \nabla \Phi) + \Gamma \cfrac{1 - f(\phi)}{l^2} + \cfrac{\Gamma}{4} (\nabla \phi, \nabla \phi) \tpoint
    \label{eq:free_energy_density}
\end{equation}
Здесь $\Gamma > 0, \; l > 0$ -- числовые параметры модели, константы.

Постулируются два уравнения, определяющие динамику системы:
\begin{equation*}
\begin{cases}
    \cfrac{\delta \pi}{\delta \Phi} = 0 \tsemicolon \\
    \cfrac{1}{m} \partt{\phi} = -\cfrac{\delta \pi}{\delta \phi} \tpoint
\end{cases}
\end{equation*}
Здесь константа $m > 0$ -- числовой параметр модели, называемый подвижностью: она имеет смысл скорости изменения $\phi$ под действием единичной <<приложенной силы>>. Говоря нестрого, согласно первому уравнению электрический потенциал $\Phi$ распределяется так, чтобы свободная энергия системы была минимальной; согласно второму -- фазовое поле $\phi$ с определенной скоростью стремится к тому, чтобы свободная энергия была минимальной.

Отыскав явно вариационные производные в двух уравнениях выше, получим следующую систему уравнений:
\begin{numcases}{}
    \Div(\epsilon[\phi] \nabla \Phi) = 0 \tsemicolon
    \label{eq:Phi} \\
    \cfrac{1}{m} \partt{\phi} = \half \epsilon'(\phi) (\nabla \Phi, \nabla \Phi) + \cfrac{\Gamma}{l^2} f'(\phi) + \cfrac{1}{2} \Gamma \triangle \phi \tpoint
    \label{eq:phi}
\end{numcases}
Здесь $(\cdot)' \equiv (\cdot)_\phi'$. Система состоит из двух уравнений: на $\phi$ и $\Phi$ соответственно; система связная, второе уравнение нелинейное, является уравнением типа Аллена--Кана.

В таблице \ref{tab:quantities}, перечислены названия и размерности величин, встречающихся в описанной модели.

\begin{table}[!t]
\captionsetup{justification=raggedright,singlelinecheck=false}
\caption[]{Величины, относящиеся к модели.}
\centering
\begin{tabular}{|c|c|m{11cm}|}
    \hline
    Величина & Размерность & Название либо физический смысл \\
    \hline \hline
    $\phi(\vx, t)$ & $1$ & фазовое поле \\
    \hline
    \rule{0mm}{\tabletopspace}
    $\Phi(\vx, t)$ & $\cfrac{\unitJ}{\unitC}$ & потенциал электрического поля \\[\tablebottomspace]
    \hline
    $\Pi(t)$ & $\unitJ$ & свободная энергия \\
    \hline
    \rule{0mm}{\tabletopspace}
    $\pi(\vx, t)$ & $\cfrac{\unitJ}{\unitm^3}$ & плотность свободной энергии \\[\tablebottomspace]
    \hline
    \rule{0mm}{\tabletopspace}
    $\epsilon(\vx, t)$ & $\cfrac{\unitF}{\unitm} = \cfrac{\unitC^2}{\unitJ \cdot \unitm}$ & диэлектрическая проницаемость среды \\[\tablebottomspace]
    \hline
    \rule{0mm}{\tabletopspace}
    $\epsilon_0(\vx)$ & $\cfrac{\unitF}{\unitm} = \cfrac{\unitC^2}{\unitJ \cdot \unitm}$ & диэлектрическая проницаемость неповрежденной среды \\[\tablebottomspace]
    \hline
    $\delta$ & $1$ & определяет диэлектрическую проницаемость полностью разрушенной среды: она равна $\epsilon_0(\vx)/\delta$ \\
    \hline
    $l$ & $\unitm$ & характерная толщина диффузной границы \\
    \hline
    \rule{0mm}{\tabletopspace}
    $\Gamma$ & $\cfrac{\unitJ}{\unitm}$ & характерная энергия образования единицы длины канала пробоя \\[\tablebottomspace]
    \hline
    \rule{0mm}{\tabletopspace}
    $m$ & $\cfrac{\units \cdot \unitJ}{\unitm^3}$ & подвижность фазового поля $\phi$ \\[\tablebottomspace]
    \hline
\end{tabular}
\label{tab:quantities}
\end{table}


\subsection{Одномерная задача}

Рассмотрим систему со следующими ограничениями. Пусть $\Omega = [0, w]_x \times [0, h]_y \times I_z$, где $w, h > 0, \; I$ -- некоторый отрезок; $\phi(\vx, 0) = \phi_0(\vx) = \phi_0(x),$ $\epsilon_0(\vx) = \epsilon_0(x)$, то есть начальное распределение фаз и диэлектрическая проницаемость неповрежденной среды зависят только от $x$. На границе $\Omega$ будем считать заданным граничное условие $\Phi|_{y = 0} = \Phi^- \in \Real, \; \Phi|_{y = h} = \Phi^+ \in \Real, \; \Phi^- \leqslant \Phi^+$. Такую систему можно представить себе как двумерный (тривиально растянутый по третьему измерению) прямоугольный конденсатор, у которого сверху и снизу обкладки с постоянным электрическим потенциалом, между ними -- диэлектрик, меняющий свойства только по горизонтали.

Попробуем искать решение системы уравнений \eqref{eq:Phi}, \eqref{eq:phi}, имеющее $\phi(\vx, t) = \phi(x, t)$, то есть полагая, что $\phi$ не будет зависеть от $y$ и $z$.

Преобразуем уравнение \eqref{eq:Phi}:
\begin{equation}
    0 = \Div(\epsilon[\phi] \nabla \Phi) = (\nabla \epsilon, \nabla \Phi) + \epsilon \triangle \Phi = \partx{\epsilon} \partx{\Phi} + \epsilon \triangle \Phi \tpoint
    \label{eq:Phi_one_dim}
\end{equation}
Заметим, что независимо от конкретных $\phi$ и $\epsilon_0$ решением является $\Phi(\vx, t) = \Phi^- + (y/h)(\Phi^+ - \Phi^-)$. В этом случае $\partial \Phi / \partial x \equiv 0, \; \triangle \Phi \equiv 0$ и уравнение \eqref{eq:Phi_one_dim} становится тождеством.

Преобразуем уравнение \eqref{eq:phi}:
\begin{equation}
    \cfrac{1}{m} \partt{\phi} = \half \epsilon'(\phi) \left( \cfrac{\Phi^+ - \Phi^-}{h} \right)^2 + \cfrac{\Gamma}{l^2} f'(\phi) + \half \Gamma \partxx{\phi} \tcomma
    \label{eq:one_dim}
\end{equation}
$\phi_0(\vx) = \phi_0(x)$. Решение этого уравнения с начальным условием $\phi_0$ будет зависеть только от $x$ и времени $t$.

Уравнение $\eqref{eq:one_dim}$ можно рассматривать как дифференциальное уравнение в частных производных на функцию $\phi(x, t)$ одной пространственной переменной и решать его на отрезке $[0, w]_x$ числовой прямой.

Для удобства введем $K_\Phi = \|\nabla \Phi\| = (\Phi^+ - \Phi^-)/h$, тогда уравнение \eqref{eq:one_dim} примет вид
\begin{equation}
    \cfrac{1}{m} \partt{\phi} = \half K_\Phi^2 \epsilon'(\phi) + \cfrac{\Gamma}{l^2} f'(\phi) + \half \Gamma \partxx{\phi} \tpoint
    \label{eq:one_dim_simpler}
\end{equation}
$\Phi^+, \Phi^-$ и $h$ перестали входить в уравнение явно -- так мы убрали последнее упоминание о втором (по $y$) измерении пространства.

Для простоты анализа везде далее будем считать $\epsilon_0$ константой.

Итак, пара из решения уравнения \eqref{eq:one_dim_simpler} и $\Phi = \Phi^- + (y/h)(\Phi^+ - \Phi^-)$ является решением исходной системы уравнений \eqref{eq:Phi}, \eqref{eq:phi}, в том случае если краевые условия имеют описанный вид.

%!TEX root = ../main.tex

\section{Stability analysis of equilibrium solutions}
\label{sec:theoretical_analysis}

Under certain conditions, the electrical breakdown can develop
form small perturbations of the undamaged medium properties.
To clarify these conditions in this section we study
stability of constant solutions~$\phi(x, t) \equiv C, \; C \in [0, 1]$,
of the equation~\eqref{eq:one_dim}.

First, one has to find stationary constant solutions of~\eqref{eq:one_dim}.
From the definition~\eqref{eq:epsilon} follows the expressions for the derivatives
of~$\epsilon(\phi)$:
\begin{equation}
	\epsilon'(\phi) = \cfrac{-\epsilon_0 f'(\phi)}{(f(\phi) + \delta)^2} \tsemicolon \qquad \epsilon''(\phi) = \epsilon_0 \cfrac{2 (f'(\phi))^2 - f''(\phi)(f(\phi) + \delta)}{(f(\phi) + \delta)^3} \tpoint
	\label{eq:epsilon_derivatives}
\end{equation}

Substituting~$\phi(x, t) \equiv C$ into~\eqref{eq:one_dim} and
taking~\eqref{eq:epsilon_derivatives} into account, one has:
\begin{equation}
	f'(C) \left( \cfrac{\Gamma}{l^2} - \half K_\Phi^2 \cfrac{\epsilon_0}{(f(C) + \delta)^2} \right) = 0 \tpoint
	\label{eq:equilibrium}
\end{equation}
First, consider the case~$f'(C) = 12C^2 (1 - C) = 0$,
which leads to~$C = 0,1$. Hence, $\phi \equiv 0$ and $\phi \equiv 1$ are
equilibrium solutions.

Second, let~$C \neq 0, 1$. Then
\[
\cfrac{\Gamma}{l^2} = \cfrac{K_\Phi^2 \epsilon_0}{2 (f(C) + \delta)^2}
\quad \text{and} \quad
f(C) + \delta = K_\Phi l \sqrt{\cfrac{\epsilon_0}{2 \Gamma}}.
\]

Note that~$f(C) \in [0, 1]$ and, moreover, $f(\phi)$ is monotonically
increasing. Therefore, in case $K_\Phi l \sqrt{\epsilon_0 / (2
  \Gamma)} \in (\delta, 1 + \delta)$ the
equation~\eqref{eq:equilibrium} has a solution $C_3\ne 0,1$ given by
\begin{equation}
	C_3 = f^{-1} \left( K_\Phi l \sqrt{\cfrac{\epsilon_0}{2 \Gamma}} - \delta \right) \tpoint
	\label{eq:equilibrium_third}
\end{equation}
Otherwise the equation~\eqref{eq:equilibrium} has only two solutions.

So, the number of constant equilibrium solutions depends on the following condition
being satisfied:
\begin{equation}
	\delta^2 < \cfrac{K_\Phi^2 l^2 \epsilon_0}{2 \Gamma} < (1 + \delta)^2 \tpoint
	\label{cond:equilibriums_number}
\end{equation}
It will be shown later how the
condition~\eqref{cond:equilibriums_number} is connected with the
stability properties of the equilibrium solutions and the
equation~\eqref{eq:one_dim} itself.

Let us now procees to the stability analysis of the equilibrium solutions.

Let~$\phi(x, t)$ be a solution of~\eqref{eq:one_dim}, $\delta \phi(x,
t)$ be its perturbation.
Writing down the equation~\eqref{eq:one_dim} for the perturbed solution~$\phi
+ \delta \phi$,
after linearizing we obtain the following equation for~$\delta\phi$:
\begin{equation}
  \cfrac{1}{m} \partt{(\delta \phi)} = \left(\half K_\Phi^2 \epsilon''(\phi) + \cfrac{\Gamma}{l^2} f''(\phi) \right) \delta \phi + \half \Gamma \partxx{(\delta \phi)} \tpoint
  \label{eq:variation}
\end{equation}
For further analysis it is convenient to write~\eqref{eq:variation} as:
%
\begin{equation}
  \partt{(\delta \phi)} = A \delta \phi + B \partxx{(\delta \phi)} \tcomma
  \label{eq:variation_common}
\end{equation}
where~$A$ and~$B  > 0$ are the respective parameters.

Choosing~$\delta \phi = e^{\alpha t} \sin(\omega x)$, one obtains from~\eqref{eq:variation_common}
the following relation for the parameters of the perturbation:
$$\alpha e^{\alpha t} \sin(\omega x) = A e^{\alpha t} \sin(\omega x) - B \omega^2 e^{\alpha t} \sin(\omega x) \tcomma$$
from where follows:
\begin{equation}
  \alpha = A - B \omega^2 \tpoint
  \label{eq:exponent_coefficient}
\end{equation}

% Summing up, let us combine now the three parts of the reasoning.
% Consider equilibrium solution~$\phi \equiv C$ of the
% equation~\eqref{eq:one_dim} perturbed by~$\delta \phi$.

% ; применим к $\delta \phi$ уравнение \eqref{eq:variation}, в
% $\epsilon''$ и $f''$ подставим $\phi = C$. Полученное уравнение
% имеет вид~\eqref{eq:variation_common}.

Now it is easy to see that, depending on the value of the coefficient
$$A = \half K_\Phi^2 \epsilon''(C) + \cfrac{\Gamma}{l^2} f''(C) \tcomma$$
three cases arise:
%
\begin{enumerate}[label=\arabic*.]
\item $A > 0$. In this case, from~$\omega^2 \in [0, A / B)$ it follows
  that~$\alpha > 0$, i.e., there exists a perturbation~$\delta \phi$
  growing in time. Hence, the equilibrium solution~$\phi \equiv C$ is
  unstable.
  
\item $A < 0$. Then, for an arbitrary~$\omega$ remains~$\alpha
  \leqslant A < 0$. Next, any perturbation~$\delta \phi$ in the
  interval~$[0, W]_x$ 
  can be represented as Fourier integral over harmonics decreasing
  at least as the harmonic for~$\omega = 0$.
  Hence, the equilibrium solution~$\phi \equiv C$ is stable.
  
\item $A = 0$. Repeating the same reasoning as in the case~$A < 0$,
  one can observe that there exist arbitrarily slowly decreasing
  harmonics (i.e., harmonics with arbitrarily small values of~$\alpha$).
  This case corresponds to neutral stability of the equilibrium
  solution, and linear analysis does not provide complete
  information.
  This case will be considered in more details later.
\end{enumerate}

We now proceed to the discussion of the particular equilibrium states.

Consider the equilibrium solution~$\phi \equiv 0$.
One has~$f''(0) = 0$, $\epsilon''(0) = 0$
(see~\eqref{eq:epsilon_derivatives}),
which leads to~$A = 0$.
As it was noted before, this case requires an elaborate analysis, which will be
performed later.

Consider the equilibrium solution~$\phi \equiv 1$.
In this case~$f''(0) = -12$, $\epsilon''(0) = 12 \epsilon_0 / (1 +
\delta)^2$ (see~\eqref{eq:epsilon_derivatives}).
As a result, we obtain:
$$A = \half K_\Phi^2 \epsilon''(C) + \cfrac{\Gamma}{l^2} f''(C) = \cfrac{6 K_\Phi^2 \epsilon_0}{(1 + \delta)^2} - \cfrac{12 \Gamma}{l^2} \tpoint$$
The equilibrium state is stable if~$A < 0$, i.e., as:
\begin{equation}
  \cfrac{K_\Phi^2 l^2 \epsilon_0}{2 \Gamma} < (1 + \delta)^2 \tpoint
  \label{cond:equilibrium_1_stable}
\end{equation}      
For this case of an unstable equilibrium, let us find~$\omega_0$ such that
increasing harmonics are replaced by the decreasing ones.
To do this, consider~\eqref{eq:exponent_coefficient} with~$\alpha =
0$, $B = \Gamma/2$ and~$A$ given above to obtain: 
$$0 = \cfrac{6 K_\Phi^2 \epsilon_0}{(1 + \delta)^2} - \cfrac{12 \Gamma}{l^2} - \cfrac{\Gamma}{2} \omega_0^2 \tcomma$$
from where follows:
$$\omega_0 = 2 \sqrt{\cfrac{3 K_\Phi^2 \epsilon_0}{\Gamma (1 + \delta)^2} - \cfrac{6}{l^2}} \tpoint$$

Note that the condition~\eqref{cond:equilibrium_1_stable} is exactly the
right-hand side of the inequality~\eqref{cond:equilibriums_number}.
To explain this and to form a complete picture of what is happening,
let us look at the equilibrium solutions from a slightly different
perspective.

Solving the equation~\eqref{eq:equilibrium}, we were finding the zeros of
the function
\begin{equation}
	\chi(\phi) = \half K_\Phi^2 \epsilon'(\phi) + \cfrac{\Gamma}{l^2} f'(\phi) \tpoint
	\label{eq:equilibruim_characteristic}
\end{equation}
Hence, each equilibrium solution $\phi \equiv C$ uniquely
corresponds to a zero~$C$ of the function~$\chi(\phi)$.
From the derivation of the equation~\eqref{eq:variation} for the perturbation
it follows that in its right-hand side the coefficient at~$\delta
\phi$ is~$\chi'(\phi)$.
Later, analyzing the equation~\eqref{eq:variation_common} for an equilibrium
solution~$\phi \equiv C$, we considered several cases depending on the
sign of the coefficient~$A$, which turns out to be exactly~$\chi'(C)$.


Summing up the results, one can state the following.
The function~$\chi(\phi)$ defined by~\eqref{eq:equilibruim_characteristic}
is smooth on~$[0, 1]$ and always has zeros~$C_1=0$ and~$C_2=1$.
The third zero~$C=C_3\in (0, 1)$ exists under the
condition~\eqref{cond:equilibriums_number}.
Each equilibrium solution~$\phi \equiv C$ uniquely corresponds to a zero
of the function~$\chi(\phi)$. Their stability properties are
described in terms of the sign of~$\chi'(\phi)$ at the zeros: positive
values of~$\chi'$
correspond to the unstable solution and negative ones~---
to the stable one.

It is also clear that in the case of vanishing~$\chi'$
(as for~$\phi = 0$) the linear analysis is not enough~---
it is necessary to analyse the sign of the first higher-order
non-vanishing
derivative of~$\chi$~--- the equilibrium solution is stable if
this derivative is negative and unstable if it is positive.


Finally we show that~$\chi(\phi)$ has a non-vanishing derivative at its
zero~$C=C_3 \in (0, 1)$ (if the latter exists).
Indeed, one has:
$$\chi(C_3) = f'(C_3) \left( \cfrac{\Gamma}{l^2} - \cfrac{K_\Phi^2 \epsilon_0}{2 (f(C_3) + \delta)^2} \right) = 0 \tpoint$$
Taking into account that~$f'(C_3) \ne 0$, one obtains:
$$\cfrac{\Gamma}{l^2} - \cfrac{K_\Phi^2 \epsilon_0}{2 (f(C_3) + \delta)^2} = 0 \tpoint$$
Then:
$$\chi'(\phi)|_{C_3} = f'(C_3) \left( \cfrac{\Gamma}{l^2} - \cfrac{K_\Phi^2 \epsilon_0}{2 (f(\phi) + \delta)^2} \right) ' \bigg|_{C_3} = (f'(C_3))^2 \cfrac{K_\Phi^2 \epsilon_0}{(f(C_3) + \delta)^3} \ne 0 \tpoint$$

Now it is possible to provide a comprehensive analysis of the behavior
of~$\chi(\phi)$ at its zeros. As it can be seen from the conditions~\eqref{cond:equilibriums_number} and~\eqref{cond:equilibrium_1_stable},
its behavior is governed by the value of the parameter
\begin{equation}
  \xi = \cfrac{K_\Phi^2 l^2 \epsilon_0}{2 \Gamma} \tpoint
  \label{char:equilibriums}
\end{equation}

First, consider the case~$0 \leqslant \xi < \delta^2$. The zeros
of~$\chi(\phi)$ are~$0$ and~$1$; $\chi'(0) = 0$, $\chi'(1) < 0$.
The qualitative behavior of~$\chi(\phi)$ is shown schematically on
Fig.~\ref{fig:equilibriums_case_1}.  It can be seen that the equilibrium
solution~$\phi \equiv 0$ is unstable and~$\phi \equiv 1$ is stable.
Such case can be conventionally called the case of ``weak electric field''.
This means that with all the parameters except the electric field being fixed,
the latter is so small that even an almost completely damaged
medium with~$\phi \approx 0$ is ``healed'' over time and evolves to the
completely undamaged state~$\phi \approx 1$.

Second, consider the case~$\delta^2 < \xi < (1 + \delta)^2$.
The zeros of~$\chi(\phi)$ are $C=0$, $C=C_3$
(see~\eqref{eq:equilibrium_third}) and $C=1$; $\chi'(0) = 0, \;
\chi'(1) < 0; \; \chi'(C_3) > 0$
(since~$\chi$ is smooth).
The behavior of~$\chi(\phi)$ in this case is shown on
Fig.~\ref{fig:equilibriums_case_2}.
The equilibrium solutions are: $\phi \equiv 0$~--- stable one, $\phi
\equiv C_3$~--- unstable one, and~$\phi \equiv 1$~--- also stable.
Such case can be conventionally called the case of the ``medium
electric field''.
This means that as the values of~$\phi$ are sufficiently close to~$0$,
the damage increases, i.e.,~$\phi$ tends to zero;
as the values of~$\phi$ are sufficiently close to~$1$,
the damage decreases, i.e.,~$\phi$ tends to one;
at certain intermediate values the equilibrium is unstable.

Finally, consider the case~ $(1 + \delta)^2 < \xi$.
The zeros of~$\chi(\phi)$ are~$C=0$ and~$C=1$; $\chi'(0) = 0$, $\chi'(1)>0$.
The qualitative behavior of~$\chi(\phi)$ is schematically shown on
Fig.~\ref{fig:equilibriums_case_3}.
The equilibrium solutions are: $\phi \equiv 0$~--- the stable one, $\phi
\equiv 1$~--- the unstable one.
This case can be conventionally called the case of ``strong
electric field''.
This means that the electric field is sufficiently strong and
any state arbitrarily close to the completely undamaged one
(i.e., any state close to~$\phi \approx 1$) evolves towards
the completely damaged state~$\phi= 0$.
Essentially this is the case where the completely damaged state develops
from arbitrarily small perturbations of the completely undamaged
equilibrium solution.

In all the three cases stability of the equilibrium solution~$\phi \equiv
0$ is defined by the higher order derivatives of~$\chi(\phi)$.

\begin{figure}[!tp]
  \centering
  \includegraphics[width=0.84\textwidth]{figures/equilibriums_case_1.png}
  \vspace{-0.3cm}
  \caption{Characteristic behavior of~$\chi(\phi)$,
    ``weak electric field'' case.}
  \label{fig:equilibriums_case_1}
  \vspace{0.7cm}
  
  \includegraphics[width=0.84\textwidth]{figures/equilibriums_case_2.png}
  \vspace{-0.3cm}
  \caption{Characteristic behavior of~$\chi(\phi)$,
    ``medium electric field'' case.}
  \label{fig:equilibriums_case_2}
  \vspace{0.7cm}
  
  \includegraphics[width=0.84\textwidth]{figures/equilibriums_case_3.png}
  \vspace{-0.3cm}
  \caption{Characteristic behavior of~$\chi(\phi)$,
    ``strong electric field'' case.}
  \label{fig:equilibriums_case_3}
\end{figure}

% EOF

%!TEX root = ../main.tex

\section{Conclusions}

In this paper we study stability properties of the phase-field model
for electrical breakdown channel evolution.
The central result is a classification of the
equilibrium solutions of the model and their stability.
From practical point of view, these results allows to
make meaningful conclusions regarding qualitative and quantitative
properties of the model. Particularly it was shown under which
conditions small perturbations of the equilibrium solutions
develop into channel-like structure typical for of electrical breakdown
process.

Besides this, a simple explicit finite-difference scheme
for solution of the model in spatially one-dimensional setting is considered.
The main question addressed here are stability conditions which guaranties
correctness of the simulations. Deep connections between
stability conditions of the model and the one of the
finite-difference scheme are shown.
The presented results of the numerical simulations confirms
predictions of the theoretical analysis of the model.

% EOF
\endinput

\newpage
\printbibliography

\newpage
\tableofcontents

\end{document}

%%%%%%%%%%%%%%%%%%%%%%%%%%%%%%%%%%%%%%%%%%%%%%%%%%%%%%%%%%%%%%%%%%%%%%%%%%%%%%%%