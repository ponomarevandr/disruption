\documentclass[a4paper,12pt]{article}

%%% Размер шрифта
\usepackage[14pt]{extsizes}

%%% Поля
\usepackage[
	left=2cm,
	right=2cm,
	top=2cm,
	bottom=3cm,
	bindingoffset=0cm
]{geometry}

%%% Работа с русским языком
\usepackage{cmap}						% поиск в PDF
\usepackage{mathtext}					% русские буквы в формулах
\usepackage[T2A]{fontenc}				% кодировка
\usepackage[utf8]{inputenc}				% кодировка исходного текста
\usepackage[english,russian]{babel}		% локализация и переносы
\usepackage{indentfirst}
\frenchspacing

%%% Дополнительная работа с математикой
\usepackage{amsmath,amsfonts,amssymb,amsthm,mathtools}  % AMS

%%% Текст в колонки
\usepackage{multicol}

%%% Списки
\usepackage{enumitem}
\setlist{nosep, leftmargin=*}
\renewcommand{\labelenumi}{\arabic*)}

%%% Системы уравнений
\usepackage{cases}

%%% Таблицы
\usepackage{array}

%%% Рисунки
\usepackage{graphicx}
\usepackage{float}

%%% Точка в подписях к рисункам
\usepackage[labelsep=period]{caption}

%%% Список литературы
\bibliographystyle{bibliography_style/gost-numeric.bbx}
\usepackage[
	natbib = true,
	style = gost-numeric,
	sorting = none,
	backend = biber,
	language = autobib,
	autolang = other
]{biblatex}
\addbibresource{references.bib}

%%% Исправление символа номера при использовании gost-numeric.bbx
\usepackage{textcomp}
\DefineBibliographyStrings{russian}{number={\textnumero}}

%%% Гиперссылки
\usepackage[pdftex,unicode]{hyperref}

%%% Перенос знаков в формулах (по Львовскому)
\newcommand*{\hm}[1]{#1\nobreak\discretionary{}{\hbox{$\mathsurround=0pt #1$}}{}}


%%% Свои команды

\newcommand*{\No}{\textnumero}

\newcommand{\vect}[1]{\boldsymbol{#1}}


%%% Свои операторы


%%% Оформление теорем

\theoremstyle{plain}
\newtheorem{proposition}{Утверждение}

\theoremstyle{remark}
\newtheorem{remark}{Замечание}


%%% Описание препринта
\newcommand{\PreprintTitle}{
	Неизотермическая консервативная модель динамики развития
	канала электрического пробоя типа <<диффузной границы>>
}
\newcommand{\PreprintTitleEnglish}{
	Neizotermicheskaya konservativnaya modelie dinamiki razvitiya
	kanala oelektricheskogo proboya tipa <<diffouznoy granicea>>
}
\newcommand{\PreprintAuthors}{
	А.С.~Пономарев, Е.Б.~Савенков, Е.В.~Зипунова
}
\newcommand{\PreprintAuthorsEnglish}{
	A.S.~Ponomarev, E.B.~Savenkov, E.V.~Zipunova
}


%%%%%%%%%%%%%%%%%%%%%%%%%%%%%%%%%%%%%%%%%%%%%%%%%%%%%%%%%%%%%%%%%%%%%%%%%%%%%%%%

\sloppy

\begin{document}

%%%%%%%%%%%%%%%%%%%%%%%%%%%%%%%%%%%%%%
\begin{titlepage}

\begin{center}
	РОССИЙСКАЯ АКАДЕМИЯ НАУК \\
	ОРДЕНА ЛЕНИНА \\
	ИНСТИТУТ ПРИКЛАДНОЙ МАТЕМАТИКИ \\
	имени М. В. КЕЛДЫША \\

	\vspace*{60mm}
	\Large{\PreprintAuthors} \\
	\vspace*{20mm}
	\textbf{\large \PreprintTitle} \\
	\vspace*{110mm}
	\Large{Москва, 2024}
	\vspace*{-50mm}
\end{center}

\end{titlepage}
%%%%%%%%%%%%%%%%%%%%%%%%%%%%%%%%%%%%%%%

\setcounter{page}{2}

\thispagestyle{empty}

\noindent \emph{\PreprintAuthors}, 
\PreprintTitle
\vspace*{3mm}

\noindent \textbf{Аннотация} \\
{
	\small
	В настоящей работе предложена модель типа <<диффузной границы>>
	для описания динамики развития канала электрического пробоя,
	которая обобщает качественную модель процесса,
	предложенную ранее в работах других авторов.
	В отличии от них, предложенная в настоящей работе модель включает в
	себя уравнение закона сохранения энергии и является энергетически
	консервативной,
	самосогласованным образом учитывая процессы преобразования
	энергии в ходе развития канала электрического пробоя.
	Вывод модели основан на применении методов рациональной
	механики сплошных сред, в частности,
	теории микросил и микронапряжений М.~Гуртина. \\[3mm]
	\textbf{Ключевые слова:}
	модели типа диффузной границы, фазовое поле,
	параметр порядка, электрический пробой \\[5mm]
}
\emph{\PreprintAuthorsEnglish},
\PreprintTitleEnglish \\[3mm]
\textbf{Abstract} \\
{
	\small
	In this paper we consider phase-field model which describes
	electric breackdown process in solid dielectrics.
	The presented model extends the earlier presented
	one.
	The derived model includes energy conservation equation,
	accounts for nonisothermal effects (e.g., Joule heating)
	and consistently describes energy transformation during
	breakdown channel propagation.
	The consistent derivaton of the model is performed
	in the context of rational thermomechanics framework and
	M.~Gurtin therory of microstresses and microforces. \\[3mm]
	\textbf{Key words and phrases:}
	diffuse interface models, phase field, order parameter, electric breakdown. \\[5mm]
}

\newpage


\section{Заключение}

В настоящей работе предложена математическая модель
типа диффузной границы для описания динамики развития канала
электрического
пробоя. Построенная модель обобщает результаты
работы \cite{zipunova_higher_codimension} (какая-то ссылка). В отличие от рассмотренной в этой работе
модели,
предложенная в настоящей работе модель
позволяет учитывать неизотермические процессы,
является консервативной и включает в себя
уравнение закона сохранения энергии.

Вывод модели выполнен в рамках методов рациональной механики сплошной
среды с применением процедуры Колмана-Нолла и теории микросил и
микронапряжений М.~Гуртина. Таким образом, построенная модель является
термодинамически корректной в смысле выполнения закона сохранения
энергии  и выполнения второго закона термодинамики
в форме неравенства Клазиуса-Дюгема.


\newpage
\printbibliography


\newpage
\tableofcontents

\end{document}

%%%%%%%%%%%%%%%%%%%%%%%%%%%%%%%%%%%%%%%%%%%%%%%%%%%%%%%%%%%%%%%%%%%%%%%%%%%%%%%%