%!TEX root = ../main.tex

\section{Разностная схема}

\subsection{Упрощающие краевые условия}

Для описанной модели будем использовать разностную схему из работы \cite{ponomarev_stability}, где предварительно делается ряд допущений, упрозающих задачу. Кратко перечислим их.

Cистема \eqref{eq:Phi},~\eqref{eq:phi} рассматривается в замкнутой области $\clOmega = [0, W]_x \hm \times [0, H]_y \times I_z$, где $W, H > 0, \; I$~-- некоторый отрезок. Пусть $\epsilon_0(\vx) = \epsilon_0(x)$, а также задано начальное условие $\phi(\vx, 0) = \phi_0(x)$, то есть диэлектрическая проницаемость неповрежденной среды и начальное распределение фаз зависят только от $x$. На $\partial \Omega$ считаем заданным следующее граничное условие на $\phi$: $\phi|_{x = 0} = \phi_l(t)$, $\phi|_{x = W} = \phi_r(t)$, а также $\partflvn{\phi}= 0$ на <<гранях>> области $\clOmega$, перпендикулярных осям $y$ и $z$; следующее граничное условие на $\Phi$: $\Phi|_{y = 0} = \Phi^-$, $\Phi|_{y = H} = \Phi^+$, где $\Phi^-, \Phi^+ \in \Real$, $\Phi^+ \geqslant \Phi^-$, а также $\partflvn{\Phi} = 0$ на <<гранях>> $\clOmega$, перпендикулярных осям $x$ и $z$.

Учитывая описанные краевые условия, решение системы уравне\forcehyphenation ний \eqref{eq:Phi},~\eqref{eq:phi} ищется в виде $\phi(\vx, t) = \phi(x, t)$, $\Phi(\vx, t) = \Phi(y, t)$. Решение \linebreak $\Phi(\vx, t) = \Phi^- + (y / H)(\Phi^+ - \Phi^-)$ известно аналитически, и система, таким образом, сводится к единственному уравнению
\begin{equation}
	\cfrac{1}{m} \partt{\phi} = -F'(\phi, |\nabla \Phi|) + \half \Gamma \partxx{\phi},
	\label{eq:one_dim}
\end{equation}
на функцию $\phi(x, t)$ в области $[0, W]_x \times [0, +\infty)_t$, с начальным условием
\begin{equation}
	\phi(x, 0) = \phi_0(x)
	\label{eq:one_dim_initial}{}
\end{equation}
и граничным условием
\begin{equation}
	\phi(0, t) = \phi_l(t), \quad \phi(W, t) = \phi_r(t).
	\label{eq:one_dim_marginal}
\end{equation}
Здесь $|\nabla \Phi| = (\Phi^+ - \Phi^-) / H$ -- константа. В дальнейшем именно $|\nabla \Phi|$ будет считаться параметром модели, конкретные значения $\Phi^+$ и $\Phi^-$ при этом неважны.

Так как в рассматриваемой постановке задачи $|\nabla \Phi|$ -- константа, то вместо $F(\phi; |\nabla \Phi|)$ будем писать $F(\phi)$.

Для простоты анализа везде далее $\epsilon_0$ считается константой.


\subsection{Разностная схема}

Для дифференциальной задачи \eqref{eq:one_dim}, \eqref{eq:one_dim_initial}, \eqref{eq:one_dim_marginal} в работе \cite{ponomarev_stability} составлено разностное уравнение
\begin{equation}
	\cfrac{1}{m} \difftau{\phi} = \half K_\phi^2 \epsilon'(\phi_j^k) + \cfrac{\Gamma}{l^2} f'(\phi_j^k) + \cfrac{\Gamma}{2} \diffhh{\phi}
	\label{eq:subtractive}
\end{equation}
и используется разностная схема
\begin{align}
	& \begin{aligned}
		\phi_j^{k + 1} = \phi_j^k + m \tau \left( -F'(\phi_j^k) + \cfrac{\Gamma}{2} \diffhh{\phi} \right), & \\
		j = \overline{1, N - 1}, \quad k \in \Natural_0; &
	\end{aligned}
	\label{sch:transition_old} \\
	& \phi_j^0 = \phi_0(jh); \quad \phi_0^k = \phi_l(k \tau); \quad \phi_N^k = \phi_r(k \tau).
	\label{sch:borders}
\end{align}
Схема четырехточечная явная, на регулярной сетке с временным шагом~$\tau$ и пространственным шагом $h$. $N = W / h$~-- целое число. $(jh, k \tau)$~-- узлы сетки, $j \hm = \overline{0, N}$, $k \in \Natural_0$. $\phi_j^k$ -- значение сеточной функции $\phi$ в узле $(jh, k \tau)$.


\subsection{Схема с адаптивным шагом по времени}

При работе с моделью возникает следующая проблема. Характерным поведением системы является резкое, стремительное образование канала пробоя, которому предшествует длительный период очень слабых и медленных изменений в системе. Шаг расчетной сетки по времени должен быть достаточно мал для адекватного моделирования быстрых процессов в системе (косвенно это выражается, например, условием устойчивости схемы, представленном в работе \cite{ponomarev_stability}); однако настолько сильное разрешение по времени оказывается избыточным в период медленных процессов. Таким образом, использование регулярной по времени сетки видится нерациональным.

Будем использовать переменный шаг по времени $\tau^k$. Уравнение \eqref{sch:transition_old} приобретает вид
\begin{equation}
	\phi_j^{k + 1} = \phi_j^k + m \tau^k \left( -F'(\phi_j^k) + \cfrac{\Gamma}{2} \diffhh{\phi} \right).
	\label{sch:transition}
\end{equation}

Далее будет рассмотрено несколько подходов к вычислению $\tau^k$. Общий вид его расчета
\begin{equation}
	\tau^k = \max \left[ \tau_{min}, \min( \tau_{max}, \widetilde{\tau}^k) \right],
	\label{sch:time_step_min_max}
\end{equation}
то есть величина $\widetilde{\tau}^k$ (рассчитываемая по своей формуле для каждого подхода) ограничивается снизу и сверху заранее выбранными значениями $\tau_{min}$ и $\tau_{max}$ соответственно.