%!TEX root = ../main.tex

\section{Адаптация по устойчивости}

\subsection{Идея подхода}

В работе \cite{ponomarev_stability} предложено следующее условие устойчивости разностной схемы~\eqref{sch:transition_old},~\eqref{sch:borders}:
\begin{equation}
	\tau \leqslant \cfrac{1}{4m} \min \left( \cfrac{\delta^{5/3}}{\norm{\nabla \Phi}^2 \epsilon_0}, \cfrac{h^2}{\Gamma} \right).
	\label{cond:stability}
\end{equation}
Это неравенство получено применением для уравнения \eqref{eq:subtractive} спектрального признака устойчивости. Строго говоря, спектральный признак не дает достаточных условий устойчивости для нелинейных задач, однако на практике ее можно ожидать.

Неравенство \eqref{cond:stability} с первым аргументом минимума эквивалентно соотношению
\begin{equation}
	m \tau \max\limits_{\phi \in [0, 1]} \left| F''(\phi) \right| \leqslant C,
	\label{cond:stability_first}
\end{equation}
где $C \approx 1.1 / 2$ -- константа, выбранная, во-первых, для создания <<запаса>> в оценке, во-вторых, для удобства формульной записи. Существенно, что \linebreak $C \leqslant 1$.

Основная идея подхода к адаптации, предлагаемого авторами в этом разделе, заключается в том, чтобы в неравенстве \eqref{cond:stability_first} заменить формальный максимум по $\phi \in [0, 1]$ на максимум по значениям сеточной функции $\phi_j^k$ на~текущем временном слое и, естественно, взять наибольшее возможное $\tau$. Таким образом получается следующая формула адаптивного шага по времени:
\begin{equation}
	\widetilde{\tau}_3^k = \cfrac{\tol_3}{m \cdot \max\limits_{j = 0}^M \left| F''(\phi_j^k) \right|}.
	\label{sch:time_step_stability_raw}
\end{equation}
Этот метод будем условно называть адаптацией по устойчивости.

Идея описанного подхода подразумевает, что для корректной работы схемы должно быть достаточно $\tol_3 = 1$, позволяя отказаться от подбора константы. При большей желаемой точности расчета можно провести подбор $\tol_3 < 1$.

Однако формула \eqref{sch:time_step_stability_raw} в чистом виде имеет критический недостаток \linebreak из-за вида функции $F''(\phi)$ и нуждается в доработке, которая будет проделана в следующем подразделе.


\subsection{Детали реализации метода}

Будем считать, что значения параметров модели относятся к случаю, представляющему наибольший практический интерес, -- случаю <<сильного напряжения>> (см. \cite{ponomarev_stability}). Функция $F(\phi)$, заданная формулой \eqref{eq:allen_cahn_potential}, строго возрастает, имея на интервале $(0, 1)$ положительную производную. Так~как на~$(0, 1)$ выполнено $f' > 0$, $\epsilon' < 0$, то $(\Gamma / l^2) \cdot |f'| < (\norm{\nabla \Phi}^2 / 2) \cdot |\epsilon'|$. Более того, вблизи точки $0$ и вовсе верно $\epsilon \approx \epsilon_0 / \delta$, $\delta \ll 1$. Исходя из этих соображений, можно полуэмпирически заключить, что поведение функции $F(\phi)$ определяется главным образом поведением функции~$\epsilon(\phi)$.

В работе \cite{ponomarev_stability} проводится анализ функции $\epsilon(\phi)$ вблизи точки $\phi = 0$. Далее будет приведено определенное обобщение предыдущих результатов, полезное для~глубокого понимания вопроса.

Приведем формулы для производных функций $f(\phi)$ и $\epsilon(\phi)$:
\begin{gather}
	f(\phi) = 4 \phi^3 - 3 \phi^4, \quad f'(\phi) = 12 \phi^2 - 12 \phi^3, \quad f''(\phi) = 24 \phi - 36 \phi^2,
	\nonumber \\
	\epsilon'(\phi) = \epsilon'_f \cdot f' = \cfrac{-\epsilon_0 f'(\phi)}{(f(\phi) + \delta)^2},
	\label{eq:epsilon_phi} \\
	\epsilon''(\phi) = \epsilon''_{ff} \cdot (f')^2 + \epsilon'_f \cdot f'' = \epsilon_0 \cfrac{2 (f'(\phi))^2 - f''(\phi)(f(\phi) + \delta)}{(f(\phi) + \delta)^3}.
	\label{eq:epsilon_phi_phi}
\end{gather}

Рассмотрим замену переменной $\phi = z\delta^{1/3}$, $\phi \in [0, 1]$, $z \in [0, \delta^{-1/3}]$. Домножим на $\delta$ функцию $\epsilon(\phi)$, определяемую выражением \eqref{eq:epsilon}; получим
\[
	\cfrac{\delta \cdot \epsilon(z \delta^{1/3})}{\epsilon_0} = \cfrac{1}{4 z^3 - 3 z^4 \delta^{1/3} + 1}.
\]
Таким образом, в результате двух преобразований получена функция, имеющая очевидный поточечный предел $1/(1 + 4 z^3)$ при $\delta \to +0$.

Производные $\delta \cdot \epsilon$ порядка $k$ по $z$ и по $\phi$ легко выражаются друг через друга:
\[
	\cfrac{d^k}{d z^k} \left[ \delta \cdot \epsilon(z \delta^{1/3}) \right] = \delta^{1 + k/3} \cdot \epsilon^{(k)}(z \delta^{1/3}),
\]
где $\epsilon^{(k)}$ обозначена $d^k \epsilon / d \phi^k$.

Очевидное утверждение о поточечной сходимости $\delta \cdot \epsilon(z \delta^{1/3}) / \epsilon_0$ может быть значительно усилено.

\begin{proposition}
	\label{prop:convergence_uniform}
	Пусть $C > 0$ произвольное. На отрезке $[0, C]_z$ для любого порядка $k$ имеет место равномерная сходимость производных
	\[
		\cfrac{d^k}{d z^k} \left[ \cfrac{\delta \cdot \epsilon(z \delta^{1/3})}{\epsilon_0} \right] \rightrightarrows \cfrac{d^k}{d z^k} \left[ \cfrac{1}{1 + 4 z^3} \right]
	\]
	при $\delta \to +0$.
\end{proposition}
\begin{proof}
	Обозначим исследуемую функцию $g(z) = \delta \cdot \epsilon(z \delta^{1/3}) / \epsilon_0$.

	Зафиксируем $\delta$. Представим $g(z)$ на отрезке $[0, \delta^{-1/3}]$ в виде функционального ряда следующим образом:
	\[
		g(z) = \cfrac{1}{1 + 4 z^3} \cdot \cfrac{1}{1 - \cfrac{3 z^4 \delta^{1/3}}{1 + 4 z^3}} = \cfrac{1}{1 + 4 z^3} \sum\limits_{n = 0}^\infty \left( \cfrac{3 z^4}{1 + 4 z^3} \right)^n \delta^{n / 3}.
	\]
	Ряд сходится равномерно по $z$. Действительно,
	\[
		\cfrac{3 z^4}{1 + 4 z^3} \delta^{1/3} = \cfrac{3 z^3}{1 + 4 z^3} \cdot z \delta^{1/3} = \cfrac{3}{4 + 1 / z^3} \cdot z \delta^{1/3} \leqslant \cfrac{3}{4} \cdot 1 = \cfrac{3}{4},
	\]
	то есть рассматриваемый ряд из положительных членов мажорируется геометрической прогрессией с основанием $3/4$.

	Обозначим
	\[
		%S_0(z) = \sum\limits_{n = 0}^\infty \left( \cfrac{3 z^4}{1 + 4 z^3} \right)^n \delta^{n / 3} \\
		%S_1(z) = \sum\limits_{n = 0}^\infty (n + 1) \left( \cfrac{3 z^4}{1 + 4 z^3} \right)^n \delta^{n / 3} \\
		%\dots \\
		S_j(z) = \sum\limits_{n = 0}^\infty \cfrac{(n + j)!}{n!} \left( \cfrac{3 z^4}{1 + 4 z^3} \right)^n \delta^{n / 3}.
	\]
	В этих обозначениях $g(z) = S_0(z) / (1 + 4 z^3)$.

	Все функциональные ряды $S_j$ также сходятся на $[0, \delta^{-1/3}]$ равномерно по~$z$, так как мажорируются сходящимися рядами
	\[
		\sum\limits_{n = 0}^\infty P_j(n) \cdot \left( \cfrac{3}{4} \right)^n,
	\]
	где $P_j(n)$ -- положительный при $n \in \Natural_0$ многочлен степени $j$.

	Рассмотрим ряд из производных членов ряда $S_0$:
	\begin{multline*}
		\sum\limits_{n = 0}^\infty \left[ \cfrac{1}{1 + 4 z^3} \cdot \left( \cfrac{3 z^4}{1 + 4 z^3} \right)^n \delta^{n / 3} \right]'_z = \\ = \left( \cfrac{1}{1 + 4 z^3} \right)'_z S_0(z) + \cfrac{1}{1 + 4 z^3} \left( \cfrac{3 z^4}{1 + 4 z^3} \right)'_z \delta^{1/3} \cdot S_1(z).
	\end{multline*}
	Указанный ряд выражается в виде суммы равномерно сходящихся $S_j$, домноженных на $\delta^{j / 3}$ и ограниченные функции, следовательно, сходится на $[0, \delta^{-1/3}]$ равномерно. Так как и сам $S_0$, и ряд из его производных равномерно сходятся, то $S_0$ можно продифференцировать почленно, то есть
	\[
		\cfrac{d S_0}{dz} = \left( \cfrac{1}{1 + 4 z^3} \right)'_z S_0(z) + \cfrac{1}{1 + 4 z^3} \left( \cfrac{3 z^4}{1 + 4 z^3} \right)'_z \delta^{1/3} \cdot S_1(z).
	\]
	Проводя аналогичное рассуждение для производных всех порядков $k$ ряда~$S_0$, по индукции заключаем, что $d^k S_0 / d z^k$ есть сумма $S_j \delta^{j / 3}$, домноженных на~ограниченные функции. То же верно и для $g(z)$, выражающейся через $S_0$.

	Теперь рассмотрим отрезок $[0, C]$ и $\delta \to +0$. Легко видеть, что на $[0, C]$ $S_j(z) \to (n + j)! / n!$ равномерно по $z$, в частности, $S_0(z) \to 1$. Тогда, очевидным образом, $g(z) \rightrightarrows 1 / (1 + 4 z^3)$. Помимо этого, так как любая производная $g(z)$ представляется в виде суммы $S_j \delta^{j / 3}$, домноженных на ограниченные функции, то верно
	\[
		\cfrac{d^k g(z)}{d z^k} \rightrightarrows \cfrac{d^k}{d z^k} \left[ \cfrac{1}{1 + 4 z^3} \right]
	\]
	на $[0, C]$ равномерно по $z$ при $\delta \to +0$.
\end{proof}

\begin{remark}
	В утверждении \ref{prop:convergence_uniform}, как можно заключить из его доказательства, равномерная по $z$ сходимость имеет асипмтотику $\bigO_{C, k}(\delta^{1/3})$.
\end{remark}

Утверждение \ref{prop:convergence_uniform} позволяет записать приближенные представления
\[
	\epsilon^{(k)}(\phi) \approx \epsilon_0 \delta^{-1 - k / 3} \cdot \cfrac{d^k}{d z^k} \left[ \cfrac{1}{1 + 4 z^3} \right] \bigg|_{z = \delta^{-1/3} \phi}.
\]
Доказанная для всех производных равномерная сходимость влечет для этих приближений в том числе сходимость нулей функций, интервалов знакопостоянства и экстремумов, а также интервалов выпуклости-вогнутости. Отсюда становится совершенно ясным описанное в работе \cite{ponomarev_stability} убывание корней $\epsilon''$ с~порядком $\delta^{1/3}$ и порядок $\delta^{-5/3}$ модулей экстремумов $\epsilon''$.

Функция $\epsilon(\phi)$ на отрезке $[0, 1]$ монотонно убывает; ее производная унимодальна: при увеличении $\phi$ сначала убывает от значения $0$, затем возрастает до значения $0$; вторая производная имеет три промежутка роста: убывает, затем возрастает, затем убывает. $\epsilon''$ вблизи $0$ меняется очень быстро, достигает больших по модулю значений и всецело определяет поведение $F''$. $\epsilon''(\phi)$ имеет ноль $\phi_0 \sim 0.5 \cdot \delta^{1/3}$, а также локальный минимум и локальный максимум в~точках
\[
	\phi_\mp \sim \cfrac{1}{\sqrt[3]{32 \mp 12 \sqrt{6}}} \cdot \delta^{1/3}
\]
соответственно.

По перечисленным выше причинам функция $F''(\phi)$ в формуле \eqref{sch:time_step_stability_raw} крайне неудобна: вблизи $0$ она достигает больших по модулю значений и к тому же имеет ноль, так что при взятии модуля в зоне больших значений возникает резкий <<провал>> до $0$. Чтобы решить проблему, мажорируем $|F''(\phi)|$ гладкой функцией, не имеющей такого недостатка.
\begin{multline*}
	\cfrac{\delta^{5/3} \epsilon''(\delta^{1/3} z)}{\epsilon_0} = \cfrac{24 (z^2 - \delta^{1/3} z^3) - (24 z - 36 \delta^{1/3} z^2)(4 z^3 - 3 \delta^{1/3} z^4 + 1)}{(4 z^3 - 3 \delta^{1/3} z^4 + 1)^3} = \\
	= \cfrac{12z (16 z^3 - 30 \delta^{1/3} z^4 + 15 \delta^{2/3} z^5 + 3z \delta^{1/3} - 2)}{(4 z^3 - 3 \delta^{1/3} z^4 + 1)^3}.
\end{multline*}
Поведение функции вблизи $0$ определяется слагаемым $-2 \cdot 12z$. Поменяем его знак и получим функцию
\begin{gather}
	\widetilde{G}(z) = \cfrac{12z (16 z^3 - 30 \delta^{1/3} z^4 + 15 \delta^{2/3} z^5 + 3z \delta^{1/3} + 2)}{(4 z^3 - 3 \delta^{1/3} z^4 + 1)^3};
	\nonumber \\
	\begin{aligned}
		G(\phi) & = \cfrac{\norm{\nabla \Phi}^2 \epsilon_0}{2} \delta^{-5/3} \widetilde{G}(\delta^{-1/3} \phi) = \\
		& = \norm{\nabla \Phi}^2 \epsilon_0 \cfrac{6 \phi (16 \phi^3 - 30 \phi^4 + 15 \phi^5 + 3 \delta \phi + 2 \delta)}{(4 \phi^3 - 3 \phi^4 + \delta)^3}.
		\label{eq:stability_majorant}
	\end{aligned}
\end{gather}
$G(\phi) \geqslant 0$ на $[0, 1]$. Имеем $G(x) = |G(x)| \geqslant |F''(\phi)|$.

Исправим формулу \eqref{sch:time_step_stability_raw} в методе адаптации по устойчивости, заменив $F(\phi)$ на мажорирующую ее $G(\phi)$, определяемую формулой \eqref{eq:stability_majorant}:
\begin{equation}
	\widetilde{\tau}_3^k = \cfrac{\tol_3}{m \cdot \max\limits_{j = 0}^M G(\phi_j^k)}.
	\label{sch:time_step_stability}
\end{equation}