%!TEX root = ../main.tex

\section{Математическая модель}

Приведем краткое описание исследуемой математической модели. По\forcehyphenation дробное описание физического смысла уравнений и параметров модели можно найти в работе \cite{ponomarev_stability}.

Рассматривается ограниченная область пространства $\Omega \subset \Real^3$. Распределение фаз вещества в ней задается гладкой функцией $\phi: \Omega \times [0, +\infty)_t \hm \to [0, 1], \; \phi(\vx, t)$~-- фазовым полем; вещество может находиться в одной из двух фаз: $\phi \approx 1$~-- <<неповрежденное>>, $\phi \approx 0$~-- <<полностью разрушенное>> (то есть относящееся к каналу пробоя),~-- а также в промежуточных состояниях в зоне диффузной границы.

Диэлектрическая проницаемость среды $\epsilon$ задается следующей формулой:
\[\epsilon(\vx, t) = \epsilon[\phi] = \cfrac{\epsilon_0(\vx)}{f(\phi(\vx, t)) + \delta}.\]
Здесь $\epsilon_0(\vx)$~-- диэлектрическая проницаемость неповрежденной среды, $f(\phi) \hm = 4\phi^3 - 3\phi^4$~-- интерполирующая функция, $0 < \delta \ll 1$~-- регуляризующий параметр.

Помимо фазового поля $\phi$, состояние системы описывает функция $\Phi: \Omega \hm \times [0, +\infty)_t \to \Real, \; \Phi(\vx, t)$~-- потенциал электрического поля.

Постулируется следующее выражение для свободной энергии системы $\Pi$:
\begin{gather*}
	\Pi = \int \limits_\Omega \pi d \vx, \\
	\pi = -\half \epsilon[\phi] \gradsq{\Phi} + \Gamma \cfrac{1 - f(\phi)}{l^2} + \cfrac{\Gamma}{4} \gradsq{\phi}.
\end{gather*}
Здесь $\Gamma > 0$, $l > 0$~-- числовые параметры модели, константы.

Постулируются два уравнения, определяющие динамику системы:
\begin{equation*}
\begin{cases}
	\cfrac{\delta \Pi}{\delta \Phi} = 0; \\[3mm]
	\cfrac{1}{m} \partt{\phi} = -\cfrac{\delta \Pi}{\delta \phi}.
\end{cases}
\end{equation*}
Здесь константа $m > 0$~-- числовой параметр модели, называемый подвижностью. Говоря нестрого, согласно первому уравнению электрический потенциал $\Phi$ распределяется так, чтобы свободная энергия была минимальной; согласно второму~-- фазовое поле $\phi$ с определенной скоростью стремится к тому, чтобы свободная энергия была минимальной.

Отыскав явно вариационные производные в двух уравнениях выше, получим следующую систему уравнений:
\begin{numcases}{}
	\Div(\epsilon[\phi] \nabla \Phi) = 0;
	\label{eq:Phi} \\
	\cfrac{1}{m} \partt{\phi} = \half \epsilon'(\phi) \gradsq{\Phi} + \cfrac{\Gamma}{l^2} f'(\phi) + \half \Gamma \Delta \phi.
	\label{eq:phi}
\end{numcases}
Здесь $(\cdot)' \equiv (\cdot)_\phi'$. Система состоит из двух уравнений: на $\phi$ и $\Phi$ соответственно; система связная.

Уравнение \eqref{eq:phi} имеет следующий вид:
\[
	\cfrac{1}{m} \partt{\phi} = -F'(\phi; |\nabla \Phi|) + \half \Gamma \Delta \phi,
\]
где $F$ -- определенная нелинейная функция от $\phi$, которая к тому же зависит от $|\nabla \Phi|$ как от параметра. Таким образом, перед нами уравнение типа Аллена--Кана, нелинейное. \absent{Ссылка}

В классической постановке Аллена--Кана $F$ -- двухъямный потенциал. В рассмотриваемой задаче $F$ меняет поведение в зависимости от величины $|\nabla \Phi|$, как было показано в работе \cite{ponomarev_stability}. Возможно три случая в зависимости от величины
\[
	\xi = \cfrac{|\nabla \Phi|^2 l^2 \epsilon_0}{2 \Gamma},
\]
а именно:
\begin{itemize}
	\item \makebox[5.3cm][l]{<<слабое напряжение>>,} \makebox[4.0cm][l]{$\xi < \delta^2$:} $F(\phi)$ монотонно убывает;
	\item \makebox[5.3cm][l]{<<среднее напряжение>>,} \makebox[4.0cm][l]{$\delta^2 < \xi < (1 + \delta)^2$:} $F(\phi)$ выпукла вверх;
	\item \makebox[5.3cm][l]{<<сильное напряжение>>,} \makebox[4.0cm][l]{$\xi > (1 + \delta)^2$:} $F(\phi)$ монотонно возрастает.
\end{itemize}
Наибольший интерес для практики моделирования представляет случай \linebreak <<сильного напряжения>>, так как именно тогда канал пробоя развивается из сколь угодно малых возмущений неповрежденной среды.