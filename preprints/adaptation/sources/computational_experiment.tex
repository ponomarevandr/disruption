%!TEX root = ../main.tex

\section{Вычислительный эксперимент}

\subsection{Параметры модели, краевые условия}

Была создана программа, реализующая в рамках разностной схе\forcehyphenation мы \eqref{sch:borders},~\eqref{sch:transition},~\eqref{sch:time_step_min_max} перечисленные ранее алгоритмы адаптации временного шага: по фазовому полю \eqref{sch:time_step_phi}, по энергии \eqref{sch:time_step_energy} и по устойчивости \eqref{sch:time_step_stability}.

Будем использовать параметры модели, отражающие реальный физический эксперимент: см. табл. \ref{tab:parameters}. Часть параметров ($| \nabla \Phi|$, $\epsilon_0$) являются полноценными физическими величинами, часть ($\Gamma$, $m$) могут быть подобраны для формального согласования модели с результатами эксперимента. Чертой отделены параметры, которые либо происходят из связанных с диффузной границей допущений ($l$, $\delta$), либо описывают расчетную сетку.

\begin{table}[!t]
\captionsetup{justification=raggedright,singlelinecheck=false}
\caption[]{Параметры модели в расчете}
\centering
\begin{tabular}{|l|c|l|}
	\hline
	Название & Переменная & Значение \\
	\hline
	электрическое напряжение		& $|\nabla \Phi|$	& $5.625 \cdot 10^6 \; \unitV / \unitm$							\\
	энергия роста канала			& $\Gamma$			& $8.118 \cdot 10^{-10} \; \unitJ / \unitm$						\\
	диэлектрическая проницаемость	& $\epsilon_0$		& $2.301 \cdot 10^{-11} \; \unitC^2 / (\unitJ \cdot \unitm)$	\\
	подвижность						& $m$				& $12 \; \unitm^3 / (\unitJ \cdot \units)$						\\
	\hline
	толщина границы					& $l$ 				& $1.5 \cdot 10^{-6} \; \unitm$									\\
	регуляризующий параметр 		& $\delta$			& $10^{-3}$														\\
	размер образца					& $W$				& $3.2 \cdot 10^{-5} \; \unitm$									\\
	продолжительность опыта			& $T$				& $2 \cdot 10^{-3} \; \units$									\\
	шаг по пространству				& $h$				& $5 \cdot 10^{-7} \; \unitm$									\\
	минимальный шаг по времени		& $\tau_{min}$		& $10^{-10} \; \units$											\\
	максимальный шаг по времени		& $\tau_{max}$		& $\leqslant 6.42 \cdot 10^{-6} \; \units$						\\
	\hline
\end{tabular}
\label{tab:parameters}
\end{table}

Заметим, что, таким образом, число узлов сетки по пространству $N_x = 64$, по времени -- $312 \leqslant N_t \leqslant 2 \cdot 10^7$.

Зададим следующие краевые условия:
\begin{gather*}
	\phi(0, t) = \phi(W, t) = 1, \qquad \phi(x, 0) = \phi_0(x), \\
	\phi_0(x) = \begin{cases}
		1 - 0.025 \cdot \left( 1 + \cos \left[ \cfrac{\pi}{0.08} \left( \cfrac{x}{W} - \half \right) \right] \right) \text{ при } \cfrac{x}{W} \in [0.42, 0.58]; \\
		1 \text{ иначе}.
	\end{cases}
\end{gather*}
Функция $\phi_0(x)$ отлична от $1$ в небольшой области вокруг $x = W / 2$, где <<прогибается>> как один период $\cos$, достигая минимума $\phi = 0.95$.