%!TEX root = ../main.tex

\section{Заключение}

В работе исследованы три различных метода адаптации расчетного шага по времени для модели развития канала электрического пробоя. Методы условно названы адаптацией по фазовому полю, по энергии и по устойчивости. Последний предложен авторами настоящей работы на основе условия устойчивости разностной схемы, полученного в одной из предыдущих работ авторов по теме.

По реузльтатам численного эксперимента лучше всех себя показывает первый метод, который к тому же выделяется своей простотой. При этом адаптация по устойчивости имеет некоторые преимущества, которые могут быть полезны при определенных требованиях к расчету.

Стоит отметить, что перечисленные методы адаптации шага по времени (особенно первые два) универсальны для моделей типа диффузной границы с уравнением Аллена--Кана и могут быть в дальнейшем применены при решении различных задач подобного класса.