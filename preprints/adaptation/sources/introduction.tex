%!TEX root = ../main.tex

\section{Введение}

Модели типа диффузной границы в настоящее время составляют целый класс подходов для решения прикладных задач гидродинамики \cite{lamorgese_flow_modeling, kim_fluid_flows, xu_hydrodynamics}, механики деформируемого тела и теории трещин \cite{ambati_fracture}, материаловедения \cite{provatas_materials}, солидификации и теории фазовых переходов \cite{boettinger_solidification, cartalade_phase_separation, gransaly_solidification}, описания кристаллических структур \cite{emmerich_crystal, asadi_crystal, provatas_crystal}. Предметом исследования авторов является модель подобного класса, предложенная в статье \cite{pitike_dielectric_breakdown}, описывающая развитие канала электрического пробоя в твердом диэлектрике. Ее подробное описание и анализ можно найти в работах \cite{zipunova_higher_codimension, zipunova_thermomechanical, ponomarev_stability}.

Вещество в моделируемой системе находится в нескольких различных состояниях -- фазах, -- причем вещество в одной и той же фазе образует некоторые однородные области. В соответствии с методом диффузной границы распределение фаз вещества описывается гладкой функцией $\phi(\vx, t)$, называемой фазовым полем. В областях однородности каждой из фаз функция $\phi$ близка к определенной константе; в переходной зоне (на <<диффузной границе>>) -- меняется пусть и быстро, но непрерывно. Характерная толщина граничной зоны определяется параметрами модели.

Исследуемая модель состоит из двух дифференциальных уравнений в частных производных; основной интерес представляет второе из них -- уравнение динамики фазового поля $\phi$ типа Аллена--Кана.

Для системы типично следующее поведение: развитие канала пробоя происходит стремительно, но ему предшествует долгий период крайне медленных изменений в системе. Такое различие временных масштабов событий вызывает проблемы при моделировании -- использование регулярной по времени расчетной сетки в методе конечных разностей видится нерациональным.

Цель настоящей работы -- исследовать различные подходы к адаптации расчетного шага по времени для описанной модели. К отбираемым подходам авторы предъявляли два основных требования: во-первых, подход должен быть не слишком сложен с точки зрения программной реализации, во-вторых, не требовать значительного объема дополнительных вычислений.

В результате было исследовано три различных подхода к адаптации: первый предложен в статье \cite{li_time_step}, второй -- в статье \cite{zhang_time_step}, третий -- авторами настоящей работы.

\absent{Что было проделано и где описано}