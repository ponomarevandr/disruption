%!TEX root = ../main.tex

\section{Адаптации по фазовому полю и по энергии}

\subsection{Формулировка методов}

Рассмотрим первые два подхода к адаптации шага по времени, предложенные в статьях \cite{li_time_step} и \cite{zhang_time_step} соответственно. Введем их вместе из-за определенной их общности:
\begin{align}
	\widetilde{\tau}_1^k & = \cfrac{tol_1}{\left\| \left[ \partt{\phi} \right]_h \right\|_C},
	\label{sch:time_step_phi} \\
	\widetilde{\tau}_2^k & = \cfrac{tol_2}{\left| \left[ \cfrac{d \Pi}{dt} \right]_h \right|}.
	\label{sch:time_step_energy}
\end{align}
Здесь $tol_1$ и $tol_2$ -- некоторые числовые константы, подбираемые на практике; символом $[\; \cdot \;]_h$ обозначены разностные производные.

В формуле \eqref{sch:time_step_phi} в качестве $[\partflt{\phi}]_h$ удобно использовать $[\partflt{\phi}]^{k + 1/2}$ из левой части разностного уравнения \eqref{eq:subtractive}. В этом случае
\[
	\left\| \left[ \partt{\phi} \right]_h^{k + 1/2} \right\|_C = \max\limits_{j = 0}^N \left| \left[ \partt{\phi} \right]_j^{k + 1/2} \right|.
\]
Если сделать этого не удается (например, из-за проблем с синхронизацией параллельных вычислений), то можно использовать $[\partflt{\phi}]^{k - 1/2}$, сохраненную с предыдущего шага.

В формуле \eqref{sch:time_step_energy} в знаменателе модуль производной полной энергии $\Pi(t)$. В силу вывода уравнений \eqref{eq:Phi},~\eqref{eq:phi} динамики системы, в адекватном расчете $[\partflt{\Pi}]_h$ либо отрицательна, либо крайне мала по модулю (сеточный эффект колебания системы вблизи минимума $\Pi$), поэтому, вообще говоря, вместо взятия модуля можно написать знак $-$.

Плотность энергии $\pi$ вычисляется из уравнения \eqref{eq:energy_density}, для чего необходима разностная производная $[\partflx{\phi}]_h$. Предлагается использовать следуюшие формулы:
\begin{gather*}
	\pi_j^k = F(\phi_j^k) + \cfrac{\Gamma}{4} \left( \left[ \partt{\phi} \right]_j^k \right)^2, \qquad \left[ \partt{\phi} \right]_j^k = \begin{cases}
		\cfrac{\phi_1^k - \phi_0^k}{\tau} & \text{при } j = 0; \\
		\cfrac{\phi_{j + 1}^k - \phi_{j - 1}^k}{2 \tau} & \text{при } j = \overline{1, N - 1}; \\
		\cfrac{\phi_N^k - \phi_{N - 1}^k}{\tau} & \text{при } j = N;
	\end{cases} \\
	\Pi^k = \cfrac{h \pi_0^k + h \pi_N^k}{2} + \sum\limits_{j = 1}^{N - 1} h \pi_j^k.
\end{gather*}
В формуле \eqref{sch:time_step_energy} будем использовать разностную производную энергии \linebreak с предыдущего шага
\[
	\left[ \cfrac{d \Pi}{dt} \right]_h^{k - 1/2} = \cfrac{\Pi^k - \Pi^{k - 1}}{\tau^{k - 1}}.
\]

Использование формулы \eqref{sch:time_step_phi} для расчета $\widetilde{\tau}^k$ будем условно называть адаптацией по фазовому полю, формулы \eqref{sch:time_step_energy} -- адаптацией по энергии.


\subsection{Связь с нормированием приращения фазового поля}

Общность описанных двух подходов и, возможно, ключ к их интуитивному пониманию заключается в следующем. Для адаптации по фазовому полю рассмотрим норму приращения $[d \phi]_h$:
\[
	\left\| [d \phi]_h^{k + 1/2} \right\|_C = \widetilde{\tau}_1^k \cdot \left\| \left[ \partt{\phi} \right]_h^{k + 1/2} \right\|_C = \cfrac{tol_1}{\left\| \left[ \partt{\phi} \right]_h^{k + 1/2} \right\|_C} \cdot \left\| \left[ \partt{\phi} \right]_h^{k + 1/2} \right\|_C = tol_1.
\]
Выходит нормирование приращения! (С оговоркой на ограничения $\tau_{min}$ и $\tau_{max}$.)

В случае адаптации по энергии можно провести похожее рассуждение. Из~вывода уравнения \eqref{eq:phi} верно следующее равенство:
\[
	\cfrac{d \Pi}{dt} = -\cfrac{1}{m} \int\limits_{\Omega} \left( \partt{\phi} \right)^2 d \vx,
\]
что для сеточных функций дает
\[
	\left| \left[ \cfrac{d \Pi}{dt} \right]_h \right| \approx \cfrac{1}{m} \left\| \left[ \partt{\phi} \right]_h \right\|_2^2.
\]
Таким образом, при адаптации по энергии
\[
	\left\| [d \phi]_h \right\|_2 = \widetilde{\tau}_2^k \cdot \left\| \left[ \partt{\phi} \right]_h^{k + 1/2} \right\|_2 \approx \cfrac{tol_2}{\cfrac{1}{m} \left\| \left[ \partt{\phi} \right]_h^{k - 1/2} \right\|_2^2} \cdot \left\| \left[ \partt{\phi} \right]_h^{k + 1/2} \right\|_2 \approx \cfrac{tol_2 \cdot m}{\left\| \left[ \partt{\phi} \right]_h \right\|_2}.
\]

Авторы не стали отклоняться от предложенного в работе \cite{zhang_time_step} метода адаптации, однако из проделанных рассуждений получается, что из модуля производной энергии в формуле \eqref{sch:time_step_energy} логичнее было бы извлечь квадратный корень, чтобы выполнялось $\left\| [d \phi]_h \right\|_2 \approx tol_2 \cdot \sqrt{m}$.