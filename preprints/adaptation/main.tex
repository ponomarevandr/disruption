\documentclass[a4paper,12pt]{article}

%%% Размер шрифта
\usepackage[14pt]{extsizes}

%%% Поля
\usepackage[
	left=2cm,
	right=2cm,
	top=2cm,
	bottom=3cm,
	bindingoffset=0cm
]{geometry}

%%% Работа с русским языком
\usepackage{cmap}						% поиск в PDF
\usepackage{mathtext}					% русские буквы в формулах
\usepackage[T2A]{fontenc}				% кодировка
\usepackage[utf8]{inputenc}				% кодировка исходного текста
\usepackage[english,russian]{babel}		% локализация и переносы
\usepackage{indentfirst}
\frenchspacing

%%% Дополнительная работа с математикой
\usepackage{amsmath,amsfonts,amssymb,amsthm,mathtools}  % AMS

%%% Текст в колонки
\usepackage{multicol}

%%% Списки
\usepackage{enumitem}
\setlist{nosep, leftmargin=*}
\renewcommand{\labelenumi}{\arabic*)}

%%% Системы уравнений
\usepackage{cases}

%%% Таблицы
\usepackage{array}

%%% Рисунки
\usepackage{graphicx}
\usepackage{float}

%%% Точка в подписях к рисункам
\usepackage[labelsep=period]{caption}

%%% Список литературы
\bibliographystyle{bibliography_style/gost-numeric.bbx}
\usepackage[
	natbib = true,
	style = gost-numeric,
	sorting = none,
	backend = biber,
	language = autobib,
	autolang = other
]{biblatex}
\addbibresource{references.bib}

%%% Исправление символа номера при использовании gost-numeric.bbx
\usepackage{textcomp}
\DefineBibliographyStrings{russian}{number={\textnumero}}

%%% Гиперссылки
\usepackage[pdftex,unicode]{hyperref}

%%% Перенос знаков в формулах (по Львовскому)
\newcommand*{\hm}[1]{#1\nobreak\discretionary{}{\hbox{$\mathsurround=0pt #1$}}{}}


%%% Свои команды

\newcommand*{\No}{\textnumero}

\newcommand{\vect}[1]{\boldsymbol{#1}}
\newcommand{\vx}{{\vect{x}}}
\newcommand{\vn}{{\vect{n}}}

\newcommand{\half}{\cfrac{1}{2}}

\newcommand{\partt}[1]{\cfrac{\partial #1}{\partial t}}
\newcommand{\partx}[1]{\cfrac{\partial #1}{\partial x}}
\newcommand{\partxx}[1]{\cfrac{\partial^2 #1}{\partial x^2}}
\newcommand{\partvn}[1]{\cfrac{\partial #1}{\partial \vn}}

\newcommand{\partflt}[1]{\partial #1 / \partial t}
\newcommand{\partflx}[1]{\partial #1 / \partial x}
\newcommand{\partflxx}[1]{\partial^2 #1 / \partial x^2}
\newcommand{\partflvn}[1]{\partial #1 / \partial \vn}

\newcommand{\difftau}[1]{\cfrac{{#1}_j^{k + 1} - {#1}_j^k}{\tau}}
\newcommand{\diffhh}[1]{\cfrac{{#1}_{j + 1}^k - 2 {#1}_j^k + {#1}_{j - 1}^k}{h^2}}

\newcommand{\Natural}{{\mathbb{N}}}
\newcommand{\Real}{{\mathbb{R}}}
\newcommand{\bigO}{{\mathcal{O}}}
\newcommand{\clOmega}{{\overline{\Omega}}}

\newcommand{\norm}[1]{\| \, #1 \, \|}
\newcommand{\enorm}{{\| \cdot \|}}

\newcommand{\forcehyphenation}{-\linebreak}

\newcommand{\multeqstart}{
	\begingroup
	\setlength{\abovedisplayshortskip}{\the\abovedisplayskip}
	\setlength{\belowdisplayshortskip}{\the\belowdisplayskip}
}
\newcommand{\multeqnext}{
	\vspace{-7mm}
}
\newcommand{\multeqfinish}{
	\endgroup
}


%%% Свои операторы
\DeclareMathOperator{\Div}{{div}}
\DeclareMathOperator{\Int}{{Int}}


%%% Оформление теорем

\theoremstyle{plain}
\newtheorem{theorem}{Теорема}
\newtheorem{proposition}{Утверждение}

\theoremstyle{remark}
\newtheorem{remark}{Замечание}


%%% Пояснение к меткам
% eq	-- equation
% cond	-- condition
% char	-- characteristic
% sch	-- scheme
% est	-- estimation
% exp	-- experiment
% fig	-- figure
% tab	-- table
% sec	-- section


%%% Описание препринта
\newcommand{\PreprintTitle}{%
	Адаптация шага по времени в модели типа диффузной границы, содержащей уравнение Аллена--Кана%
}
\newcommand{\PreprintTitleFormatted}{
	Адаптация шага по времени \\ в модели типа диффузной границы, \\ содержащей уравнение Аллена--Кана
}
\newcommand{\PreprintTitleEnglish}{%
	Time step adaptation in a diffuse interface model including a Allen--Cahn equation%
}
\newcommand{\PreprintAuthors}{%
	А.~С.~Пономарев, Е.~В.~Зипунова, Е.~Б.~Савенков%
}
\newcommand{\PreprintAuthorsEnglish}{%
	A.~S.~Ponomarev, E.~V.~Zipunova, E.~B.~Savenkov%
}


%%%%%%%%%%%%%%%%%%%%%%%%%%%%%%%%%%%%%%%%%%%%%%%%%%%%%%%%%%%%%%%%%%%%%%%%%%%%%%%%

\begin{document}

%%%%%%%%%%%%%%%%%%%%%%%%%%%%%%%%%%%%%%
\begin{titlepage}

\begin{center}
	РОССИЙСКАЯ АКАДЕМИЯ НАУК \\
	ОРДЕНА ЛЕНИНА \\
	ИНСТИТУТ ПРИКЛАДНОЙ МАТЕМАТИКИ \\
	имени М. В. КЕЛДЫША \\

	\vspace*{60mm}
	{
		\Large{\PreprintAuthors} \\
	}
	\vspace*{20mm}
	{
		\large \textbf{\PreprintTitleFormatted} \\
	}
	\vspace*{110mm}
	\Large{Москва, 2025}
	\vspace*{-50mm}
\end{center}

\end{titlepage}
%%%%%%%%%%%%%%%%%%%%%%%%%%%%%%%%%%%%%%%

\setcounter{page}{2}

\thispagestyle{empty}

\noindent \emph{\PreprintAuthors,} \PreprintTitle \\[3mm]
\textbf{Аннотация} \\
{
	\small
	В работе исследованы три различных подхода к адаптации шага по времени в модели развития канала электрического пробоя типа диффузной границы. Один из подходов предложен авторами настоящей работы, дано его теоретическое обоснование. Для всех трех алгоритмов адаптации проведены численные эксперименты; выявлен наиболее эффективный из них. \\
	Исследованные алгоритмы адаптации универсальны – они могут использоваться и в других моделях типа диффузной границы с уравнением Аллена–Кана. \\[3mm]
	\textbf{Ключевые слова:} модель типа диффузной границы, уравнение Аллена--Кана, адаптивная сетка \par
	\vspace{5mm}
}
\begin{otherlanguage}{english}
\noindent \emph{\PreprintAuthorsEnglish,} \PreprintTitleEnglish \\[3mm]
\textbf{Abstract} \\
{
	\small
	[...Annotaceaya rabotea na angliyskom...] \\[3mm]
	\textbf{Key words and phrases:} [Klucheveaye slova na angliyskom] \par
	\vspace{5mm}
}
\end{otherlanguage}

\clearpage

[...Текст...]

Ссылка: \cite{zipunova_higher_codimension}.

\clearpage
\printbibliography[
	heading=bibintoc
]

\clearpage
\tableofcontents

\end{document}

%%%%%%%%%%%%%%%%%%%%%%%%%%%%%%%%%%%%%%%%%%%%%%%%%%%%%%%%%%%%%%%%%%%%%%%%%%%%%%%%