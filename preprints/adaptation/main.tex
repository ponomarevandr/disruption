\documentclass[a4paper,12pt]{article}

%%% Размер шрифта
\usepackage[14pt]{extsizes}

%%% Поля
\usepackage[
	left=2cm,
	right=2cm,
	top=2cm,
	bottom=3cm,
	bindingoffset=0cm
]{geometry}

%%% Работа с русским языком
\usepackage{cmap}						% поиск в PDF
\usepackage{mathtext}					% русские буквы в формулах
\usepackage[T2A]{fontenc}				% кодировка
\usepackage[utf8]{inputenc}				% кодировка исходного текста
\usepackage[english,russian]{babel}		% локализация и переносы
\usepackage{indentfirst}
\frenchspacing

%%% Дополнительная работа с математикой
\usepackage{amsmath,amsfonts,amssymb,amsthm,mathtools}  % AMS

%%% Текст в колонки
\usepackage{multicol}

%%% Списки
\usepackage{enumitem}
\setlist{nosep, leftmargin=*}
\renewcommand{\labelenumi}{\arabic*)}

%%% Системы уравнений
\usepackage{cases}

%%% Таблицы
\usepackage{array}

%%% Рисунки
\usepackage{graphicx}
\usepackage{float}

%%% Точка в подписях к рисункам
\usepackage[labelsep=period]{caption}

%%% Список литературы
\bibliographystyle{bibliography_style/gost-numeric.bbx}
\usepackage[
	natbib = true,
	style = gost-numeric,
	sorting = none,
	backend = biber,
	language = autobib,
	autolang = other
]{biblatex}
\addbibresource{references.bib}

%%% Исправление символа номера при использовании gost-numeric.bbx
\usepackage{textcomp}
\DefineBibliographyStrings{russian}{number={\textnumero}}

%%% Гиперссылки
\usepackage[pdftex,unicode]{hyperref}

%%% Перенос знаков в формулах (по Львовскому)
\newcommand*{\hm}[1]{#1\nobreak\discretionary{}{\hbox{$\mathsurround=0pt #1$}}{}}


%%% Свои команды

\newcommand*{\No}{\textnumero}

\newcommand{\vect}[1]{\boldsymbol{#1}}
\newcommand{\vx}{{\vect{x}}}
\newcommand{\vn}{{\vect{n}}}

\newcommand{\half}{\cfrac{1}{2}}

\newcommand{\partt}[1]{\cfrac{\partial #1}{\partial t}}
\newcommand{\partx}[1]{\cfrac{\partial #1}{\partial x}}
\newcommand{\partxx}[1]{\cfrac{\partial^2 #1}{\partial x^2}}
\newcommand{\partvn}[1]{\cfrac{\partial #1}{\partial \vn}}

\newcommand{\partflt}[1]{\partial #1 / \partial t}
\newcommand{\partflx}[1]{\partial #1 / \partial x}
\newcommand{\partflxx}[1]{\partial^2 #1 / \partial x^2}
\newcommand{\partflvn}[1]{\partial #1 / \partial \vn}

\newcommand{\gradsq}[1]{(\nabla #1, \nabla #1)}

\newcommand{\difftau}[1]{\cfrac{{#1}_j^{k + 1} - {#1}_j^k}{\tau}}
\newcommand{\diffhh}[1]{\cfrac{{#1}_{j + 1}^k - 2 {#1}_j^k + {#1}_{j - 1}^k}{h^2}}

\newcommand{\Natural}{{\mathbb{N}}}
\newcommand{\Real}{{\mathbb{R}}}
\newcommand{\bigO}{{\mathcal{O}}}
\newcommand{\clOmega}{{\overline{\Omega}}}

\newcommand{\norm}[1]{\| \, #1 \, \|}
\newcommand{\enorm}{{\| \cdot \|}}

\newcommand{\unitm}{{\text{м}}}
\newcommand{\units}{{\text{с}}}
\newcommand{\unitJ}{{\text{Дж}}}
\newcommand{\unitC}{{\text{Кл}}}
\newcommand{\unitV}{{\text{В}}}
\newcommand{\unitF}{{\text{Ф}}}

\newcommand{\forcehyphenation}{-\linebreak}

\newcommand{\multeqstart}{
	\begingroup
	\setlength{\abovedisplayshortskip}{\the\abovedisplayskip}
	\setlength{\belowdisplayshortskip}{\the\belowdisplayskip}
}
\newcommand{\multeqnext}{
	\vspace{-7mm}
}
\newcommand{\multeqfinish}{
	\endgroup
}

\newcommand{\absent}[1]{[...#1...]}


%%% Свои операторы
\DeclareMathOperator{\Div}{{div}}
\DeclareMathOperator{\Int}{{Int}}


%%% Оформление теорем

\theoremstyle{plain}
\newtheorem{theorem}{Теорема}
\newtheorem{proposition}{Утверждение}

\theoremstyle{remark}
\newtheorem{remark}{Замечание}


%%% Пояснение к меткам
% eq	-- equation
% cond	-- condition
% char	-- characteristic
% sch	-- scheme
% est	-- estimation
% exp	-- experiment
% fig	-- figure
% tab	-- table
% sec	-- section


%%% Описание препринта
% Комментирование конца строки убирает паразитные пробелы
\newcommand{\PreprintTitle}{%
	Адаптация шага по времени в модели типа <<диффузной границы>>, содержащей уравнение Аллена--Кана%
}
\newcommand{\PreprintTitleFormatted}{%
	Адаптация шага по времени \\ в модели типа <<диффузной границы>>, \\ содержащей уравнение Аллена--Кана%
}
\newcommand{\PreprintTitleEnglish}{
	Time step adaptation in a diffuse interface model including an Allen--Cahn equation%
}
\newcommand{\PreprintAuthors}{%
	А.~С.~Пономарев, Е.~В.~Зипунова, Е.~Б.~Савенков%
}
\newcommand{\PreprintAuthorsEnglish}{%
	A.~S.~Ponomarev, E.~V.~Zipunova, E.~B.~Savenkov%
}


%%%%%%%%%%%%%%%%%%%%%%%%%%%%%%%%%%%%%%%%%%%%%%%%%%%%%%%%%%%%%%%%%%%%%%%%%%%%%%%%

\begin{document}

%%%%%%%%%%%%%%%%%%%%%%%%%%%%%%%%%%%%%%
\begin{titlepage}

\begin{center}
	РОССИЙСКАЯ АКАДЕМИЯ НАУК \\
	ОРДЕНА ЛЕНИНА \\
	ИНСТИТУТ ПРИКЛАДНОЙ МАТЕМАТИКИ \\
	имени М. В. КЕЛДЫША \par

	\vspace*{60mm}
	{
		\Large{\PreprintAuthors} \par
	}
	\vspace*{20mm}
	{
		\large \textbf{\PreprintTitleFormatted} \par
	}
	\vspace*{\fill}
	\Large{Москва, 2025}
	\vspace*{-15mm}
\end{center}

\end{titlepage}
%%%%%%%%%%%%%%%%%%%%%%%%%%%%%%%%%%%%%%%

\setcounter{page}{2}

\thispagestyle{empty}

\noindent \emph{\PreprintAuthors,} \PreprintTitle \\[3mm]
\textbf{Аннотация} \par
{
	\noindent \small
	В работе исследованы три различных подхода к адаптации шага по времени в модели развития канала электрического пробоя типа диффузной границы. Один из подходов предложен авторами настоящей работы, дано его теоретическое обоснование. Для всех трех алгоритмов адаптации проведены численные эксперименты; выявлен наиболее эффективный из них. \\
	Исследованные алгоритмы адаптации универсальны – они могут использоваться и в других моделях типа диффузной границы с уравнением Аллена–Кана. \\[3mm]
	\textbf{Ключевые слова:} модель типа диффузной границы, уравнение Аллена--Кана, адаптация шага по времени \par
	\vspace{5mm}
}
\begin{otherlanguage}{english}
\noindent \emph{\PreprintAuthorsEnglish,} \PreprintTitleEnglish \\[3mm]
\textbf{Abstract} \par
{
	\noindent \small
	\absent{Annotaceaya rabotea na angliyskom} \\[3mm]
	\textbf{Key words and phrases:} diffuse interface model, Allen--Cahn equation, adaptive time-step\-ping method \par
	\vspace{5mm}
}
\end{otherlanguage}

\clearpage

%!TEX root = ../main.tex

\section{Введение}

\begin{frame}{Физическое явление}
\begin{block}{Электрический пробой}
	Явление резкого возрастания тока в диэлектрике при приложении электрического напряжения
	выше критического.
\end{block}
\begin{itemize}
	\item Рассматриваем твердый диэлектрик
	\item Деградация диэлектрических свойств материала
	\item Процесс развивается в ограниченной зоне -- канале пробоя
	\item Сложная физическая природа
\end{itemize}
\end{frame}


\begin{frame}{Математическая модель}
\begin{block}{Модель типа диффузной границы}
	Вещество находится в разных фазах. Состояние вещества описывается гладкой функцией
	$\phi(\vx, t)$ -- фазовым полем.
\end{block}
\begin{itemize}
	\item $\phi = 1$ -- неповрежденная среда
	\item $\phi = 0$ -- полностью разрушенная среда
	\item Зона $\phi \in (0, 1)$ -- диффузная граница
	\item На разрушение среды тратится энергия
\end{itemize}
\begin{figure}
	\includegraphics[width=0.5\textwidth]{figures/diffuse_edge.jpg}
\end{figure}
\end{frame}


\begin{frame}{Математическая модель}
Модель, предложенная в работе \cite{pitike_dielectric_breakdown}:
\begin{itemize}
	\item $\pi = \textcolor{red}{-\half \epsilon[\phi] (\nabla \Phi, \nabla \Phi)} +
	\Gamma \left( \cfrac{1 - f(\phi)}{l^2} + \cfrac{1}{4} (\nabla \phi, \nabla \phi) \right)$
	-- плотность свободной энергии
	\item $\Gamma$ -- энегрия роста канала пробоя на единицу длины
	\item $l$ -- величина <<размытия>> канала
	\item $\epsilon(\vx, t)$ -- диэлектрическая проницаемость среды
	\item $f(\phi)$ -- интерполирующая функция
\end{itemize}
\end{frame}


\begin{frame}{Математическая модель}
\vspace{-0.2cm}
\begin{itemize}
	\item $\epsilon(\vx, t) = \cfrac{\epsilon_0(\vx)}{f(\phi(\vx, t)) +
	\delta}$ -- диэлектрическая проницаемость среды
	\item $f(\phi) = 4 \phi^3 - 3 \phi^4$ -- интерполирующая функция
\end{itemize}
\begin{columns}
\column{0.5\textwidth}
\begin{figure}
	\hspace*{1.4cm}
	\includegraphics[width=0.65\textwidth]{figures/f_form.png}
\end{figure}
\column{0.5\textwidth}
\begin{figure}
	\hspace*{-2cm}
	\includegraphics[width=0.60\textwidth]{figures/eps_form.png}
\end{figure}
\end{columns}
\end{frame}


\begin{frame}{Математическая модель}
\vspace{-0.5cm}
\begin{block}{Уравнения модели}
\begin{itemize}
	\item Уравнение электрического потенциала $\Phi$:
	\begin{equation}
		\Div(\epsilon[\phi] \nabla \Phi) = 0
		\label{equation_potential}
	\end{equation}
	\item Уравнение фазового поля $\phi$:
	\begin{equation}
		\cfrac{1}{m} \partt{\phi} = \half \epsilon'(\phi) \gradscalsq{\Phi} + \cfrac{\Gamma}{l^2} f'(\phi) + \half \Gamma \Delta \phi
		\label{equation_phase}
	\end{equation}
\end{itemize}
\end{block}
Свойства:
\begin{itemize}
	\item связанная система уравнений на $\phi$ и $\Phi$;
	\item уравнение для $\phi$ типа Аллена--Кана, нелинейное.
\end{itemize}
\end{frame}


\begin{frame}{Пример вычислительного эксперимента}
\begin{columns}
\column{0.32\textwidth}
\begin{figure}
	\includegraphics[width=\textwidth]{figures/model_example_1.png}
\end{figure}
\column{0.32\textwidth}
\begin{figure}
	\includegraphics[width=\textwidth]{figures/model_example_2.png}
\end{figure}
\column{0.32\textwidth}
\begin{figure}
	\includegraphics[width=\textwidth]{figures/model_example_3.png}
\end{figure}
\end{columns}
\begin{center}
	Расчет из работы \cite{zipunova_experiment}
\end{center}
\end{frame}


\begin{frame}{Цель работы}
\begin{block}{Цель работы}
	Исследовать качественные характеристики системы уравнений \eqref{equation_potential},
	\eqref{equation_phase} и выполнить ее численный анализ.
\end{block}
Для этого рассмотрим задачу в определенных краевых условиях, упрощающих ее, но позволяющих
установить интересующие свойства.
\end{frame}

%!TEX root = ../main.tex

\section{Математическая модель}

Приведем краткое описание исследуемой математической модели. По\forcehyphenation дробное описание физического смысла уравнений и параметров модели можно найти в работе \cite{ponomarev_stability}.

Рассматривается ограниченная область пространства $\Omega \subset \Real^3$. Распределение фаз вещества в ней задается гладкой функцией $\phi: \Omega \times [0, +\infty)_t \hm \to [0, 1], \; \phi(\vx, t)$~-- фазовым полем; вещество может находиться в одной из двух фаз: $\phi \approx 1$~-- <<неповрежденное>>, $\phi \approx 0$~-- <<полностью разрушенное>> (то есть относящееся к каналу пробоя),~-- а также в промежуточных состояниях в зоне диффузной границы.

Диэлектрическая проницаемость среды $\epsilon$ задается следующей формулой:
\[\epsilon(\vx, t) = \epsilon[\phi] = \cfrac{\epsilon_0(\vx)}{f(\phi(\vx, t)) + \delta}.\]
Здесь $\epsilon_0(\vx)$~-- диэлектрическая проницаемость неповрежденной среды, $f(\phi) \hm = 4\phi^3 - 3\phi^4$~-- интерполирующая функция, $0 < \delta \ll 1$~-- регуляризующий параметр.

Помимо фазового поля $\phi$, состояние системы описывает функция $\Phi: \Omega \hm \times [0, +\infty)_t \to \Real, \; \Phi(\vx, t)$~-- потенциал электрического поля.

Постулируется следующее выражение для свободной энергии системы $\Pi$:
\begin{gather*}
	\Pi = \int \limits_\Omega \pi d \vx, \\
	\pi = -\half \epsilon[\phi] \gradsq{\Phi} + \Gamma \cfrac{1 - f(\phi)}{l^2} + \cfrac{\Gamma}{4} \gradsq{\phi}.
\end{gather*}
Здесь $\Gamma > 0$, $l > 0$~-- числовые параметры модели, константы.

Постулируются два уравнения, определяющие динамику системы:
\begin{equation*}
\begin{cases}
	\cfrac{\delta \Pi}{\delta \Phi} = 0; \\[3mm]
	\cfrac{1}{m} \partt{\phi} = -\cfrac{\delta \Pi}{\delta \phi}.
\end{cases}
\end{equation*}
Здесь константа $m > 0$~-- числовой параметр модели, называемый подвижностью. Говоря нестрого, согласно первому уравнению электрический потенциал $\Phi$ распределяется так, чтобы свободная энергия была минимальной; согласно второму~-- фазовое поле $\phi$ с определенной скоростью стремится к тому, чтобы свободная энергия была минимальной.

Отыскав явно вариационные производные в двух уравнениях выше, получим следующую систему уравнений:
\begin{numcases}{}
	\Div(\epsilon[\phi] \nabla \Phi) = 0;
	\label{eq:Phi} \\
	\cfrac{1}{m} \partt{\phi} = \half \epsilon'(\phi) \gradsq{\Phi} + \cfrac{\Gamma}{l^2} f'(\phi) + \half \Gamma \Delta \phi.
	\label{eq:phi}
\end{numcases}
Здесь $(\cdot)' \equiv (\cdot)_\phi'$. Система состоит из двух уравнений: на $\phi$ и $\Phi$ соответственно; система связная.

Уравнение \eqref{eq:phi} имеет следующий вид:
\[
	\cfrac{1}{m} \partt{\phi} = -F'(\phi; |\nabla \Phi|) + \half \Gamma \Delta \phi,
\]
где $F$ -- определенная нелинейная функция от $\phi$, которая к тому же зависит от $|\nabla \Phi|$ как от параметра. Таким образом, перед нами уравнение типа Аллена--Кана, нелинейное. \absent{Ссылка}

В классической постановке Аллена--Кана $F$ -- двухъямный потенциал. В рассмотриваемой задаче $F$ меняет поведение в зависимости от величины $|\nabla \Phi|$, как было показано в работе \cite{ponomarev_stability}. Возможно три случая в зависимости от величины
\[
	\xi = \cfrac{|\nabla \Phi|^2 l^2 \epsilon_0}{2 \Gamma},
\]
а именно:
\begin{itemize}
	\item \makebox[5.3cm][l]{<<слабое напряжение>>,} \makebox[4.0cm][l]{$\xi < \delta^2$:} $F(\phi)$ монотонно убывает;
	\item \makebox[5.3cm][l]{<<среднее напряжение>>,} \makebox[4.0cm][l]{$\delta^2 < \xi < (1 + \delta)^2$:} $F(\phi)$ выпукла вверх;
	\item \makebox[5.3cm][l]{<<сильное напряжение>>,} \makebox[4.0cm][l]{$\xi > (1 + \delta)^2$:} $F(\phi)$ монотонно возрастает.
\end{itemize}
Наибольший интерес для практики моделирования представляет случай \linebreak <<сильного напряжения>>, так как именно тогда канал пробоя развивается из сколь угодно малых возмущений неповрежденной среды.

%!TEX root = ../main.tex

\section{The finite-difference scheme}
\label{sec:differential_scheme}

In this section, we present a finite-difference scheme for solving
the equation~\eqref{eq:one_dim} in the domain~$[0, W]_x \times [0,
+\infty)_t$. The equation is
subjected to initial conditions~\eqref{eq:one_dim_initial} and boundary conditions~\eqref{eq:one_dim_marginal}.

Consider a regular mesh with a time step~$\tau$ and
spatial step~$h$. Let~$W = Nh$ with $N$ being the number of
nodes. The nodes of the spatiotemporal grid are given by~$(jh, k \tau)$,
$j = \overline{0, N}$, $k \in \Natural_0$. Define by~$\phi_j^k$
the value of a mesh function~$\phi$ at the node~$(jh, k \tau)$.
Then the finite-difference approximations read
\begin{equation}
  \cfrac{1}{m} \difftau{\phi} = \half K_\phi^2 \epsilon'(\phi_j^k) + \cfrac{\Gamma}{l^2} f'(\phi_j^k) + \cfrac{\Gamma}{2} \diffhh{\phi} \tpoint
  \label{eq:subtractive}
\end{equation}
or, in the explicit form,
\begin{gather}
  \begin{aligned}
    \phi_j^{k + 1} = \phi_j^k + m \tau \left( \half K_\Phi^2 \epsilon'(\phi_j^k) + \cfrac{\Gamma}{l^2} f'(\phi_j^k) + \cfrac{\Gamma}{2} \diffhh{\phi} \right), \\ j = \overline{1, N - 1}, \quad k \in \Natural_0 \tsemicolon
  \end{aligned}
  \label{sch:transition} \\
  \phi_j^0 = \phi_0(jh); \quad \phi_0^k = \phi_l(k \tau); \quad \phi_N^k = \phi_r(k \tau) \tpoint
  \label{sch:borders}
\end{gather}

It is easy to see that the scheme has the first order of approximation in
time and the second order of approximation in spatial terms.

To study properties of the scheme~\eqref{sch:transition}, \eqref{sch:borders},
the linear theory can be used (see, e.g.,
\cite[Chapter~10]{bahvalov_computational_methods}
or~\cite[Chapter~IX]{kalitkin_computational_methods}).
The central result of the theory states, in a somewhat simplified
form, that if a finite-difference scheme is stable and approximates a
continuous problem then the solution of the finite-dimensional problem
converges to the solution of the continuous one with order
not lower then the order of approximation.

To apply this result for the nonlinear setting~\eqref{sch:transition}, \eqref{sch:borders},
we proceed as follows:
(i) linearize the equation~\eqref{eq:subtractive}
for a fixed~$\phi$ and then (ii) apply the spectral stability
argument~\cite{bahvalov_computational_methods} to the
derived linearized equation. As the stability criteria are
satisfied for the linearized equation, stability should be expected for the
complete, nonlinear, problem. In this case, convergence of the
approximate solution should be expected as well~--- since
the finite-difference problem is stable and approximates the continuous
one.
The results of such non-rigorous analysis will be further confirmed by
numerical computations in the fully nonlinear setting.


\subsection{Stability estimate}

In this section we derive a stability condition for the
finite-difference scheme~\eqref{sch:transition}, \eqref{sch:borders}
using the so-called principal of ``frozen coefficients''
(see, e.g.,~\cite{bahvalov_computational_methods}).
Let~$\phi_j^k$ and~$\phi_j^k + \delta_j^k$ be solutions of the
finite-difference equation~\eqref{eq:subtractive}.
Substitute~$\phi_j^k + \delta_j^k$ into~\eqref{eq:subtractive} to obtain:
\begin{multline*}
  \cfrac{1}{m} \cfrac{(\phi_j^{k + 1} + \delta_j^{k + 1}) - (\phi_j^k + \delta_j^k)}{\tau} = \half K_\Phi^2 [\epsilon'(\phi_j^k) + \epsilon''(\phi_j^k) \delta_j^k + o(\delta_j^k)] + \\ + \cfrac{\Gamma}{l^2} [f'(\phi_j^k) + f''(\phi_j^k) \delta_j^k + o(\delta_j^k)] + \cfrac{\Gamma}{2} \cfrac{(\phi_{j + 1}^k + \delta_{j + 1}^k) - 2 (\phi_j^k + \delta_j^k) + (\phi_{j - 1}^k + \delta_{j - 1}^k)}{h^2} \tpoint
\end{multline*}
Linearizing this equation around~ $\phi_j^k = P$, assuming that
perturbations~$\delta_j^k$ are small and taking into account
that~$\phi_j^k$ is a solution of the finite-difference problem, we obtain:
\begin{equation}
  \delta_j^{k + 1} = \delta_j^k + m \tau \left( \half K_\Phi^2 \epsilon''(P) \delta_j^k + \cfrac{\Gamma}{l^2} f''(P) \delta_j^k + \cfrac{\Gamma}{2} \diffhh{\delta} \right) \tpoint
  \label{eq:scheme_variation}
\end{equation} 

We now apply spectral stability analysis to the derived equation for
perturbations.
Let~$\delta_j^k = \lambda(\theta)^k \cdot \exp(\imath j \theta)$, $\imath^2 = -1$.
Substituting this representation into~\eqref{eq:scheme_variation} one obtains:
$$\lambda(\theta) = 1 + m \tau \left( \half K_\Phi^2 \epsilon''(P) + \cfrac{\Gamma}{l^2} f''(P) + \cfrac{\Gamma}{2} \cfrac{\exp(\imath \theta) - 2 + \exp(-\imath \theta)}{h^2} \right) \tcomma$$
or
\begin{equation}
  \lambda(\theta) = 1 + m \tau \left( \half K_\Phi^2 \epsilon''(P) + \cfrac{\Gamma}{l^2} f''(P) - \cfrac{2 \Gamma}{h^2} \sin^2 \cfrac{\theta}{2} \right) \tpoint
  \label{eq:spectral}
\end{equation}

According to the spectral stability argument, a time
step~$\tau = \tau(h)$ provides stability of the scheme in the
domain~$[0, W]_x \times [0, T]_t$ with~$T<+\infty$ as~$\tau, h \to 0$ if there
exists~$C > 0$ such that for an arbitrary~$\theta$ it
holds~$|\lambda(\theta)| \leqslant \exp(C\tau)$. Note that here it is
also possible to use more strict condition~$|\lambda(\theta)| \leqslant 1 + C\tau$.
If for an arbitrary~$\theta$ it holds~$|\lambda(\theta)| \leqslant 1$,
then stability will be provided also for an unbounded time interval, i.e.,
for~$[0, W]_x \times [0, +\infty)_t$.
Strictly speaking, the spectral argument does not provide a sufficient
stability condition; however, stability should be expected in practice.

First, consider the expression~\eqref{eq:spectral} for~$P=0$.
We have~$f''(0) = 0$, $\epsilon''(0) = 0$, and the equation~\eqref{eq:spectral}
takes the form of
$$\lambda(\theta) = 1 - \cfrac{2 \tau m \Gamma}{h^2} \sin^2 \cfrac{\theta}{2} \tpoint$$
Hence, for an arbitrary~$\theta$ it holds~$|\lambda(\theta)| \leqslant 1$
if and only if
\begin{equation}
  \tau \leqslant \cfrac{h^2}{m \Gamma} \tpoint
  \label{cond:spectral_0}
\end{equation}
As the condition~\eqref{cond:spectral_0} is satisfied, one can expect stability of the scheme
when the solution describes an almost completely damaged state~$\phi\approx0$
in the domain~$[0, W]_x \times [0, +\infty)_t$.

Note that under the condition~\eqref{cond:spectral_0} one also can expect stable computations
for~$[0, W]_x \times [0, T]_t$ for an arbitrary value~$P \in [0, 1]$.
In this case the following is true:
$$
|\lambda(\theta)| \leqslant \left| 1 - \cfrac{2 \tau m \Gamma}{h^2} \sin^2 \cfrac{\theta}{2} \right| + m \tau \left| \half K_\Phi^2 \epsilon''(P) + \cfrac{\Gamma}{l^2} f''(P) \right| \leqslant 1 + m \tau \left| \half K_\Phi^2 \epsilon''(P) + \cfrac{\Gamma}{l^2} f''(P) \right| \tpoint
$$
Hence, there exists~$C$ such that
$|\lambda(\theta)| \leqslant 1 + C \tau$ holds,~--- since~$\epsilon''(\phi)$ and~$f''(\phi)$
are continuous on~$[0, 1]$.
It should be noted that, despite such versatility, the estimate~\eqref{cond:spectral_0}
is poorly applicable in practice and requires clarification, which will be done later.

We now consider the expression~\eqref{eq:spectral} at the value~$P=1$.
Note that~$f''(1) < 0$, $\epsilon''(1) > 0$.
We see that for~$(K_\Phi^2 / 2) \epsilon''(1) + (\Gamma / l^2) f''(1) \leqslant 0$
it is possible to achieve~$|\lambda(\theta)| \leqslant 1$ with demanded sufficiently small
values of~$\tau$ and the condition $\tau \leqslant h^2 / (2m \Gamma)$,
similar to the one for~\eqref{cond:spectral_0}.
Substituting~$f''(1) = -12, \; \epsilon''(1) = 12 \epsilon_0 / (1 + \delta)^2$
(see~\eqref{eq:epsilon_derivatives}),
we obtain
\begin{equation}
  \cfrac{K_\Phi^2 l^2 \epsilon_0}{2 \Gamma (1 + \delta)^2} \leqslant 1 \tpoint
  \label{cond:spectral_possible_1}
\end{equation}

So, under the condition~\eqref{cond:spectral_possible_1}, it is expected that
there exist such values of~$\tau$ и $h$
that the difference scheme is stable for~$\phi \approx 1$
and~$T=+\infty$.
Naturally the condition~\eqref{cond:spectral_possible_1}
is equivalent to the stability condition~\eqref{cond:equilibrium_1_stable}
for the equilibrium state~$\phi \equiv 1$ of the equation~\eqref{eq:one_dim}.

\subsection{Improved stability estimate}

In the previous section form the analysis of equation~\eqref{eq:spectral}
it was derived stability condition~\eqref{cond:spectral_0}
for finite-difference scheme~\eqref{sch:transition} and~\eqref{sch:borders} for~$\phi \approx 0$.
The assumption of its usefulness is based on the fact that typical ``'natural'' solution of the model
will has a form of the transition process from the undamaged state~$\phi=1$ to the completely
damaged state~$\phi=0$ occurring in a finite time interval and then infinitely long staying in the
damaged state~$\phi \approx 0$.


However the performed analysis of the equation~\eqref{eq:spectral}
is not sufficient at~$\phi = 0$. Indeed, it was used that at~$\phi=0$,
$\epsilon''(0) = 0$ (see expression~\eqref{eq:epsilon_derivatives}),~---
but it was not accounted that~$\epsilon''(\phi)$ growth fast and reaches
large values  for small values of~$\delta\approx 0$,
see Fig.~\ref{fig:eps_phi_phi}.
This means that the equations of the model are stable at~$\phi=0$,
but can be unstable in the small neighbourhood of~$\phi=0$.
Such situation is not satisfactory and we now try to improve
the obtained stability estimates.
%
\begin{figure}[!t]
	\centering
	\includegraphics[width=\textwidth]{figures/eps_phi_phi.png}
	\vspace{-0.7cm}
	\caption{Typical behavior of~$\epsilon''(\phi)$ in the vicinity of~$0$.}
	\label{fig:eps_phi_phi}
\end{figure}

To proceed let us estimate extremums of~$\epsilon''(\phi)$ in the neighbourhood of~$0$.
First, find zeros of~$\epsilon'''(\phi)$. We have
\begin{equation}
	\epsilon''' = \epsilon_0 \cfrac{-6 (f')^3 + 6 (f + \delta) f' f'' - (f + \delta)^2 f'''}{(f + \delta)^4},
	\label{eq:epsilon_phi_phi_phi}
\end{equation}
form where:
$$\epsilon'''(\phi) = -6 (f')^3 + 6 (f + \delta) f' f'' - (f + \delta)^2 f''' = 0 \tcomma$$
or, taking~\eqref{eq:epsilon} into account:
$$-3 \cdot 12^2 (1 - \phi)^3 + 36 \left(4 - 3\phi + \cfrac{\delta}{\phi^3} \right)(1 - \phi)(2 - 3\phi) - \left(4 - 3 \phi + \cfrac{\delta}{\phi^3} \right)^2 (1 - 3 \phi) = 0 \tpoint$$

Let~$\delta_n \to +0$ and~$\phi_n \to +0$ such that~$\delta_n / \phi_n^3$ is bounded.
Then:
\begin{gather*}
	-3 \cdot 12^2 \cdot 1^3 + 36 \left(4 + \cfrac{\delta_n}{\phi_n^3} \right) \cdot 1 \cdot 2 - \left(4 + \cfrac{\delta_n}{\phi_n^3} \right)^2 \cdot 1 \to 0 \tcomma \\
	\left(4 + \cfrac{\delta_n}{\phi_n^3} \right)^2 - 72 \left(4 + \cfrac{\delta_n}{\phi_n^3} \right) + 3 \cdot 12^2 \to 0 \tpoint
\end{gather*}
Hence, a sequence~$4 + \delta_n / \phi_n^3$ has not more than two partial limits~$\xi_+$ and~$\xi_-$~---
which are zeros of the equation~$\xi^2 - 72 \xi + 432 = 0$.
To the first zero~$\xi_+ = 36 + 12 \sqrt{6}$ it corresponds
$$\phi_+ = \cfrac{1}{\sqrt[3]{32 + 12 \sqrt{6}}} \sqrt[3]{\delta_n} \approx \cfrac{1}{3.945} \sqrt[3]{\delta_n} \tsemicolon$$
to the second zero~$\xi_- = 36 - 12 \sqrt{6}$ it corresponds
$$\phi_- = \cfrac{1}{\sqrt[3]{32 - 12 \sqrt{6}}} \sqrt[3]{\delta_n} \approx \cfrac{1}{1.376} \sqrt[3]{\delta_n} \tpoint$$

From here it can be seen that for~$\delta \to +0$ the function~$\epsilon'''(\phi)$ has two zeros in the neighbourhood of~$0$:
\begin{equation}
  \phi_{\pm} = \cfrac{1}{\sqrt[3]{32 \pm 12 \sqrt{6}}} \sqrt[3]{\delta} [1 + o(1)] \tpoint
  \label{eq:epsilon_phi_phi_phi_roots}
\end{equation}

We now estimate~$\epsilon''(\phi)$ at~$\phi_{\pm}$  for~$\delta \to +0$. Let~$\phi = (1 / c) \sqrt[3]{\delta}$, $c \in \Real$.
Then:
$$\epsilon'' = \epsilon_0 \cfrac{24 c^5 (8 - c^3)}{(4 + c^3)^3} \delta^{-5 / 3} [1 + o(1)],$$
and:
\begin{equation}
  \epsilon''(\phi_+) \approx -4.378 \epsilon_0 \delta^{-5 / 3}; \quad \epsilon''(\phi_-) \approx 2.216 \epsilon_0 \delta^{-5 / 3} \tpoint
  \label{est:epsilon_phi_phi_bounds}
\end{equation}
The derived estimates are shown as black dashed lines on Fig.~\ref{fig:eps_phi_phi_multiplied}.

\begin{figure}[!t]
	\centering
	\includegraphics[width=\textwidth]{figures/eps_phi_phi_multiplied.png}
	\caption{Qualitative behavior of~$\delta^{5 / 3} \epsilon''(\phi)$ for small values of~$\delta$.}
	\label{fig:eps_phi_phi_multiplied}
\end{figure}

Now, to derive new stability estimate we consider equation~\eqref{eq:spectral} at~$\phi = \phi_+$.
Note that~$\epsilon''(\phi_+) \approx -4.4 \epsilon_0 \delta^{-5 / 3}$.
The term inside braces in~\eqref{eq:spectral} is negative since
$\delta$ is small and~$\epsilon''(\phi_+)$ is negative and large in its absolute value.
Therefore~$f''(\phi_+) > 0$ can be estimated as~$0$~--- such estimate makes inequality stronger.
Then from inequality~\eqref{eq:spectral} it follows that 
$$\lambda(\theta) = 1 + m \tau \left( -\cfrac{2.2 K_\Phi^2 \epsilon_0}{\delta^{5 / 3}} - \cfrac{2 \Gamma}{h^2} \sin^2 \cfrac{\theta}{2} \right) \tpoint$$
Condition~$|\lambda(\theta)| \leqslant 1$ is satisfied for an arbitrary~$\theta$, if and only if
\begin{equation}
  \tau \leqslant \cfrac{1}{m} \left( \cfrac{1.1 K_\Phi^2 \epsilon_0}{\delta^{5 / 3}} + \cfrac{\Gamma}{h^2} \right)^{-1} \tpoint
  \label{cond:spectral_better_theoretical}
\end{equation}

Numerical experiments described in the next sections indicates that
more strong version of the estimate~\eqref{cond:spectral_better_theoretical} is also valid
(note the doubled denominator):
%
\begin{equation}
  \tau \leqslant \cfrac{1}{2m} \left( \cfrac{K_\Phi^2 \epsilon_0}{\delta^{5 / 3}} + \cfrac{\Gamma}{h^2} \right)^{-1} \tpoint
  \label{cond:spectral_better}
\end{equation}

Finally, more simple estimate not weaker then~\eqref{cond:spectral_better} is:
\begin{equation}
  \tau \leqslant \cfrac{1}{4m} \min \left(\cfrac{\delta^{5 / 3}}{K_\Phi^2 \epsilon_0}, \; \cfrac{h^2}{\Gamma} \right) \tpoint
  \label{cond:spectral_better_simpler}
\end{equation}

Note that the derived stability estimate~\eqref{cond:spectral_better}
for finite-difference scheme~\eqref{sch:transition},\eqref{sch:borders}
includes all the parameters of the equation~\eqref{eq:one_dim}, except~$l$.
Notably, this is the only parameter of the model which has somehow artificial nature and can not be
related directly to the underlying physics.

\endinput
% EOF

%!TEX root = ../main.tex

\section{Адаптации по фазовому полю и по энергии}

\subsection{Формулировка методов}

Рассмотрим первые два подхода к адаптации шага по времени, предложенные в статьях \cite{li_time_step} и \cite{zhang_time_step} соответственно. Введем их вместе из-за определенной их общности:
\begin{align}
	\widetilde{\tau}_1^k & = \cfrac{tol_1}{\left\| \left[ \partt{\phi} \right]_h \right\|_C},
	\label{sch:time_step_phi} \\
	\widetilde{\tau}_2^k & = \cfrac{tol_2}{\left| \left[ \cfrac{d \Pi}{dt} \right]_h \right|}.
	\label{sch:time_step_energy}
\end{align}
Здесь $tol_1$ и $tol_2$ -- некоторые числовые константы, подбираемые на практике; символом $[\; \cdot \;]_h$ обозначены разностные производные.

В формуле \eqref{sch:time_step_phi} в качестве $[\partflt{\phi}]_h$ удобно использовать $[\partflt{\phi}]^{k + 1/2}$ из левой части разностного уравнения \eqref{eq:subtractive}. В этом случае
\[
	\left\| \left[ \partt{\phi} \right]_h^{k + 1/2} \right\|_C = \max\limits_{j = 0}^N \left| \left[ \partt{\phi} \right]_j^{k + 1/2} \right|.
\]
Если сделать этого не удается (например, из-за проблем с синхронизацией параллельных вычислений), то можно использовать $[\partflt{\phi}]^{k - 1/2}$, сохраненную с предыдущего шага.

В формуле \eqref{sch:time_step_energy} в знаменателе модуль производной полной энергии $\Pi(t)$. В силу вывода уравнений \eqref{eq:Phi},~\eqref{eq:phi} динамики системы, в адекватном расчете $[\partflt{\Pi}]_h$ либо отрицательна, либо крайне мала по модулю (сеточный эффект колебания системы вблизи минимума $\Pi$), поэтому, вообще говоря, вместо взятия модуля можно написать знак $-$.

Плотность энергии $\pi$ вычисляется из уравнения \eqref{eq:energy_density}, для чего необходима разностная производная $[\partflx{\phi}]_h$. Предлагается использовать следуюшие формулы:
\begin{gather*}
	\pi_j^k = F(\phi_j^k) + \cfrac{\Gamma}{4} \left( \left[ \partt{\phi} \right]_j^k \right)^2, \qquad \left[ \partt{\phi} \right]_j^k = \begin{cases}
		\cfrac{\phi_1^k - \phi_0^k}{\tau} & \text{при } j = 0; \\
		\cfrac{\phi_{j + 1}^k - \phi_{j - 1}^k}{2 \tau} & \text{при } j = \overline{1, N - 1}; \\
		\cfrac{\phi_N^k - \phi_{N - 1}^k}{\tau} & \text{при } j = N;
	\end{cases} \\
	\Pi^k = \cfrac{h \pi_0^k + h \pi_N^k}{2} + \sum\limits_{j = 1}^{N - 1} h \pi_j^k.
\end{gather*}
В формуле \eqref{sch:time_step_energy} будем использовать разностную производную энергии \linebreak с предыдущего шага
\[
	\left[ \cfrac{d \Pi}{dt} \right]_h^{k - 1/2} = \cfrac{\Pi^k - \Pi^{k - 1}}{\tau^{k - 1}}.
\]

Использование формулы \eqref{sch:time_step_phi} для расчета $\widetilde{\tau}^k$ будем условно называть адаптацией по фазовому полю, формулы \eqref{sch:time_step_energy} -- адаптацией по энергии.


\subsection{Связь с нормированием приращения фазового поля}

Общность описанных двух подходов и, возможно, ключ к их интуитивному пониманию заключается в следующем. Для адаптации по фазовому полю рассмотрим норму приращения $[d \phi]_h$:
\[
	\left\| [d \phi]_h^{k + 1/2} \right\|_C = \widetilde{\tau}_1^k \cdot \left\| \left[ \partt{\phi} \right]_h^{k + 1/2} \right\|_C = \cfrac{tol_1}{\left\| \left[ \partt{\phi} \right]_h^{k + 1/2} \right\|_C} \cdot \left\| \left[ \partt{\phi} \right]_h^{k + 1/2} \right\|_C = tol_1.
\]
Выходит нормирование приращения! (С оговоркой на ограничения $\tau_{min}$ и $\tau_{max}$.)

В случае адаптации по энергии можно провести похожее рассуждение. Из~вывода уравнения \eqref{eq:phi} верно следующее равенство:
\[
	\cfrac{d \Pi}{dt} = -\cfrac{1}{m} \int\limits_{\Omega} \left( \partt{\phi} \right)^2 d \vx,
\]
что для сеточных функций дает
\[
	\left| \left[ \cfrac{d \Pi}{dt} \right]_h \right| \approx \cfrac{1}{m} \left\| \left[ \partt{\phi} \right]_h \right\|_2^2.
\]
Таким образом, при адаптации по энергии
\[
	\left\| [d \phi]_h \right\|_2 = \widetilde{\tau}_2^k \cdot \left\| \left[ \partt{\phi} \right]_h^{k + 1/2} \right\|_2 \approx \cfrac{tol_2}{\cfrac{1}{m} \left\| \left[ \partt{\phi} \right]_h^{k - 1/2} \right\|_2^2} \cdot \left\| \left[ \partt{\phi} \right]_h^{k + 1/2} \right\|_2 \approx \cfrac{tol_2 \cdot m}{\left\| \left[ \partt{\phi} \right]_h \right\|_2}.
\]

Авторы не стали отклоняться от предложенного в работе \cite{zhang_time_step} метода адаптации, однако из проделанных рассуждений получается, что из модуля производной энергии в формуле \eqref{sch:time_step_energy} логичнее было бы извлечь квадратный корень, чтобы выполнялось $\left\| [d \phi]_h \right\|_2 \approx tol_2 \cdot \sqrt{m}$.

%!TEX root = ../main.tex

\section{Адаптация по устойчивости}

\subsection{Идея подхода}

В работе \cite{ponomarev_stability} получено следующее условие устойчивости разностной схемы~\eqref{sch:transition_old},~\eqref{sch:borders}:
\[
	\tau \leqslant \cfrac{1}{4m} \min \left( \cfrac{\delta^{5/3}}{|\nabla \Phi|^2 \epsilon_0}, \cfrac{h^2}{\Gamma} \right).
\]
Неравенство с первым аргументом минимума эквивалентно соотношению
\begin{equation}
	m \tau \max\limits_{\phi \in [0, 1]} |F''(\phi)| \leqslant C,
	\label{cond:stability_first}
\end{equation}
где $1 \geqslant C \approx 1.1 / 2$ -- константа, выбранная, во-первых, для создания <<запаса>> в оценке, во-вторых, для удобства формульной записи. Неравенство это, в свою очередь, получено применением для схемы спектрального признака устойчивости. Строго говоря, спектральный признак не дает достаточных условий устойчивости для нелинейных задач, однако на практике ее следует ожидать.

Основная идея подхода к адаптации, предлагаемого авторами в этом разделе, заключается в том, чтобы в неравенстве \eqref{cond:stability_first} заменить формальный максимум по $\phi \in [0, 1]$ на максимум по значениям сеточной функции $\phi_j^k$ и, естественно, взять наибольшее возможное $\tau$. Таким образом получается следующая формула адаптивного шага по времени:
\begin{equation}
	\widetilde{\tau}_3^k = \cfrac{tol_3}{m \cdot \max\limits_{j = 0}^N |F''(\phi_j^k)|}.
	\label{sch:time_step_stability_raw}
\end{equation}
Этот метод будем условно называть методом адаптации по устойчивости.

Идея описанного подхода подразумевает, что для корректной работы схемы должно быть достаточно $tol_3 = 1$, позволяя отказаться от подбора значения. При большей желаемой точности расчета можно провести подбор $tol_3 < 1$.

Однако формула \eqref{sch:time_step_stability_raw} в чистом виде имеет критический недостаток из-за вида функции $F''(\phi)$ и нуждается в доработке, которая будет проделана в следующем подразделе.


\subsection{Доработка метода}

Будем считать, что конфигурация модели относится к случаю, представляющему наибольший практический интерес, -- случаю <<сильного напряжения>> (см. \cite{ponomarev_stability}). Функция $F(\phi)$, заданная формулой \eqref{eq:allen_cahn_potential}, имеет на интервале $(0, 1)$ положительную производную, а значит, строго возрастает. Так как на $(0, 1)$ выполнено $f' > 0$, $\epsilon' < 0$, то $|f'| < |\epsilon'|$. Более того, вблизи точки $0$ верно $\epsilon \approx \epsilon_0 / \delta$, $\delta \ll 1$, то есть $\epsilon(\phi)$ вместе со своими производными много больше $f(\phi)$ с ее производными. Исходя из этого, поведение функции $F(\phi)$ определяется главным образом поведением функции $\epsilon(\phi)$.

В работе \cite{ponomarev_stability} проводится анализ функции $\epsilon(\phi)$ вблизи точки $\phi = 0$. Далее будет приведено определенное обобщение старых результатов, не слишком сложное, но полезное для глубокого понимания вопроса.

Приведем формулы для производных функций $f(\phi)$ и $\epsilon(\phi)$:
\[
	f(\phi) = 4 \phi^3 - 3 \phi^4, \quad f'(\phi) = 12 \phi^2 - 12 \phi^3, \quad f''(\phi) = 24 \phi - 36 \phi^2,
\]
откуда
\begin{gather}
	\epsilon'(\phi) = \epsilon'_f \cdot f' = \cfrac{-\epsilon_0 f'(\phi)}{(f(\phi) + \delta)^2},
	\label{eq:epsilon_phi} \\
	\epsilon''(\phi) = \epsilon''_{ff} \cdot (f')^2 + \epsilon'_f \cdot f'' = \epsilon_0 \cfrac{2 (f'(\phi))^2 - f''(\phi)(f(\phi) + \delta)}{(f(\phi) + \delta)^3}.
	\label{eq:epsilon_phi_phi}
\end{gather}

Рассмотрим замену переменной $\phi = \delta^{1/3} z$, $z \in [0, \delta^{-1/3}]$.

\begin{proposition}
	\label{prop:convergence}
	\begin{gather*}
		\cfrac{\delta \epsilon(\delta^{1/3} z)}{\epsilon_0} \to \cfrac{1}{4 z^3 + 1} = g(z), \qquad \delta^{4/3} \cfrac{\epsilon'(\delta^{1/3} z)}{\epsilon_0} \to \cfrac{-12 z^2}{(4 z^3 + 1)^2} = g'(z), \\
		\cfrac{\delta^{5/3} \epsilon''(\delta^{1/3} z)}{\epsilon_0} \to \cfrac{24z (8 z^3 - 1)}{(4 z^3 + 1)^3} = g''(z)
	\end{gather*}
	поточечно на луче $[0, +\infty)_z$ при $\delta \to +0$.
\end{proposition}

\begin{proposition}
	\label{prop:convergence_uniform}
	В утверждении \ref{prop:convergence} сходимость на отрезке $[0, \delta^{-1/3}]_z$ с подвижной правой границей равномерная, с порядком $\bigO(\delta^{1/3})$. \absent{Так ли это?}
\end{proposition}

Утверждения \ref{prop:convergence} и \ref{prop:convergence_uniform} позволяют записать приближенные представления $\epsilon(\phi) \approx \delta^{-1} g(\delta^{-1/3} \phi)$, $\epsilon'(\phi) \approx \delta^{-4/3} g'(\delta^{-1/3} \phi)$, $\epsilon''(\phi) \approx \delta^{-5/3} g(\delta^{-1/3} \phi)$. Отсюда становится совершенно ясным описанное в работе \cite{ponomarev_stability} убывание корней $\epsilon''$ с порядком $\delta^{1/3}$ и порядок $\delta^{-5/3}$ модулей экстремумов $\epsilon''$.

Функция $\epsilon(\phi)$ на отрезке $[0, 1]$ монотонно убывает; ее производная унимодальна: вначале убывает от $0$, затем возрастает до $0$; вторая производная имеет три промежутка роста: убывает, затем возрастает, затем убывает. $\epsilon''$ вблизи $0$ меняется очень быстро, достигает больших по модулю значений и всецело определяет поведение $F''$. $\epsilon''(\phi)$ имеет ноль $\phi_0 \sim 0.5 \cdot \delta^{1/3}$, а также локальный минимум и локальный максимум в точках
\[
	\phi_\pm \sim \cfrac{1}{\sqrt[3]{32 \pm 12 \sqrt{6}}} \cdot \delta^{1/3}
\]
соответственно, что следует из равномерной сходимости $\epsilon'' \rightrightarrows g''$.

По перечисленным выше причинам функция $F''(\phi)$ в формуле \eqref{sch:time_step_stability_raw} крайне неудобна: вблизи $0$ она достигает больших по модулю значений и к тому же имеет ноль, так что при взятии модуля в зоне больших значений возникает резкий <<провал>> до $0$. Чтобы решить проблему, мажорируем $|F''(\phi)|$ гладкой функцией, не имеющей такого недостатка.
\begin{multline*}
	\cfrac{\delta^{5/3} \epsilon''(\delta^{1/3} t)}{\epsilon_0} = \cfrac{24 (t^2 - \delta^{1/3} t^3) - (24 t - 36 \delta^{1/3} t^2)(4 t^3 - 3 \delta^{1/3} t^4 + 1)}{(4 t^3 - 3 \delta^{1/3} t^4 + 1)^3} = \\
	= \cfrac{12t (16 t^3 - 30 \delta^{1/3} t^4 + 15 \delta^{2/3} t^5 + 3t \delta^{1/3} - 2)}{(4 t^3 - 3 \delta^{1/3} t^4 + 1)^3}.
\end{multline*}
Поведение функции вблизи $0$ определяется слагаемым $-2 \cdot 12t$. Поменяем его знак и получим функцию
\begin{gather*}
	\widetilde{G}(t) = \cfrac{12t (16 t^3 - 30 \delta^{1/3} t^4 + 15 \delta^{2/3} t^5 + 3t \delta^{1/3} + 2)}{(4 t^3 - 3 \delta^{1/3} t^4 + 1)^3}; \\
	G(\phi) = \cfrac{|\nabla \Phi|^2 \epsilon_0}{2} \delta^{-5/3} \widetilde{G}(\delta^{-1/3} \phi) = |\nabla \Phi|^2 \epsilon_0 \cfrac{6 \phi (16 \phi^3 - 30 \phi^4 + 15 \phi^5 + 3 \delta \phi + 2 \delta)}{(4 \phi^3 - 3 \phi^4 + \delta)^3}.
\end{gather*}
$G(\phi) \geqslant 0$ на $[0, 1]$. Имеем $G(x) = |G(x)| \geqslant |F''(\phi)|$.

Исправим формулу \eqref{sch:time_step_stability_raw} в методе адаптации по устойчивости:
\begin{equation}
	\widetilde{\tau}_3^k = \cfrac{tol_3}{m \cdot \max\limits_{j = 0}^N G(\phi_j^k)}.
	\label{sch:time_step_stability}
\end{equation}

%!TEX root = ../main.tex

\section{Вычислительный эксперимент}

\subsection{Параметры модели, краевые условия}

Была создана программа, реализующая в рамках разностной схе\forcehyphenation мы \eqref{sch:borders},~\eqref{sch:transition},~\eqref{sch:time_step_min_max} перечисленные ранее алгоритмы адаптации временного шага: по фазовому полю \eqref{sch:time_step_phi}, по энергии \eqref{sch:time_step_energy} и по устойчивости \eqref{sch:time_step_stability}.

Будем использовать параметры модели, отражающие реальный физический эксперимент: см. табл. \ref{tab:parameters}. Часть параметров ($| \nabla \Phi|$, $\epsilon_0$) являются полноценными физическими величинами, часть ($\Gamma$, $m$) могут быть подобраны для согласования модели с результатами эксперимента. Чертой отделены параметры, которые либо происходят из связанных с диффузной границей допущений ($l$, $\delta$), либо описывают расчетную сетку.

\begin{table}[!t]
\captionsetup{justification=raggedright,singlelinecheck=false}
\caption[]{Параметры модели в расчете}
\centering
\begin{tabular}{|l|c|l|}
	\hline
	Название & Параметр & Значение \\
	\hline
	электрическое напряжение		& $|\nabla \Phi|$	& $5.625 \cdot 10^6 \; \unitV / \unitm$							\\
	энергия роста ед. длины канала	& $\Gamma$			& $8.118 \cdot 10^{-10} \; \unitJ / \unitm$						\\
	диэлектрическая проницаемость	& $\epsilon_0$		& $2.301 \cdot 10^{-11} \; \unitC^2 / (\unitJ \cdot \unitm)$	\\
	подвижность						& $m$				& $12 \; \unitm^3 / (\unitJ \cdot \units)$						\\
	\hline
	характерная толщина границы		& $l$ 				& $1.5 \cdot 10^{-6} \; \unitm$									\\
	регуляризующий параметр 		& $\delta$			& $10^{-3}$														\\
	размер образца					& $W$				& $3.2 \cdot 10^{-5} \; \unitm$									\\
	продолжительность опыта			& $T$				& $2 \cdot 10^{-3} \; \units$									\\
	шаг по пространству				& $h$				& $5 \cdot 10^{-7} \; \unitm$									\\
	минимальный шаг по времени		& $\taumin$		& $10^{-10} \; \units$											\\
	максимальный шаг по времени		& $\taumax$		& $\leqslant 6.42 \cdot 10^{-6} \; \units$						\\
	\hline
\end{tabular}
\label{tab:parameters}
\end{table}

Число узлов сетки по пространству $N_x \equiv M = 64$, по времени -- $312 \hm \leqslant N_t \leqslant 2 \cdot 10^7$.

При подстановке значений параметров в условие \eqref{cond:stability} устойчивости разностной схемы получаем $\tau \leqslant \min(2.86 \cdot 10^{-10}, 6.42 \cdot 10^{-6}) \, \units$. Неравенство с первым аргументом минимума является ограничением, происходящим из свойств функции $F(\phi)$ (см. формулу \eqref{cond:stability_first}). Оно может быть ослаблено в зависимости от значений $\phi_j^k$ на текущем временном слое. Неравенство со вторым аргументом носит безусловный характер. Поэтому качестве $\taumin$ взято значение~$6.42 \cdot 10^{-6} \; \units$; $\taumax = 10^{-10} \; \units$ удовлетворяет обоим неравенствам и может рассматриваться как шаг по времени до введения адаптации.

Зададим следующие краевые условия:
\begin{gather}
	\phi(0, t) = \phi(W, t) = 1, \qquad \phi(x, 0) = \phi_0(x),
	\label{exp:border} \\
	\phi_0(x) = \begin{cases}
		1 - 0.025 \cdot \left( 1 + \cos \left[ \cfrac{\pi}{0.08} \left( \cfrac{x}{W} - \half \right) \right] \right) \text{ при } \cfrac{x}{W} \in [0.42, 0.58]; \\
		1 \text{ иначе}.
	\end{cases}
	\nonumber
\end{gather}
Функция $\phi_0(x)$ отлична от $1$ в небольшой области вокруг $x = W / 2$, где <<прогибается>> как один период синусоиды, достигая минимума $\phi = 0.95$.


\subsection{Структура сетки с переменным шагом по времени}

Итак, на каждом временном слое используется своя величина шага~$\tau^k$. Таким образом, расчетная сетка теряет регулярность по времени, и сравнение разных решений по сеточной норме становится нетривиальной задачей. Введем у нерегулярной сетки определенную структуру, в которой сконцентрируем всю сложность вопроса, избегая при этом использования сеточной интерполяции для результатов расчетов.

Пусть $N = N_{t, max} = T / \taumin \in \Natural$, то есть временной промежуток $[0, T]$ разбит $N + 1$ узлом на $N$ равных отрезков длиной $\taumin$ каждый. Над этим разбиением введем структуру <<типа дерева отрезков>>. Говоря формально, будем считать допустимыми лишь разбиения вида $D = (0, p_1 \taumin, p_2 \taumin, \dots$, $p_{n - 1} \taumin, N \taumin)$, где $p_k \in \Natural_0$, $k = \overline{0, n}$, $p_k$ строго возрастают, $L_k = p_k - p_{k - 1} \hm = 2^{s_k}$, $s_k \in \Natural_0$, и к тому же $p_{k - 1} \divby L_k$.

Описанная структура замечательна тем, что если из любых двух допустимых разбиений $D_1$ и $D_2$ выбрать по интервалу, то либо эти интервалы не пересекаются, либо совпадают, либо один строго вложен в другой. Следовательно, любые два соседних узла объемлющего разбиения $D = D_1 \cap D_2$ (пересечение в смысле множеств) соседствуют также в $D_1$ или в $D_2$ -- в таком ключе $D$ оптимально.

При адаптации шага по времени в разностной схеме \eqref{sch:borders},~\eqref{sch:transition},~\eqref{sch:time_step_min_max} на слое $k$ будем использовать не рассчитываемое $\tau^k$ напрямую, а максимальное ${\tau'}^k = 2^s \taumin \leqslant \tau^k$, $s \in \Natural_0$, к тому же допустимое описанным разбиением <<типа дерева отрезков>> временного промежутка $T$ на $N$ отрезков, а именно:
\begin{gather*}
	p_0 = 0, \quad p_k = p_{k - 1} + 2^{s_k} \leqslant N, \; s_k \in \Natural_0; \\
	{\tau'}^k = 2^{s_k} \cdot \taumin \leqslant \tau^k, \quad p_{k - 1} \divby 2^{s_k}; \\
	s_k \to \max.
\end{gather*}

Для сравнения по равномерной норме $\enorm_{C, h}$ двух сеточных решений~$\phi_1$ и~$\phi_2$ на разбиениях $D_1$ и $D_2$ соответственно ограничим их оба на объемлющем разбиении $D = D_1 \cap D_2$.


\subsection{Результаты расчетов}

На рис. \ref{fig:solution_basic} изображен результат расчета с параметрами из табл. \ref{tab:parameters} и краевыми условиями \eqref{exp:border} для разностной схемы \eqref{sch:transition_old}, \eqref{sch:borders} без адаптации шага по времени. Видно, как из малого начального возмущения фазового поля $\phi$ постепенно растет канал электрического пробоя. В момент времени $t \approx 1.82 \cdot 10^{-3}$ в точке $x = W / 2$ фазовое поле $\phi$  становится мало отличимо от $0$ -- происходит <<пробой насквозь>>. Обратим внимание, что значение фазового поля упало от $\phi \approx 0.6$ до $\phi \approx 0$ менее чем за время $10^{-5}$, то есть $0.5 \%$ от всей продолжительности эксперимента. Далее канал пробоя растет в толщину примерно с постоянной скоростью.

\begin{figure}[!t]
	\centering
	\includegraphics[width=\textwidth]{figures/solution_basic.png}
	\vspace{-0.8cm}
	\caption{Решение задачи (расчет без адаптации)}
	\label{fig:solution_basic}
\end{figure}

Теперь проведем расчеты с той же конфигурацией системы, но используя схему \eqref{sch:borders}, \eqref{sch:transition}, \eqref{sch:time_step_min_max} с переменным шагом по времени для каждого из трех методов адаптации: по фазовому полю~\eqref{sch:time_step_phi}, по энергии~\eqref{sch:time_step_energy}, по устойчивости~\eqref{sch:time_step_stability}.

При слишком больших константах $\tol_1$, $\tol_2$, $\tol_3$ разностная схема закономерно теряет устойчивость и результаты расчетов оказываются неадекватны. В таких случаях либо значения $\phi_j^k$ уходят на бесконечность, либо на графиках явно прослеживаются колебания по узлам сетки.

Для первых двух методов адаптации были опытным путем подобраны минимальные значения констант, при которых описанный вычислительный эксперимент завершается успешно: $\tol_1 = 5 \cdot 10^{-4}$, $\tol_2 = 2 \cdot 10^{-7}$. Адаптация по устойчивости дает адекватный расчет сразу, при $\tol_3 = 1$, что соответствует идее метода.

Провести сравнение решений по стандартной сеточной равномерной норме~$\enorm_{C, h}$ не удается. При введении адаптивного временного шага решение разностной задачи начинает <<отставать>> от исходного -- так проявляет себя ошибка аппроксимации по времени. Так как канал пробоя развивается стремительно, то даже небольшое <<отставание>> приводит к тому, что норма разности решений становится порядка $1$ и не несет значимой информации.

Для сравнения решений будем использовать следующую величину:
\begin{gather*}
	\rho(\phi, \psi) = \max\limits_{k = 0}^n \rho(\phi, \psi; k), \\
	\rho(\phi, \psi; k) = \min\limits_{s = 0}^n \| \phi^k - \psi^s \|_{C, x} = \min\limits_{s = 0}^n \max\limits_{j = 0}^M |\phi_j^k - \psi_j^s|.
\end{gather*}
Формула означает, что каждому моменту времени $t_{1, k}$ первого расчета сопоставляется момент времени $t_{2, s}$ второго расчета, в который сеточное решение~$\psi_j^s$ наиболее близко к $\phi_j^k$ по пространственной равномерной норме.

В описанном выше смысле <<отставание>> $\phi_j^k$ от $\psi_j^s$ есть
\begin{gather*}
	\zeta(\phi, \psi) = \max\limits_{k = 0}^n \zeta(\phi, \psi; k), \\
	\zeta(\phi, \psi; k) = t_2 \left( \argmin\limits_{s = 0}^n \| \phi^k - \psi^s \|_{C, x} \right) - t_{1, k}.
\end{gather*}
Относительным отставанием будем называть величину $\zeta(\phi, \psi) / T$, где $T$ -- длительность эксперимента.

На рис. \ref{fig:peak_error} и \ref{fig:peak_lag} показаны графики отклонения $\rho(\phi, \psi; k)$ и отстава\forcehyphenation ния $\zeta(\phi, \psi; k)$ в зависимости от $t_k$ для расчетов с максимальным наблюдаемым ускорением для каждого из трех методов адаптации. Ускорением считается уменьшение числа временных шагов после введения адаптивного шага; $\phi$ обозначает расчет с адаптией, $\psi$ -- исходный, эталонный, расчет. При построении графика $\rho(\phi, \psi; k)$ был применен фильтр оконного минимума для избавления от сеточных артефактов, связанных, по-видимому, с используемым экономичным сохранением результатов расчетов. Такой прием уместен, поскольку порядок ошибки решения куда более важен, чем ее точное значение.

\begin{figure}[!tp]
	\centering
	\includegraphics[width=\textwidth]{figures/adaptation_peak_error.png}
	\vspace{-0.8cm}
	\caption{Отклонение $\rho(\phi, \psi; k)$ решения с адаптацией от исходного решения}
	\label{fig:peak_error}
	\vspace{1cm}

	\includegraphics[width=\textwidth]{figures/adaptation_peak_lag.png}
	\vspace{-0.8cm}
	\caption{Отставание $\zeta(\phi, \psi; k)$ решения с адаптацией от исходного решения}
	\label{fig:peak_lag}
\end{figure}

В табл. \ref{tab:results_max} перечислены значения $\rho(\phi, \psi)$ и $\zeta(\phi, \psi)$ для указанных выше расчетов. При вычислении максимума в формуле отклонения использовались значения $\rho(\phi, \psi; k)$ с упомянутым оконным фильтром.

В табл. \ref{tab:results_100} перечислены те же характеристики, но для расчетов с ускорением примерно в $100$ раз.

\begin{table}[!t]
\captionsetup{justification=raggedright,singlelinecheck=false}
\caption[]{Результаты расчетов с максимальным ускорением}
\centering
\begin{tabular}{|l|c|c|c|c|}
	\hline
	Тип адаптации & Ускорение (раз) & $\| \phi - \psi \|_{C, h}$ & $\rho(\phi, \psi)$ & $\zeta(\phi, \psi) / T$ \\
	\hline
	по фазовому полю	& $800$		& $0.65$	& $3.64 \cdot 10^{-4}$	& $0.29\%$	\\
	по энергии			& $107$		& $0.68$	& $5.38 \cdot 10^{-4}$	& $0.36\%$	\\
	по устойчивости		& $1474$	& $0.77$	& $1.51 \cdot 10^{-2}$	& $0.71\%$	\\
	\hline
\end{tabular}
\label{tab:results_max}
\end{table}

\begin{table}[!t]
\captionsetup{justification=raggedright,singlelinecheck=false}
\caption[]{Результаты расчетов с ускорением примерно в 100 раз}
\centering
\begin{tabular}{|l|c|c|c|c|}
	\hline
	Тип адаптации & Ускорение (раз) & $\| \phi - \psi \|_{C, h}$ & $\rho(\phi, \psi)$ & $\zeta(\phi, \psi) / T$ \\
	\hline
	по фазовому полю	& $101$	& $0.21$	& $1.23 \cdot 10^{-5}$	& $0.0043\%$	\\
	по энергии			& $101$	& $0.60$	& $3.25 \cdot 10^{-4}$	& $0.19\%$		\\
	по устойчивости		& $100$	& $0.24$	& $2.23 \cdot 10^{-5}$	& $0.0049\%$	\\
	\hline
\end{tabular}
\label{tab:results_100}
\end{table}

Согласно проведенному сравнению, лучше всех себя показал первый метод (он же самый простой) -- адаптация временного шага по фазовому полю. Она показывает высокое ускорение и наименьшую ошибку решения. У адаптации по энергии обе эти характеристики хуже. Адаптация по устойчивости уступает первой в точности, однако имеет определенные особые преимущества: наибольшее пиковое ускорение и возможность использования без подбора коэффициента $\tol_3$ (пусть и при низкой точности решения).

\clearpage
\printbibliography[
	heading=bibintoc
]

\clearpage
\tableofcontents

\end{document}

%%%%%%%%%%%%%%%%%%%%%%%%%%%%%%%%%%%%%%%%%%%%%%%%%%%%%%%%%%%%%%%%%%%%%%%%%%%%%%%%