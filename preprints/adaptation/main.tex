\documentclass[a4paper,12pt]{article}

%%% Размер шрифта
\usepackage[14pt]{extsizes}

%%% Поля
\usepackage[
	left=2cm,
	right=2cm,
	top=2cm,
	bottom=3cm,
	bindingoffset=0cm
]{geometry}

%%% Работа с русским языком
\usepackage{cmap}						% поиск в PDF
\usepackage{mathtext}					% русские буквы в формулах
\usepackage[T2A]{fontenc}				% кодировка
\usepackage[utf8]{inputenc}				% кодировка исходного текста
\usepackage[english,russian]{babel}		% локализация и переносы
\usepackage{indentfirst}
\frenchspacing

%%% Дополнительная работа с математикой
\usepackage{amsmath,amsfonts,amssymb,amsthm,mathtools}  % AMS

%%% Текст в колонки
\usepackage{multicol}

%%% Списки
\usepackage{enumitem}
\setlist{nosep, leftmargin=*}
\renewcommand{\labelenumi}{\arabic*)}

%%% Системы уравнений
\usepackage{cases}

%%% Таблицы
\usepackage{array}

%%% Рисунки
\usepackage{graphicx}
\usepackage{float}

%%% Точка в подписях к рисункам
\usepackage[labelsep=period]{caption}

%%% Список литературы
\bibliographystyle{bibliography_style/gost-numeric.bbx}
\usepackage[
	natbib = true,
	style = gost-numeric,
	sorting = none,
	backend = biber,
	language = autobib,
	autolang = other
]{biblatex}
\addbibresource{references.bib}

%%% Исправление символа номера при использовании gost-numeric.bbx
\usepackage{textcomp}
\DefineBibliographyStrings{russian}{number={\textnumero}}

%%% Гиперссылки
\usepackage[pdftex,unicode]{hyperref}

%%% Перенос знаков в формулах (по Львовскому)
\newcommand*{\hm}[1]{#1\nobreak\discretionary{}{\hbox{$\mathsurround=0pt #1$}}{}}


%%% Свои команды

\newcommand*{\No}{\textnumero}

\newcommand{\vect}[1]{\boldsymbol{#1}}
\newcommand{\vx}{{\vect{x}}}
\newcommand{\vn}{{\vect{n}}}

\newcommand{\half}{\cfrac{1}{2}}

\newcommand{\partt}[1]{\cfrac{\partial #1}{\partial t}}
\newcommand{\partx}[1]{\cfrac{\partial #1}{\partial x}}
\newcommand{\partxx}[1]{\cfrac{\partial^2 #1}{\partial x^2}}
\newcommand{\partvn}[1]{\cfrac{\partial #1}{\partial \vn}}

\newcommand{\partflt}[1]{\partial #1 / \partial t}
\newcommand{\partflx}[1]{\partial #1 / \partial x}
\newcommand{\partflxx}[1]{\partial^2 #1 / \partial x^2}
\newcommand{\partflvn}[1]{\partial #1 / \partial \vn}

\newcommand{\gradsq}[1]{(\nabla #1, \nabla #1)}

\newcommand{\difftau}[1]{\cfrac{{#1}_j^{k + 1} - {#1}_j^k}{\tau}}
\newcommand{\diffhh}[1]{\cfrac{{#1}_{j + 1}^k - 2 {#1}_j^k + {#1}_{j - 1}^k}{h^2}}

\newcommand{\Natural}{{\mathbb{N}}}
\newcommand{\Real}{{\mathbb{R}}}
\newcommand{\bigO}{{\mathcal{O}}}
\newcommand{\clOmega}{{\overline{\Omega}}}

\newcommand{\norm}[1]{\| \, #1 \, \|}
\newcommand{\enorm}{{\| \cdot \|}}

\newcommand{\forcehyphenation}{-\linebreak}

\newcommand{\multeqstart}{
	\begingroup
	\setlength{\abovedisplayshortskip}{\the\abovedisplayskip}
	\setlength{\belowdisplayshortskip}{\the\belowdisplayskip}
}
\newcommand{\multeqnext}{
	\vspace{-7mm}
}
\newcommand{\multeqfinish}{
	\endgroup
}

\newcommand{\absent}[1]{[...#1...]}


%%% Свои операторы
\DeclareMathOperator{\Div}{{div}}
\DeclareMathOperator{\Int}{{Int}}


%%% Оформление теорем

\theoremstyle{plain}
\newtheorem{theorem}{Теорема}
\newtheorem{proposition}{Утверждение}

\theoremstyle{remark}
\newtheorem{remark}{Замечание}


%%% Пояснение к меткам
% eq	-- equation
% cond	-- condition
% char	-- characteristic
% sch	-- scheme
% est	-- estimation
% exp	-- experiment
% fig	-- figure
% tab	-- table
% sec	-- section


%%% Описание препринта
% Комментирование конца строки убирает паразитные пробелы
\newcommand{\PreprintTitle}{%
	Адаптация шага по времени в модели типа <<диффузной границы>>, содержащей уравнение Аллена--Кана%
}
\newcommand{\PreprintTitleFormatted}{%
	Адаптация шага по времени \\ в модели типа <<диффузной границы>>, \\ содержащей уравнение Аллена--Кана%
}
\newcommand{\PreprintTitleEnglish}{
	Time step adaptation in a diffuse interface model including an Allen--Cahn equation%
}
\newcommand{\PreprintAuthors}{%
	А.~С.~Пономарев, Е.~В.~Зипунова, Е.~Б.~Савенков%
}
\newcommand{\PreprintAuthorsEnglish}{%
	A.~S.~Ponomarev, E.~V.~Zipunova, E.~B.~Savenkov%
}


%%%%%%%%%%%%%%%%%%%%%%%%%%%%%%%%%%%%%%%%%%%%%%%%%%%%%%%%%%%%%%%%%%%%%%%%%%%%%%%%

\begin{document}

%%%%%%%%%%%%%%%%%%%%%%%%%%%%%%%%%%%%%%
\begin{titlepage}

\begin{center}
	РОССИЙСКАЯ АКАДЕМИЯ НАУК \\
	ОРДЕНА ЛЕНИНА \\
	ИНСТИТУТ ПРИКЛАДНОЙ МАТЕМАТИКИ \\
	имени М. В. КЕЛДЫША \par

	\vspace*{60mm}
	{
		\Large{\PreprintAuthors} \par
	}
	\vspace*{20mm}
	{
		\large \textbf{\PreprintTitleFormatted} \par
	}
	\vspace*{\fill}
	\Large{Москва, 2025}
	\vspace*{-15mm}
\end{center}

\end{titlepage}
%%%%%%%%%%%%%%%%%%%%%%%%%%%%%%%%%%%%%%%

\setcounter{page}{2}

\thispagestyle{empty}

\noindent \emph{\PreprintAuthors,} \PreprintTitle \\[3mm]
\textbf{Аннотация} \par
{
	\noindent \small
	В работе исследованы три различных подхода к адаптации шага по времени в модели развития канала электрического пробоя типа диффузной границы. Один из подходов предложен авторами настоящей работы, дано его теоретическое обоснование. Для всех трех алгоритмов адаптации проведены численные эксперименты; выявлен наиболее эффективный из них. \\
	Исследованные алгоритмы адаптации универсальны – они могут использоваться и в других моделях типа диффузной границы с уравнением Аллена–Кана. \\[3mm]
	\textbf{Ключевые слова:} модель типа диффузной границы, уравнение Аллена--Кана, адаптация шага по времени \par
	\vspace{5mm}
}
\begin{otherlanguage}{english}
\noindent \emph{\PreprintAuthorsEnglish,} \PreprintTitleEnglish \\[3mm]
\textbf{Abstract} \par
{
	\noindent \small
	\absent{Annotaceaya rabotea na angliyskom} \\[3mm]
	\textbf{Key words and phrases:} diffuse interface model, Allen--Cahn equation, adaptive time-step\-ping method \par
	\vspace{5mm}
}
\end{otherlanguage}

\clearpage

%!TEX root = ../main.tex

\section{Введение}

Электрический пробой~-- это явление резкого возрастания тока в диэлектрике при приложении электрического напряжения выше некоторого критического значения. Механизм разрушения диэлектрика под действием электрического поля сложен и многообразен: оно может иметь различные причины, характер развития, сопутствующие физические процессы \cite{vorobiev_dielectric_physics}.

Среди многообразия математических моделей, созданных для описания развития канала электрического пробоя, выделим предложенную в работе \cite{pitike_dielectric_breakdown} модель типа диффузной границы.

В настоящее время модели типа диффузной границы составляют целый класс подходов для решения задач в различных областях науки и техники. В частности, описанная в работе \cite{pitike_dielectric_breakdown} модель построена как формальное обобщение ранее известных моделей типа диффузной границы, применяемых в теории трещин.

Исследование и дальнейшее развитие упомянутой модели можно найти в работах \cite{zipunova_higher_codimension, zipunova_conservative, zipunova_thermomechanical, ponomarev_stability}. Основные положения метода диффузной границы в применении к моделированию развития канала электрического пробоя перечислены в работе \cite{ponomarev_stability}.

Модели типа диффузной границы используются для описания систем, в которых вещество может находиться в нескольких различных состояниях~-- фазах,~-- причем вещество в одной и той же фазе образует некоторые однородные области. В моделях типа диффузной границы распределение фаз вещества задается гладкой функцией $\phi$~-- фазовым полем,~-- которая в каждой области однородности близка к постоянной. Характерная толщина разделяющего слоя (<<диффузной границы>>) и, соответственно, скорость изменения~$\phi$ при переходе от одной фазы к другой определяется параметрами модели.

В работе \cite{zipunova_higher_codimension} проводится исследование свойства упомянутой модели развития канала электрического пробоя, которое можно назвать коразмерностью <<включений>>. Для задач теории трещин естественным будет двумерное включение (плоская трещина) в трехмерной среде вещества~-- в таком случае говорят, что коразмерность объекта равна 1. Обратим внимание, что, хотя исследуемая модель, как было сказано, получена на основе моделей из теории трещин, для нее характерным будет одномерное включение (канал пробоя), то есть имеющее коразмерность 2. В работе \cite{zipunova_higher_codimension} указано, что это может привести к нетривиальным последствиям, и предложено определенное обобщение исходной модели, которое предположительно делает ее более адекватной.

Суть обобщения состоит в формальном добавлении в уравнения модели двух слагаемых высших порядков с некоторыми коэффициентами. Целью настоящей работы является численная проверка поведения модели при различных значениях коэффициентов. Для этого ищется стационарное распределение фазового поля $\phi$ в нескольких характеристических случаях. Построение разностной схемы для задачи несет определенные сложности, связанные с необходимостью задать граничные условия на множествах коразмерности~2 и~3 в трехмерном пространстве. Предполагается, что точках этих множеств функция фазового поля $\phi$ имеет особенность.

Авторами применена модификация метода конечных объемов. Для части конфигураций обобщенной модели она позволила составить разнотную схему. Создана компьютерная программа, реализующая схему; проделаны расчеты, их результаты приведены в виде графиков. Для остальных конфигураций модели в процессе применения метода возникли фундаментальные проблемы, что позволяет выдвинуть гипотезу о некорректной постановке дифференциальной задачи в этих случаях.

%!TEX root = ../main.tex

\section{Математическая модель}

Приведем краткое описание исследуемой математической модели. По\forcehyphenation дробное описание физического смысла уравнений и параметров модели можно найти в работе \cite{ponomarev_stability}.

Рассматривается ограниченная область пространства $\Omega \subset \Real^3$. Распределение фаз вещества в ней задается гладкой функцией $\phi: \Omega \times [0, +\infty)_t \hm \to [0, 1], \; \phi(\vx, t)$~-- фазовым полем; вещество может находиться в одной из двух фаз: $\phi \approx 1$~-- <<неповрежденное>>, $\phi \approx 0$~-- <<полностью разрушенное>> (то есть относящееся к каналу пробоя),~-- а также в промежуточных состояниях в зоне диффузной границы.

Диэлектрическая проницаемость среды $\epsilon$ задается следующей формулой:
\[
	\epsilon(\vx, t) = \epsilon[\phi] = \cfrac{\epsilon_0(\vx)}{f(\phi(\vx, t)) + \delta}.
\]
Здесь $\epsilon_0(\vx)$~-- диэлектрическая проницаемость неповрежденной среды, $f(\phi) \hm = 4\phi^3 - 3\phi^4$~-- интерполирующая функция, $0 < \delta \ll 1$~-- регуляризующий параметр. Запись $\epsilon[\phi]$ означает функциональную зависимость $\epsilon$ от $\phi$.

Помимо фазового поля $\phi$, состояние системы описывает функция $\Phi: \Omega \hm \times [0, +\infty)_t \to \Real, \; \Phi(\vx, t)$~-- потенциал электрического поля.

Постулируется следующее выражение для свободной энергии системы $\Pi$:
\begin{gather*}
	\Pi = \int \limits_\Omega \pi d \vx, \\
	\pi = -\half \epsilon[\phi] \gradsq{\Phi} + \Gamma \cfrac{1 - f(\phi)}{l^2} + \cfrac{\Gamma}{4} \gradsq{\phi}.
\end{gather*}
Здесь $\Gamma > 0$, $l > 0$~-- числовые параметры модели, константы.

Постулируются два уравнения, определяющие динамику системы:
\begin{equation*}
\begin{cases}
	\cfrac{\delta \Pi}{\delta \Phi} = 0; \\[3mm]
	\cfrac{1}{m} \partt{\phi} = -\cfrac{\delta \Pi}{\delta \phi}.
\end{cases}
\end{equation*}
Здесь константа $m > 0$~-- числовой параметр модели, называемый подвижностью. Говоря нестрого, согласно первому уравнению электрический потенциал $\Phi$ распределяется так, чтобы свободная энергия была минимальной; согласно второму~-- фазовое поле $\phi$ с определенной скоростью стремится к тому, чтобы свободная энергия была минимальной.

Отыскав явно вариационные производные в двух уравнениях выше, получим следующую систему уравнений:
\begin{numcases}{}
	\Div(\epsilon[\phi] \nabla \Phi) = 0;
	\label{eq:Phi} \\
	\cfrac{1}{m} \partt{\phi} = \half \epsilon'(\phi) \gradsq{\Phi} + \cfrac{\Gamma}{l^2} f'(\phi) + \half \Gamma \Delta \phi.
	\label{eq:phi}
\end{numcases}
Здесь $(\cdot)' \equiv (\cdot)_\phi'$. Система состоит из двух уравнений: на $\phi$ и $\Phi$ соответственно; система связная.

Уравнение \eqref{eq:phi} имеет вид
\[
	\cfrac{1}{m} \partt{\phi} = -F'(\phi; |\nabla \Phi|) + \half \Gamma \Delta \phi,
\]
где
\begin{equation}
	F(\phi; |\nabla \Phi|) = -\half \epsilon[\phi] |\nabla \Phi|^2 + \Gamma \cfrac{1 - f(\phi)}{l^2}
	\label{eq:allen_cahn_potential}
\end{equation}
есть определенная нелинейная функция от $\phi$, которая к тому же зависит от~$|\nabla \Phi|$ как от параметра. Таким образом, перед нами нелинейное уравнение типа Аллена--Кана. \absent{Ссылка}

Из вывода модели очевидна следующая запись формулы для плотности свободной энергии:
\begin{equation}
	\pi = F(\phi; |\nabla \Phi|) + \cfrac{\Gamma}{4} \gradsq{\phi}.
	\label{eq:energy_density}
\end{equation}

В классической постановке Аллена--Кана $F$ -- двухъямный потенциал. В рассмотриваемой задаче $F$ меняет поведение в зависимости от значения~$|\nabla \Phi|$, как было показано в работе \cite{ponomarev_stability}. Возможны три случая в зависимости от величины
\[
	\xi = \cfrac{|\nabla \Phi|^2 l^2 \epsilon_0}{2 \Gamma},
\]
а именно:
\begin{itemize}
	\item \makebox[5.3cm][l]{<<слабое напряжение>>,} \makebox[4.0cm][l]{$\xi < \delta^2$:} $F(\phi)$ монотонно убывает;
	\item \makebox[5.3cm][l]{<<среднее напряжение>>,} \makebox[4.0cm][l]{$\delta^2 < \xi < (1 + \delta)^2$:} $F(\phi)$ унимодальна, убывание \par {\raggedleft сменяется возрастанием; \par}
	\item \makebox[5.3cm][l]{<<сильное напряжение>>,} \makebox[4.0cm][l]{$\xi > (1 + \delta)^2$:} $F(\phi)$ монотонно возрастает.
\end{itemize}
Наибольший интерес для практики моделирования представляет случай \linebreak <<сильного напряжения>>, так как именно тогда канал пробоя развивается из сколь угодно малых возмущений неповрежденной среды.

\clearpage
\printbibliography[
	heading=bibintoc
]

\clearpage
\tableofcontents

\end{document}

%%%%%%%%%%%%%%%%%%%%%%%%%%%%%%%%%%%%%%%%%%%%%%%%%%%%%%%%%%%%%%%%%%%%%%%%%%%%%%%%