%!TEX root = ../main.tex

\section{Введение}

Электрический пробой~-- это явление резкого возрастания тока в диэлектрике при приложении электрического напряжения выше некоторого критического значения. Механизм разрушения диэлектрика под действием электрического поля сложен и многообразен: оно может иметь различные причины, характер развития, сопутствующие физические процессы \cite{vorobiev_dielectric_physics}.

Среди многообразия математических моделей, созданных для описания развития канала электрического пробоя, выделим предложенную в работе \cite{pitike_dielectric_breakdown} модель типа диффузной границы.

В настоящее время модели типа диффузной границы составляют целый класс подходов для решения задач в различных областях науки и техники. В частности, описанная в работе \cite{pitike_dielectric_breakdown} модель построена как формальное обобщение ранее известных моделей типа диффузной границы, применяемых в теории трещин.

Исследование и дальнейшее развитие упомянутой модели можно найти в работах \cite{zipunova_higher_codimension, zipunova_conservative, zipunova_thermomechanical, ponomarev_stability}. Основные положения метода диффузной границы в применении к моделированию развития канала электрического пробоя перечислены в работе \cite{ponomarev_stability}.

Модели типа диффузной границы используются для описания систем, в которых вещество может находиться в нескольких различных состояниях~-- фазах,~-- причем вещество в одной и той же фазе образует некоторые однородные области. В моделях типа диффузной границы распределение фаз вещества задается гладкой функцией $\phi$~-- фазовым полем,~-- которая в каждой области однородности близка к постоянной. Характерная толщина разделяющего слоя (<<диффузной границы>>) и, соответственно, скорость изменения~$\phi$ при переходе от одной фазы к другой определяется параметрами модели.

В работе \cite{zipunova_higher_codimension} проводится исследование свойства упомянутой модели развития канала электрического пробоя, которое можно назвать коразмерностью <<включений>>. Для задач теории трещин естественным будет двумерное включение (плоская трещина) в трехмерной среде вещества~-- в таком случае говорят, что коразмерность объекта равна 1. Обратим внимание, что, хотя исследуемая модель, как было сказано, получена на основе моделей из теории трещин, для нее характерным будет одномерное включение (канал пробоя), то есть имеющее коразмерность 2. В работе \cite{zipunova_higher_codimension} указано, что это может привести к нетривиальным последствиям, и предложено определенное обобщение исходной модели, которое предположительно делает ее более адекватной.

Суть обобщения состоит в формальном добавлении в уравнения модели двух слагаемых высших порядков с некоторыми коэффициентами. Целью настоящей работы является численная проверка поведения модели при различных значениях коэффициентов. Для этого ищется стационарное распределение фазового поля $\phi$ в нескольких характеристических случаях. Построение разностной схемы для задачи несет определенные сложности, связанные с необходимостью задать граничные условия на множествах коразмерности~2~и~3 в трехмерном пространстве. Предполагается, что в точках этих множеств функция фазового поля $\phi$ имеет особенность.

Авторами применена модификация метода конечных объемов. Для части конфигураций обобщенной модели она позволила составить разностную схему. Создана компьютерная программа, реализующая схему; проделаны расчеты, их результаты приведены в виде графиков. Для остальных конфигураций модели в процессе применения метода возникли фундаментальные проблемы, что позволяет выдвинуть гипотезу о некорректной постановке дифференциальной задачи в этих случаях.