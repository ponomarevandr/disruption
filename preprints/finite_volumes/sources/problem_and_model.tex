%!TEX root = ../main.tex

\section{Постановка задачи и модель}
\label{sec:problem_and_model}

Приведем краткое описание математической модели, предложенной в работе \cite{pitike_dielectric_breakdown}. Подробное описание модели и физического смысла ее уравнений и параметров можно найти в работе \cite{ponomarev_stability}.

Рассматривается ограниченная область пространства $\Omega \subset \Real^3$. Распределение фаз вещества в ней задается гладкой функцией $\phi: \Omega \times [0, +\infty)_t \hm \to [0, 1], \; \phi(\vx, t)$~-- фазовым полем; вещество может находиться в одной из двух фаз: $\phi \approx 1$~-- <<неповрежденное>>, $\phi \approx 0$~-- <<полностью разрушенное>> (то есть относящееся к каналу пробоя),~-- а также в промежуточных состояниях в зоне диффузной границы.

Диэлектрическую проницаемость среды $\epsilon$ предлагается описать следующей формулой:
\begin{equation}
	\epsilon(\vx, t) = \epsilon[\phi] = \cfrac{\epsilon_0(\vx)}{f(\phi(\vx, t)) + \delta} \tpoint
	\label{eq:epsilon}
\end{equation}
Здесь $\epsilon_0(\vx)$~-- диэлектрическая проницаемость неповрежденной среды, $f(\phi) \hm = 4\phi^3 - 3\phi^4$~-- интерполирующая функция, $0 < \delta \ll 1$~-- регуляризующий параметр.

Помимо фазового поля $\phi$, состояние системы описывает также функция $\Phi: \Omega \times [0, +\infty)_t \to \Real, \; \Phi(\vx, t)$~-- потенциал электрического поля.

Постулируется следующее выражение для свободной энергии системы $\Pi$:
\begin{gather*}
	\Pi = \int \limits_\Omega \pi d \vx \tcomma \\
	\pi = -\half \epsilon[\phi] \scalsq{\Phi} + \Gamma \cfrac{1 - f(\phi)}{l^2} + \cfrac{\Gamma}{4} \scalsq{\phi} \tpoint
\end{gather*}
Здесь $\Gamma > 0, \; l > 0$~-- числовые параметры модели, константы.

Постулируются два уравнения, определяющие динамику системы:
\begin{equation*}
\begin{cases}
	\cfrac{\delta \Pi}{\delta \Phi} = 0 \tsemicolon \\[3mm]
	\cfrac{1}{m} \partt{\phi} = -\cfrac{\delta \Pi}{\delta \phi} \tpoint
\end{cases}
\end{equation*}
Здесь константа $m > 0$~-- числовой параметр модели, называемый подвижностью. Говоря нестрого, согласно первому уравнению электрический потенциал $\Phi$ распределяется так, чтобы свободная энергия была минимальной; согласно второму~-- фазовое поле $\phi$ с определенной скоростью стремится к тому, чтобы свободная энергия была минимальной.

Отыскав явно вариационные производные в двух уравнениях выше, получим следующую систему уравнений:
\begin{numcases}{}
	\Div(\epsilon[\phi] \nabla \Phi) = 0 \tsemicolon
	\label{eq:Phi} \\
	\cfrac{1}{m} \partt{\phi} = \half \epsilon'(\phi) \scalsq{\Phi} + \cfrac{\Gamma}{l^2} f'(\phi) + \half \Gamma \triangle \phi \tpoint
	\label{eq:phi}
\end{numcases}
Здесь $(\cdot)' \equiv (\cdot)_\phi'$. Система состоит из двух уравнений: на $\phi$ и $\Phi$ соответственно; система связная, второе уравнение нелинейное, является уравнением типа Аллена--Кана.