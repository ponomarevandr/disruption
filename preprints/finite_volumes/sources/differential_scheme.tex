%!TEX root = ../main.tex

\section{Построение разностной схемы}

\subsection{О методе конечных объемов}

Нашей целью будет численно исследовать систему уравнений \eqref{eq:Phi_corrected}, \eqref{eq:phi_corrected} в трех характеристических случаях: в двух описанных в подразделе \ref{ssec:matter_of_problem} и третьем, им подобном.

Итак, мы ищем стационарное решение задачи \eqref{eq:Phi_corrected}, \eqref{eq:phi_corrected} с $\Phi \equiv 0$ для трех различных граничных условий:
\begin{enumerate}
	\item $\clOmega = [0, +\infty)_x \times I_y \times J_z, \; \phi|_{x = 0} = 0, \; \phi \to 1$ при $x \to +\infty$ -- плоский случай;
	\item $\clOmega = \Real_x \times \Real_y \times J_z, \; \phi|_{x, y = 0} = 0, \; \phi \to 1$ при $r = \sqrt{x^2 + y^2} \to +\infty$ -- цилиндрический случай;
	\item $\clOmega = \Real_x \times \Real_y \times \Real_z, \; \phi|_{x, y, z = 0} = 0, \; \phi \to 1$ при $r = \sqrt{x^2 + y^2 + z^2} \to +\infty$ -- сферический случай.
\end{enumerate}
Подробно случаи 1 и 2 были описаны в подразделе \ref{ssec:matter_of_problem}. Случай 3 закономерно продолжает ряд: в нем граничное условие задано в точке -- объекте коразмерности 3.

Стационарное решение соответствует минимуму свободной энергии $\Pi$; \linebreak уравнения динамики системы выведены таковыми, что система стремится к минимуму энергии $\Pi$ в ходе эволюции (подробнее см. в разделе \ref{sec:problem_and_model}). Поэтому будем проводить расчет на достаточно долгое время, тогда установившееся положение равновесия и будет искомым стационарным решением $\phi$.

В случаях 2 и 3 естественно перейти в цилиндрические и сферические координаты соответственно и считать решение $\phi$ зависящим только от радиуса $r$. В случае 1 для единообразия пространственную переменную также назовем $r$. Итак, $\phi(\vx) = \phi(r)$.

Для численного решения задачи воспользуемся методом конечных объемов. Классический метод конечных разностей, встречая ряд проблем, подходит плохо. К примеру, в уравнениях разностной схемы могут возникать ситуации деления на 0 в узле $r = 0$ в цилиндрическом и сферическом случае (см. формулу \eqref{eq:stationary_cylindrical}).

При моделировании ограничим область $\Omega$ некоторым конечным размером -- граничные условия превращаются в $\phi(0) = 0, \; \phi(R) = 1$, где $R > 0$ -- внешний радиус $\Omega$, такой что $R / l \gg 1$.

Разобьем область $\Omega$ на $n + 1$ ячейку (прямоугольную либо в форме цилиндрического или сферического слоя), обозначим их $\Omega_0, ..., \Omega_n$. Пусть границы ячеек имеют радиусы $0 = r_{-1/2}, r_{1/2}, ..., r_{n + 1/2} = R$.

Обозначим $V(r)$ объем прямоугольника, цилиндра или сферы (в зависимости от случая), который заполняет область $\Omega$ от радиуса $0$ до $r$. Пусть $S(r)$ -- площадь внешней (разделяющей область $\Omega$) поверхности подобного прямоугольника, цилиндра или сферы. Тогда объем ячейки $\Omega_i$ равен $dV_i = V(r_{i + 1/2}) - V(r_{i - 1/2})$, площадь внутренней и внешней границ -- $S_{i - 1/2} = S(r_{i - 1/2})$ и $S_{i + 1/2} = S(r_{i + 1/2})$ соответственно.

\begin{enumerate}[label=\arabic*.]
	\item Плоский случай. $V(r) = r \cdot |I_y| \cdot |J_z|, \; S(r) = |I_y| \cdot |J_z|$. Сократив, можно считать $V(r) = r, \; S(r) = 1$.
	\item Цилиндрический случай. $V(r) = \pi r^2 \cdot |J_z|, \; S(r) = 2 \pi r \cdot |J_z|$. Сократив, можно считать $V(r) = r^2, \; S(r) = 2r$.
	\item Сферический случай. $V(r) = (4/3) \pi r^3, \; S(r) = 4 \pi r^2$. Домножив оба выражения на $3/(4\pi)$, можно считать $V(r) = r^3, \; S(r) = 3 r^2$.
\end{enumerate}

Итак, было показано, что можно считать $V(r) = r^{k + 1}, \; S(r) = (k + 1)r^k$, где $k = 0$ для плоского случая, $k = 1$ для цилиндрического, $k = 2$ для сферического.

Проведем преобразование решаемого уравнения \eqref{eq:phi_corrected}, обычное для метода конечных объемов. Учтем, что $\Phi \equiv 0$. Уравнение можно представить в следующей форме:
\begin{equation}
	\cfrac{1}{m} \partt{\phi} = \cfrac{\Gamma}{l^2} f'(\phi) + \Gamma \Div \vrho \tcomma
	\label{eq:phi_for_integration}
\end{equation}
где
\begin{equation}
	\vrho = \half \nabla \phi - \alpha \cfrac{l^2}{4} \nabla (\triangle \phi) + \beta l^2 \norm{\nabla \phi}_2^2 \nabla \phi \tpoint
	\label{eq:rho}
\end{equation}
Проинтегрируем уравнение \eqref{eq:phi_for_integration} вначале по некоторому промежутку времени [$t_j, t_{j + 1}]$, затем по ячейке $\Omega_i$. Преобразуем левую часть:
$$\intOmega \intt \cfrac{1}{m} \partt{\phi} dt d \vx = \cfrac{1}{m} \intOmega [\phi(\vx, t_{j + 1}) - \phi(\vx, t_j)] d \vx = \cfrac{dV_i}{m} [\avphi_i(t_{j + 1}) - \avphi_i(t_j)] \tcomma$$
где $\avphi_i$ -- это интегральное среднее функции $\phi$ по ячейке $\Omega_i$. Преобразуем правую часть, предварительно поменяв порядок интегрирования:
$$\intt \intOmega \left( \cfrac{\Gamma}{l^2} f'(\phi) + \Gamma \Div \vrho \right) d \vx dt = \Gamma \intt \left( \cfrac{1}{l^2} \intOmega f'(\phi) d \vx + \int\limits_{\partial \Omega_i} (\vrho, \vn) dS \right) dt \tpoint$$
К интегралу слагаемого $\Div \vrho$ была применена формула Гаусса--Остроград\-ского. Функция $\phi$ зависит только от $r$, следовательно, вектор $\vrho$ всегда параллелен оси $r$. Граница ячейки $\partial \Omega_i$ складывается из внешней (где вектор нормали $\vn$ и ось $r$ сонаправлены) и внутренней (где $\vn$ и ось $r$ противоположно направлены). Обозначим $F_{i \pm 1/2}(t)$ поток в положительном направлении оси $r$ через соответствующую границу с радиусом $r_{i \pm 1/2}$:
$$F_{i \pm 1/2}(t) = \int\limits_{r = r_{i \pm 1/2}} (\vrho, \vn) dS = \int\limits_{r = r_{i \pm 1/2}} \rho_r dS = \rho_r S(r_{i \pm 1/2}) = \rho_{i \pm 1/2}(t) \cdot S_{i \pm 1/2} \tcomma$$
$$\int\limits_{\partial \Omega_i} (\vrho, \vn) dS = F_{i + 1/2} - F_{i - 1/2} = \rho_{i + 1/2} S_{i + 1/2} - \rho_{i - 1/2} S_{i - 1/2} \tpoint$$
Здесь $\rho_r$ обозначена $r$-координата вектора $\vrho$ (она же единственная ненулевая). Величину $\rho_{i \pm 1/2}(t) = \rho_r(r_{i \pm 1/2}, t)$ будем называть плотностью потока через соответствующую границу. Таким образом, выведено следующее интегральное соотношение:
\begin{equation}
	\cfrac{dV_i}{m} [\avphi_i(t_{j + 1}) - \avphi_i(t_j)] = \Gamma \intt \left[ \cfrac{1}{l^2} \intOmega f'(\phi) d \vx + \rho_{i + 1/2} S_{i + 1/2} - \rho_{i - 1/2} S_{i - 1/2} \right] dt \tpoint
	\label{eq:finite_volume_integral}
\end{equation}

Первое слагаемое в подынтегральном выражении в правой части равенства \eqref{eq:finite_volume_integral} приблизим выражением $l^{-2} \cdot dV_i \cdot f'[\avphi_i(t_j)]$. При построении разностной схемы от интеграла по отрезку $[t_j, t_{j + 1}]$ перейдем к умножению на $(t_{j + 1} - t_j)$ значения подынтегрального выражения в точке $t_j$.

Выясним, как вычислить плотность потока $\rho_{i \pm 1/2}$ в оставшихся двух слагаемых. Если некоторая функция $\psi(\vx) = \psi(r)$, то
$$(\nabla \psi)_r = \partr{\psi}, \quad (\nabla \psi)_\theta = (\nabla \psi)_z = 0 \tcomma$$
где $\nabla \psi$ также зависит только от $r$. Следовательно, из выражения \eqref{eq:rho} получаем:
\begin{equation}
	\rho_r = \half \partr{\phi} - \alpha \cfrac{l^2}{4} \partr (\triangle \phi) + \beta l^2 \left( \partr{\phi} \right)^3 \tpoint
	\label{eq:finite_volumes_density}
\end{equation}

Традиционно в методе конечных объемов принимается, что локальное восполнение решения в ячейке -- постоянная функция. В силу того, что рассматриваемая задача требует постановки граничных условий при $r = 0$, а решение задачи вблизи этой точки может иметь большие производные, обобщим традиционный подход. А именно, будем считать, что в окрестности нуля решение представляется в виде линейной комбинации двух специально выбранных базисных функций, а его производные, соответственно, приближаются линейной комбинацией производных базисных функций с теми же коэффициентами. Опишем в общем виде поиск коэффициентов разложения.

Построим приближение для некоторой функции $\psi(r)$ в соседних ячейках $\Omega_i$ и $\Omega_{i + 1}$ по известным интегральным средним $\avpsi_i$ и $\avpsi_{i + 1}$ в этих ячейках. Пусть
$$g(r) = a \cdot \ga(r) + b \cdot \gb (r)$$
есть функция с двумя числовыми параметрами $a$ и $b$; $\ga$ и $\gb$ -- базисные функции, используемые для локального представления $\psi$. Найдем такие $a$ и $b$, что интегральные средние $g(r)$ по ячейкам $\Omega_i$ и $\Omega_{i + 1}$ были бы равны $\avpsi_i$ и $\avpsi_{i + 1}$ соответственно. Это эквивалентно системе уравнений
\begin{equation}
	\int\limits_{r_{i - 1/2}}^{r_{i + 1/2}} [\avpsi_i - g(r)] S(r) dr = 0 \tsemicolon \qquad \int\limits_{r_{i + 1/2}}^{r_{i + 3/2}} [\avpsi_{i + 1} - g(r)] S(r) dr = 0 \tpoint
	\label{eq:interpolation}
\end{equation}
Пусть
$$\int\limits_{r_{i - 1/2}}^{r_{i + 1/2}} \ga(r) S(r) dr = I_i^{(a)}; \qquad \int\limits_{r_{i - 1/2}}^{r_{i + 1/2}} \gb(r) S(r) dr = I_i^{(b)} \tpoint$$
Считаем, что интегралы $I_i^{(a)}$ и $I_i^{(b)}$ найдены аналитически. Система \eqref{eq:interpolation} эквивалентна системе
$$\begin{cases}
	a I_i^{(a)} + b I_i^{(b)} = (r_{i + 1/2}^{k + 1} - r_{i - 1/2}^{k + 1}) \avpsi_i \tsemicolon \\
	a I_{i + 1}^{(a)} + b I_{i + 1}^{(b)} = (r_{i + 3/2}^{k + 1} - r_{i + 1/2}^{k + 1}) \avpsi_{i + 1} \tpoint
\end{cases}$$
Получена система двух линейных уравнений с двумя неизвестными -- решим методом Крамера:
$$\varDelta = I_i^{(a)} I_{i + 1}^{(b)} - I_i^{(b)} I_{i + 1}^{(a)} \tsemicolon$$
\begin{equation}
	a = \cfrac{(r_{i + 1/2}^{k + 1} - r_{i - 1/2}^{k + 1}) I_{i + 1}^{(b)}}{\varDelta} \cdot \avpsi_i + \cfrac{-(r_{i + 3/2}^{k + 1} - r_{i + 1/2}^{k + 1}) I_i^{(b)}}{\varDelta} \cdot \avpsi_{i + 1} \tsemicolon
	\label{eq:interpolation_a}
\end{equation}
\begin{equation}
	b = \cfrac{-(r_{i + 1/2}^{k + 1} - r_{i - 1/2}^{k + 1}) I_{i + 1}^{(a)}}{\varDelta} \cdot \avpsi_i + \cfrac{(r_{i + 3/2}^{k + 1} - r_{i + 1/2}^{k + 1}) I_i^{(a)}}{\varDelta} \cdot \avpsi_{i + 1} \tpoint
	\label{eq:interpolation_b}
\end{equation}
Теперь можно легко вычислять $a$ и $b$ при различных значениях $\avpsi_i, \; \avpsi_{i + 1}$, если рассчитать заранее и сохранить четыре коэффициента, стоящие в формулах \eqref{eq:interpolation_a}, \eqref{eq:interpolation_b}. 

Приблизим $\partflr{\psi}$ на границе с радиусом $r_{i + 1/2}$ (между ячейками $\Omega_i$ и $\Omega_{i + 1}$) с помощью функции $g(r)$ с известными числовыми параметрами:
$$\left[ \partr{\psi} \right]_{i + 1/2} = g'(r_{i + 1/2}) = a \cdot (\ga)'(r_{i + 1/2}) + b \cdot (\gb)'(r_{i + 1/2}) \tpoint$$
Производные $(\ga)'$ и $(\gb)'$ считаем найденными аналитически.

Отыщем приближение производной $\partflr{\phi}$ на границах с радиусами $r_{3/2},$ $..., r_{n - 1/2}$ (то есть на всех, кроме первых двух внутренних и крайней внешней). Используем описанный выше метод с $\avpsi = \avphi$ и базисными функциями $\ga(r) = r, \; \gb(r) = 1$. $g' \equiv a$, вывести формулы для $I^{(a)}, \; I^{(b)}$ также не составляет труда. Таким образом, происходит приближение функции $\phi$ на парах соседних ячеек линейной функцией.

Задание граничных условий, в том числе приближение $\partflr{\phi}$ на крайних границах $\Omega$, будет подробно описано в следующем разделе.

Без ответа остался только вопрос вычисления $\triangle \phi$ и его производной по $r$. Оно требуется лишь в случае ненулевой константы $\alpha$ (см. уравнение \eqref{eq:phi_corrected}). Проведя рассуждение, аналогичное проделанному для вывода соотношения \eqref{eq:finite_volume_integral}, но без интегрирования по времени, получим следующее выражение для интегрального среднего лапласиана $\phi$ по ячейке:
\begin{equation}
	dV_i \cdot \avlaplphi_i = \partr{\phi} (r_{i + 1/2}) \cdot S_{i + 1/2} - \partr{\phi} (r_{i - 1/2}) \cdot S_{i - 1/2} \tpoint
	\label{eq:finite_volume_laplasian}
\end{equation}

По средним $\avlaplphi_i$ вычислим приближение $\partflr{(\triangle \phi)}$ на границах ячеек с радиусами $r_{1/2}, ..., r_{n - 1/2}$ (то есть на всех, кроме крайних внутренней и внешней) тем же способом, который ранее использовался для $\partflr{\phi}$, с $\avpsi = \avlaplphi, \; \ga(r) = r, \; \gb(r) = 1$.


\subsection{Задание граничных условий}

Граничные условия, подробно описанные в предыдущем разделе, имеют следующий вид: $\phi(0) = 0, \; \phi(R) = 1$, где $R > 0$ такое, что $R / l \gg 1$. Если коэффициент $\alpha$ в уравнении \eqref{eq:phi_corrected} при слагаемом с билапласианом ненулевой, то этих граничных условий недостаточно в силу повышения порядка уравнения -- необходимо добавить условия на производную $\phi$. По логике задачи их следует сделать таковыми:
$$\partr{\phi} \bigg|_{r = 0} = 0, \qquad \partr{\phi} \bigg|_{r = R} = 0 \tpoint$$

Граничные условия в точке $r = R$ в разностной схеме задаются легко: $\avphi_n = 1, \; \avlaplphi_n = 0$. Этого оказывается достаточно ввиду того, что функция $\phi$ вблизи точки $R$ меняется очень слабо.

С граничными условиями в точке $r = 0$ дела обстоят намного сложнее. Для начала, функции перехода в цилиндрическую и сферическую систему координат имеют в этой точке особенность. К тому же ожидается, что $\phi$ в окрестности $r = 0$ довольно быстро растет. Более того, как выяснится позже, в цилиндрическом и сферическом случае функции $\phi$ и $\triangle \phi$ в точке $0$ имеют особенность.

Зададим граничные условия в точке $0$ следующим образом. Выберем для приближения $\phi$ в ячейках $\Omega_0$ и $\Omega_1$ такие базисные функции $\ga$ и $\gb$, что каждая из них удовлетворяет граничным условиям при $r = 0$ и, при необходимости, одна из них имеет в точке $0$ предположительно тот же вид особенности, что решение $\phi$. С помощью функции $g(r)$ с известными параметрами получим искомые приближения:
$$\gridpartphi_{-1/2} = g'(0); \qquad \gridpartphi_{1/2} = g'(r_{1/2}) \tsemicolon$$
$$\gridpartlaplphi_{-1/2} = \partr{} \left[ r^k \partr{} \left(\cfrac{1}{r^k} \partr{g} \right) \right] \Bigg|_{r = 0} \tpoint$$
В последнем выражении используется общий вид формулы для лапласиана функции, зависящей только от $r$, в прямоугольных, цилиндрических и сферических координатах:
\begin{equation}
	\triangle g = \cfrac{1}{r^k} \partr{} \left( r^k \partr{g} \right) \tpoint
	\label{eq:laplasian_common}
\end{equation}

Определимся с выбором базисных функций для приближения $\phi$ в первых двух ячейках в зависимости от случая задачи и значений параметров $\alpha$ и $\beta$.

Рассмотрим плоский случай. Здесь у производной $\phi$ (и $\triangle \phi$, входящей в формулы, только если $\alpha \neq 0$) в точке $0$ не ожидается особенностей. Обоснуем это так: $S(0) = 1 \neq 0$, поэтому для ненулевого потока $F_{-1/2}$ через крайнюю внутреннюю границу достаточно ненулевой конечной плотности потока $\rho_{-1/2}$. При $\alpha = 0$ граничное условие $\phi(0) = 0$ -- возьмем $\ga = r^2, \; \gb = r$. При $\alpha \neq 0$ добавляется граничное условие $\partflr{\phi} |_{r = 0} = 0$ -- используем $\ga = r^3, \; \gb = r^2$.

Теперь рассмотрим цилиндрический случай. Имеем $S(0) = 0$, поэтому, казалось бы, поток $F_{-1/2}$ всегда нулевой. Однако если допустить у $\rho_r$ особенность вида $1/r$, то получается $\rho_r(r) S(r) \to 2$ при $r \to +0$ -- конечный ненулевой поток! Если $\rho_r$ по модулю растет асимптотически медленнее, то поток будет нулевым, если быстрее, то бесконечным (что не имеет смысла).

Пусть $\alpha = \beta = 0$. Тогда в выражении \eqref{eq:finite_volumes_density} для плотности потока встречается лишь $\partflr{\phi}$ в первой степени. Значит, мы ищем $\ga$, такую что $(\ga)' = C_0/r$. Проинтегрировав, получим $\ga = C_0 \ln r + C_1$. $\ga \to \infty$ при $r \to +0$, то есть граничное условие $\ga(0) = 0$ выполнить невозможно. Это косвенно подтверждает вывод из работы \cite{zipunova_higher_codimension}, что в этом случае решения дифференциальной задачи не существует.

Пусть $\alpha = 0, \; \beta \neq 0$. В выражение \eqref{eq:finite_volumes_density} для плотности потока входит $\partflr{\phi}$ в первой и третьей степени. Следовательно, мы ищем такую $\ga$, что $(\ga)' = C_0 r^{-1/3}$. Первая степень производной даст нулевой вклад в поток, третья -- конечный ненулевой. Проинтегрировав, получим $\ga = C_0 \cdot (3/2) \cdot r^{2/3} + C_1$. С учетом граничного условия выберем $\ga = r^{2/3}$. Вторую базисную функцию возьмем без особенности в точке $0$, например, $\gb = r$.

Пусть $\alpha \neq 0, \; \beta$ произвольное. Тогда в выражение \eqref{eq:finite_volumes_density} для плотности потока входит $\partflr{(\triangle \phi)}$. Встречается там и $\partflr{\phi}$, но она согласно граничному условию равна~$0$ при $r = 0$. Отыщем такую базисную функцию $\ga$, что $\partflr{(\triangle \ga)} = C_0 / r$. Учтем, что лапласиан функции $\ga$, зависящей только от $r$, в цилиндрических координатах вычисляется по формуле \eqref{eq:laplasian_common} с $k = 1$. Трижды проинтегрировав уравнение на производную лапласиана, получим $\ga = (C_0/4) r^2 (\ln r - 1) + C_1 r^2 / 4 + C_2 \ln r + C_3$. Требуется $\ga(0) = 0$, следовательно, $C_2 = C_3 = 0$. Особенность содержит первое слагаемое, поэтому выберем $\ga = r^2 \ln r$. $\ga$ удовлетворяет граничному условию на производную $(\ga)'(0) = 0$. Второй базисной функцией сделаем $\gb = r^2$.

Наконец, рассмотрим сферический случай. Аналогично цилиндрическому случаю будем подбирать базисную функцию так, чтобы поток $F_{-1/2}$ был конечным ненулевым. Для этого $\rho_r$ должна иметь особенность вида $1/r^2$, так как тогда $\rho_r(r) S(r) \to 3$ при $r \to 0$.

Пусть $\alpha = \beta = 0$. Ищем $\ga$, такую что $(\ga)' = C_0 / r^2$. Проинтегрировав, получим $\ga = -C_0 / r + C_1$. $\ga \to \infty$ при $r \to +0$, то есть граничное условие $\ga(0) = 0$ выполнить невозможно.

Пусть $\alpha = 0, \; \beta \neq 0$. Аналогично рассмотренному ранее подобному случаю ищем $\ga$, такую что $(\ga)' = C_0 r^{-2/3}$. Проинтегрировав, получим $\ga = C_0 \cdot 3 \cdot r^{1/3} + C_1$. С учетом граничного условия выберем $\ga = r^{1/3}, \; \gb = r$.

Пусть $\alpha \neq 0, \; \beta$ произвольное. Ищем $\ga$, такую что $\partr{(\triangle \ga)} = C_0 / r^2$. Лапласиан функции $\ga$, зависящей только от $r$, в сферических координатах вычисляется по формуле \eqref{eq:laplasian_common} с $k = 2$. Дважды проинтегрировав уравнение на производную лапласиана, получим $(\ga)' = -C_0 / 2 + C_1 r / 3 + C_2 / r^2$. $C_0 \neq 0$, так как иначе у функции пропадет особенность. Получается, что $(\ga)'(0) \neq 0$ -- невозможно удовлетворить граничному условию на производную!

Примечательно, что выбранный подход не удалось применить в сферическом случае не только при $\alpha = \beta = 0$, но и при $\alpha \neq 0$. Выдвинем гипотезу, что в обеих этих конфигурациях исследуемая дифференциальная задача поставлена некорректно и не имеет решения.


\subsection{Разностная схема}

Везде в двух предыдущих подразделах допускалось, что радиусы границ ячеек $0 = r_{-1/2}, r_{1/2}, ..., r_{n - 1/2}, r_{n + 1/2} = R$ могут быть произвольной возрастающей последовательностью чисел. В дальнейшем, на практике, мы сделаем их структуру регулярной с шагом $h$: $r_{i - 1/2} = ih$.

При моделировании система будет проходить моменты времени $t_j = j \tau$, где $\tau$ -- фиксированный шаг по времени. Функцию времени с аргументом $t_j$ будем обозначать верхним индексом $j$.

В качестве начального условия задаются $\{ \avphi_i^0 \}_{i = 0}^n$.

Как было принято ранее, $k = 0$ в плоском случае, $k = 1$ в цилиндрическом, $k = 2$ в сферическом.

Подводя итог рассуждениям, проделанным в двух предыдущих подразделах, выпишем построенную методом конечных объемов разностную схему.

\multeqstart
\begin{subequations}
\begin{gather}
	\cfrac{1}{m} (\avphi_i^{j + 1} - \avphi_i^j) = \tau \cfrac{\Gamma}{l^2} f'(\avphi_i^j) + \cfrac{\tau}{dV_i} \Gamma (\rho_{i + 1/2}^j S_{i + 1/2} - \rho_{i - 1/2}^j S_{i - 1/2}) \; \text{при} \; i = \overline{0, n - 1} \tsemicolon
	\nonumber \\
	\avphi_n^j = 1 \tsemicolon
	\label{sch:phi} \\
	%
	dV_i = (r_{i + 1/2})^{k + 1} - (r_{i - 1/2})^{k + 1}; \qquad S_{i \pm 1/2} = (k + 1) (r_{i \pm 1/2})^k \tsemicolon
	\nonumber \\
	%
	\rho_{i \pm 1/2}^j = \half \gridpartphi_{i \pm 1/2}^j - \alpha \cfrac{l^2}{4} \gridpartlaplphi_{i \pm 1/2}^j + \beta l^2 \left( \gridpartphi_{i \pm 1/2}^j \right)^3 \tsemicolon
	\label{sch:density} \\
	%
	\gridpartphi_{i + 1/2}^j = a_{i + 1/2}^j \; \text{при} \; i = \overline{1, n - 1} \tsemicolon
	\label{sch:derivative} \\
	\{\avphi_i^j\}_{i = 1}^n \leadsto \{a_{i + 1/2}^j\}_{i = 1}^{n - 1}, \; \text{где} \; \ga = r, \; \gb = 1 \tsemicolon
	\nonumber \\
	%
	\begin{gathered}
		\avlaplphi_i^j = \cfrac{1}{dV_i} \left( \gridpartphi_{i + 1/2}^j S_{i + 1/2} - \gridpartphi_{i - 1/2}^j S_{i - 1/2} \right) \; \text{при} \; i = \overline{0, n - 1} \tsemicolon \\
		\avlaplphi_n^j = 0 \tsemicolon
	\end{gathered}
	\label{sch:laplasian} \\
	%
	\gridpartlaplphi_{i + 1/2}^j = c_{i + 1/2}^j \; \text{при} \; i = \overline{0, n - 1} \tsemicolon
	\label{sch:laplasian_derivative} \\
	\{\avlaplphi_i^j\}_{i = 0}^n \leadsto \{c_{i + 1/2}^j\}_{i = 0}^{n - 1}, \; \text{где} \; g^{(c)} = r, \; g^{(d)} = 1 \tsemicolon
	\nonumber
\end{gather}
\multeqnext
\begin{gather}
	\gridpartphi_{i - 1/2}^j = a_{1/2}^j (g_{1/2}^{(a)})'(r_{i - 1/2}) + b_{1/2}^j (g_{1/2}^{(b)})'(r_{i - 1/2}) \; \text{при} \; i = \overline{0, 1} \tsemicolon
	\label{sch:derivative_border} \\
	\gridpartlaplphi_{-1/2}^j = a_{1/2}^j \partr{(\triangle \ga)} \bigg|_{r = 0} + b_{1/2}^j \partr{(\triangle \gb} \bigg|_{r = 0} \tsemicolon
	\label{sch:laplasian_derivative_border} \\
	\{ \avphi_0^j, \avphi_1^j \} \leadsto \{ a_{1/2}^j, b_{1/2}^j \}, \; \text{где} \; g_{1/2}^{(a)}, \; g_{1/2}^{(b)} \; \text{определены ниже} \tpoint
	\nonumber
\end{gather}
\label{sch:all}
\end{subequations}
\multeqfinish
1. Плоский случай при $\alpha = 0$:
\multeqstart
\begin{align*}
	g_{1/2}^{(a)} &= r^2, & (g_{1/2}^{(a)})' &= 2r, & I^{(a)} &= \cfrac{1}{3} r^3  \tsemicolon \\
	g_{1/2}^{(b)} &= r, & (g_{1/2}^{(b)})' &= 1, & I^{(b)} &= \half r^2 \anylimits \tpoint
\end{align*}
2. Плоский случай при $\alpha \neq 0$:
\begin{align*}
	g_{1/2}^{(a)} &= r^3, & (g_{1/2}^{(a)})' &= 3r^2, & \cfrac{\partial (\triangle g_{1/2}^{(a)})}{\partial r} &= 6, & I^{(a)} &= \cfrac{1}{4} r^4 \anylimits \tsemicolon \\
	g_{1/2}^{(b)} &= r^2, & (g_{1/2}^{(b)})' &= 2r, & \cfrac{\partial (\triangle g_{1/2}^{(b)})}{\partial r} &= 0, & I^{(b)} &= \cfrac{1}{3} r^3 \anylimits \tpoint
\end{align*}
3. Цилиндрический случай при $\alpha = 0, \; \beta \neq 0$:
\begin{align*}
g_{1/2}^{(a)} &= r^{2/3}, & (g_{1/2}^{(a)})' &= \cfrac{2}{3} r^{-1/3}, & I^{(a)} &= \cfrac{3}{4} r^{8/3} \anylimits \tsemicolon \\
g_{1/2}^{(b)} &= r, & (g_{1/2}^{(b)})' &= 1, & I^{(b)} &= \cfrac{2}{3} r^3 \anylimits \tsemicolon
\end{align*}
\multeqnext
$$\rho_{-1/2}^j S_{-1/2} = \cfrac{16}{27} (a_{1/2}^j)^3 \tpoint$$
4. Цилиндрический случай при $\alpha \neq 0$:
\begin{align*}
	g_{1/2}^{(a)} &= r^2 \ln r, & (g_{1/2}^{(a)})' &= r (2 \ln r + 1), & \cfrac{\partial (\triangle g_{1/2}^{(a)})}{\partial r} &= \cfrac{4}{r}, & I^{(a)} &= \cfrac{1}{8} r^4 (4 \ln r - 1) \anylimits \tsemicolon \\
	g_{1/2}^{(b)} &= r^2, & (g_{1/2}^{(b)})' &= 2r, & \cfrac{\partial (\triangle g_{1/2}^{(a)})}{\partial r} &= 0, & I^{(b)} &= \half r^4 \anylimits \tsemicolon
\end{align*}
\multeqnext
$$\rho_{-1/2}^j S_{-1/2} = 8 a_{1/2}^j \tpoint$$
5. Сферический случай при $\alpha = 0, \; \beta \neq 0$:
\begin{align*}
	g_{1/2}^{(a)} &= r^{1/3}, & (g_{1/2}^{(a)})' &= \cfrac{1}{3} r^{-2/3}, & I^{(a)} &= \cfrac{9}{10} r^{10/3} \anylimits  \tsemicolon \\
	g_{1/2}^{(b)} &= r, & (g_{1/2}^{(b)})' &= 1, & I^{(b)} &= \cfrac{3}{4} r^4 \anylimits \tsemicolon
\end{align*}
\multeqnext
$$\rho_{-1/2}^j S_{-1/2} = \cfrac{1}{9} (a_{1/2}^j)^3 \tpoint$$
\multeqfinish