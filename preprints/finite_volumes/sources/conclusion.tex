%!TEX root = ../main.tex

\section{Заключение}

Настоящая работа продолжает исследование, начатое в статье \cite{zipunova_higher_codimension}. Как было отмечено ее авторами, исследование это хотя и проводится для конкретной задачи, но, вероятно, затрагивает вопросы, содержащиеся в методе диффузной границы как таковом. Суть этих вопросов в том, позволяют ли уравнения среды с диффузной границей в своей <<классической>> редакции адекватно описывать включения, по своей природе являющиеся объектами высшей коразмерности. В качестве возможного ответа авторы работы \cite{zipunova_higher_codimension} предлагают определенного вида обобщение исходной модели.

Целью настоящей работы было численно исследовать упомянутое обобщение. В этом достигнуты определенные успехи. С помощью модификации метода конечных объемов преодолены трудности, связанные с необходимостью задавать граничные условия на множествах коразмерности 2 и 3 в трехмерном пространстве и с наличием у функции-решения  особенности в точках этих множеств. Указанный подход существенно не привязан к рассматриваемой модели~-- в дальнейшем он может быть использован и в других задачах.

В некоторых случаях при построении разностной схемы возникли фундаментальные препятствия: оказалось, что необходимых базисных функций попросту не существует. На основании этого выдвинута гипотеза, что в указанных случаях рассматриваемая дифференциальная задача поставлена некорректно и не имеет решения. Рассуждения вполне согласуются с теоретическими результатами работы \cite{zipunova_higher_codimension}. В будущем возможно строгое обоснование представленной гипотезы.