%!TEX root = ../main.tex

\section{Суть проблемы}

Как говорилось ранее, исследуемая модель канала электрического пробоя, предложенная в работе \cite{pitike_dielectric_breakdown}, создана на основе подобных моделей из теории трещин. Учитывая это, ознакомимся с анализом модели в следующем характеристическом случае.

Рассмотрим систему уравнений \eqref{eq:Phi}, \eqref{eq:phi} с упрощающими краевыми условиями, описанными в подразделе \ref{subsection_one_dimensional_problem}. Как было показано, система сведется к единственному нетривиальному уравнению \eqref{eq:one_dim_simpler} одной пространственной переменной $x$ и времени $t$ на фазовое поле $\phi$. Пусть $K_\Phi = 0$ (электрическое напряжение нулевое); будем искать стационарное решение следующей краевой задачи: $\phi(0) = 0, \; \phi(x) \to 1$ при $x \to +\infty$.

Уравнение \eqref{eq:one_dim_simpler} принимает следующий вид:
$$f'(\phi) = - \cfrac{l^2}{2} \cfrac{\partial^2 \phi}{\partial x^2} \tpoint$$
Домножим обе части уравнения на $\phi'_x$. Учитывая, что $f'_\phi \phi'_x = f'_x$ и $2 \phi'_x \phi''_{xx} = [(\phi'_x)^2]'_x$, проинтегрируем и, с учетом $\phi \to 1, \; \phi'_x \to 0$ при $x \to \infty$, получим:
\begin{equation}
    \cfrac{\partial \phi}{\partial x} = \cfrac{2}{l} \sqrt{1 - f(\phi)} \tpoint
    \label{eq:stationary_rectangular}
\end{equation}
Итак, мы перешли к обыкновенной задаче Коши с уравнением \eqref{eq:stationary_rectangular} и условием $\phi(0) = 0$.

Рассмотренный случай системы имеет следующий смысл: найдено распределение фазового поля в полупространстве сбоку от проводящей пластины (состоящей из полностью разрушенного вещества). Этот случай был ранее назван характеристическим, так как показывает влияние на систему параметра $l$: видно, что $l$ в уравнении~\eqref{eq:stationary_rectangular} есть коэффициент <<растяжения>> решения вдоль оси $x$. Можно показать \cite{zipunova_higher_codimension}, что при $x > l$ решение рассмотренной задачи Коши $\phi \approx 1$. Другими словами, решение есть распределение фазового поля, локализованное на отрезке $[0, l]$.

Подобный анализ вполне подходит для задачи из теории трещин (вместо проводящей пластины была бы плоская трещина). Однако характерный канал пробоя -- объект не двумерный, а одномерный. Проверим, можно ли провести аналогичное рассуждение не для пластины, а для тонкого прямого проводника.

Рассмотрим систему уравнений \eqref{eq:Phi}, \eqref{eq:phi} в области $\Omega = \Real_x \times \Real_y \times I_z$, где $I$ -- некоторый отрезок, при условии, что $\Phi \equiv 0$ (электрическое напряжение нулевое). Уравнение~\eqref{eq:Phi} выполняется тождественно, осталось единственное нетривиальное уравнение \eqref{eq:phi} на $\phi$. Будем искать стационарное решение следующей краевой задачи: $\phi|_{x, y = 0} = 0, \; \phi \to 1$ при $r = \sqrt{x^2 + y^2} \to +\infty$.

Удобно перейти в цилиндрическую систему координат: $x, y, z \longmapsto r, \theta, z$. Граничные условия однородны по $\theta$ и $z$, поэтому естественно искать решение, зависящее только от $r$. Так как $\phi(\mathbf{x}) = \phi(r)$, выражение для лапласиана $\phi$ в цилиндрических координатах принимает следующий вид:
$$\triangle \phi = \cfrac{1}{r} \cfrac{\partial}{\partial r} \left( r \cfrac{\partial \phi}{\partial r} \right) = \cfrac{1}{r} \cfrac{\partial \phi}{\partial r} + \cfrac{\partial ^2 \phi}{\partial r^2} \tpoint$$
С учетом этого уравнение \eqref{eq:phi} преобразуется в
\begin{equation}
    \cfrac{2}{l^2} f'(\phi) + \cfrac{1}{r} \cfrac{\partial \phi}{\partial r} + \cfrac{\partial ^2 \phi}{\partial r^2} = 0 \tpoint
    \label{eq:stationary_cylindrical}
\end{equation}

Подобное рассуждение проделано в работе \cite{zipunova_higher_codimension}; за ним следует анализ уравнения~\eqref{eq:stationary_cylindrical}. На основании теоретических результатов из работы \cite{cirstea_elliptic_equations} заключается, что поставленная краевая задача некорректна и решения не имеет. Даже на уровне интуиции постановка задачи выглядит необычно: условие $\phi|_{x, y = 0} = 0$ задано не на двумерной, а на одномерной <<внутренней>> границе области $\Omega$.

Возникает желание формально изменить модель так, чтобы описанная краевая задача имела решение. При моделировании канал пробоя невозможно явно представить сегментом линии, за исключением тривиальных случаев, -- однако естественно считать его <<нитевидной>> областью соответствующей формы, радиус которой может стремиться к нулю. Распределение фазового поля вблизи канала пробоя в таком случае должно приближаться к решению рассмотренной краевой задачи.