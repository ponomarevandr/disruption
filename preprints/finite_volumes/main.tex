\documentclass[a4paper,12pt]{article}

%%% Размер шрифта
\usepackage[14pt]{extsizes}

%%% Поля
\usepackage[
	left=2cm,
	right=2cm,
	top=2cm,
	bottom=3cm,
	bindingoffset=0cm
]{geometry}

%%% Работа с русским языком
\usepackage{cmap}						% поиск в PDF
\usepackage{mathtext}					% русские буквы в формулах
\usepackage[T2A]{fontenc}				% кодировка
\usepackage[utf8]{inputenc}				% кодировка исходного текста
\usepackage[english,russian]{babel}		% локализация и переносы
\usepackage{indentfirst}
\frenchspacing

%%% Дополнительная работа с математикой
\usepackage{amsmath,amsfonts,amssymb,amsthm,mathtools}  % AMS

%%% Текст в колонки
\usepackage{multicol}

%%% Списки
\usepackage{enumitem}
\setlist{nosep, leftmargin=*}
\renewcommand{\labelenumi}{\arabic*)}

%%% Системы уравнений
\usepackage{cases}

%%% Таблицы
\usepackage{array}

%%% Рисунки
\usepackage{graphicx}
\usepackage{float}

%%% Точка в подписях к рисункам
\usepackage[labelsep=period]{caption}

%%% Список литературы
\bibliographystyle{bibliography_style/gost-numeric.bbx}
\usepackage[
	natbib = true,
	style = gost-numeric,
	sorting = none,
	backend = biber,
	language = autobib,
	autolang = other
]{biblatex}
\addbibresource{references.bib}

%%% Исправление символа номера при использовании gost-numeric.bbx
\usepackage{textcomp}
\DefineBibliographyStrings{russian}{number={\textnumero}}

%%% Гиперссылки
\usepackage[pdftex,unicode]{hyperref}

%%% Перенос знаков в формулах (по Львовскому)
\newcommand*{\hm}[1]{#1\nobreak\discretionary{}{\hbox{$\mathsurround=0pt #1$}}{}}


%%% Свои команды

\newcommand*{\No}{\textnumero}

\newcommand{\vect}[1]{\boldsymbol{#1}}
\newcommand{\vx}{{\vect{x}}}

\newcommand{\half}{\cfrac{1}{2}}
\newcommand{\partt}[1]{\cfrac{\partial #1}{\partial t}}
\newcommand{\partx}[1]{\cfrac{\partial #1}{\partial x}}
\newcommand{\partxx}[1]{\cfrac{\partial^2 #1}{\partial x^2}}

\newcommand{\Natural}{{\mathbb{N}}}
\newcommand{\Real}{{\mathbb{R}}}

\newcommand{\norm}[1]{\| \, #1 \, \|}
\newcommand{\enorm}{{\| \cdot \|}}

\newcommand{\tabletopspace}{9mm}
\newcommand{\tablebottomspace}{3mm}

\newcommand{\tpoint}{{\text{.}}}
\newcommand{\tcomma}{{\text{,}}}
\newcommand{\tsemicolon}{{\text{;}}}

\newcommand{\unitm}{{\text{м}}}
\newcommand{\units}{{\text{с}}}
\newcommand{\unitJ}{{\text{Дж}}}
\newcommand{\unitC}{{\text{Кл}}}
\newcommand{\unitF}{{\text{Ф}}}


%%% Свои операторы
\DeclareMathOperator{\Div}{{div}}
\DeclareMathOperator{\Int}{{Int}}


%%% Оформление теорем

\theoremstyle{plain}
\newtheorem{proposition}{Утверждение}

\theoremstyle{remark}
\newtheorem{remark}{Замечание}


%%% Пояснение к меткам
% eq	-- equation
% cond	-- condition
% char	-- characteristic
% sch	-- scheme
% est	-- estimation
% exp	-- experiment
% fig	-- figure
% tab	-- table
% sec	-- section


%%% Описание препринта
\newcommand{\PreprintTitle}{
	Применение метода конечных объемов в модели развития канала электрического пробоя типа диффузной границы
}
\newcommand{\PreprintTitleEnglish}{
	Application of finite volume method in a phase-field model of electrical breakdown process
}
\newcommand{\PreprintAuthors}{
	А.С.~Пономарев, Е.Б.~Савенков, Е.В.~Зипунова
}
\newcommand{\PreprintAuthorsEnglish}{
	A.S.~Ponomarev, E.B.~Savenkov, E.V.~Zipunova
}


%%%%%%%%%%%%%%%%%%%%%%%%%%%%%%%%%%%%%%%%%%%%%%%%%%%%%%%%%%%%%%%%%%%%%%%%%%%%%%%%

\begin{document}

%%%%%%%%%%%%%%%%%%%%%%%%%%%%%%%%%%%%%%
\begin{titlepage}

\begin{center}
	РОССИЙСКАЯ АКАДЕМИЯ НАУК \\
	ОРДЕНА ЛЕНИНА \\
	ИНСТИТУТ ПРИКЛАДНОЙ МАТЕМАТИКИ \\
	имени М. В. КЕЛДЫША \\

	\vspace*{60mm}
	\Large{\PreprintAuthors} \\
	\vspace*{20mm}
	\textbf{\large \PreprintTitle} \\
	\vspace*{110mm}
	\Large{Москва, 2024}
	\vspace*{-50mm}
\end{center}

\end{titlepage}
%%%%%%%%%%%%%%%%%%%%%%%%%%%%%%%%%%%%%%%

\setcounter{page}{2}

\thispagestyle{empty}

\noindent \emph{\PreprintAuthors}, 
\PreprintTitle
\vspace*{3mm}

\noindent \textbf{Аннотация} \\
{
	\small
	В настоящей работе исследуется применение метода конечных объемов при отыскании стационарных решений в модели типа диффузной границы, описывающей развитие канала пробоя в твердом диэлектрике. Задача решается в трех различных случаях: плоском, цилиндрическом и сферическом. При построении разностной схемы особое внимание уделено заданию граничных условий на внутренней границе области моделирования, где решение системы уравнений может иметь особенность. \\[3mm]
	\textbf{Ключевые слова:} модель типа диффузной границы, фазовое поле, метод конечных объемов, электрический пробой. \\[5mm]
}
\emph{\PreprintAuthorsEnglish},
\PreprintTitleEnglish \\[3mm]
\textbf{Abstract} \\
{
	\small
	[Annotaciya rabotea na angliyskom yazeake.] \\[3mm]
	\textbf{Key words and phrases:} diffuse interface models, phase field, finite volume method, electrical breakdown. \\[5mm]
}

\newpage

%!TEX root = ../main.tex

\section{Введение}

\begin{frame}{Физическое явление}
\begin{block}{Электрический пробой}
	Явление резкого возрастания тока в диэлектрике при приложении электрического напряжения
	выше критического.
\end{block}
\begin{itemize}
	\item Рассматриваем твердый диэлектрик
	\item Деградация диэлектрических свойств материала
	\item Процесс развивается в ограниченной зоне -- канале пробоя
	\item Сложная физическая природа
\end{itemize}
\end{frame}


\begin{frame}{Математическая модель}
\begin{block}{Модель типа диффузной границы}
	Вещество находится в разных фазах. Состояние вещества описывается гладкой функцией
	$\phi(\vx, t)$ -- фазовым полем.
\end{block}
\begin{itemize}
	\item $\phi = 1$ -- неповрежденная среда
	\item $\phi = 0$ -- полностью разрушенная среда
	\item Зона $\phi \in (0, 1)$ -- диффузная граница
	\item На разрушение среды тратится энергия
\end{itemize}
\begin{figure}
	\includegraphics[width=0.5\textwidth]{figures/diffuse_edge.jpg}
\end{figure}
\end{frame}


\begin{frame}{Математическая модель}
Модель, предложенная в работе \cite{pitike_dielectric_breakdown}:
\begin{itemize}
	\item $\pi = \textcolor{red}{-\half \epsilon[\phi] (\nabla \Phi, \nabla \Phi)} +
	\Gamma \left( \cfrac{1 - f(\phi)}{l^2} + \cfrac{1}{4} (\nabla \phi, \nabla \phi) \right)$
	-- плотность свободной энергии
	\item $\Gamma$ -- энегрия роста канала пробоя на единицу длины
	\item $l$ -- величина <<размытия>> канала
	\item $\epsilon(\vx, t)$ -- диэлектрическая проницаемость среды
	\item $f(\phi)$ -- интерполирующая функция
\end{itemize}
\end{frame}


\begin{frame}{Математическая модель}
\vspace{-0.2cm}
\begin{itemize}
	\item $\epsilon(\vx, t) = \cfrac{\epsilon_0(\vx)}{f(\phi(\vx, t)) +
	\delta}$ -- диэлектрическая проницаемость среды
	\item $f(\phi) = 4 \phi^3 - 3 \phi^4$ -- интерполирующая функция
\end{itemize}
\begin{columns}
\column{0.5\textwidth}
\begin{figure}
	\hspace*{1.4cm}
	\includegraphics[width=0.65\textwidth]{figures/f_form.png}
\end{figure}
\column{0.5\textwidth}
\begin{figure}
	\hspace*{-2cm}
	\includegraphics[width=0.60\textwidth]{figures/eps_form.png}
\end{figure}
\end{columns}
\end{frame}


\begin{frame}{Математическая модель}
\vspace{-0.5cm}
\begin{block}{Уравнения модели}
\begin{itemize}
	\item Уравнение электрического потенциала $\Phi$:
	\begin{equation}
		\Div(\epsilon[\phi] \nabla \Phi) = 0
		\label{equation_potential}
	\end{equation}
	\item Уравнение фазового поля $\phi$:
	\begin{equation}
		\cfrac{1}{m} \partt{\phi} = \half \epsilon'(\phi) \gradscalsq{\Phi} + \cfrac{\Gamma}{l^2} f'(\phi) + \half \Gamma \Delta \phi
		\label{equation_phase}
	\end{equation}
\end{itemize}
\end{block}
Свойства:
\begin{itemize}
	\item связанная система уравнений на $\phi$ и $\Phi$;
	\item уравнение для $\phi$ типа Аллена--Кана, нелинейное.
\end{itemize}
\end{frame}


\begin{frame}{Пример вычислительного эксперимента}
\begin{columns}
\column{0.32\textwidth}
\begin{figure}
	\includegraphics[width=\textwidth]{figures/model_example_1.png}
\end{figure}
\column{0.32\textwidth}
\begin{figure}
	\includegraphics[width=\textwidth]{figures/model_example_2.png}
\end{figure}
\column{0.32\textwidth}
\begin{figure}
	\includegraphics[width=\textwidth]{figures/model_example_3.png}
\end{figure}
\end{columns}
\begin{center}
	Расчет из работы \cite{zipunova_experiment}
\end{center}
\end{frame}


\begin{frame}{Цель работы}
\begin{block}{Цель работы}
	Исследовать качественные характеристики системы уравнений \eqref{equation_potential},
	\eqref{equation_phase} и выполнить ее численный анализ.
\end{block}
Для этого рассмотрим задачу в определенных краевых условиях, упрощающих ее, но позволяющих
установить интересующие свойства.
\end{frame}

%!TEX root = ../main.tex

\section{Обобщение модели}

\subsection{Предложенное обобщение}

В ответ на описанную в предыдущем подразделе проблему в работе \cite{zipunova_higher_codimension}, на основании теоретических результатов работ \cite{sobolev_functional_analysis}, \cite{oleynik_biharmonic_equations}, \cite{sternin_elliptic_equations}, \cite{lewis_quasi_linear}, предлагается следующая обобщенная модель, для которой постановка условий на границах размерности 1 (соответственно, коразмерности 2) является математически корректной:
\begin{equation}
    \Pi = \int \limits_\Omega \pi d \mathbf{x} \tcomma
    \label{eq:free_energy_corrected}
\end{equation}
\begin{equation}
    \pi = -\half \epsilon[\phi] (\nabla \Phi, \nabla \Phi) + \Gamma \cfrac{1 - f(\phi)}{l^2} + \cfrac{\Gamma}{4} (\nabla \phi, \nabla \phi) + \alpha \cfrac{\Gamma l^2}{8} (\triangle \phi)^2 + \beta \cfrac{1}{p} \Gamma l^{p - 2} \| \, \nabla \phi \, \|_2^p \tsemicolon
    \label{eq:free_energy_density_corrected}
\end{equation}
\begin{numcases}{}
    \Div(\epsilon[\phi] \nabla \Phi) = 0 \tsemicolon \label{equation_Phi_corrected} \\
    \cfrac{1}{m} \cfrac{\partial \phi}{\partial t} = \half \epsilon'(\phi) (\nabla \Phi, \nabla \Phi) + \cfrac{\Gamma}{l^2} f'(\phi) + \half \Gamma \triangle \phi - \alpha \cfrac{\Gamma l^2}{4} \triangle^2 \phi + \beta \Gamma l^{p - 2} \Div (\| \, \nabla \phi \, \|_2^{p - 2} \nabla \phi) \tpoint
    \label{eq:phi_corrected}
\end{numcases}
Здесь $\alpha, \beta \geqslant 0$ -- некоторые константы, $p$ -- четное натуральное число, не меньшее~4. Дифференциальный оператор $\Div (\| \, \nabla \phi \, \|_2^{p - 2} \nabla \phi)$ принято называть \emph{$p$-лапласианом}, $\triangle^2 \phi = \triangle(\triangle \phi)$ -- \emph{билапласианом}. В дальнейшем для простоты будем считать $p = 4$.


\subsection{О методе конечных объемов}

Нашей целью будет численно исследовать систему уравнений \eqref{eq:Phi_corrected}, \eqref{eq:phi_corrected} в трех характеристических случаях, подобных описанному в подразделе \ref{subsection_matter_of_problem}.

Итак, мы ищем стационарное решение задачи \eqref{eq:Phi_corrected}, \eqref{eq:phi_corrected} с $\Phi \equiv 0$ для трех различных граничных условий:
\begin{enumerate}
    \item $\Omega = [0, +\infty)_x \times I_y \times I_z, \; \phi|_{x = 0} = 0, \; \phi \to 1$ при $x \to +\infty$ -- плоский случай;
    \item $\Omega = \Real_x \times \Real_y \times I_z, \; \phi|_{x, y = 0} = 0, \; \phi \to 1$ при $r = \sqrt{x^2 + y^2} \to +\infty$ -- цилиндрический случай;
    \item $\Omega = \Real_x \times \Real_y \times \Real_z, \; \phi|_{x, y, z = 0} = 0, \; \phi \to 1$ при $r = \sqrt{x^2 + y^2 + z^2} \to +\infty$ -- сферический случай.
\end{enumerate}
Подробно случаи 1 и 2 были описаны в подразделе \ref{subsection_matter_of_problem}. Случай 3 закономерно продолжает ряд: в нем граничное условие задано в точке -- объекте коразмерности 3.

Стационарное решение соответствует минимуму свободной энергии $\Pi$. Уравнения динамики системы выведены таковыми, что система стремится к минимуму энергии $\Pi$ в ходе эволюции. Поэтому будем проводить расчет на достаточно долгое время, тогда установившееся положение равновесия и будет искомым стационарным решением $\phi$.

В случаях 2 и 3 естественно перейти в цилиндрические и сферические координаты соответственно и считать решение $\phi$ зависящим только от радиуса $r$. В случае 1 для единообразия пространственную переменную также назовем $r$. Итак, $\phi(\mathbf{x}) = \phi(r)$.

Для численного решения задачи воспользуемся методом конечных объемов. Классический метод конечных разностей, встречая ряд проблем, подходит плохо. К примеру, в уравнениях разностной схемы могут возникать ситуации деления на 0 в узле $r = 0$ в цилиндрическом и сферическом случае (см. формулу \eqref{eq:stationary_cylindrical}).

При моделировании мы ограничим область $\Omega$ некоторым конечным размером -- граничные условия превращаются в $\phi(0) = 0, \; \phi(R) = 1, \; R > 0$ -- внешний радиус $\Omega$, такой что $R / l \gg 1$.

Разобьем область $\Omega$ на $n + 1$ ячейку (прямоугольную либо в форме цилиндрического или сферического слоя), обозначим их $\Omega_0, ..., \Omega_n$. Пусть границы ячеек имеют радиусы $0 = r_{-1/2}, r_{1/2}, ..., r_{n + 1/2} = R$.

Обозначим $V(r)$ объем прямоугольника, цилиндра или сферы (в зависимости от случая), который заполняет область $\Omega$ от радиуса $0$ до $r$. Пусть $S(r)$ -- площадь внешней (разделяющей область $\Omega$) поверхности подобного прямоугольника, цилиндра или сферы. Тогда объем ячейки $\Omega_i$ равен $dV_i = V(r_{i + 1/2}) - V(r_{i - 1/2})$, площадь внутренней и внешней границ -- $S(r_{i - 1/2})$ и $S(r_{i + 1/2})$ соответственно.

\begin{enumerate}[label=\arabic*.]
    \item Плоский случай. $V(r) = r \cdot |I_y| \cdot |I_z|, \; S(r) = |I_y| \cdot |I_z|$. Сократив, можно считать $V(r) = r, \; S(r) = 1$.
    \item Цилиндрический случай. $V(r) = \pi r^2 \cdot |I_z|, \; S(r) = 2 \pi r \cdot |I_z|$. Сократив, можно считать $V(r) = r^2, \; S(r) = 2r$.
    \item Сферический случай. $V(r) = (4/3) \pi r^3, \; S(r) = 4 \pi r^2$. Домножив оба выражения на $3/(4\pi)$, можно считать $V(r) = r^3, \; S(r) = 3 r^2$.
\end{enumerate}

Итак, было показано, что можно считать $V(r) = r^{k + 1}, \; S(r) = (k + 1)r^k$, где $k = 0$ для плоского случая, $k = 1$ для цилиндрического, $k = 2$ для сферического.

Проведем преобразование решаемого уравнения \eqref{eq:phi_corrected}, обычное для метода конечных объемов. Учтем, что $\Phi \equiv 0$. Уравнение представляется в следующей форме:
\begin{equation}
    \cfrac{1}{m} \cfrac{\partial \phi}{\partial t} = \cfrac{\Gamma}{l^2} f'(\phi) + \Gamma \Div \overline{\rho} \tcomma
    \label{eq:phi_for_integration}
\end{equation}
где
$$\overline{\rho} = \half \nabla \phi - \alpha \cfrac{l^2}{4} \nabla (\triangle \phi) + \beta l^2 \| \, \nabla \phi \, \|_2^2 \nabla \phi \tpoint$$
Проинтегрируем уравнение \eqref{eq:phi_for_integration} вначале по некоторому промежутку времени [$t_j, t_{j + 1}]$, затем по ячейке $\Omega_i$. Преобразуем левую часть:
$$\int\limits_{\Omega_i} \int\limits_{t_j}^{t_{j + 1}} \cfrac{1}{m} \cfrac{\partial \phi}{\partial t} dt d \mathbf{x} = \cfrac{1}{m} \int\limits_{\Omega_i} [\phi(\mathbf{x}, t_{j + 1}) - \phi(\mathbf{x}, t_j)] d \mathbf{x} = \cfrac{dV_i}{m} [\widetilde{\phi}_i(t_{j + 1}) - \widetilde{\phi}_i(t_j)] \tcomma$$
где $\widetilde{\phi}_i$ -- это интегральное среднее функции $\phi$ по ячейке $\Omega_i$. Преобразуем правую часть, предварительно поменяв порядок интегрирования:
$$\int\limits_{t_j}^{t_{j + 1}} \int\limits_{\Omega_i} \left( \cfrac{\Gamma}{l^2} f'(\phi) + \Gamma \Div \overline{\rho} \right) d \mathbf{x} dt = \int\limits_{t_j}^{t_{j + 1}} \left( \cfrac{\Gamma}{l^2} \int\limits_{\Omega_i} f'(\phi) d \mathbf{x} + \Gamma \int\limits_{\partial \Omega_i} (\overline{\rho}, \overline{n}) dS \right) dt \tpoint$$
К интегралу слагаемого $\Gamma \Div \overline{\rho}$ была применена формула Гаусса--Остроградского. Функция $\phi$ зависит только от $r$, следовательно, вектор $\overline{\rho}$ всегда параллелен оси $r$. Граница ячейки $\partial \Omega_i$ складывается из внешней (где вектор нормали $\overline{n}$ и ось $r$ сонаправлены) и внутренней (где $\overline{n}$ и ось $r$ противоположно направлены). Обозначим $F_{i \pm 1/2}(t)$ поток в положительном направлении оси $r$ через соответствующую границу с радиусом $r_{i \pm 1/2}$:
$$F_{i \pm 1/2}(t) = \int\limits_{r = r_{i \pm 1/2}} (\overline{\rho}, \overline{n}) dS = \int\limits_{r = r_{i \pm 1/2}} \overline{\rho}_r dS = \overline{\rho}_r S(r_{i \pm 1/2}) = \rho_{i \pm 1/2}(t) \cdot S(r_{i \pm 1/2}) \tcomma$$
$$\int\limits_{\partial \Omega_i} (\overline{\rho}, \overline{n}) dS = F_{i + 1/2} - F_{i - 1/2} = \rho_{i + 1/2} S(r_{i + 1/2}) - \rho_{i - 1/2} S(r_{i - 1/2}) \tpoint$$
Здесь $\overline{\rho}_r$ обозначена $r$-координата вектора $\overline{\rho}$ (она же единственная ненулевая). Величину $\rho_{i \pm 1/2}(t) = \overline{\rho}_r(r_{i \pm 1/2}, t)$ будем называть плотностью потока через соответствующую границу. Таким образом, выведено следующее интегральное соотношение:
\begin{equation}
    \cfrac{dV_i}{m} [\widetilde{\phi}_i(t_{j + 1}) - \widetilde{\phi}_i(t_j)] = \int\limits_{t_j}^{t_{j + 1}} \left( \cfrac{\Gamma}{l^2} \int\limits_{\Omega_i} f'(\phi) d \mathbf{x} + \Gamma \left[ \rho_{i + 1/2} S(r_{i + 1/2}) - \rho_{i - 1/2} S(r_{i - 1/2}) \right] \right) dt \tpoint
    \label{eq:finite_volume_integral}
\end{equation}

Первое слагаемое в подынтегральном выражении в правой части равенства \eqref{eq:finite_volume_integral} приблизим выражением $(\Gamma/l^2) \cdot dV_i \cdot f'[\widetilde{\phi}_i(t_j)]$. При построении разностной схемы от интеграла по $[t_j, t_{j + 1}]$ перейдем к умножению на $(t_{j + 1} - t_j)$ значения подынтегрального выражения в точке $t_j$.

Выясним, как вычислить плотность потока $\rho_{i \pm 1/2}$ во втором слагаемом. Если некоторая функция $\psi(\mathbf{x}) = \psi(r)$, то
$$(\nabla \psi)_r = \cfrac{\partial \psi}{\partial r} \tcomma$$
где $\nabla \psi$ также зависит только от $r$. Таким образом:
\begin{equation}
    \rho = \half \cfrac{\partial \phi}{\partial r} - \alpha \cfrac{l^2}{4} \cfrac{\partial}{\partial r} (\triangle \phi) + \beta l^2 \left( \cfrac{\partial \phi}{\partial r} \right)^3 \tpoint
    \label{eq:finite_volumes_density}
\end{equation}

Традиционно в методе конечных объемов принимается, что локальное восполнение решения в ячейке -- постоянная функция. В силу того, что рассматриваемая задача требует постановки граничных условий при $r = 0$, а решение задачи вблизи этой точки может иметь большие производные, обобщим традиционный подход. А именно, будем считать, что в окрестности нуля решение представляется в виде линейной комбинации двух специально выбранных базисных функций, а его производные, соответственно, приближаются линейной комбинацией производных базисных функций с теми же коэффициентами. Опишем в общем виде поиск коэффициентов разложения.

Построим приближение для некоторой функции $\psi(r)$ в соседних ячейках $\Omega_i$ и $\Omega_{i + 1}$ по известным интегральным средним $\widetilde{\psi}_i$ и $\widetilde{\psi}_{i + 1}$ в этих ячейках. Пусть
$$g(r) = a \cdot g^{(a)}(r) + b \cdot g^{(b)} (r)$$
есть функция с двумя числовыми параметрами $a$ и $b$, $g^{(a)}$ и $g^{(b)}$ -- базисные функции, используемые для локального представления $\psi$. Найдем такие $a$ и $b$, что интегральные средние $g(r)$ по ячейкам $\Omega_i$ и $\Omega_{i + 1}$ были бы равны $\widetilde{\psi}_i$ и $\widetilde{\psi}_{i + 1}$ соответственно. Это эквивалентно системе уравнений
\begin{numcases}{}
    \int\limits_{r_{i - 1/2}}^{r_{i + 1/2}} [\widetilde{\psi}_i - g(r)] S(r) dr = 0 \tsemicolon
    \label{eq:interpolation_first} \\
    \int\limits_{r_{i + 1/2}}^{r_{i + 3/2}} [\widetilde{\psi}_{i + 1} - g(r)] S(r) dr = 0 \tpoint
    \label{eq:interpolation_second}
\end{numcases}
Пусть
$$\int\limits_{r_{i - 1/2}}^{r_{i + 1/2}} g^{(a)}(r) S(r) dr = I_i^{(a)}; \qquad \int\limits_{r_{i - 1/2}}^{r_{i + 1/2}} g^{(b)}(r) S(r) dr = I_i^{(b)} \tpoint$$
Считаем, что интегралы $I_i^{(a)}$ и $I_i^{(b)}$ найдены аналитически. Тогда система \eqref{eq:interpolation_first}, \eqref{eq:interpolation_second} эквивалентна системе
$$\begin{cases}
    a I_i^{(a)} + b I_i^{(b)} = (r_{i + 1/2}^{k + 1} - r_{i - 1/2}^{k + 1}) \widetilde{\psi}_i \tsemicolon \\
    a I_{i + 1}^{(a)} + b I_{i + 1}^{(b)} = (r_{i + 3/2}^{k + 1} - r_{i + 1/2}^{k + 1}) \widetilde{\psi}_{i + 1} \tpoint
\end{cases}$$
Это система двух линейных уравнений с двумя неизвестными -- решим методом Крамера. Получим:
$$\varDelta = I_i^{(a)} I_{i + 1}^{(b)} - I_i^{(b)} I_{i + 1}^{(a)} \tsemicolon$$
\begin{equation}
    a = \cfrac{(r_{i + 1/2}^{k + 1} - r_{i - 1/2}^{k + 1}) I_{i + 1}^{(b)}}{\varDelta} \cdot \widetilde{\psi}_i + \cfrac{-(r_{i + 3/2}^{k + 1} - r_{i + 1/2}^{k + 1}) I_i^{(b)}}{\varDelta} \cdot \widetilde{\psi}_{i + 1} \tsemicolon
    \label{eq:interpolation_a}
\end{equation}
\begin{equation}
    b = \cfrac{-(r_{i + 1/2}^{k + 1} - r_{i - 1/2}^{k + 1}) I_{i + 1}^{(a)}}{\varDelta} \cdot \widetilde{\psi}_i + \cfrac{(r_{i + 3/2}^{k + 1} - r_{i + 1/2}^{k + 1}) I_i^{(a)}}{\varDelta} \cdot \widetilde{\psi}_{i + 1} \tpoint
    \label{eq:interpolation_b}
\end{equation}
Теперь можно легко находить $a$ и $b$ при различных значениях $\widetilde{\psi}_i, \; \widetilde{\psi}_{i + 1}$, если вычислить заранее и сохранить четыре коэффициента, стоящие в формулах \eqref{eq:interpolation_a}, \eqref{eq:interpolation_b}. 

Приблизим $\partial \psi / \partial r$ на границе с радиусом $r_{i + 1/2}$ (между ячейками $\Omega_i$ и $\Omega_{i + 1}$) с помощью функции $g(r)$ с известными числовыми параметрами:
$$\left[ \cfrac{\partial \psi}{\partial r} \right]_{i + 1/2} = g'(r_{i + 1/2}) = a \cdot (g^{(a)})'(r_{i + 1/2}) + b \cdot (g^{(b)})'(r_{i + 1/2}) \tpoint$$
Производные $(g^{(a)})'$ и $(g^{(b)})'$ считаем найденными аналитически.

Отыщем приближение производной $\partial \phi / \partial r$ на границах с радиусами $r_{3/2}, ...$, $r_{n - 1/2}$ (то есть на всех, кроме первых двух внутренних и крайней внешней). Используем описанный выше метод с $\widetilde{\psi} = \widetilde{\phi}$ и базисными функциями $g^{(a)}(r) = r, \; g^{(b)}(r) = 1$. $g' \equiv a$, вывести формулы для $I^{(a)}$, $I^{(b)}$ также не составляет труда. Таким образом, происходит приближение функции $\phi$ на парах соседних ячеек линейной функцией.

Задание граничных условий, в том числе приближение $\partial \phi / \partial r$ на крайних границах $\Omega$, будет подробно описано в следующем разделе.

Без ответа остался только вопрос вычисления $\triangle \phi$ и его производной по $r$. Оно требуется лишь в случае ненулевой константы $\alpha$ (см. уравнение \eqref{eq:phi_corrected}). Проведя рассуждение, аналогичное проделанному для вывода соотношения \eqref{eq:finite_volume_integral}, но без интегрирования по времени, получим следующее выражение для интегрального среднего лапласиана $\phi$ по ячейке:
\begin{equation}
    dV_i \cdot \widetilde{\triangle \phi}_i = \cfrac{\partial \phi}{\partial r} \bigg|_{r = r_{i + 1/2}} S(r_{i + 1/2}) - \cfrac{\partial \phi}{\partial r} \bigg|_{r = r_{i - 1/2}} S(r_{i - 1/2}) \tpoint
    \label{eq:finite_volume_laplasian}
\end{equation}

По средним $\widetilde{\triangle \phi}$ вычислим приближение $\partial (\triangle \phi) / \partial r$ на границах ячеек с радиусами $r_{1/2}, ..., r_{n - 1/2}$ (то есть на всех, кроме крайних внутренней и внешней) тем же способом, который ранее использовался для $\partial \phi / \partial r$, с $\widetilde{\psi} = \widetilde{\triangle \phi}, \; g^{(a)}(r) = r, \; g^{(b)}(r) = 1$.


\subsection{Задание граничных условий}

Граничные условия, подробно описанные в предыдущем разделе, имеют следующий вид: $\phi(0) = 0, \; \phi(R) = 1, \; R > 0$, где $R$ такое, что $R / l \gg 1$. Если коэффициент $\alpha$ в уравнении \eqref{eq:phi_corrected} при слагаемом с билапласианом ненулевой, то этих граничных условий недостаточно в силу повышения порядка уравнения -- необходимо добавить условия на производные. По логике задачи их следует сделать таковыми:
$$\cfrac{\partial \phi}{\partial r} \bigg|_{r = 0} = 0, \qquad \cfrac{\partial \phi}{\partial r} \bigg|_{r = 1} = 0 \tpoint$$

Граничные условия в точке $r = R$ в разностной схеме задаются легко: $\widetilde{\phi}_n \equiv 1, \; \widetilde{\triangle \phi}_n \equiv 0$. Этого оказывается достаточно ввиду того, что функция $\phi$ вблизи точки $R$ меняется очень слабо.

С граничными условиями в точке $r = 0$ дела обстоят намного сложнее. Для начала, функции перехода в цилиндрическую и сферическую систему координат имеют в этой точке особенность. К тому же ожидается, что $\phi$ в окрестности точки $r = 0$ довольно быстро растет. Более того, как выяснится позже, в цилиндрическом и сферическом случае в точке $0$ имеют особенность производные функций $\phi$ и $\triangle \phi$.

Зададим граничные условия в точке $0$ следующим образом. Выберем для приближения $\phi$ в ячейках $\Omega_0$ и $\Omega_1$ такие базисные функции $g^{(a)}$ и $g^{(b)}$, что каждая из них удовлетворяет граничным условиям при $r = 0$ и, при необходимости, одна из них имеет в точке $0$ предположительно тот же вид особенности, что решение $\phi$. С помощью функции $g(r)$ с известными параметрами получим искомые приближения:
$$\left[ \cfrac{\partial \phi}{\partial r} \right]_{-1/2} = g'(0); \qquad \left[ \cfrac{\partial \phi}{\partial r} \right]_{1/2} = g'(r_{1/2}) \tsemicolon$$
$$\left[ \cfrac{\partial (\triangle \phi)}{\partial r} \right]_{-1/2} = \cfrac{\partial}{\partial r} \left[ r^k \cfrac{\partial}{\partial r} \left(\cfrac{1}{r^k} \cfrac{\partial g}{\partial r} \right) \right] \Bigg|_{r = 0} \tpoint$$
В последнем выражении используется общий вид формулы для лапласиана функции, зависящей только от $r$, в прямоугольных, цилиндрических и сферических координатах:
\begin{equation}
    \triangle g = \cfrac{1}{r^k} \cfrac{\partial}{\partial r} \left( r^k \cfrac{\partial g}{\partial r} \right) \tpoint
    \label{eq:laplasian_common}
\end{equation}

Определимся с выбором базисных функций для приближения $\phi$ в первых двух ячейках в зависимости от случая задачи и значений параметров $\alpha$ и $\beta$.

Рассмотрим плоский случай. Здесь у производной $\phi$ (и $\triangle \phi$, входящей в формулы, только если $\alpha \neq 0$) в точке $0$ не ожидается особенностей. Обоснуем это так: $S(0) = 1 \neq 0$, поэтому для ненулевого потока $F_{-1/2}$ через крайнюю внутреннюю границу нужна ненулевая конечная плотность потока $\rho_{-1/2}$. При $\alpha = 0$ граничное условие $\phi(0) = 0$ -- возьмем $g^{(a)} = r^2, \; g^{(b)} = r$. При $\alpha \neq 0$ добавляется граничное условие $\partial \phi / \partial r |_{r = 0} = 0$ -- используем $g^{(a)} = r^3, \; g^{(b)} = r^2$.

Теперь рассмотрим цилиндрический случай. Имеем $S(0) = 0$, поэтому, казалось бы, поток $F_{-1/2}$ всегда нулевой. Однако если допустить у $\rho$ особенность вида $1/r$, то получается $\rho(r) S(r) \to 2$ при $r \to +0$ -- конечный ненулевой поток! Если $\rho$ по модулю растет асимптотически медленнее, то поток будет нулевым, если быстрее, то бесконечным (что не имеет смысла).

Пусть $\alpha = \beta = 0$. Тогда в выражении \eqref{eq:finite_volumes_density} для плотности потока встречается лишь $\partial \phi / \partial r$ в первой степени. Значит, мы ищем $g^{(a)}$, такую что $(g^{(a)})' = C_0/r$. Проинтегрировав, получим $g^{(a)} = C_0 \ln r + C_1$. $g^{(a)} \to \infty$ при $r \to +0$, то есть граничное условие $g^{(a)}(0) = 0$ выполнить невозможно. Это косвенно подтверждает вывод из работы \cite{zipunova_higher_codimension}, что в этом случае решения дифференциальной задачи не существует.

Пусть $\alpha = 0, \; \beta \neq 0$. В выражение \eqref{eq:finite_volumes_density} для плотности потока входит $\partial \phi / \partial r$ в первой и третьей степени. Следовательно, мы ищем $g^{(a)}$, такую что $(g^{(a)})' = C_0 r^{-1/3}$. Ее первая степень даст нулевой вклад в поток, а третья -- конечный ненулевой. Проинтегрировав, получим $g^{(a)} = C_0 \cdot (3/2) \cdot r^{2/3} + C_1$. С учетом граничного условия выберем $g^{(a)} = r^{2/3}$. Вторую базисную функцию возьмем без особенности в точке $0$, например, $g^{(b)} = r$.

Пусть $\alpha \neq 0, \; \beta$ произвольное. Тогда в выражение \eqref{eq:finite_volumes_density} для плотности потока входит $\partial (\triangle \phi) / \partial r$. Встречается там и $\partial \phi / \partial r$, но она, согласно граничному условию, равна~$0$ при $r = 0$. Отыщем такую базисную функцию $g^{(a)}$, что $\partial (\triangle g^{(a)}) / \partial r = C_0 / r$. Учтем, что лапласиан функции $g^{(a)}$, зависящей только от $r$, в цилиндрических координатах вычисляется по формуле \eqref{eq:laplasian_common} с $k = 1$. Трижды проинтегрировав уравнение на производную лапласиана, получим $g^{(a)} = (C_0/4) r^2 (\ln r - 1) + C_1 r^2 / 4 + C_2 \ln r + C_3$. Требуется $g^{(a)}(0) = 0$, следовательно, $C_2 = C_3 = 0$. Особенность содержит первое слагаемое, поэтому выберем $g^{(a)} = r^2 (\ln r - 1)$. $g^{(a)}$ удовлетворяет граничному условию на производную $\partial g^{(a)} / \partial r|_{r = 0} = 0$. Второй базисной функцией сделаем $g^{(b)} = r^2$.

Наконец, рассмотрим сферический случай. Аналогично цилиндрическому случаю будем подбирать базисную функцию так, чтобы поток $F_{-1/2}$ был конечным. Для этого $\rho$ должна иметь особенность вида $1/r^2$, так как тогда $\rho(r) S(r) \to 3$ при $r \to 0$.

Пусть $\alpha = \beta = 0$. Аналогично рассмотренному ранее случаю ищем $g^{(a)}$, такую что $(g^{(a)})' = C_0 / r^2$. Проинтегрировав, получим $g^{(a)} = -C_0 / r + C_1$. $g^{(a)} \to \infty$ при $r \to +0$, то есть граничное условие $g^{(a)}(0) = 0$ выполнить невозможно.

Пусть $\alpha = 0, \; \beta \neq 0$. Аналогично рассмотренному ранее случаю ищем $g^{(a)}$, такую что $(g^{(a)})' = C_0 / r^{-2/3}$. Проинтегрировав, получим $g^{(a)} = C_0 \cdot 3 \cdot r^{1/3} + C_1$. С учетом граничного условия выберем $g^{(a)} = r^{1/3}, \; g^{(b)} = r$.

Пусть $\alpha \neq 0, \; \beta$ произвольное. Аналогично рассмотренному ранее случаю ищем $g^{(a)}$, такую что $\partial (\triangle g^{(a)}) / \partial r = C_0 / r^2$. Лапласиан функции $g^{(a)}$, зависящей только от $r$, в сферических координатах вычисляется по формуле \eqref{eq:laplasian_common} с $k = 2$. Дважды проинтегрировав уравнение на производную лапласиана, получим $(g^{(a)})' = -C_0 / 2 + C_1 r / 3 + C_2 / r^2$. $C_0 \neq 0$, так как иначе у функции пропадет особенность. Получается, что $(g^{(a)})'(0) \neq 0$ -- невозможно удовлетворить граничному условию на производную!

Примечательно, что выбранный подход не удалось применить в сферическом случае не только при $\alpha = \beta = 0$, но и при $\alpha \neq 0$. Выдвинем гипотезу, что в обеих этих конфигурациях исследуемая дифференциальная задача поставлена некорректно и не имеет решения.


\subsection{Разностная схема}

Везде в двух предыдущих подразделах допускалось, что радиусы границ ячеек $0 = r_{-1/2}, r_{1/2}, ..., r_{n - 1/2}, r_{n + 1/2} = R$ могут быть произвольной возрастающей последовательностью чисел. В дальнейшем, на практике, мы сделаем их структуру регулярной с шагом $h$: $r_{i - 1/2} = ih$.

При моделировании система будет проходить моменты времени $t_j = j \tau$, где $\tau$ -- фиксированный шаг по времени. Функцию времени с аргументом $t_j$ будем обозначать верхним индексом $j$.

В качестве начального условия задаются $\{ \widetilde{\phi}_i^0 \}_{i = 0}^n$.

Как было принято ранее, $k = 0$ в плоском случае, $k = 1$ в цилиндрическом, $k = 2$ в сферическом.

Подводя итог рассуждениям, проделанным в двух предыдущих подразделах, выпишем построенную методом конечных объемов разностную схему.

\begin{equation}
    \begin{gathered}
        \cfrac{1}{m} (\widetilde{\phi}_i^{j + 1} - \widetilde{\phi}_i^j) = \tau \cfrac{\Gamma}{l^2} f'(\widetilde{\phi}_i^j) + \cfrac{\tau}{dV_i} \Gamma (\rho_{i + 1/2}^j S_{i + 1/2} - \rho_{i - 1/2}^j S_{i - 1/2}) \; \text{при} \; i = \overline{0, n - 1} \tsemicolon \\
        \widetilde{\phi}_n^j = 1 \tsemicolon
    \end{gathered}
    \label{sch:volime_phi}
\end{equation}
$$dV_i = r_{i + 1/2}^{k + 1} - r_{i - 1/2}^{k + 1}; \qquad S_{i \pm 1/2} = (k + 1) r_{i \pm 1/2}^k \tsemicolon$$
\begin{equation}
    \rho_{i \pm 1/2}^j = \half \left[ \cfrac{\partial \phi}{\partial r} \right]_{i \pm 1/2}^j - \alpha \cfrac{l^2}{4} \left[ \cfrac{\partial (\triangle \phi)}{\partial r} \right]_{i \pm 1/2}^j + \beta l^2 \left( \left[ \cfrac{\partial \phi}{\partial r} \right]_{i \pm 1/2}^j \right)^3 \tsemicolon
    \label{sch:volume_density}
\end{equation}
\begin{equation}
    \left[ \cfrac{\partial \phi}{\partial r} \right]_{i + 1/2}^j = a_{i + 1/2}^j \; \text{при} \; i = \overline{1, n - 1} \tsemicolon
    \label{sch:volume_phi_derivative}
\end{equation}
$$\{\widetilde{\phi}_i^j\}_{i = 1}^n \leadsto \{a_{i + 1/2}^j\}_{i = 1}^{n - 1}, \; \text{где} \; g^{(a)} = r, \; g^{(b)} = 1 \tsemicolon$$
\begin{equation}
    \begin{gathered}
        \widetilde{\triangle \phi}_i^j = \cfrac{1}{dV_i} \left( \left[ \cfrac{\partial \phi}{\partial r} \right]_{i + 1/2}^j S_{i + 1/2} - \left[ \cfrac{\partial \phi}{\partial r} \right]_{i - 1/2}^j S_{i - 1/2} \right) \; \text{при} \; i = \overline{0, n - 1} \tsemicolon \\
        \widetilde{\triangle \phi}_n^j = 0 \tsemicolon
    \end{gathered}
    \label{sch:volume_laplasian}
\end{equation}
\begin{equation}
    \left[ \cfrac{\partial (\triangle \phi)}{\partial r} \right]_{i + 1/2}^j = c_{i + 1/2}^j \; \text{при} \; i = \overline{0, n - 1} \tsemicolon
    \label{sch:volume_laplasian_derivative}
\end{equation}
$$\{\widetilde{\triangle \phi}_i^j\}_{i = 0}^n \leadsto \{c_{i + 1/2}^j\}_{i = 0}^{n - 1}, \; \text{где} \; g^{(c)} = r, \; g^{(d)} = 1 \tsemicolon$$
\begin{equation}
    \left[ \cfrac{\partial \phi}{\partial r} \right]_{i - 1/2}^j = a_{1/2}^j (g_{1/2}^{(a)})'(r_{i - 1/2}) + b_{1/2}^j (g_{1/2}^{(b)})'(r_{i - 1/2}) \; \text{при} \; i = \overline{0, 1} \tsemicolon
    \label{sch:volume_phi_derivetive_border}
\end{equation}
\begin{equation}
    \left[ \cfrac{\partial (\triangle \phi)}{\partial r} \right]_{-1/2}^j = a_{1/2}^j \cfrac{\partial (\triangle g^{(a)})}{\partial r} \bigg|_{r = 0} + b_{1/2}^j \cfrac{\partial (\triangle g^{(b)})}{\partial r} \bigg|_{r = 0} \tsemicolon
    \label{sch:volume_laplasian_derivative_border}
\end{equation}
$$\{ \widetilde{\phi}_0^j, \widetilde{\phi}_1^j \} \leadsto \{ a_{1/2}^j, b_{1/2}^j \} \tpoint$$
1. Плоский случай при $\alpha = 0$:
\begingroup
\setlength{\abovedisplayskip}{5pt}
\setlength{\belowdisplayskip}{5pt}
\begin{align*}
    g_{1/2}^{(a)} &= r^2, & (g_{1/2}^{(a)})' &= 2r, & I^{(a)} &= \cfrac{1}{3} r^3 \bigg|_{...}^{...} \tsemicolon \\
    g_{1/2}^{(b)} &= r, & (g_{1/2}^{(b)})' &= 1, & I^{(b)} &= \half r^2 \bigg|_{...}^{...} \tpoint
\end{align*}
2. Плоский случай при $\alpha \neq 0$:
\begin{align*}
    g_{1/2}^{(a)} &= r^3, & (g_{1/2}^{(a)})' &= 3r^2, & \cfrac{\partial (\triangle g_{1/2}^{(a)})}{\partial r} &= 6, & I^{(a)} &= \cfrac{1}{4} r^4 \bigg|_{...}^{...} \tsemicolon \\
    g_{1/2}^{(b)} &= r^2, & (g_{1/2}^{(b)})' &= 2r, & \cfrac{\partial (\triangle g_{1/2}^{(b)})}{\partial r} &= 0, & I^{(b)} &= \cfrac{1}{3} r^3 \bigg|_{...}^{...} \tpoint
\end{align*}
3. Цилиндрический случай при $\alpha = 0, \; \beta \neq 0$:
\begin{align*}
g_{1/2}^{(a)} &= r^{2/3}, & (g_{1/2}^{(a)})' &= \cfrac{2}{3} r^{-1/3}, & I^{(a)} &= \cfrac{3}{4} r^{8/3} \bigg|_{...}^{...} \tsemicolon \\
g_{1/2}^{(b)} &= r, & (g_{1/2}^{(b)})' &= 1, & I^{(b)} &= \cfrac{2}{3} r^3 \bigg|_{...}^{...} \tsemicolon
\end{align*}
$$\rho_{-1/2}^j S_{-1/2} = \cfrac{16}{27} (a_{1/2}^j)^3 \tpoint$$
4. Цилиндрический случай при $\alpha \neq 0$:
\begin{align*}
    g_{1/2}^{(a)} &= r^2 (\ln r - 1), & (g_{1/2}^{(a)})' &= r (2 \ln r - 1), & \cfrac{\partial (\triangle g_{1/2}^{(a)})}{\partial r} &= \cfrac{4}{r}, & I^{(a)} &= \cfrac{1}{8} r^4 (4 \ln r - 5) \bigg|_{...}^{...} \tsemicolon \\
    g_{1/2}^{(b)} &= r^2, & (g_{1/2}^{(b)})' &= 2r, & \cfrac{\partial (\triangle g_{1/2}^{(a)})}{\partial r} &= 0, & I^{(b)} &= \half r^4 \bigg|_{...}^{...} \tsemicolon
\end{align*}
$$\left[ \cfrac{\partial (\triangle \phi)}{\partial r} \right]_{-1/2}^j S_{-1/2} = 8 a_{1/2}^j \tpoint$$
5. Сферический случай при $\alpha = 0, \; \beta \neq 0$:
\begin{align*}
    g_{1/2}^{(a)} &= r^{1/3}, & (g_{1/2}^{(a)})' &= \cfrac{1}{3} r^{-2/3}, & I^{(a)} &= \cfrac{9}{10} r^{10/3} \bigg|_{...}^{...}  \tsemicolon \\
    g_{1/2}^{(b)} &= r, & (g_{1/2}^{(b)})' &= 1, & I^{(b)} &= \cfrac{3}{4} r^4 \bigg|_{...}^{...} \tsemicolon
\end{align*}
$$\rho_{-1/2}^j S_{-1/2} = \cfrac{1}{9} (a_{1/2}^j)^3 \tpoint$$
\endgroup


\subsection{Вычислительный эксперимент}

Была написана программа, реализующая разностную схему \eqref{sch:volime_phi}, \eqref{sch:volume_density}, \eqref{sch:volume_phi_derivative}, \eqref{sch:volume_laplasian}, \eqref{sch:volume_laplasian_derivative}, \eqref{sch:volume_phi_derivetive_border}, \eqref{sch:volume_laplasian_derivative_border}.

Все предыдущие рассуждения были универсальны для плоского, цилиндрического и сферического случаев, насколько это возможно. Все три случая моделируются одной и той же программой, принимающей $k = \overline{0, 2}$ как параметр.

Зададим параметры модели:
$$\epsilon_0 = 0.2, \; \delta = 0.04, \; l = 1.0, \; \Gamma = 1.0, \; m = 0.5 \tpoint$$

Пусть $R = 5$. По виду графиков будет понятно, что для использованного набора параметров такое $R$ достаточно велико.

Выберем число ячеек $n = 300$ при $\alpha = 0$ и $n = 150$ иначе. Шаг по пространству~$h$, соответственно, равен $1/30 \approx 0.0333$ либо $1/60 \approx 0.0167$. При этом придется брать шаг по времени $\tau = 5 \cdot 10^{-7}$. Если шаг по времени кратно больше, то схема оказывается неустойчива и программа завершается с ошибкой переполнения (в данных возникает значение NaN). Как и ожидалось, $p$-лапласиан и билапласиан (особенно) требуют относительно малых шагов по времени.

Выбор начальных условий не имеет значения. Пусть, например, значения $0$ и $1$ гладко соединяются ветвью синусоиды: $\widetilde{\phi}_i^0 = \sin[(h/2 + ih) \pi / 2], \; h/2 + ih < 1$; далее все значения равны $1$.

Будем останавливать расчет, когда вектор <<воздействия>> на систему достаточно мал, а именно:
$$\max\limits_{i = 0}^n \cfrac{\widetilde{\phi}_i^{j + 1} - \widetilde{\phi}_i^j}{m} < 10^{-9} \tpoint$$
Все конфигурации модели достигли названного условия не более чем через $7.4$ единицы времени.

Результаты вычислений изображены на рис. \ref{fig:result_volumes} и \ref{fig:result_volumes_p} (случай 1 из описания разностной схемы), рис. \ref{fig:result_volumes_bi} (случай 2), рис. \ref{fig:result_volumes_cyl_p} (случай 3), рис. \ref{fig:result_volumes_cyl_bi} (случай 4), рис.~\ref{fig:result_volumes_sph_p} (случай~5). Графики функций показаны до зримого момента выхода на примерно постоянное значение $1$, после чего они в действительности продолжаются до $R = 5$. Графики состоят из соединенных значений средних $\widetilde{\phi}_i^j$, размещенных в серединах ячеек; вблизи~$0$ к ним добавлено несколько значений приближающей функции $g^{(a)}$, имеющей особенность в $0$, если того требует случай задачи.

\begin{figure}[!tp]
    \centering
    \includegraphics[width=0.89\textwidth]{figures/result_volumes.png}
    \vspace{-0.3cm}
    \caption{Решение $\phi$ в плоском случае при $\alpha = 0, \; \beta = 0$.}
    \label{fig:result_volumes}
    \vspace{0.5cm}

    \includegraphics[width=0.89\textwidth]{figures/result_volumes_p.png}
    \vspace{-0.3cm}
    \caption{Решение $\phi$ в плоском случае при $\alpha = 0, \; \beta = 1$.}
    \label{fig:result_volumes_p}
    \vspace{0.5cm}
    
    \includegraphics[width=0.89\textwidth]{figures/result_volumes_bi.png}
    \vspace{-0.3cm}
    \caption{Решение $\phi$ в плоском случае при $\alpha = 1, \; \beta = 0$.}
    \label{fig:result_volumes_bi}
\end{figure}

\begin{figure}[!tp]
    \centering
    \includegraphics[width=0.89\textwidth]{figures/result_volumes_cyl_p.png}
    \vspace{-0.3cm}
    \caption{Решение $\phi$ в цилиндрическом случае при $\alpha = 0, \; \beta = 1$.}
    \label{fig:result_volumes_cyl_p}
    \vspace{0.5cm}

    \includegraphics[width=0.89\textwidth]{figures/result_volumes_cyl_bi.png}
    \vspace{-0.3cm}
    \caption{Решение $\phi$ в цилиндрическом случае при $\alpha = 1, \; \beta = 0$.}
    \label{fig:result_volumes_cyl_bi}
    \vspace{0.5cm}
    
    \includegraphics[width=0.89\textwidth]{figures/result_volumes_sph_p.png}
    \vspace{-0.3cm}
    \caption{Решение $\phi$ в сферическом случае при $\alpha = 0, \; \beta = 1$.}
    \label{fig:result_volumes_sph_p}
\end{figure}

Отметим, что если в уравнения входит билапласиан ($\alpha \neq 0$), то функция $\phi$ может быть не монотонной и в некоторых точках превышать значение $1$ (см. рис. \ref{fig:result_volumes_bi}, \ref{fig:result_volumes_cyl_bi}). В работе \cite{zipunova_higher_codimension} это было отмечено и указано, что монотонность $\phi$ следует ожидать при достаточно малых $\alpha$.

Итак, эксперимент подтверждает, что, несмотря на некоторую громоздкость формулировок, предложенная модификация метода конечных объемов позволяет эффективно моделировать решение $\phi$, даже если на границе области оно имеет особенность.

%!TEX root = ../main.tex

\section{Conclusions}

In this paper we study stability properties of the phase-field model
for electrical breakdown channel evolution.
The central result is a classification of the
equilibrium solutions of the model and their stability.
From practical point of view, these results allows to
make meaningful conclusions regarding qualitative and quantitative
properties of the model. Particularly it was shown under which
conditions small perturbations of the equilibrium solutions
develop into channel-like structure typical for of electrical breakdown
process.

Besides this, a simple explicit finite-difference scheme
for solution of the model in spatially one-dimensional setting is considered.
The main question addressed here are stability conditions which guaranties
correctness of the simulations. Deep connections between
stability conditions of the model and the one of the
finite-difference scheme are shown.
The presented results of the numerical simulations confirms
predictions of the theoretical analysis of the model.

% EOF
\endinput

\newpage
\printbibliography

\newpage
\tableofcontents

\end{document}

%%%%%%%%%%%%%%%%%%%%%%%%%%%%%%%%%%%%%%%%%%%%%%%%%%%%%%%%%%%%%%%%%%%%%%%%%%%%%%%%