\documentclass[a4paper,12pt]{article}

%%% Размер шрифта
\usepackage[14pt]{extsizes}

%%% Поля
\usepackage[
	left=2cm,
	right=2cm,
	top=2cm,
	bottom=3cm,
	bindingoffset=0cm
]{geometry}

%%% Работа с русским языком
\usepackage{cmap}						% поиск в PDF
\usepackage{mathtext}					% русские буквы в формулах
\usepackage[T2A]{fontenc}				% кодировка
\usepackage[utf8]{inputenc}				% кодировка исходного текста
\usepackage[english,russian]{babel}		% локализация и переносы
\usepackage{indentfirst}
\frenchspacing

%%% Дополнительная работа с математикой
\usepackage{amsmath,amsfonts,amssymb,amsthm,mathtools}  % AMS

%%% Текст в колонки
\usepackage{multicol}

%%% Списки
\usepackage{enumitem}
\setlist{nosep, leftmargin=*}
\renewcommand{\labelenumi}{\arabic*)}

%%% Системы уравнений
\usepackage{cases}

%%% Таблицы
\usepackage{array}

%%% Рисунки
\usepackage{graphicx}
\usepackage{float}

%%% Точка в подписях к рисункам
\usepackage[labelsep=period]{caption}

%%% Список литературы
\bibliographystyle{bibliography_style/gost-numeric.bbx}
\usepackage[
	natbib = true,
	style = gost-numeric,
	sorting = none,
	backend = biber,
	language = autobib,
	autolang = other
]{biblatex}
\addbibresource{references.bib}

%%% Исправление символа номера при использовании gost-numeric.bbx
\usepackage{textcomp}
\DefineBibliographyStrings{russian}{number={\textnumero}}

%%% Гиперссылки
\usepackage[pdftex,unicode]{hyperref}

%%% Перенос знаков в формулах (по Львовскому)
\newcommand*{\hm}[1]{#1\nobreak\discretionary{}{\hbox{$\mathsurround=0pt #1$}}{}}


%%% Свои команды

\newcommand*{\No}{\textnumero}

\newcommand{\vect}[1]{\boldsymbol{#1}}
\newcommand{\vx}{{\vect{x}}}
\newcommand{\vn}{{\vect{n}}}

\newcommand{\half}{\cfrac{1}{2}}

\newcommand{\partt}[1]{\cfrac{\partial #1}{\partial t}}
\newcommand{\partx}[1]{\cfrac{\partial #1}{\partial x}}
\newcommand{\partxx}[1]{\cfrac{\partial^2 #1}{\partial x^2}}
\newcommand{\partvn}[1]{\cfrac{\partial #1}{\partial \vn}}

\newcommand{\partflt}[1]{\partial #1 / \partial t}
\newcommand{\partflx}[1]{\partial #1 / \partial x}
\newcommand{\partflxx}[1]{\partial^2 #1 / \partial x^2}
\newcommand{\partflvn}[1]{\partial #1 / \partial \vn}

\newcommand{\difftau}[1]{\cfrac{{#1}_j^{k + 1} - {#1}_j^k}{\tau}}
\newcommand{\diffhh}[1]{\cfrac{{#1}_{j + 1}^k - 2 {#1}_j^k + {#1}_{j - 1}^k}{h^2}}

\newcommand{\scalsq}[1]{\left( \nabla #1, \nabla #1 \right)}

\newcommand{\Natural}{{\mathbb{N}}}
\newcommand{\Real}{{\mathbb{R}}}
\newcommand{\bigO}{{\mathcal{O}}}
\newcommand{\clOmega}{{\overline{\Omega}}}

\newcommand{\norm}[1]{\| \, #1 \, \|}
\newcommand{\enorm}{{\| \cdot \|}}

\newcommand{\tabletopspace}{9mm}
\newcommand{\tablebottomspace}{3mm}

\newcommand{\tpoint}{{\text{.}}}
\newcommand{\tcomma}{{\text{,}}}
\newcommand{\tsemicolon}{{\text{;}}}

\newcommand{\unitm}{{\text{м}}}
\newcommand{\units}{{\text{с}}}
\newcommand{\unitJ}{{\text{Дж}}}
\newcommand{\unitC}{{\text{Кл}}}
\newcommand{\unitF}{{\text{Ф}}}

\newcommand{\forcehyphenation}{-\linebreak}


%%% Свои операторы
\DeclareMathOperator{\Div}{{div}}
\DeclareMathOperator{\Int}{{Int}}


%%% Оформление теорем

\theoremstyle{plain}
\newtheorem{theorem}{Теорема}
\newtheorem{proposition}{Утверждение}

\theoremstyle{remark}
\newtheorem{remark}{Замечание}


%%% Пояснение к меткам
% eq	-- equation
% cond	-- condition
% char	-- characteristic
% sch	-- scheme
% est	-- estimation
% exp	-- experiment
% fig	-- figure
% tab	-- table
% sec	-- section


%%% Описание препринта
\newcommand{\PreprintTitle}{
	Применение метода конечных объемов в модели развития канала электрического пробоя типа <<диффузной границы>>
}
\newcommand{\PreprintTitleEnglish}{
	Application of the finite volume method in a diffuse interface model for the electrical breakdown process
}
\newcommand{\PreprintAuthors}{
	А.~С.~Пономарев, Е.~В.~Зипунова, Е.~Б.~Савенков
}
\newcommand{\PreprintAuthorsEnglish}{
	A.~S.~Ponomarev, E.~V.~Zipunova, E.~B.~Savenkov
}


%%%%%%%%%%%%%%%%%%%%%%%%%%%%%%%%%%%%%%%%%%%%%%%%%%%%%%%%%%%%%%%%%%%%%%%%%%%%%%%%

\begin{document}

%%%%%%%%%%%%%%%%%%%%%%%%%%%%%%%%%%%%%%
\begin{titlepage}

\begin{center}
	РОССИЙСКАЯ АКАДЕМИЯ НАУК \\
	ОРДЕНА ЛЕНИНА \\
	ИНСТИТУТ ПРИКЛАДНОЙ МАТЕМАТИКИ \\
	имени М. В. КЕЛДЫША \\

	\vspace*{60mm}
	\Large{\PreprintAuthors} \\
	\vspace*{20mm}
	\textbf{\large \PreprintTitle} \\
	\vspace*{110mm}
	\Large{Москва, 2024}
	\vspace*{-50mm}
\end{center}

\end{titlepage}
%%%%%%%%%%%%%%%%%%%%%%%%%%%%%%%%%%%%%%%

\setcounter{page}{2}

\thispagestyle{empty}

\noindent \emph{\PreprintAuthors}, \PreprintTitle \\[3mm]
\textbf{Аннотация} \\
{
	\small
	В настоящей работе исследуется применение метода конечных объемов при отыскании стационарных решений в модели типа диффузной границы, описывающей развитие канала пробоя в твердом диэлектрике. Задача решается в трех различных случаях: плоском, цилиндрическом и сферическом. При построении разностной схемы особое внимание уделено заданию граничных условий на внутренней границе области моделирования, где решение системы уравнений может иметь особенность. \\[3mm]
	\textbf{Ключевые слова:} модель типа диффузной границы, фазовое поле, метод конечных объемов, электрический пробой. \\[5mm]
}
\begin{otherlanguage}{english}
\emph{\PreprintAuthorsEnglish}, \PreprintTitleEnglish \\[3mm]
\textbf{Abstract} \\
{
	\small
	[Annotaciya na angliyskom yazeake] \\[3mm]
	\textbf{Key words and phrases:} diffuse interface model, phase field, finite volume method, electrical breakdown. \\[5mm]
}
\end{otherlanguage}

\clearpage

%!TEX root = ../main.tex

\section{Введение}

Электрический пробой~-- это явление резкого возрастания тока в диэлектрике при приложении электрического напряжения выше некоторого критического значения. Механизм разрушения диэлектрика под действием электрического поля сложен и многообразен: оно может иметь различные причины, характер развития, сопутствующие физические процессы \cite{vorobiev_dielectric_physics}.

Среди многообразия математических моделей, созданных для описания развития канала электрического пробоя, выделим предложенную в работе \cite{pitike_dielectric_breakdown} модель типа диффузной границы.

В настоящее время модели типа диффузной границы составляют целый класс подходов для решения задач в различных областях науки и техники. В частности, описанная в работе \cite{pitike_dielectric_breakdown} модель построена как формальное обобщение ранее известных моделей типа диффузной границы, применяемых в теории трещин.

Исследование и дальнейшее развитие упомянутой модели можно найти в работах \cite{zipunova_higher_codimension, zipunova_conservative, zipunova_thermomechanical, ponomarev_stability}. Основные положения метода диффузной границы в применении к моделированию развития канала электрического пробоя перечислены в работе \cite{ponomarev_stability}.

Модели типа диффузной границы используются для описания систем, в которых вещество может находиться в нескольких различных состояниях~-- фазах,~-- причем вещество в одной и той же фазе образует некоторые однородные области. В моделях типа диффузной границы распределение фаз вещества задается гладкой функцией $\phi$~-- фазовым полем,~-- которая в каждой области однородности близка к постоянной. Характерная толщина разделяющего слоя (<<диффузной границы>>) и, соответственно, скорость изменения~$\phi$ при переходе от одной фазы к другой определяется параметрами модели.

В работе \cite{zipunova_higher_codimension} проводится исследование свойства упомянутой модели развития канала электрического пробоя, которое можно назвать коразмерностью <<включений>>. Для задач теории трещин естественным будет двумерное включение (плоская трещина) в трехмерной среде вещества~-- в таком случае говорят, что коразмерность объекта равна 1. Обратим внимание, что, хотя исследуемая модель, как было сказано, получена на основе моделей из теории трещин, для нее характерным будет одномерное включение (канал пробоя), то есть имеющее коразмерность 2. В работе \cite{zipunova_higher_codimension} указано, что это может привести к нетривиальным последствиям, и предложено определенное обобщение исходной модели, которое предположительно делает ее более адекватной.

Суть обобщения состоит в формальном добавлении в уравнения модели двух слагаемых высших порядков с некоторыми коэффициентами. Целью настоящей работы является численная проверка поведения модели при различных значениях коэффициентов. Для этого ищется стационарное распределение фазового поля $\phi$ в нескольких характеристических случаях. Построение разностной схемы для задачи несет определенные сложности, связанные с необходимостью задать граничные условия на множествах коразмерности~2 и~3 в трехмерном пространстве. Предполагается, что точках этих множеств функция фазового поля $\phi$ имеет особенность.

Авторами применена модификация метода конечных объемов. Для части конфигураций обобщенной модели она позволила составить разнотную схему. Создана компьютерная программа, реализующая схему; проделаны расчеты, их результаты приведены в виде графиков. Для остальных конфигураций модели в процессе применения метода возникли фундаментальные проблемы, что позволяет выдвинуть гипотезу о некорректной постановке дифференциальной задачи в этих случаях.

%!TEX root = ../main.tex

\section{Постановка задачи и модель}
\label{sec:problem_and_model}

Приведем краткое описание математической модели, предложенной в работе \cite{pitike_dielectric_breakdown}. Подробное описание модели и физического смысла ее уравнений и параметров можно найти в работе \cite{ponomarev_stability}.

Рассматривается ограниченная область пространства $\Omega \subset \Real^3$. Распределение фаз вещества в ней задается гладкой функцией $\phi: \Omega \times [0, +\infty)_t \hm \to [0, 1], \; \phi(\vx, t)$~-- фазовым полем; вещество может находиться в одной из двух фаз: $\phi \approx 1$~-- <<неповрежденное>>, $\phi \approx 0$~-- <<полностью разрушенное>> (то есть относящееся к каналу пробоя),~-- а также в промежуточных состояниях в зоне диффузной границы.

Диэлектрическую проницаемость среды $\epsilon$ предлагается описать следующей формулой:
\begin{equation}
	\epsilon(\vx, t) = \epsilon[\phi] = \cfrac{\epsilon_0(\vx)}{f(\phi(\vx, t)) + \delta} \tpoint
	\label{eq:epsilon}
\end{equation}
Здесь $\epsilon_0(\vx)$~-- диэлектрическая проницаемость неповрежденной среды, $f(\phi) \hm = 4\phi^3 - 3\phi^4$~-- интерполирующая функция, $0 < \delta \ll 1$~-- регуляризующий параметр.

Помимо фазового поля $\phi$, состояние системы описывает также функция $\Phi: \Omega \times [0, +\infty)_t \to \Real, \; \Phi(\vx, t)$~-- потенциал электрического поля.

Постулируется следующее выражение для свободной энергии системы $\Pi$:
\begin{gather*}
	\Pi = \int \limits_\Omega \pi d \vx \tcomma \\
	\pi = -\half \epsilon[\phi] \scalsq{\Phi} + \Gamma \cfrac{1 - f(\phi)}{l^2} + \cfrac{\Gamma}{4} \scalsq{\phi} \tpoint
\end{gather*}
Здесь $\Gamma > 0, \; l > 0$~-- числовые параметры модели, константы.

Постулируются два уравнения, определяющие динамику системы:
\begin{equation*}
\begin{cases}
	\cfrac{\delta \Pi}{\delta \Phi} = 0 \tsemicolon \\[3mm]
	\cfrac{1}{m} \partt{\phi} = -\cfrac{\delta \Pi}{\delta \phi} \tpoint
\end{cases}
\end{equation*}
Здесь константа $m > 0$~-- числовой параметр модели, называемый подвижностью. Говоря нестрого, согласно первому уравнению электрический потенциал $\Phi$ распределяется так, чтобы свободная энергия была минимальной; согласно второму~-- фазовое поле $\phi$ с определенной скоростью стремится к тому, чтобы свободная энергия была минимальной.

Отыскав явно вариационные производные в двух уравнениях выше, получим следующую систему уравнений:
\begin{numcases}{}
	\Div(\epsilon[\phi] \nabla \Phi) = 0 \tsemicolon
	\label{eq:Phi} \\
	\cfrac{1}{m} \partt{\phi} = \half \epsilon'(\phi) \scalsq{\Phi} + \cfrac{\Gamma}{l^2} f'(\phi) + \half \Gamma \triangle \phi \tpoint
	\label{eq:phi}
\end{numcases}
Здесь $(\cdot)' \equiv (\cdot)_\phi'$. Система состоит из двух уравнений: на $\phi$ и $\Phi$ соответственно; система связная, второе уравнение нелинейное, является уравнением типа Аллена--Кана.

%!TEX root = ../main.tex

\section{Суть проблемы}

Как говорилось ранее, исследуемая модель канала электрического пробоя, предложенная в работе \cite{pitike_dielectric_breakdown}, создана на основе подобных моделей из теории трещин. Учитывая это, ознакомимся с анализом системы уравнений \eqref{eq:Phi}, \eqref{eq:phi} в следующем характеристическом случае.

Пусть $\Phi \equiv 0$, что соответствует нулевому электрическому напряжению в системе; тогда уравнение \eqref{eq:Phi} выполнено тождественно. В уравнении \eqref{eq:phi} первое слагаемое тождественно равно нулю, так как $\nabla \Phi \equiv 0$. В дальнейшем будем искать стационарное во времени распределение фазового поля, так что $\phi'_t \equiv 0$. В таком случае задача сводится к следующему уравнению на фазовое поле $\phi$:
\begin{equation}
    \cfrac{2}{l^2} f'(\phi) + \triangle \phi = 0 \tpoint
    \label{eq:stationary}
\end{equation}

Пусть задача решается в замкнутой области $\clOmega = [0, +\infty)_x \times I_y \times J_z$, где $I$ и $J$ -- некоторые отрезки. Пусть $\epsilon_0(\vx) = \epsilon_0(x)$, то есть диэлектрическая проницаемость неповрежденной среды зависит только от $x$. Будем искать стационарное решение следующей краевой задачи: $\phi|_{x = 0} = 0, \; \phi \to 1$ при $x \to +\infty$, а также $\partflvn{\phi} = 0$ на <<гранях>> области $\clOmega$, перпендикулярных осям $y$ и $z$. Второе условие интуитивно означает <<однородность>> системы по $y$ и $z$; символом $\partflvn{}$ обозначена производная по вектору нормали $\vn$ к границе $\clOmega$. Учитывая описанные краевые условия, будем искать решение уравнения \eqref{eq:stationary}, имеющее $\phi(\vx) = \phi(x)$, то есть полагая, что $\phi$ зависит только от пространственной переменной $x$.

С учетом описанных допущений уравнение \eqref{eq:stationary} принимает вид
$$\cfrac{\partial^2 \phi}{\partial x^2} = -\cfrac{2}{l^2} f'(\phi) \tpoint$$
Домножим обе части уравнения на $\phi'_x$. Учитывая, что $f'_\phi \phi'_x = f'_x$ и $2 \phi'_x \phi''_{xx} = [(\phi'_x)^2]'_x$, проинтегрируем уравнение. При $x \to +\infty$ согласно граничному условию $\phi \to 1$; естественно также считать, что при этом $\phi'_x \to 0$. С учетом этого получим:
\begin{equation}
    \cfrac{\partial \phi}{\partial x} = \cfrac{2}{l} \sqrt{1 - f(\phi)} \tpoint
    \label{eq:stationary_rectangular}
\end{equation}
Итак, мы перешли к обыкновенной задаче Коши с уравнением \eqref{eq:stationary_rectangular} и условием $\phi(0) = 0$, решение которой существует и единственно.

Рассмотренный случай системы имеет следующий смысл: найдено распределение фазового поля в полупространстве сбоку от проводящей пластины (состоящей из полностью разрушенного вещества). Этот случай был ранее назван характеристическим, так как показывает влияние на систему параметра $l$: видно, что $l$ в уравнении~\eqref{eq:stationary_rectangular} есть коэффициент <<растяжения>> решения вдоль оси $x$. Можно показать \cite{zipunova_higher_codimension}, что при $x > l$ решение рассматриваемой задачи Коши $\phi \approx 1$. Другими словами, ее решение есть распределение фазового поля, локализованное на отрезке $[0, l]$.

Подобный анализ вполне подходит для задачи из теории трещин (вместо проводящей пластины была бы плоская трещина). Однако характерный канал пробоя -- объект не двумерный, а одномерный. Проверим, можно ли провести аналогичное рассуждение не для пластины, а для тонкого прямого проводника.

Как и ранее, электрическое напряжение нулевое -- $\Phi \equiv 0$. Рассмотрим уравнение \eqref{eq:stationary} в области $\clOmega = \Real_x \times \Real_y \times J_z$, где $J$ -- некоторый отрезок. Аналогично будем искать решение следующей краевой задачи: $\phi|_{x, y = 0} = 0, \; \phi \to 1$ при $r = \sqrt{x^2 + y^2} \to +\infty$, а также $\partflvn{\phi} = 0$ на <<гранях>> области $\clOmega$, перпендикулярных оси $z$.

Удобно перейти в цилиндрическую систему координат: $x, y, z \mapsto r, \theta, z$. Граничные условия однородны по $\theta$ и $z$, поэтому естественно искать решение, зависящее только от $r$. Так как $\phi(\vx) = \phi(r)$, выражение для лапласиана $\phi$ в цилиндрических координатах принимает вид
$$\triangle \phi = \cfrac{1}{r} \cfrac{\partial}{\partial r} \left( r \cfrac{\partial \phi}{\partial r} \right) = \cfrac{1}{r} \cfrac{\partial \phi}{\partial r} + \cfrac{\partial ^2 \phi}{\partial r^2} \tpoint$$
С учетом этого уравнение \eqref{eq:stationary} преобразуется в
\begin{equation}
    \cfrac{2}{l^2} f'(\phi) + \cfrac{1}{r} \cfrac{\partial \phi}{\partial r} + \cfrac{\partial ^2 \phi}{\partial r^2} = 0 \tpoint
    \label{eq:stationary_cylindrical}
\end{equation}

Подобное рассуждение проделано в работе \cite{zipunova_higher_codimension}; за ним следует анализ уравнения~\eqref{eq:stationary_cylindrical}. На основании теоретических результатов из работы \cite{cirstea_elliptic_equations} заключается, что поставленная краевая задача некорректна и решения не имеет. Даже на уровне интуиции постановка задачи выглядит необычно: условие $\phi|_{x, y = 0} = 0$ задано не на двумерной, а на одномерной <<внутренней>> границе области $\Omega$.

Возникает желание формально изменить модель так, чтобы описанная краевая задача имела решение. При моделировании канал пробоя невозможно явно представить сегментом линии, за исключением тривиальных случаев, -- однако естественно считать его <<нитевидной>> областью соответствующей формы, радиус которой может стремиться к нулю. Распределение фазового поля вблизи канала пробоя в таком случае должно приближаться к решению рассмотренной краевой задачи.

%!TEX root = ../main.tex

\section{Обобщение модели}

\subsection{Суть проблемы}
\label{ssec:matter_of_problem}

Как говорилось ранее, исследуемая модель канала электрического пробоя, предложенная в работе \cite{pitike_dielectric_breakdown}, создана на основе подобных моделей из теории трещин. Учитывая это, ознакомимся с анализом системы уравнений \eqref{eq:Phi}, \eqref{eq:phi} в следующем характеристическом случае.

Пусть $\Phi \equiv 0$, что соответствует нулевому электрическому напряжению в системе; тогда уравнение \eqref{eq:Phi} выполнено тождественно. В уравнении \eqref{eq:phi} первое слагаемое тождественно равно нулю, так как $\nabla \Phi \equiv 0$. В дальнейшем будем искать стационарное во времени распределение фазового поля, так что $\phi'_t \equiv 0$. В таком случае задача сводится к следующему уравнению на фазовое поле $\phi$:
\begin{equation}
	\cfrac{2}{l^2} f'(\phi) + \triangle \phi = 0 \tpoint
	\label{eq:stationary}
\end{equation}

Пусть задача решается в замкнутой области $\clOmega = [0, +\infty)_x \times I_y \times J_z$, где $I$ и $J$ -- некоторые отрезки. Пусть $\epsilon_0(\vx) = \epsilon_0(x)$, то есть диэлектрическая проницаемость неповрежденной среды зависит только от $x$. Будем искать стационарное решение следующей краевой задачи: $\phi|_{x = 0} = 0, \; \phi \to 1$ при $x \to +\infty$, а также $\partflvn{\phi} = 0$ на <<гранях>> области $\clOmega$, перпендикулярных осям $y$ и $z$. Второе условие интуитивно означает <<однородность>> системы по $y$ и $z$; символом $\partflvn{}$ обозначена производная по вектору нормали $\vn$ к границе $\clOmega$. Учитывая описанные краевые условия, будем искать решение уравнения \eqref{eq:stationary}, имеющее $\phi(\vx) = \phi(x)$, то есть полагая, что $\phi$ зависит только от пространственной переменной $x$.

С учетом описанных допущений уравнение \eqref{eq:stationary} принимает вид
$$\partxx{\phi} = -\cfrac{2}{l^2} f'(\phi) \tpoint$$
Домножим обе части уравнения на $\phi'_x$. Учитывая, что $f'_\phi \phi'_x = f'_x$ и $2 \phi'_x \phi''_{xx} = [(\phi'_x)^2]'_x$, проинтегрируем уравнение. При $x \to +\infty$ согласно граничному условию $\phi \to 1$; естественно также считать, что при этом $\phi'_x \to 0$. С учетом этого получим:
\begin{equation}
	\partx{\phi} = \cfrac{2}{l} \sqrt{1 - f(\phi)} \tpoint
	\label{eq:stationary_rectangular}
\end{equation}
Итак, мы перешли к обыкновенной задаче Коши с уравнением \eqref{eq:stationary_rectangular} и условием $\phi(0) = 0$, решение которой существует и единственно.

Рассмотренный случай системы имеет следующий смысл: найдено распределение фазового поля в полупространстве сбоку от проводящей пластины (состоящей из полностью разрушенного вещества). Этот случай был ранее назван характеристическим, так как показывает влияние на систему параметра $l$: видно, что $l$ в уравнении~\eqref{eq:stationary_rectangular} есть коэффициент <<растяжения>> решения вдоль оси $x$. Можно показать \cite{zipunova_higher_codimension}, что при $x > l$ решение рассматриваемой задачи Коши $\phi \approx 1$. Другими словами, ее решение есть распределение фазового поля, локализованное на отрезке $[0, l]$.

Подобный анализ вполне подходит для задачи из теории трещин (вместо проводящей пластины была бы плоская трещина). Однако характерный канал пробоя -- объект не двумерный, а одномерный. Проверим, можно ли провести аналогичное рассуждение не для пластины, а для тонкого прямого проводника.

Как и ранее, электрическое напряжение нулевое -- $\Phi \equiv 0$. Рассмотрим уравнение \eqref{eq:stationary} в области $\clOmega = \Real_x \times \Real_y \times J_z$, где $J$ -- некоторый отрезок. Аналогично будем искать решение следующей краевой задачи: $\phi|_{x, y = 0} = 0, \; \phi \to 1$ при $r = \sqrt{x^2 + y^2} \to +\infty$, а также $\partflvn{\phi} = 0$ на <<гранях>> области $\clOmega$, перпендикулярных оси $z$.

Удобно перейти в цилиндрическую систему координат: $x, y, z \mapsto r, \theta, z$. Граничные условия однородны по $\theta$ и $z$, поэтому естественно искать решение, зависящее только от $r$. Так как $\phi(\vx) = \phi(r)$, выражение для лапласиана $\phi$ в цилиндрических координатах принимает вид
$$\triangle \phi = \cfrac{1}{r} \partr{} \left( r \partr{\phi} \right) = \cfrac{1}{r} \partr{\phi} + \partrr{\phi} \tpoint$$
С учетом этого уравнение \eqref{eq:stationary} преобразуется в
\begin{equation}
	\cfrac{2}{l^2} f'(\phi) + \cfrac{1}{r} \partr{\phi} + \partrr{\phi} = 0 \tpoint
	\label{eq:stationary_cylindrical}
\end{equation}

Подобное рассуждение проделано в работе \cite{zipunova_higher_codimension}; за ним следует анализ уравнения~\eqref{eq:stationary_cylindrical}. На основании теоретических результатов из работы \cite{cirstea_elliptic_equations} заключается, что поставленная краевая задача некорректна и решения не имеет. Даже на уровне интуиции постановка задачи выглядит необычно: условие $\phi|_{x, y = 0} = 0$ задано не на двумерной, а на одномерной <<внутренней>> границе области $\Omega$.

Возникает желание формально изменить модель так, чтобы описанная краевая задача имела решение. При моделировании канал пробоя невозможно явно представить сегментом линии, за исключением тривиальных случаев, -- однако естественно считать его <<нитевидной>> областью соответствующей формы, радиус которой может стремиться к нулю. Распределение фазового поля вблизи канала пробоя в таком случае должно приближаться к решению рассмотренной краевой задачи.


\subsection{Предложенное обобщение}

В ответ на описанную в предыдущем подразделе проблему в работе \cite{zipunova_higher_codimension}, на основании теоретических результатов работ \cite{sobolev_functional_analysis}, \cite{oleynik_biharmonic_equations}, \cite{sternin_elliptic_equations}, \cite{lewis_quasi_linear}, предлагается следующая обобщенная модель, для которой постановка условий на границах размерности 1 (соответственно, коразмерности 2) является математически корректной:
\begin{gather}
	\Pi = \int \limits_\Omega \pi d \vx \tcomma
	\label{eq:energy_corrected} \\
	\begin{aligned}
		\pi = -\half \epsilon[\phi] \scalsq{\Phi} + \Gamma \cfrac{1 - f(\phi)}{l^2} & + \cfrac{\Gamma}{4} \scalsq{\phi} + \\ & + \alpha \cfrac{\Gamma l^2}{8} (\triangle \phi)^2 + \beta \cfrac{1}{p} \Gamma l^{p - 2} \norm{\nabla \phi}_2^p \tsemicolon
	\end{aligned}
	\label{eq:energy_density_corrected}
\end{gather}
\begin{numcases}{}
	\Div(\epsilon[\phi] \nabla \Phi) = 0 \tsemicolon
	\label{eq:Phi_corrected} \\
	\begin{aligned}
		\cfrac{1}{m} \partt{\phi} = \half \epsilon'(\phi) \scalsq{\Phi} & + \cfrac{\Gamma}{l^2} f'(\phi) + \half \Gamma \triangle \phi \: - \\ & - \alpha \cfrac{\Gamma l^2}{4} \bilapl{\phi} + \beta \Gamma l^{p - 2} \plapl{\phi}{p - 2} \tpoint
	\end{aligned}
	\label{eq:phi_corrected}
\end{numcases}
Здесь $\alpha, \beta \geqslant 0$ -- некоторые константы, $p$ -- четное натуральное число, не меньшее~4. Дифференциальный оператор $\plapl{\phi}{p - 2}$ принято называть \emph{$p$-лапласианом}, $\bilapl{\phi} = \triangle(\triangle \phi)$ -- \emph{билапласианом}. В дальнейшем для простоты будем считать $p = 4$.

%!TEX root = ../main.tex

\section{Заключение}

Настоящая работа продолжает исследование, начатое в статье \cite{zipunova_higher_codimension}. Как было отмечено ее авторами, исследование это хотя и проводится для конкретной задачи, но, вероятно, затрагивает вопросы, содержащиеся в методе диффузной границы как таковом. Суть этих вопросов в том, позволяют ли уравнения среды с диффузной границей в своей <<классической>> редакции адекватно описывать включения, по своей природе являющиеся объектами высшей коразмерности. В качестве возможного ответа авторы работы \cite{zipunova_higher_codimension} предлагают определенного вида обобщение исходной модели.

Целью настоящей работы было численно исследовать упомянутое обобщение. В этом достигнуты определенные успехи. С помощью модификации метода конечных объемов преодолены трудности, связанные с необходимостью задавать граничные условия на множествах коразмерности 2 и 3 в трехмерном пространстве и с наличием у функции-решения  особенности в точках этих множеств. Указанный подход существенно не привязан к рассматриваемой модели -- в дальнейшем он может быть использован и в других задачах.

В некоторых случаях при построении разностной схемы возникали фундаментальные препятствия: оказывалось, что необходимых базисных функций попросту не существует. На основании этого выдвинута гипотеза, что в указанных случаях рассматриваемая дифференциальная задача поставлена некорректно и не имеет решения. Рассуждения вполне согласуется с теоретическими результатами работы \cite{zipunova_higher_codimension}. В будущем возможно строгое обоснование представленной гипотезы.

\clearpage
\printbibliography

\clearpage
\tableofcontents

\end{document}

%%%%%%%%%%%%%%%%%%%%%%%%%%%%%%%%%%%%%%%%%%%%%%%%%%%%%%%%%%%%%%%%%%%%%%%%%%%%%%%%