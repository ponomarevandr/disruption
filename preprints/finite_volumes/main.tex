\documentclass[a4paper,12pt]{article}

%%% Размер шрифта
\usepackage[14pt]{extsizes}

%%% Поля
\usepackage[
	left=2cm,
	right=2cm,
	top=2cm,
	bottom=3cm,
	bindingoffset=0cm
]{geometry}

%%% Работа с русским языком
\usepackage{cmap}						% поиск в PDF
\usepackage{mathtext}					% русские буквы в формулах
\usepackage[T2A]{fontenc}				% кодировка
\usepackage[utf8]{inputenc}				% кодировка исходного текста
\usepackage[english,russian]{babel}		% локализация и переносы
\usepackage{indentfirst}
\frenchspacing

%%% Дополнительная работа с математикой
\usepackage{amsmath,amsfonts,amssymb,amsthm,mathtools}  % AMS

%%% Текст в колонки
\usepackage{multicol}

%%% Списки
\usepackage{enumitem}
\setlist{nosep, leftmargin=*}
\renewcommand{\labelenumi}{\arabic*)}

%%% Системы уравнений
\usepackage{cases}

%%% Таблицы
\usepackage{array}

%%% Рисунки
\usepackage{graphicx}
\usepackage{float}

%%% Точка в подписях к рисункам
\usepackage[labelsep=period]{caption}

%%% Список литературы
\bibliographystyle{bibliography_style/gost-numeric.bbx}
\usepackage[
	natbib = true,
	style = gost-numeric,
	sorting = none,
	backend = biber,
	language = autobib,
	autolang = other
]{biblatex}
\addbibresource{references.bib}

%%% Исправление символа номера при использовании gost-numeric.bbx
\usepackage{textcomp}
\DefineBibliographyStrings{russian}{number={\textnumero}}

%%% Гиперссылки
\usepackage[pdftex,unicode]{hyperref}

%%% Перенос знаков в формулах (по Львовскому)
\newcommand*{\hm}[1]{#1\nobreak\discretionary{}{\hbox{$\mathsurround=0pt #1$}}{}}


%%% Свои команды

\newcommand*{\No}{\textnumero}

\newcommand{\vect}[1]{\boldsymbol{#1}}
\newcommand{\vx}{{\vect{x}}}
\newcommand{\vn}{{\vect{n}}}
\newcommand{\vrho}{{\vect{\rho}}}

\newcommand{\aver}[1]{\widetilde{#1}}
\newcommand{\avphi}{{\aver{\phi}}}
\newcommand{\avpsi}{{\aver{\psi}}}
\newcommand{\avlaplphi}{{\aver{\Delta \phi}}}

\newcommand{\ga}{{g^{(a)}}}
\newcommand{\gb}{{g^{(b)}}}

\newcommand{\half}{\cfrac{1}{2}}

\newcommand{\partderiv}[2]{\cfrac{\partial #1}{\partial #2}}
\newcommand{\partsecondderiv}[2]{\cfrac{\partial^2 #1}{\partial {#2}^2}}
\newcommand{\partt}[1]{\partderiv{#1}{t}}
\newcommand{\partx}[1]{\partderiv{#1}{x}}
\newcommand{\partxx}[1]{\partsecondderiv{#1}{x}}
\newcommand{\partvn}[1]{\partderiv{#1}{\vn}}
\newcommand{\partr}[1]{\partderiv{#1}{r}}
\newcommand{\partrr}[1]{\partsecondderiv{#1}{r}}

\newcommand{\partflderiv}[2]{\partial #1 / \partial #2}
\newcommand{\partflsecondderiv}[2]{\partial^2 #1 / \partial {#2}^2}
\newcommand{\partflt}[1]{\partflderiv{#1}{t}}
\newcommand{\partflx}[1]{\partflderiv{#1}{x}}
\newcommand{\partflxx}[1]{\partflsecondderiv{#1}{x}}
\newcommand{\partflvn}[1]{\partflderiv{#1}{\vn}}
\newcommand{\partflr}[1]{\partflderiv{#1}{r}}
\newcommand{\partflrr}[1]{\partflsecondderiv{#1}{r}}

\newcommand{\scalsq}[1]{\left( \nabla #1, \nabla #1 \right)}

\newcommand{\norm}[1]{\| \, #1 \, \|}
\newcommand{\enorm}{{\| \cdot \|}}

\newcommand{\plapl}[2]{\Div(\norm{\nabla #1}_2^{#2} \nabla #1)}
\newcommand{\bilapl}[1]{\Delta^2 #1}

\newcommand{\intt}{{\int\limits_{t_j}^{t_{j + 1}}}}
\newcommand{\intOmega}{{\int\limits_{\Omega_i}}}

\newcommand{\gridfunc}[1]{\left[ \cfrac{\partial #1}{\partial r} \right]}
\newcommand{\gridpartphi}{{\gridfunc{\phi}}}
\newcommand{\gridpartlaplphi}{{\gridfunc{(\Delta \phi)}}}

\newcommand{\Natural}{{\mathbb{N}}}
\newcommand{\Real}{{\mathbb{R}}}
\newcommand{\bigO}{{\mathcal{O}}}
\newcommand{\clOmega}{{\overline{\Omega}}}

\newcommand{\anylimits}{{\bigg|_{...}^{...}}}

\newcommand{\tpoint}{{\text{.}}}
\newcommand{\tcomma}{{\text{,}}}
\newcommand{\tsemicolon}{{\text{;}}}

\newcommand{\forcehyphenation}{-\linebreak}

\newcommand{\multeqstart}{
	\begingroup
	\setlength{\abovedisplayshortskip}{\the\abovedisplayskip}
	\setlength{\belowdisplayshortskip}{\the\belowdisplayskip}
}
\newcommand{\multeqnext}{
	\vspace{-7mm}
}
\newcommand{\multeqfinish}{
	\endgroup
}


%%% Свои операторы
\DeclareMathOperator{\Div}{{div}}


%%% Оформление теорем

\theoremstyle{plain}
\newtheorem{theorem}{Теорема}
\newtheorem{proposition}{Утверждение}

\theoremstyle{remark}
\newtheorem{remark}{Замечание}


%%% Пояснение к меткам
% eq	-- equation
% cond	-- condition
% char	-- characteristic
% sch	-- scheme
% est	-- estimation
% exp	-- experiment
% fig	-- figure
% tab	-- table
% sec	-- section
% ssec	-- subsection


%%% Описание препринта
\newcommand{\PreprintTitle}{%
	Численное исследование обобщения модели развития канала электрического пробоя типа <<диффузной границы>>%
}
\newcommand{\PreprintTitleFormatted}{
	Численное исследование обобщения \\ модели развития канала электрического пробоя \\ типа <<диффузной границы>>
}
\newcommand{\PreprintTitleEnglish}{%
	Numerical analysis of a generalized diffuse interface model for the electrical breakdown process%
}
\newcommand{\PreprintAuthors}{%
	А.~С.~Пономарев, Е.~В.~Зипунова, Е.~Б.~Савенков%
}
\newcommand{\PreprintAuthorsEnglish}{%
	A.~S.~Ponomarev, E.~V.~Zipunova, E.~B.~Savenkov%
}


%%%%%%%%%%%%%%%%%%%%%%%%%%%%%%%%%%%%%%%%%%%%%%%%%%%%%%%%%%%%%%%%%%%%%%%%%%%%%%%%

\begin{document}

%%%%%%%%%%%%%%%%%%%%%%%%%%%%%%%%%%%%%%
\begin{titlepage}

\begin{center}
	РОССИЙСКАЯ АКАДЕМИЯ НАУК \\
	ОРДЕНА ЛЕНИНА \\
	ИНСТИТУТ ПРИКЛАДНОЙ МАТЕМАТИКИ \\
	имени М. В. КЕЛДЫША \\

	\vspace*{60mm}
	{
		\Large{\PreprintAuthors} \\
	}
	\vspace*{20mm}
	{
		\large \textbf{\PreprintTitleFormatted} \\
	}
	\vspace*{110mm}
	\Large{Москва, 2024}
	\vspace*{-50mm}
\end{center}

\end{titlepage}
%%%%%%%%%%%%%%%%%%%%%%%%%%%%%%%%%%%%%%%

\setcounter{page}{2}

\thispagestyle{empty}

\noindent \emph{\PreprintAuthors,} \PreprintTitle \\[3mm]
\textbf{Аннотация} \\
{
	\small
	В настоящей работе проводится численное исследование ранее предложенного обобщения модели типа диффузной границы, описывающей развитие канала электрического пробоя в твердом диэлектрике. Для этого ищется стационарное распределение фазового поля в нескольких характеристических случаях задачи. При построении разностной схемы применена модификация метода конечных объемов, так что схема позволяет задать граничные условия на множествах высшей коразмерности в трехмерном пространстве, притом что у решения задачи в точках этих множеств допускается особенность. \\[3mm]
	\textbf{Ключевые слова:} модель типа диффузной границы, фазовое поле, метод конечных объемов, электрический пробой \\[5mm]
}
\begin{otherlanguage}{english}
\emph{\PreprintAuthorsEnglish,} \PreprintTitleEnglish \\[3mm]
\textbf{Abstract} \\
{
	\small
	The subject of the present work is a previously suggested generalization of a diffuse interface model describing the development of an electrical breakdown channel in a solid dielectric. Numerical analysis for the model is performed. In the analysis, the stationary distribution of the phase field is found in several characteristic cases of the problem. A modification of the finite volume method is applied so that the numerical scheme is available for determining the boundary conditions at sets of higher codimension in three-dimensional space. At that the solution function is expected to have a singularity at the points of these sets. \\[3mm]
	\textbf{Key words and phrases:} diffuse interface model, phase field, finite volume method, electrical breakdown \\[5mm]
}
\end{otherlanguage}

\clearpage

%!TEX root = ../main.tex

\section{Введение}

\begin{frame}{Физическое явление}
\begin{block}{Электрический пробой}
	Явление резкого возрастания тока в диэлектрике при приложении электрического напряжения
	выше критического.
\end{block}
\begin{itemize}
	\item Рассматриваем твердый диэлектрик
	\item Деградация диэлектрических свойств материала
	\item Процесс развивается в ограниченной зоне -- канале пробоя
	\item Сложная физическая природа
\end{itemize}
\end{frame}


\begin{frame}{Математическая модель}
\begin{block}{Модель типа диффузной границы}
	Вещество находится в разных фазах. Состояние вещества описывается гладкой функцией
	$\phi(\vx, t)$ -- фазовым полем.
\end{block}
\begin{itemize}
	\item $\phi = 1$ -- неповрежденная среда
	\item $\phi = 0$ -- полностью разрушенная среда
	\item Зона $\phi \in (0, 1)$ -- диффузная граница
	\item На разрушение среды тратится энергия
\end{itemize}
\begin{figure}
	\includegraphics[width=0.5\textwidth]{figures/diffuse_edge.jpg}
\end{figure}
\end{frame}


\begin{frame}{Математическая модель}
Модель, предложенная в работе \cite{pitike_dielectric_breakdown}:
\begin{itemize}
	\item $\pi = \textcolor{red}{-\half \epsilon[\phi] (\nabla \Phi, \nabla \Phi)} +
	\Gamma \left( \cfrac{1 - f(\phi)}{l^2} + \cfrac{1}{4} (\nabla \phi, \nabla \phi) \right)$
	-- плотность свободной энергии
	\item $\Gamma$ -- энегрия роста канала пробоя на единицу длины
	\item $l$ -- величина <<размытия>> канала
	\item $\epsilon(\vx, t)$ -- диэлектрическая проницаемость среды
	\item $f(\phi)$ -- интерполирующая функция
\end{itemize}
\end{frame}


\begin{frame}{Математическая модель}
\vspace{-0.2cm}
\begin{itemize}
	\item $\epsilon(\vx, t) = \cfrac{\epsilon_0(\vx)}{f(\phi(\vx, t)) +
	\delta}$ -- диэлектрическая проницаемость среды
	\item $f(\phi) = 4 \phi^3 - 3 \phi^4$ -- интерполирующая функция
\end{itemize}
\begin{columns}
\column{0.5\textwidth}
\begin{figure}
	\hspace*{1.4cm}
	\includegraphics[width=0.65\textwidth]{figures/f_form.png}
\end{figure}
\column{0.5\textwidth}
\begin{figure}
	\hspace*{-2cm}
	\includegraphics[width=0.60\textwidth]{figures/eps_form.png}
\end{figure}
\end{columns}
\end{frame}


\begin{frame}{Математическая модель}
\vspace{-0.5cm}
\begin{block}{Уравнения модели}
\begin{itemize}
	\item Уравнение электрического потенциала $\Phi$:
	\begin{equation}
		\Div(\epsilon[\phi] \nabla \Phi) = 0
		\label{equation_potential}
	\end{equation}
	\item Уравнение фазового поля $\phi$:
	\begin{equation}
		\cfrac{1}{m} \partt{\phi} = \half \epsilon'(\phi) \gradscalsq{\Phi} + \cfrac{\Gamma}{l^2} f'(\phi) + \half \Gamma \Delta \phi
		\label{equation_phase}
	\end{equation}
\end{itemize}
\end{block}
Свойства:
\begin{itemize}
	\item связанная система уравнений на $\phi$ и $\Phi$;
	\item уравнение для $\phi$ типа Аллена--Кана, нелинейное.
\end{itemize}
\end{frame}


\begin{frame}{Пример вычислительного эксперимента}
\begin{columns}
\column{0.32\textwidth}
\begin{figure}
	\includegraphics[width=\textwidth]{figures/model_example_1.png}
\end{figure}
\column{0.32\textwidth}
\begin{figure}
	\includegraphics[width=\textwidth]{figures/model_example_2.png}
\end{figure}
\column{0.32\textwidth}
\begin{figure}
	\includegraphics[width=\textwidth]{figures/model_example_3.png}
\end{figure}
\end{columns}
\begin{center}
	Расчет из работы \cite{zipunova_experiment}
\end{center}
\end{frame}


\begin{frame}{Цель работы}
\begin{block}{Цель работы}
	Исследовать качественные характеристики системы уравнений \eqref{equation_potential},
	\eqref{equation_phase} и выполнить ее численный анализ.
\end{block}
Для этого рассмотрим задачу в определенных краевых условиях, упрощающих ее, но позволяющих
установить интересующие свойства.
\end{frame}

%!TEX root = main.tex

\section{Постановка задачи и модель}

\subsection{Математическая модель}

Приведем описание математической модели, предложенной в работе \cite{pitike_dielectric_breakdown}.

Итак, рассматривается ограниченная область пространства $\Omega \subset \Real^3$. Распределение фаз вещества в ней задается гладкой функцией $\phi: \Omega \times [0, +\infty)_t \to [0, 1], \; \phi(\vx, t)$ -- фазовым полем; вещество может находиться в одной из двух фаз: $\phi \approx 1$ -- <<неповрежденное>>, $\phi \approx 0$ -- <<полностью разрушенное>> (то есть относящееся к каналу пробоя), -- а также в промежуточных состояниях в зоне диффузной границы.

Диэлектрическую проницаемость среды $\epsilon$ предлагается описать следующей формулой:
\begin{equation}
    \epsilon(\vx, t) = \epsilon[\phi] = \cfrac{\epsilon_0(\vx)}{f(\phi(\vx, t)) + \delta} \tpoint
    \label{eq:epsilon}
\end{equation}
Здесь $\epsilon_0(\vx)$ -- диэлектрическая проницаемость неповрежденной среды; $f(\phi) = 4\phi^3 - 3\phi^4$ -- интерполирующая функция, гладко соединяющая значения $0$ и $1$ ($f(0) = 0, \; f(1) = 1, \; f'(0) = f'(1) = 0$); $0 < \delta \ll 1$ -- регуляризующий параметр. Обратим внимание, что при $\phi = 1 \;\; \epsilon(\vx, t) \approx \epsilon_0(\vx)$, что соответствует диэлектрику; при $\phi = 0 \;\; \epsilon(\vx, t) = \epsilon_0(\vx) / \delta$ (в $\delta^{-1} \gg 1$ раз больше), что соответствует проводнику.

Помимо фазового поля $\phi$, состояние системы описывает также функция $\Phi: \Omega \times [0, +\infty)_t \to \Real, \; \Phi(\vx, t)$ -- потенциал электрического поля.

Постулируется следующее выражение для свободной энергии:
\begin{equation}
    \Pi = \int \limits_\Omega \pi d \vx \tcomma
    \label{eq:free_energy}
\end{equation}
\begin{equation}
    \pi = -\cfrac{1}{2} \epsilon[\phi] (\nabla \Phi, \nabla \Phi) + \Gamma \cfrac{1 - f(\phi)}{l^2} + \cfrac{\Gamma}{4} (\nabla \phi, \nabla \phi) \tpoint
    \label{eq:free_energy_density}
\end{equation}
Здесь $\Gamma > 0, \; l > 0$ -- числовые параметры модели, константы.

Постулируются два уравнения, определяющие динамику системы:
\begin{equation*}
\begin{cases}
    \cfrac{\delta \pi}{\delta \Phi} = 0 \tsemicolon \\
    \cfrac{1}{m} \partt{\phi} = -\cfrac{\delta \pi}{\delta \phi} \tpoint
\end{cases}
\end{equation*}
Здесь константа $m > 0$ -- числовой параметр модели, называемый подвижностью: она имеет смысл скорости изменения $\phi$ под действием единичной <<приложенной силы>>. Говоря нестрого, согласно первому уравнению электрический потенциал $\Phi$ распределяется так, чтобы свободная энергия системы была минимальной; согласно второму -- фазовое поле $\phi$ с определенной скоростью стремится к тому, чтобы свободная энергия была минимальной.

Отыскав явно вариационные производные в двух уравнениях выше, получим следующую систему уравнений:
\begin{numcases}{}
    \Div(\epsilon[\phi] \nabla \Phi) = 0 \tsemicolon
    \label{eq:Phi} \\
    \cfrac{1}{m} \partt{\phi} = \half \epsilon'(\phi) (\nabla \Phi, \nabla \Phi) + \cfrac{\Gamma}{l^2} f'(\phi) + \cfrac{1}{2} \Gamma \triangle \phi \tpoint
    \label{eq:phi}
\end{numcases}
Здесь $(\cdot)' \equiv (\cdot)_\phi'$. Система состоит из двух уравнений: на $\phi$ и $\Phi$ соответственно; система связная, второе уравнение нелинейное, является уравнением типа Аллена--Кана.

В таблице \ref{tab:quantities}, перечислены названия и размерности величин, встречающихся в описанной модели.

\begin{table}[!t]
\captionsetup{justification=raggedright,singlelinecheck=false}
\caption[]{Величины, относящиеся к модели.}
\centering
\begin{tabular}{|c|c|m{11cm}|}
    \hline
    Величина & Размерность & Название либо физический смысл \\
    \hline \hline
    $\phi(\vx, t)$ & $1$ & фазовое поле \\
    \hline
    \rule{0mm}{\tabletopspace}
    $\Phi(\vx, t)$ & $\cfrac{\unitJ}{\unitC}$ & потенциал электрического поля \\[\tablebottomspace]
    \hline
    $\Pi(t)$ & $\unitJ$ & свободная энергия \\
    \hline
    \rule{0mm}{\tabletopspace}
    $\pi(\vx, t)$ & $\cfrac{\unitJ}{\unitm^3}$ & плотность свободной энергии \\[\tablebottomspace]
    \hline
    \rule{0mm}{\tabletopspace}
    $\epsilon(\vx, t)$ & $\cfrac{\unitF}{\unitm} = \cfrac{\unitC^2}{\unitJ \cdot \unitm}$ & диэлектрическая проницаемость среды \\[\tablebottomspace]
    \hline
    \rule{0mm}{\tabletopspace}
    $\epsilon_0(\vx)$ & $\cfrac{\unitF}{\unitm} = \cfrac{\unitC^2}{\unitJ \cdot \unitm}$ & диэлектрическая проницаемость неповрежденной среды \\[\tablebottomspace]
    \hline
    $\delta$ & $1$ & определяет диэлектрическую проницаемость полностью разрушенной среды: она равна $\epsilon_0(\vx)/\delta$ \\
    \hline
    $l$ & $\unitm$ & характерная толщина диффузной границы \\
    \hline
    \rule{0mm}{\tabletopspace}
    $\Gamma$ & $\cfrac{\unitJ}{\unitm}$ & характерная энергия образования единицы длины канала пробоя \\[\tablebottomspace]
    \hline
    \rule{0mm}{\tabletopspace}
    $m$ & $\cfrac{\units \cdot \unitJ}{\unitm^3}$ & подвижность фазового поля $\phi$ \\[\tablebottomspace]
    \hline
\end{tabular}
\label{tab:quantities}
\end{table}


\subsection{Одномерная задача}

Рассмотрим систему со следующими ограничениями. Пусть $\Omega = [0, w]_x \times [0, h]_y \times I_z$, где $w, h > 0, \; I$ -- некоторый отрезок; $\phi(\vx, 0) = \phi_0(\vx) = \phi_0(x),$ $\epsilon_0(\vx) = \epsilon_0(x)$, то есть начальное распределение фаз и диэлектрическая проницаемость неповрежденной среды зависят только от $x$. На границе $\Omega$ будем считать заданным граничное условие $\Phi|_{y = 0} = \Phi^- \in \Real, \; \Phi|_{y = h} = \Phi^+ \in \Real, \; \Phi^- \leqslant \Phi^+$. Такую систему можно представить себе как двумерный (тривиально растянутый по третьему измерению) прямоугольный конденсатор, у которого сверху и снизу обкладки с постоянным электрическим потенциалом, между ними -- диэлектрик, меняющий свойства только по горизонтали.

Попробуем искать решение системы уравнений \eqref{eq:Phi}, \eqref{eq:phi}, имеющее $\phi(\vx, t) = \phi(x, t)$, то есть полагая, что $\phi$ не будет зависеть от $y$ и $z$.

Преобразуем уравнение \eqref{eq:Phi}:
\begin{equation}
    0 = \Div(\epsilon[\phi] \nabla \Phi) = (\nabla \epsilon, \nabla \Phi) + \epsilon \triangle \Phi = \partx{\epsilon} \partx{\Phi} + \epsilon \triangle \Phi \tpoint
    \label{eq:Phi_one_dim}
\end{equation}
Заметим, что независимо от конкретных $\phi$ и $\epsilon_0$ решением является $\Phi(\vx, t) = \Phi^- + (y/h)(\Phi^+ - \Phi^-)$. В этом случае $\partial \Phi / \partial x \equiv 0, \; \triangle \Phi \equiv 0$ и уравнение \eqref{eq:Phi_one_dim} становится тождеством.

Преобразуем уравнение \eqref{eq:phi}:
\begin{equation}
    \cfrac{1}{m} \partt{\phi} = \half \epsilon'(\phi) \left( \cfrac{\Phi^+ - \Phi^-}{h} \right)^2 + \cfrac{\Gamma}{l^2} f'(\phi) + \half \Gamma \partxx{\phi} \tcomma
    \label{eq:one_dim}
\end{equation}
$\phi_0(\vx) = \phi_0(x)$. Решение этого уравнения с начальным условием $\phi_0$ будет зависеть только от $x$ и времени $t$.

Уравнение $\eqref{eq:one_dim}$ можно рассматривать как дифференциальное уравнение в частных производных на функцию $\phi(x, t)$ одной пространственной переменной и решать его на отрезке $[0, w]_x$ числовой прямой.

Для удобства введем $K_\Phi = \|\nabla \Phi\| = (\Phi^+ - \Phi^-)/h$, тогда уравнение \eqref{eq:one_dim} примет вид
\begin{equation}
    \cfrac{1}{m} \partt{\phi} = \half K_\Phi^2 \epsilon'(\phi) + \cfrac{\Gamma}{l^2} f'(\phi) + \half \Gamma \partxx{\phi} \tpoint
    \label{eq:one_dim_simpler}
\end{equation}
$\Phi^+, \Phi^-$ и $h$ перестали входить в уравнение явно -- так мы убрали последнее упоминание о втором (по $y$) измерении пространства.

Для простоты анализа везде далее будем считать $\epsilon_0$ константой.

Итак, пара из решения уравнения \eqref{eq:one_dim_simpler} и $\Phi = \Phi^- + (y/h)(\Phi^+ - \Phi^-)$ является решением исходной системы уравнений \eqref{eq:Phi}, \eqref{eq:phi}, в том случае если краевые условия имеют описанный вид.

%!TEX root = ../main.tex

\section{Обобщение модели}

\subsection{Предложенное обобщение}

В ответ на описанную в предыдущем подразделе проблему в работе \cite{zipunova_higher_codimension}, на основании теоретических результатов работ \cite{sobolev_functional_analysis}, \cite{oleynik_biharmonic_equations}, \cite{sternin_elliptic_equations}, \cite{lewis_quasi_linear}, предлагается следующая обобщенная модель, для которой постановка условий на границах размерности 1 (соответственно, коразмерности 2) является математически корректной:
\begin{equation}
    \Pi = \int \limits_\Omega \pi d \mathbf{x} \tcomma
    \label{eq:free_energy_corrected}
\end{equation}
\begin{equation}
    \pi = -\half \epsilon[\phi] (\nabla \Phi, \nabla \Phi) + \Gamma \cfrac{1 - f(\phi)}{l^2} + \cfrac{\Gamma}{4} (\nabla \phi, \nabla \phi) + \alpha \cfrac{\Gamma l^2}{8} (\triangle \phi)^2 + \beta \cfrac{1}{p} \Gamma l^{p - 2} \| \, \nabla \phi \, \|_2^p \tsemicolon
    \label{eq:free_energy_density_corrected}
\end{equation}
\begin{numcases}{}
    \Div(\epsilon[\phi] \nabla \Phi) = 0 \tsemicolon \label{equation_Phi_corrected} \\
    \cfrac{1}{m} \cfrac{\partial \phi}{\partial t} = \half \epsilon'(\phi) (\nabla \Phi, \nabla \Phi) + \cfrac{\Gamma}{l^2} f'(\phi) + \half \Gamma \triangle \phi - \alpha \cfrac{\Gamma l^2}{4} \triangle^2 \phi + \beta \Gamma l^{p - 2} \Div (\| \, \nabla \phi \, \|_2^{p - 2} \nabla \phi) \tpoint
    \label{eq:phi_corrected}
\end{numcases}
Здесь $\alpha, \beta \geqslant 0$ -- некоторые константы, $p$ -- четное натуральное число, не меньшее~4. Дифференциальный оператор $\Div (\| \, \nabla \phi \, \|_2^{p - 2} \nabla \phi)$ принято называть \emph{$p$-лапласианом}, $\triangle^2 \phi = \triangle(\triangle \phi)$ -- \emph{билапласианом}. В дальнейшем для простоты будем считать $p = 4$.


\subsection{О методе конечных объемов}

Нашей целью будет численно исследовать систему уравнений \eqref{eq:Phi_corrected}, \eqref{eq:phi_corrected} в трех характеристических случаях, подобных описанному в подразделе \ref{subsection_matter_of_problem}.

Итак, мы ищем стационарное решение задачи \eqref{eq:Phi_corrected}, \eqref{eq:phi_corrected} с $\Phi \equiv 0$ для трех различных граничных условий:
\begin{enumerate}
    \item $\Omega = [0, +\infty)_x \times I_y \times I_z, \; \phi|_{x = 0} = 0, \; \phi \to 1$ при $x \to +\infty$ -- плоский случай;
    \item $\Omega = \Real_x \times \Real_y \times I_z, \; \phi|_{x, y = 0} = 0, \; \phi \to 1$ при $r = \sqrt{x^2 + y^2} \to +\infty$ -- цилиндрический случай;
    \item $\Omega = \Real_x \times \Real_y \times \Real_z, \; \phi|_{x, y, z = 0} = 0, \; \phi \to 1$ при $r = \sqrt{x^2 + y^2 + z^2} \to +\infty$ -- сферический случай.
\end{enumerate}
Подробно случаи 1 и 2 были описаны в подразделе \ref{subsection_matter_of_problem}. Случай 3 закономерно продолжает ряд: в нем граничное условие задано в точке -- объекте коразмерности 3.

Стационарное решение соответствует минимуму свободной энергии $\Pi$. Уравнения динамики системы выведены таковыми, что система стремится к минимуму энергии $\Pi$ в ходе эволюции. Поэтому будем проводить расчет на достаточно долгое время, тогда установившееся положение равновесия и будет искомым стационарным решением $\phi$.

В случаях 2 и 3 естественно перейти в цилиндрические и сферические координаты соответственно и считать решение $\phi$ зависящим только от радиуса $r$. В случае 1 для единообразия пространственную переменную также назовем $r$. Итак, $\phi(\mathbf{x}) = \phi(r)$.

Для численного решения задачи воспользуемся методом конечных объемов. Классический метод конечных разностей, встречая ряд проблем, подходит плохо. К примеру, в уравнениях разностной схемы могут возникать ситуации деления на 0 в узле $r = 0$ в цилиндрическом и сферическом случае (см. формулу \eqref{eq:stationary_cylindrical}).

При моделировании мы ограничим область $\Omega$ некоторым конечным размером -- граничные условия превращаются в $\phi(0) = 0, \; \phi(R) = 1, \; R > 0$ -- внешний радиус $\Omega$, такой что $R / l \gg 1$.

Разобьем область $\Omega$ на $n + 1$ ячейку (прямоугольную либо в форме цилиндрического или сферического слоя), обозначим их $\Omega_0, ..., \Omega_n$. Пусть границы ячеек имеют радиусы $0 = r_{-1/2}, r_{1/2}, ..., r_{n + 1/2} = R$.

Обозначим $V(r)$ объем прямоугольника, цилиндра или сферы (в зависимости от случая), который заполняет область $\Omega$ от радиуса $0$ до $r$. Пусть $S(r)$ -- площадь внешней (разделяющей область $\Omega$) поверхности подобного прямоугольника, цилиндра или сферы. Тогда объем ячейки $\Omega_i$ равен $dV_i = V(r_{i + 1/2}) - V(r_{i - 1/2})$, площадь внутренней и внешней границ -- $S(r_{i - 1/2})$ и $S(r_{i + 1/2})$ соответственно.

\begin{enumerate}[label=\arabic*.]
    \item Плоский случай. $V(r) = r \cdot |I_y| \cdot |I_z|, \; S(r) = |I_y| \cdot |I_z|$. Сократив, можно считать $V(r) = r, \; S(r) = 1$.
    \item Цилиндрический случай. $V(r) = \pi r^2 \cdot |I_z|, \; S(r) = 2 \pi r \cdot |I_z|$. Сократив, можно считать $V(r) = r^2, \; S(r) = 2r$.
    \item Сферический случай. $V(r) = (4/3) \pi r^3, \; S(r) = 4 \pi r^2$. Домножив оба выражения на $3/(4\pi)$, можно считать $V(r) = r^3, \; S(r) = 3 r^2$.
\end{enumerate}

Итак, было показано, что можно считать $V(r) = r^{k + 1}, \; S(r) = (k + 1)r^k$, где $k = 0$ для плоского случая, $k = 1$ для цилиндрического, $k = 2$ для сферического.

Проведем преобразование решаемого уравнения \eqref{eq:phi_corrected}, обычное для метода конечных объемов. Учтем, что $\Phi \equiv 0$. Уравнение представляется в следующей форме:
\begin{equation}
    \cfrac{1}{m} \cfrac{\partial \phi}{\partial t} = \cfrac{\Gamma}{l^2} f'(\phi) + \Gamma \Div \overline{\rho} \tcomma
    \label{eq:phi_for_integration}
\end{equation}
где
$$\overline{\rho} = \half \nabla \phi - \alpha \cfrac{l^2}{4} \nabla (\triangle \phi) + \beta l^2 \| \, \nabla \phi \, \|_2^2 \nabla \phi \tpoint$$
Проинтегрируем уравнение \eqref{eq:phi_for_integration} вначале по некоторому промежутку времени [$t_j, t_{j + 1}]$, затем по ячейке $\Omega_i$. Преобразуем левую часть:
$$\int\limits_{\Omega_i} \int\limits_{t_j}^{t_{j + 1}} \cfrac{1}{m} \cfrac{\partial \phi}{\partial t} dt d \mathbf{x} = \cfrac{1}{m} \int\limits_{\Omega_i} [\phi(\mathbf{x}, t_{j + 1}) - \phi(\mathbf{x}, t_j)] d \mathbf{x} = \cfrac{dV_i}{m} [\widetilde{\phi}_i(t_{j + 1}) - \widetilde{\phi}_i(t_j)] \tcomma$$
где $\widetilde{\phi}_i$ -- это интегральное среднее функции $\phi$ по ячейке $\Omega_i$. Преобразуем правую часть, предварительно поменяв порядок интегрирования:
$$\int\limits_{t_j}^{t_{j + 1}} \int\limits_{\Omega_i} \left( \cfrac{\Gamma}{l^2} f'(\phi) + \Gamma \Div \overline{\rho} \right) d \mathbf{x} dt = \int\limits_{t_j}^{t_{j + 1}} \left( \cfrac{\Gamma}{l^2} \int\limits_{\Omega_i} f'(\phi) d \mathbf{x} + \Gamma \int\limits_{\partial \Omega_i} (\overline{\rho}, \overline{n}) dS \right) dt \tpoint$$
К интегралу слагаемого $\Gamma \Div \overline{\rho}$ была применена формула Гаусса--Остроградского. Функция $\phi$ зависит только от $r$, следовательно, вектор $\overline{\rho}$ всегда параллелен оси $r$. Граница ячейки $\partial \Omega_i$ складывается из внешней (где вектор нормали $\overline{n}$ и ось $r$ сонаправлены) и внутренней (где $\overline{n}$ и ось $r$ противоположно направлены). Обозначим $F_{i \pm 1/2}(t)$ поток в положительном направлении оси $r$ через соответствующую границу с радиусом $r_{i \pm 1/2}$:
$$F_{i \pm 1/2}(t) = \int\limits_{r = r_{i \pm 1/2}} (\overline{\rho}, \overline{n}) dS = \int\limits_{r = r_{i \pm 1/2}} \overline{\rho}_r dS = \overline{\rho}_r S(r_{i \pm 1/2}) = \rho_{i \pm 1/2}(t) \cdot S(r_{i \pm 1/2}) \tcomma$$
$$\int\limits_{\partial \Omega_i} (\overline{\rho}, \overline{n}) dS = F_{i + 1/2} - F_{i - 1/2} = \rho_{i + 1/2} S(r_{i + 1/2}) - \rho_{i - 1/2} S(r_{i - 1/2}) \tpoint$$
Здесь $\overline{\rho}_r$ обозначена $r$-координата вектора $\overline{\rho}$ (она же единственная ненулевая). Величину $\rho_{i \pm 1/2}(t) = \overline{\rho}_r(r_{i \pm 1/2}, t)$ будем называть плотностью потока через соответствующую границу. Таким образом, выведено следующее интегральное соотношение:
\begin{equation}
    \cfrac{dV_i}{m} [\widetilde{\phi}_i(t_{j + 1}) - \widetilde{\phi}_i(t_j)] = \int\limits_{t_j}^{t_{j + 1}} \left( \cfrac{\Gamma}{l^2} \int\limits_{\Omega_i} f'(\phi) d \mathbf{x} + \Gamma \left[ \rho_{i + 1/2} S(r_{i + 1/2}) - \rho_{i - 1/2} S(r_{i - 1/2}) \right] \right) dt \tpoint
    \label{eq:finite_volume_integral}
\end{equation}

Первое слагаемое в подынтегральном выражении в правой части равенства \eqref{eq:finite_volume_integral} приблизим выражением $(\Gamma/l^2) \cdot dV_i \cdot f'[\widetilde{\phi}_i(t_j)]$. При построении разностной схемы от интеграла по $[t_j, t_{j + 1}]$ перейдем к умножению на $(t_{j + 1} - t_j)$ значения подынтегрального выражения в точке $t_j$.

Выясним, как вычислить плотность потока $\rho_{i \pm 1/2}$ во втором слагаемом. Если некоторая функция $\psi(\mathbf{x}) = \psi(r)$, то
$$(\nabla \psi)_r = \cfrac{\partial \psi}{\partial r} \tcomma$$
где $\nabla \psi$ также зависит только от $r$. Таким образом:
\begin{equation}
    \rho = \half \cfrac{\partial \phi}{\partial r} - \alpha \cfrac{l^2}{4} \cfrac{\partial}{\partial r} (\triangle \phi) + \beta l^2 \left( \cfrac{\partial \phi}{\partial r} \right)^3 \tpoint
    \label{eq:finite_volumes_density}
\end{equation}

Традиционно в методе конечных объемов принимается, что локальное восполнение решения в ячейке -- постоянная функция. В силу того, что рассматриваемая задача требует постановки граничных условий при $r = 0$, а решение задачи вблизи этой точки может иметь большие производные, обобщим традиционный подход. А именно, будем считать, что в окрестности нуля решение представляется в виде линейной комбинации двух специально выбранных базисных функций, а его производные, соответственно, приближаются линейной комбинацией производных базисных функций с теми же коэффициентами. Опишем в общем виде поиск коэффициентов разложения.

Построим приближение для некоторой функции $\psi(r)$ в соседних ячейках $\Omega_i$ и $\Omega_{i + 1}$ по известным интегральным средним $\widetilde{\psi}_i$ и $\widetilde{\psi}_{i + 1}$ в этих ячейках. Пусть
$$g(r) = a \cdot g^{(a)}(r) + b \cdot g^{(b)} (r)$$
есть функция с двумя числовыми параметрами $a$ и $b$, $g^{(a)}$ и $g^{(b)}$ -- базисные функции, используемые для локального представления $\psi$. Найдем такие $a$ и $b$, что интегральные средние $g(r)$ по ячейкам $\Omega_i$ и $\Omega_{i + 1}$ были бы равны $\widetilde{\psi}_i$ и $\widetilde{\psi}_{i + 1}$ соответственно. Это эквивалентно системе уравнений
\begin{numcases}{}
    \int\limits_{r_{i - 1/2}}^{r_{i + 1/2}} [\widetilde{\psi}_i - g(r)] S(r) dr = 0 \tsemicolon
    \label{eq:interpolation_first} \\
    \int\limits_{r_{i + 1/2}}^{r_{i + 3/2}} [\widetilde{\psi}_{i + 1} - g(r)] S(r) dr = 0 \tpoint
    \label{eq:interpolation_second}
\end{numcases}
Пусть
$$\int\limits_{r_{i - 1/2}}^{r_{i + 1/2}} g^{(a)}(r) S(r) dr = I_i^{(a)}; \qquad \int\limits_{r_{i - 1/2}}^{r_{i + 1/2}} g^{(b)}(r) S(r) dr = I_i^{(b)} \tpoint$$
Считаем, что интегралы $I_i^{(a)}$ и $I_i^{(b)}$ найдены аналитически. Тогда система \eqref{eq:interpolation_first}, \eqref{eq:interpolation_second} эквивалентна системе
$$\begin{cases}
    a I_i^{(a)} + b I_i^{(b)} = (r_{i + 1/2}^{k + 1} - r_{i - 1/2}^{k + 1}) \widetilde{\psi}_i \tsemicolon \\
    a I_{i + 1}^{(a)} + b I_{i + 1}^{(b)} = (r_{i + 3/2}^{k + 1} - r_{i + 1/2}^{k + 1}) \widetilde{\psi}_{i + 1} \tpoint
\end{cases}$$
Это система двух линейных уравнений с двумя неизвестными -- решим методом Крамера. Получим:
$$\varDelta = I_i^{(a)} I_{i + 1}^{(b)} - I_i^{(b)} I_{i + 1}^{(a)} \tsemicolon$$
\begin{equation}
    a = \cfrac{(r_{i + 1/2}^{k + 1} - r_{i - 1/2}^{k + 1}) I_{i + 1}^{(b)}}{\varDelta} \cdot \widetilde{\psi}_i + \cfrac{-(r_{i + 3/2}^{k + 1} - r_{i + 1/2}^{k + 1}) I_i^{(b)}}{\varDelta} \cdot \widetilde{\psi}_{i + 1} \tsemicolon
    \label{eq:interpolation_a}
\end{equation}
\begin{equation}
    b = \cfrac{-(r_{i + 1/2}^{k + 1} - r_{i - 1/2}^{k + 1}) I_{i + 1}^{(a)}}{\varDelta} \cdot \widetilde{\psi}_i + \cfrac{(r_{i + 3/2}^{k + 1} - r_{i + 1/2}^{k + 1}) I_i^{(a)}}{\varDelta} \cdot \widetilde{\psi}_{i + 1} \tpoint
    \label{eq:interpolation_b}
\end{equation}
Теперь можно легко находить $a$ и $b$ при различных значениях $\widetilde{\psi}_i, \; \widetilde{\psi}_{i + 1}$, если вычислить заранее и сохранить четыре коэффициента, стоящие в формулах \eqref{eq:interpolation_a}, \eqref{eq:interpolation_b}. 

Приблизим $\partial \psi / \partial r$ на границе с радиусом $r_{i + 1/2}$ (между ячейками $\Omega_i$ и $\Omega_{i + 1}$) с помощью функции $g(r)$ с известными числовыми параметрами:
$$\left[ \cfrac{\partial \psi}{\partial r} \right]_{i + 1/2} = g'(r_{i + 1/2}) = a \cdot (g^{(a)})'(r_{i + 1/2}) + b \cdot (g^{(b)})'(r_{i + 1/2}) \tpoint$$
Производные $(g^{(a)})'$ и $(g^{(b)})'$ считаем найденными аналитически.

Отыщем приближение производной $\partial \phi / \partial r$ на границах с радиусами $r_{3/2}, ...$, $r_{n - 1/2}$ (то есть на всех, кроме первых двух внутренних и крайней внешней). Используем описанный выше метод с $\widetilde{\psi} = \widetilde{\phi}$ и базисными функциями $g^{(a)}(r) = r, \; g^{(b)}(r) = 1$. $g' \equiv a$, вывести формулы для $I^{(a)}$, $I^{(b)}$ также не составляет труда. Таким образом, происходит приближение функции $\phi$ на парах соседних ячеек линейной функцией.

Задание граничных условий, в том числе приближение $\partial \phi / \partial r$ на крайних границах $\Omega$, будет подробно описано в следующем разделе.

Без ответа остался только вопрос вычисления $\triangle \phi$ и его производной по $r$. Оно требуется лишь в случае ненулевой константы $\alpha$ (см. уравнение \eqref{eq:phi_corrected}). Проведя рассуждение, аналогичное проделанному для вывода соотношения \eqref{eq:finite_volume_integral}, но без интегрирования по времени, получим следующее выражение для интегрального среднего лапласиана $\phi$ по ячейке:
\begin{equation}
    dV_i \cdot \widetilde{\triangle \phi}_i = \cfrac{\partial \phi}{\partial r} \bigg|_{r = r_{i + 1/2}} S(r_{i + 1/2}) - \cfrac{\partial \phi}{\partial r} \bigg|_{r = r_{i - 1/2}} S(r_{i - 1/2}) \tpoint
    \label{eq:finite_volume_laplasian}
\end{equation}

По средним $\widetilde{\triangle \phi}$ вычислим приближение $\partial (\triangle \phi) / \partial r$ на границах ячеек с радиусами $r_{1/2}, ..., r_{n - 1/2}$ (то есть на всех, кроме крайних внутренней и внешней) тем же способом, который ранее использовался для $\partial \phi / \partial r$, с $\widetilde{\psi} = \widetilde{\triangle \phi}, \; g^{(a)}(r) = r, \; g^{(b)}(r) = 1$.


\subsection{Задание граничных условий}

Граничные условия, подробно описанные в предыдущем разделе, имеют следующий вид: $\phi(0) = 0, \; \phi(R) = 1, \; R > 0$, где $R$ такое, что $R / l \gg 1$. Если коэффициент $\alpha$ в уравнении \eqref{eq:phi_corrected} при слагаемом с билапласианом ненулевой, то этих граничных условий недостаточно в силу повышения порядка уравнения -- необходимо добавить условия на производные. По логике задачи их следует сделать таковыми:
$$\cfrac{\partial \phi}{\partial r} \bigg|_{r = 0} = 0, \qquad \cfrac{\partial \phi}{\partial r} \bigg|_{r = 1} = 0 \tpoint$$

Граничные условия в точке $r = R$ в разностной схеме задаются легко: $\widetilde{\phi}_n \equiv 1, \; \widetilde{\triangle \phi}_n \equiv 0$. Этого оказывается достаточно ввиду того, что функция $\phi$ вблизи точки $R$ меняется очень слабо.

С граничными условиями в точке $r = 0$ дела обстоят намного сложнее. Для начала, функции перехода в цилиндрическую и сферическую систему координат имеют в этой точке особенность. К тому же ожидается, что $\phi$ в окрестности точки $r = 0$ довольно быстро растет. Более того, как выяснится позже, в цилиндрическом и сферическом случае в точке $0$ имеют особенность производные функций $\phi$ и $\triangle \phi$.

Зададим граничные условия в точке $0$ следующим образом. Выберем для приближения $\phi$ в ячейках $\Omega_0$ и $\Omega_1$ такие базисные функции $g^{(a)}$ и $g^{(b)}$, что каждая из них удовлетворяет граничным условиям при $r = 0$ и, при необходимости, одна из них имеет в точке $0$ предположительно тот же вид особенности, что решение $\phi$. С помощью функции $g(r)$ с известными параметрами получим искомые приближения:
$$\left[ \cfrac{\partial \phi}{\partial r} \right]_{-1/2} = g'(0); \qquad \left[ \cfrac{\partial \phi}{\partial r} \right]_{1/2} = g'(r_{1/2}) \tsemicolon$$
$$\left[ \cfrac{\partial (\triangle \phi)}{\partial r} \right]_{-1/2} = \cfrac{\partial}{\partial r} \left[ r^k \cfrac{\partial}{\partial r} \left(\cfrac{1}{r^k} \cfrac{\partial g}{\partial r} \right) \right] \Bigg|_{r = 0} \tpoint$$
В последнем выражении используется общий вид формулы для лапласиана функции, зависящей только от $r$, в прямоугольных, цилиндрических и сферических координатах:
\begin{equation}
    \triangle g = \cfrac{1}{r^k} \cfrac{\partial}{\partial r} \left( r^k \cfrac{\partial g}{\partial r} \right) \tpoint
    \label{eq:laplasian_common}
\end{equation}

Определимся с выбором базисных функций для приближения $\phi$ в первых двух ячейках в зависимости от случая задачи и значений параметров $\alpha$ и $\beta$.

Рассмотрим плоский случай. Здесь у производной $\phi$ (и $\triangle \phi$, входящей в формулы, только если $\alpha \neq 0$) в точке $0$ не ожидается особенностей. Обоснуем это так: $S(0) = 1 \neq 0$, поэтому для ненулевого потока $F_{-1/2}$ через крайнюю внутреннюю границу нужна ненулевая конечная плотность потока $\rho_{-1/2}$. При $\alpha = 0$ граничное условие $\phi(0) = 0$ -- возьмем $g^{(a)} = r^2, \; g^{(b)} = r$. При $\alpha \neq 0$ добавляется граничное условие $\partial \phi / \partial r |_{r = 0} = 0$ -- используем $g^{(a)} = r^3, \; g^{(b)} = r^2$.

Теперь рассмотрим цилиндрический случай. Имеем $S(0) = 0$, поэтому, казалось бы, поток $F_{-1/2}$ всегда нулевой. Однако если допустить у $\rho$ особенность вида $1/r$, то получается $\rho(r) S(r) \to 2$ при $r \to +0$ -- конечный ненулевой поток! Если $\rho$ по модулю растет асимптотически медленнее, то поток будет нулевым, если быстрее, то бесконечным (что не имеет смысла).

Пусть $\alpha = \beta = 0$. Тогда в выражении \eqref{eq:finite_volumes_density} для плотности потока встречается лишь $\partial \phi / \partial r$ в первой степени. Значит, мы ищем $g^{(a)}$, такую что $(g^{(a)})' = C_0/r$. Проинтегрировав, получим $g^{(a)} = C_0 \ln r + C_1$. $g^{(a)} \to \infty$ при $r \to +0$, то есть граничное условие $g^{(a)}(0) = 0$ выполнить невозможно. Это косвенно подтверждает вывод из работы \cite{zipunova_higher_codimension}, что в этом случае решения дифференциальной задачи не существует.

Пусть $\alpha = 0, \; \beta \neq 0$. В выражение \eqref{eq:finite_volumes_density} для плотности потока входит $\partial \phi / \partial r$ в первой и третьей степени. Следовательно, мы ищем $g^{(a)}$, такую что $(g^{(a)})' = C_0 r^{-1/3}$. Ее первая степень даст нулевой вклад в поток, а третья -- конечный ненулевой. Проинтегрировав, получим $g^{(a)} = C_0 \cdot (3/2) \cdot r^{2/3} + C_1$. С учетом граничного условия выберем $g^{(a)} = r^{2/3}$. Вторую базисную функцию возьмем без особенности в точке $0$, например, $g^{(b)} = r$.

Пусть $\alpha \neq 0, \; \beta$ произвольное. Тогда в выражение \eqref{eq:finite_volumes_density} для плотности потока входит $\partial (\triangle \phi) / \partial r$. Встречается там и $\partial \phi / \partial r$, но она, согласно граничному условию, равна~$0$ при $r = 0$. Отыщем такую базисную функцию $g^{(a)}$, что $\partial (\triangle g^{(a)}) / \partial r = C_0 / r$. Учтем, что лапласиан функции $g^{(a)}$, зависящей только от $r$, в цилиндрических координатах вычисляется по формуле \eqref{eq:laplasian_common} с $k = 1$. Трижды проинтегрировав уравнение на производную лапласиана, получим $g^{(a)} = (C_0/4) r^2 (\ln r - 1) + C_1 r^2 / 4 + C_2 \ln r + C_3$. Требуется $g^{(a)}(0) = 0$, следовательно, $C_2 = C_3 = 0$. Особенность содержит первое слагаемое, поэтому выберем $g^{(a)} = r^2 (\ln r - 1)$. $g^{(a)}$ удовлетворяет граничному условию на производную $\partial g^{(a)} / \partial r|_{r = 0} = 0$. Второй базисной функцией сделаем $g^{(b)} = r^2$.

Наконец, рассмотрим сферический случай. Аналогично цилиндрическому случаю будем подбирать базисную функцию так, чтобы поток $F_{-1/2}$ был конечным. Для этого $\rho$ должна иметь особенность вида $1/r^2$, так как тогда $\rho(r) S(r) \to 3$ при $r \to 0$.

Пусть $\alpha = \beta = 0$. Аналогично рассмотренному ранее случаю ищем $g^{(a)}$, такую что $(g^{(a)})' = C_0 / r^2$. Проинтегрировав, получим $g^{(a)} = -C_0 / r + C_1$. $g^{(a)} \to \infty$ при $r \to +0$, то есть граничное условие $g^{(a)}(0) = 0$ выполнить невозможно.

Пусть $\alpha = 0, \; \beta \neq 0$. Аналогично рассмотренному ранее случаю ищем $g^{(a)}$, такую что $(g^{(a)})' = C_0 / r^{-2/3}$. Проинтегрировав, получим $g^{(a)} = C_0 \cdot 3 \cdot r^{1/3} + C_1$. С учетом граничного условия выберем $g^{(a)} = r^{1/3}, \; g^{(b)} = r$.

Пусть $\alpha \neq 0, \; \beta$ произвольное. Аналогично рассмотренному ранее случаю ищем $g^{(a)}$, такую что $\partial (\triangle g^{(a)}) / \partial r = C_0 / r^2$. Лапласиан функции $g^{(a)}$, зависящей только от $r$, в сферических координатах вычисляется по формуле \eqref{eq:laplasian_common} с $k = 2$. Дважды проинтегрировав уравнение на производную лапласиана, получим $(g^{(a)})' = -C_0 / 2 + C_1 r / 3 + C_2 / r^2$. $C_0 \neq 0$, так как иначе у функции пропадет особенность. Получается, что $(g^{(a)})'(0) \neq 0$ -- невозможно удовлетворить граничному условию на производную!

Примечательно, что выбранный подход не удалось применить в сферическом случае не только при $\alpha = \beta = 0$, но и при $\alpha \neq 0$. Выдвинем гипотезу, что в обеих этих конфигурациях исследуемая дифференциальная задача поставлена некорректно и не имеет решения.


\subsection{Разностная схема}

Везде в двух предыдущих подразделах допускалось, что радиусы границ ячеек $0 = r_{-1/2}, r_{1/2}, ..., r_{n - 1/2}, r_{n + 1/2} = R$ могут быть произвольной возрастающей последовательностью чисел. В дальнейшем, на практике, мы сделаем их структуру регулярной с шагом $h$: $r_{i - 1/2} = ih$.

При моделировании система будет проходить моменты времени $t_j = j \tau$, где $\tau$ -- фиксированный шаг по времени. Функцию времени с аргументом $t_j$ будем обозначать верхним индексом $j$.

В качестве начального условия задаются $\{ \widetilde{\phi}_i^0 \}_{i = 0}^n$.

Как было принято ранее, $k = 0$ в плоском случае, $k = 1$ в цилиндрическом, $k = 2$ в сферическом.

Подводя итог рассуждениям, проделанным в двух предыдущих подразделах, выпишем построенную методом конечных объемов разностную схему.

\begin{equation}
    \begin{gathered}
        \cfrac{1}{m} (\widetilde{\phi}_i^{j + 1} - \widetilde{\phi}_i^j) = \tau \cfrac{\Gamma}{l^2} f'(\widetilde{\phi}_i^j) + \cfrac{\tau}{dV_i} \Gamma (\rho_{i + 1/2}^j S_{i + 1/2} - \rho_{i - 1/2}^j S_{i - 1/2}) \; \text{при} \; i = \overline{0, n - 1} \tsemicolon \\
        \widetilde{\phi}_n^j = 1 \tsemicolon
    \end{gathered}
    \label{sch:volime_phi}
\end{equation}
$$dV_i = r_{i + 1/2}^{k + 1} - r_{i - 1/2}^{k + 1}; \qquad S_{i \pm 1/2} = (k + 1) r_{i \pm 1/2}^k \tsemicolon$$
\begin{equation}
    \rho_{i \pm 1/2}^j = \half \left[ \cfrac{\partial \phi}{\partial r} \right]_{i \pm 1/2}^j - \alpha \cfrac{l^2}{4} \left[ \cfrac{\partial (\triangle \phi)}{\partial r} \right]_{i \pm 1/2}^j + \beta l^2 \left( \left[ \cfrac{\partial \phi}{\partial r} \right]_{i \pm 1/2}^j \right)^3 \tsemicolon
    \label{sch:volume_density}
\end{equation}
\begin{equation}
    \left[ \cfrac{\partial \phi}{\partial r} \right]_{i + 1/2}^j = a_{i + 1/2}^j \; \text{при} \; i = \overline{1, n - 1} \tsemicolon
    \label{sch:volume_phi_derivative}
\end{equation}
$$\{\widetilde{\phi}_i^j\}_{i = 1}^n \leadsto \{a_{i + 1/2}^j\}_{i = 1}^{n - 1}, \; \text{где} \; g^{(a)} = r, \; g^{(b)} = 1 \tsemicolon$$
\begin{equation}
    \begin{gathered}
        \widetilde{\triangle \phi}_i^j = \cfrac{1}{dV_i} \left( \left[ \cfrac{\partial \phi}{\partial r} \right]_{i + 1/2}^j S_{i + 1/2} - \left[ \cfrac{\partial \phi}{\partial r} \right]_{i - 1/2}^j S_{i - 1/2} \right) \; \text{при} \; i = \overline{0, n - 1} \tsemicolon \\
        \widetilde{\triangle \phi}_n^j = 0 \tsemicolon
    \end{gathered}
    \label{sch:volume_laplasian}
\end{equation}
\begin{equation}
    \left[ \cfrac{\partial (\triangle \phi)}{\partial r} \right]_{i + 1/2}^j = c_{i + 1/2}^j \; \text{при} \; i = \overline{0, n - 1} \tsemicolon
    \label{sch:volume_laplasian_derivative}
\end{equation}
$$\{\widetilde{\triangle \phi}_i^j\}_{i = 0}^n \leadsto \{c_{i + 1/2}^j\}_{i = 0}^{n - 1}, \; \text{где} \; g^{(c)} = r, \; g^{(d)} = 1 \tsemicolon$$
\begin{equation}
    \left[ \cfrac{\partial \phi}{\partial r} \right]_{i - 1/2}^j = a_{1/2}^j (g_{1/2}^{(a)})'(r_{i - 1/2}) + b_{1/2}^j (g_{1/2}^{(b)})'(r_{i - 1/2}) \; \text{при} \; i = \overline{0, 1} \tsemicolon
    \label{sch:volume_phi_derivetive_border}
\end{equation}
\begin{equation}
    \left[ \cfrac{\partial (\triangle \phi)}{\partial r} \right]_{-1/2}^j = a_{1/2}^j \cfrac{\partial (\triangle g^{(a)})}{\partial r} \bigg|_{r = 0} + b_{1/2}^j \cfrac{\partial (\triangle g^{(b)})}{\partial r} \bigg|_{r = 0} \tsemicolon
    \label{sch:volume_laplasian_derivative_border}
\end{equation}
$$\{ \widetilde{\phi}_0^j, \widetilde{\phi}_1^j \} \leadsto \{ a_{1/2}^j, b_{1/2}^j \} \tpoint$$
1. Плоский случай при $\alpha = 0$:
\begingroup
\setlength{\abovedisplayskip}{5pt}
\setlength{\belowdisplayskip}{5pt}
\begin{align*}
    g_{1/2}^{(a)} &= r^2, & (g_{1/2}^{(a)})' &= 2r, & I^{(a)} &= \cfrac{1}{3} r^3 \bigg|_{...}^{...} \tsemicolon \\
    g_{1/2}^{(b)} &= r, & (g_{1/2}^{(b)})' &= 1, & I^{(b)} &= \half r^2 \bigg|_{...}^{...} \tpoint
\end{align*}
2. Плоский случай при $\alpha \neq 0$:
\begin{align*}
    g_{1/2}^{(a)} &= r^3, & (g_{1/2}^{(a)})' &= 3r^2, & \cfrac{\partial (\triangle g_{1/2}^{(a)})}{\partial r} &= 6, & I^{(a)} &= \cfrac{1}{4} r^4 \bigg|_{...}^{...} \tsemicolon \\
    g_{1/2}^{(b)} &= r^2, & (g_{1/2}^{(b)})' &= 2r, & \cfrac{\partial (\triangle g_{1/2}^{(b)})}{\partial r} &= 0, & I^{(b)} &= \cfrac{1}{3} r^3 \bigg|_{...}^{...} \tpoint
\end{align*}
3. Цилиндрический случай при $\alpha = 0, \; \beta \neq 0$:
\begin{align*}
g_{1/2}^{(a)} &= r^{2/3}, & (g_{1/2}^{(a)})' &= \cfrac{2}{3} r^{-1/3}, & I^{(a)} &= \cfrac{3}{4} r^{8/3} \bigg|_{...}^{...} \tsemicolon \\
g_{1/2}^{(b)} &= r, & (g_{1/2}^{(b)})' &= 1, & I^{(b)} &= \cfrac{2}{3} r^3 \bigg|_{...}^{...} \tsemicolon
\end{align*}
$$\rho_{-1/2}^j S_{-1/2} = \cfrac{16}{27} (a_{1/2}^j)^3 \tpoint$$
4. Цилиндрический случай при $\alpha \neq 0$:
\begin{align*}
    g_{1/2}^{(a)} &= r^2 (\ln r - 1), & (g_{1/2}^{(a)})' &= r (2 \ln r - 1), & \cfrac{\partial (\triangle g_{1/2}^{(a)})}{\partial r} &= \cfrac{4}{r}, & I^{(a)} &= \cfrac{1}{8} r^4 (4 \ln r - 5) \bigg|_{...}^{...} \tsemicolon \\
    g_{1/2}^{(b)} &= r^2, & (g_{1/2}^{(b)})' &= 2r, & \cfrac{\partial (\triangle g_{1/2}^{(a)})}{\partial r} &= 0, & I^{(b)} &= \half r^4 \bigg|_{...}^{...} \tsemicolon
\end{align*}
$$\left[ \cfrac{\partial (\triangle \phi)}{\partial r} \right]_{-1/2}^j S_{-1/2} = 8 a_{1/2}^j \tpoint$$
5. Сферический случай при $\alpha = 0, \; \beta \neq 0$:
\begin{align*}
    g_{1/2}^{(a)} &= r^{1/3}, & (g_{1/2}^{(a)})' &= \cfrac{1}{3} r^{-2/3}, & I^{(a)} &= \cfrac{9}{10} r^{10/3} \bigg|_{...}^{...}  \tsemicolon \\
    g_{1/2}^{(b)} &= r, & (g_{1/2}^{(b)})' &= 1, & I^{(b)} &= \cfrac{3}{4} r^4 \bigg|_{...}^{...} \tsemicolon
\end{align*}
$$\rho_{-1/2}^j S_{-1/2} = \cfrac{1}{9} (a_{1/2}^j)^3 \tpoint$$
\endgroup


\subsection{Вычислительный эксперимент}

Была написана программа, реализующая разностную схему \eqref{sch:volime_phi}, \eqref{sch:volume_density}, \eqref{sch:volume_phi_derivative}, \eqref{sch:volume_laplasian}, \eqref{sch:volume_laplasian_derivative}, \eqref{sch:volume_phi_derivetive_border}, \eqref{sch:volume_laplasian_derivative_border}.

Все предыдущие рассуждения были универсальны для плоского, цилиндрического и сферического случаев, насколько это возможно. Все три случая моделируются одной и той же программой, принимающей $k = \overline{0, 2}$ как параметр.

Зададим параметры модели:
$$\epsilon_0 = 0.2, \; \delta = 0.04, \; l = 1.0, \; \Gamma = 1.0, \; m = 0.5 \tpoint$$

Пусть $R = 5$. По виду графиков будет понятно, что для использованного набора параметров такое $R$ достаточно велико.

Выберем число ячеек $n = 300$ при $\alpha = 0$ и $n = 150$ иначе. Шаг по пространству~$h$, соответственно, равен $1/30 \approx 0.0333$ либо $1/60 \approx 0.0167$. При этом придется брать шаг по времени $\tau = 5 \cdot 10^{-7}$. Если шаг по времени кратно больше, то схема оказывается неустойчива и программа завершается с ошибкой переполнения (в данных возникает значение NaN). Как и ожидалось, $p$-лапласиан и билапласиан (особенно) требуют относительно малых шагов по времени.

Выбор начальных условий не имеет значения. Пусть, например, значения $0$ и $1$ гладко соединяются ветвью синусоиды: $\widetilde{\phi}_i^0 = \sin[(h/2 + ih) \pi / 2], \; h/2 + ih < 1$; далее все значения равны $1$.

Будем останавливать расчет, когда вектор <<воздействия>> на систему достаточно мал, а именно:
$$\max\limits_{i = 0}^n \cfrac{\widetilde{\phi}_i^{j + 1} - \widetilde{\phi}_i^j}{m} < 10^{-9} \tpoint$$
Все конфигурации модели достигли названного условия не более чем через $7.4$ единицы времени.

Результаты вычислений изображены на рис. \ref{fig:result_volumes} и \ref{fig:result_volumes_p} (случай 1 из описания разностной схемы), рис. \ref{fig:result_volumes_bi} (случай 2), рис. \ref{fig:result_volumes_cyl_p} (случай 3), рис. \ref{fig:result_volumes_cyl_bi} (случай 4), рис.~\ref{fig:result_volumes_sph_p} (случай~5). Графики функций показаны до зримого момента выхода на примерно постоянное значение $1$, после чего они в действительности продолжаются до $R = 5$. Графики состоят из соединенных значений средних $\widetilde{\phi}_i^j$, размещенных в серединах ячеек; вблизи~$0$ к ним добавлено несколько значений приближающей функции $g^{(a)}$, имеющей особенность в $0$, если того требует случай задачи.

\begin{figure}[!tp]
    \centering
    \includegraphics[width=0.89\textwidth]{figures/result_volumes.png}
    \vspace{-0.3cm}
    \caption{Решение $\phi$ в плоском случае при $\alpha = 0, \; \beta = 0$.}
    \label{fig:result_volumes}
    \vspace{0.5cm}

    \includegraphics[width=0.89\textwidth]{figures/result_volumes_p.png}
    \vspace{-0.3cm}
    \caption{Решение $\phi$ в плоском случае при $\alpha = 0, \; \beta = 1$.}
    \label{fig:result_volumes_p}
    \vspace{0.5cm}
    
    \includegraphics[width=0.89\textwidth]{figures/result_volumes_bi.png}
    \vspace{-0.3cm}
    \caption{Решение $\phi$ в плоском случае при $\alpha = 1, \; \beta = 0$.}
    \label{fig:result_volumes_bi}
\end{figure}

\begin{figure}[!tp]
    \centering
    \includegraphics[width=0.89\textwidth]{figures/result_volumes_cyl_p.png}
    \vspace{-0.3cm}
    \caption{Решение $\phi$ в цилиндрическом случае при $\alpha = 0, \; \beta = 1$.}
    \label{fig:result_volumes_cyl_p}
    \vspace{0.5cm}

    \includegraphics[width=0.89\textwidth]{figures/result_volumes_cyl_bi.png}
    \vspace{-0.3cm}
    \caption{Решение $\phi$ в цилиндрическом случае при $\alpha = 1, \; \beta = 0$.}
    \label{fig:result_volumes_cyl_bi}
    \vspace{0.5cm}
    
    \includegraphics[width=0.89\textwidth]{figures/result_volumes_sph_p.png}
    \vspace{-0.3cm}
    \caption{Решение $\phi$ в сферическом случае при $\alpha = 0, \; \beta = 1$.}
    \label{fig:result_volumes_sph_p}
\end{figure}

Отметим, что если в уравнения входит билапласиан ($\alpha \neq 0$), то функция $\phi$ может быть не монотонной и в некоторых точках превышать значение $1$ (см. рис. \ref{fig:result_volumes_bi}, \ref{fig:result_volumes_cyl_bi}). В работе \cite{zipunova_higher_codimension} это было отмечено и указано, что монотонность $\phi$ следует ожидать при достаточно малых $\alpha$.

Итак, эксперимент подтверждает, что, несмотря на некоторую громоздкость формулировок, предложенная модификация метода конечных объемов позволяет эффективно моделировать решение $\phi$, даже если на границе области оно имеет особенность.

%!TEX root = ../main.tex

\section{The finite-difference scheme}
\label{sec:differential_scheme}

In this section, we present a finite-difference scheme for solving
the equation~\eqref{eq:one_dim} in the domain~$[0, W]_x \times [0,
+\infty)_t$. The equation is
subjected to initial conditions~\eqref{eq:one_dim_initial} and boundary conditions~\eqref{eq:one_dim_marginal}.

Consider a regular mesh with a time step~$\tau$ and
spatial step~$h$. Let~$W = Nh$ with $N$ being the number of
nodes. The nodes of the spatiotemporal grid are given by~$(jh, k \tau)$,
$j = \overline{0, N}$, $k \in \Natural_0$. Define by~$\phi_j^k$
the value of a mesh function~$\phi$ at the node~$(jh, k \tau)$.
Then the finite-difference approximations read
\begin{equation}
  \cfrac{1}{m} \difftau{\phi} = \half K_\phi^2 \epsilon'(\phi_j^k) + \cfrac{\Gamma}{l^2} f'(\phi_j^k) + \cfrac{\Gamma}{2} \diffhh{\phi} \tpoint
  \label{eq:subtractive}
\end{equation}
or, in the explicit form,
\begin{gather}
  \begin{aligned}
    \phi_j^{k + 1} = \phi_j^k + m \tau \left( \half K_\Phi^2 \epsilon'(\phi_j^k) + \cfrac{\Gamma}{l^2} f'(\phi_j^k) + \cfrac{\Gamma}{2} \diffhh{\phi} \right), \\ j = \overline{1, N - 1}, \quad k \in \Natural_0 \tsemicolon
  \end{aligned}
  \label{sch:transition} \\
  \phi_j^0 = \phi_0(jh); \quad \phi_0^k = \phi_l(k \tau); \quad \phi_N^k = \phi_r(k \tau) \tpoint
  \label{sch:borders}
\end{gather}

It is easy to see that the scheme has the first order of approximation in
time and the second order of approximation in spatial terms.

To study properties of the scheme~\eqref{sch:transition}, \eqref{sch:borders},
the linear theory can be used (see, e.g.,
\cite[Chapter~10]{bahvalov_computational_methods}
or~\cite[Chapter~IX]{kalitkin_computational_methods}).
The central result of the theory states, in a somewhat simplified
form, that if a finite-difference scheme is stable and approximates a
continuous problem then the solution of the finite-dimensional problem
converges to the solution of the continuous one with order
not lower then the order of approximation.

To apply this result for the nonlinear setting~\eqref{sch:transition}, \eqref{sch:borders},
we proceed as follows:
(i) linearize the equation~\eqref{eq:subtractive}
for a fixed~$\phi$ and then (ii) apply the spectral stability
argument~\cite{bahvalov_computational_methods} to the
derived linearized equation. As the stability criteria are
satisfied for the linearized equation, stability should be expected for the
complete, nonlinear, problem. In this case, convergence of the
approximate solution should be expected as well~--- since
the finite-difference problem is stable and approximates the continuous
one.
The results of such non-rigorous analysis will be further confirmed by
numerical computations in the fully nonlinear setting.


\subsection{Stability estimate}

In this section we derive a stability condition for the
finite-difference scheme~\eqref{sch:transition}, \eqref{sch:borders}
using the so-called principal of ``frozen coefficients''
(see, e.g.,~\cite{bahvalov_computational_methods}).
Let~$\phi_j^k$ and~$\phi_j^k + \delta_j^k$ be solutions of the
finite-difference equation~\eqref{eq:subtractive}.
Substitute~$\phi_j^k + \delta_j^k$ into~\eqref{eq:subtractive} to obtain:
\begin{multline*}
  \cfrac{1}{m} \cfrac{(\phi_j^{k + 1} + \delta_j^{k + 1}) - (\phi_j^k + \delta_j^k)}{\tau} = \half K_\Phi^2 [\epsilon'(\phi_j^k) + \epsilon''(\phi_j^k) \delta_j^k + o(\delta_j^k)] + \\ + \cfrac{\Gamma}{l^2} [f'(\phi_j^k) + f''(\phi_j^k) \delta_j^k + o(\delta_j^k)] + \cfrac{\Gamma}{2} \cfrac{(\phi_{j + 1}^k + \delta_{j + 1}^k) - 2 (\phi_j^k + \delta_j^k) + (\phi_{j - 1}^k + \delta_{j - 1}^k)}{h^2} \tpoint
\end{multline*}
Linearizing this equation around~ $\phi_j^k = P$, assuming that
perturbations~$\delta_j^k$ are small and taking into account
that~$\phi_j^k$ is a solution of the finite-difference problem, we obtain:
\begin{equation}
  \delta_j^{k + 1} = \delta_j^k + m \tau \left( \half K_\Phi^2 \epsilon''(P) \delta_j^k + \cfrac{\Gamma}{l^2} f''(P) \delta_j^k + \cfrac{\Gamma}{2} \diffhh{\delta} \right) \tpoint
  \label{eq:scheme_variation}
\end{equation} 

We now apply spectral stability analysis to the derived equation for
perturbations.
Let~$\delta_j^k = \lambda(\theta)^k \cdot \exp(\imath j \theta)$, $\imath^2 = -1$.
Substituting this representation into~\eqref{eq:scheme_variation} one obtains:
$$\lambda(\theta) = 1 + m \tau \left( \half K_\Phi^2 \epsilon''(P) + \cfrac{\Gamma}{l^2} f''(P) + \cfrac{\Gamma}{2} \cfrac{\exp(\imath \theta) - 2 + \exp(-\imath \theta)}{h^2} \right) \tcomma$$
or
\begin{equation}
  \lambda(\theta) = 1 + m \tau \left( \half K_\Phi^2 \epsilon''(P) + \cfrac{\Gamma}{l^2} f''(P) - \cfrac{2 \Gamma}{h^2} \sin^2 \cfrac{\theta}{2} \right) \tpoint
  \label{eq:spectral}
\end{equation}

According to the spectral stability argument, a time
step~$\tau = \tau(h)$ provides stability of the scheme in the
domain~$[0, W]_x \times [0, T]_t$ with~$T<+\infty$ as~$\tau, h \to 0$ if there
exists~$C > 0$ such that for an arbitrary~$\theta$ it
holds~$|\lambda(\theta)| \leqslant \exp(C\tau)$. Note that here it is
also possible to use more strict condition~$|\lambda(\theta)| \leqslant 1 + C\tau$.
If for an arbitrary~$\theta$ it holds~$|\lambda(\theta)| \leqslant 1$,
then stability will be provided also for an unbounded time interval, i.e.,
for~$[0, W]_x \times [0, +\infty)_t$.
Strictly speaking, the spectral argument does not provide a sufficient
stability condition; however, stability should be expected in practice.

First, consider the expression~\eqref{eq:spectral} for~$P=0$.
We have~$f''(0) = 0$, $\epsilon''(0) = 0$, and the equation~\eqref{eq:spectral}
takes the form of
$$\lambda(\theta) = 1 - \cfrac{2 \tau m \Gamma}{h^2} \sin^2 \cfrac{\theta}{2} \tpoint$$
Hence, for an arbitrary~$\theta$ it holds~$|\lambda(\theta)| \leqslant 1$
if and only if
\begin{equation}
  \tau \leqslant \cfrac{h^2}{m \Gamma} \tpoint
  \label{cond:spectral_0}
\end{equation}
As the condition~\eqref{cond:spectral_0} is satisfied, one can expect stability of the scheme
when the solution describes an almost completely damaged state~$\phi\approx0$
in the domain~$[0, W]_x \times [0, +\infty)_t$.

Note that under the condition~\eqref{cond:spectral_0} one also can expect stable computations
for~$[0, W]_x \times [0, T]_t$ for an arbitrary value~$P \in [0, 1]$.
In this case the following is true:
$$
|\lambda(\theta)| \leqslant \left| 1 - \cfrac{2 \tau m \Gamma}{h^2} \sin^2 \cfrac{\theta}{2} \right| + m \tau \left| \half K_\Phi^2 \epsilon''(P) + \cfrac{\Gamma}{l^2} f''(P) \right| \leqslant 1 + m \tau \left| \half K_\Phi^2 \epsilon''(P) + \cfrac{\Gamma}{l^2} f''(P) \right| \tpoint
$$
Hence, there exists~$C$ such that
$|\lambda(\theta)| \leqslant 1 + C \tau$ holds,~--- since~$\epsilon''(\phi)$ and~$f''(\phi)$
are continuous on~$[0, 1]$.
It should be noted that, despite such versatility, the estimate~\eqref{cond:spectral_0}
is poorly applicable in practice and requires clarification, which will be done later.

We now consider the expression~\eqref{eq:spectral} at the value~$P=1$.
Note that~$f''(1) < 0$, $\epsilon''(1) > 0$.
We see that for~$(K_\Phi^2 / 2) \epsilon''(1) + (\Gamma / l^2) f''(1) \leqslant 0$
it is possible to achieve~$|\lambda(\theta)| \leqslant 1$ with demanded sufficiently small
values of~$\tau$ and the condition $\tau \leqslant h^2 / (2m \Gamma)$,
similar to the one for~\eqref{cond:spectral_0}.
Substituting~$f''(1) = -12, \; \epsilon''(1) = 12 \epsilon_0 / (1 + \delta)^2$
(see~\eqref{eq:epsilon_derivatives}),
we obtain
\begin{equation}
  \cfrac{K_\Phi^2 l^2 \epsilon_0}{2 \Gamma (1 + \delta)^2} \leqslant 1 \tpoint
  \label{cond:spectral_possible_1}
\end{equation}

So, under the condition~\eqref{cond:spectral_possible_1}, it is expected that
there exist such values of~$\tau$ и $h$
that the difference scheme is stable for~$\phi \approx 1$
and~$T=+\infty$.
Naturally the condition~\eqref{cond:spectral_possible_1}
is equivalent to the stability condition~\eqref{cond:equilibrium_1_stable}
for the equilibrium state~$\phi \equiv 1$ of the equation~\eqref{eq:one_dim}.

\subsection{Improved stability estimate}

In the previous section form the analysis of equation~\eqref{eq:spectral}
it was derived stability condition~\eqref{cond:spectral_0}
for finite-difference scheme~\eqref{sch:transition} and~\eqref{sch:borders} for~$\phi \approx 0$.
The assumption of its usefulness is based on the fact that typical ``'natural'' solution of the model
will has a form of the transition process from the undamaged state~$\phi=1$ to the completely
damaged state~$\phi=0$ occurring in a finite time interval and then infinitely long staying in the
damaged state~$\phi \approx 0$.


However the performed analysis of the equation~\eqref{eq:spectral}
is not sufficient at~$\phi = 0$. Indeed, it was used that at~$\phi=0$,
$\epsilon''(0) = 0$ (see expression~\eqref{eq:epsilon_derivatives}),~---
but it was not accounted that~$\epsilon''(\phi)$ growth fast and reaches
large values  for small values of~$\delta\approx 0$,
see Fig.~\ref{fig:eps_phi_phi}.
This means that the equations of the model are stable at~$\phi=0$,
but can be unstable in the small neighbourhood of~$\phi=0$.
Such situation is not satisfactory and we now try to improve
the obtained stability estimates.
%
\begin{figure}[!t]
	\centering
	\includegraphics[width=\textwidth]{figures/eps_phi_phi.png}
	\vspace{-0.7cm}
	\caption{Typical behavior of~$\epsilon''(\phi)$ in the vicinity of~$0$.}
	\label{fig:eps_phi_phi}
\end{figure}

To proceed let us estimate extremums of~$\epsilon''(\phi)$ in the neighbourhood of~$0$.
First, find zeros of~$\epsilon'''(\phi)$. We have
\begin{equation}
	\epsilon''' = \epsilon_0 \cfrac{-6 (f')^3 + 6 (f + \delta) f' f'' - (f + \delta)^2 f'''}{(f + \delta)^4},
	\label{eq:epsilon_phi_phi_phi}
\end{equation}
form where:
$$\epsilon'''(\phi) = -6 (f')^3 + 6 (f + \delta) f' f'' - (f + \delta)^2 f''' = 0 \tcomma$$
or, taking~\eqref{eq:epsilon} into account:
$$-3 \cdot 12^2 (1 - \phi)^3 + 36 \left(4 - 3\phi + \cfrac{\delta}{\phi^3} \right)(1 - \phi)(2 - 3\phi) - \left(4 - 3 \phi + \cfrac{\delta}{\phi^3} \right)^2 (1 - 3 \phi) = 0 \tpoint$$

Let~$\delta_n \to +0$ and~$\phi_n \to +0$ such that~$\delta_n / \phi_n^3$ is bounded.
Then:
\begin{gather*}
	-3 \cdot 12^2 \cdot 1^3 + 36 \left(4 + \cfrac{\delta_n}{\phi_n^3} \right) \cdot 1 \cdot 2 - \left(4 + \cfrac{\delta_n}{\phi_n^3} \right)^2 \cdot 1 \to 0 \tcomma \\
	\left(4 + \cfrac{\delta_n}{\phi_n^3} \right)^2 - 72 \left(4 + \cfrac{\delta_n}{\phi_n^3} \right) + 3 \cdot 12^2 \to 0 \tpoint
\end{gather*}
Hence, a sequence~$4 + \delta_n / \phi_n^3$ has not more than two partial limits~$\xi_+$ and~$\xi_-$~---
which are zeros of the equation~$\xi^2 - 72 \xi + 432 = 0$.
To the first zero~$\xi_+ = 36 + 12 \sqrt{6}$ it corresponds
$$\phi_+ = \cfrac{1}{\sqrt[3]{32 + 12 \sqrt{6}}} \sqrt[3]{\delta_n} \approx \cfrac{1}{3.945} \sqrt[3]{\delta_n} \tsemicolon$$
to the second zero~$\xi_- = 36 - 12 \sqrt{6}$ it corresponds
$$\phi_- = \cfrac{1}{\sqrt[3]{32 - 12 \sqrt{6}}} \sqrt[3]{\delta_n} \approx \cfrac{1}{1.376} \sqrt[3]{\delta_n} \tpoint$$

From here it can be seen that for~$\delta \to +0$ the function~$\epsilon'''(\phi)$ has two zeros in the neighbourhood of~$0$:
\begin{equation}
  \phi_{\pm} = \cfrac{1}{\sqrt[3]{32 \pm 12 \sqrt{6}}} \sqrt[3]{\delta} [1 + o(1)] \tpoint
  \label{eq:epsilon_phi_phi_phi_roots}
\end{equation}

We now estimate~$\epsilon''(\phi)$ at~$\phi_{\pm}$  for~$\delta \to +0$. Let~$\phi = (1 / c) \sqrt[3]{\delta}$, $c \in \Real$.
Then:
$$\epsilon'' = \epsilon_0 \cfrac{24 c^5 (8 - c^3)}{(4 + c^3)^3} \delta^{-5 / 3} [1 + o(1)],$$
and:
\begin{equation}
  \epsilon''(\phi_+) \approx -4.378 \epsilon_0 \delta^{-5 / 3}; \quad \epsilon''(\phi_-) \approx 2.216 \epsilon_0 \delta^{-5 / 3} \tpoint
  \label{est:epsilon_phi_phi_bounds}
\end{equation}
The derived estimates are shown as black dashed lines on Fig.~\ref{fig:eps_phi_phi_multiplied}.

\begin{figure}[!t]
	\centering
	\includegraphics[width=\textwidth]{figures/eps_phi_phi_multiplied.png}
	\caption{Qualitative behavior of~$\delta^{5 / 3} \epsilon''(\phi)$ for small values of~$\delta$.}
	\label{fig:eps_phi_phi_multiplied}
\end{figure}

Now, to derive new stability estimate we consider equation~\eqref{eq:spectral} at~$\phi = \phi_+$.
Note that~$\epsilon''(\phi_+) \approx -4.4 \epsilon_0 \delta^{-5 / 3}$.
The term inside braces in~\eqref{eq:spectral} is negative since
$\delta$ is small and~$\epsilon''(\phi_+)$ is negative and large in its absolute value.
Therefore~$f''(\phi_+) > 0$ can be estimated as~$0$~--- such estimate makes inequality stronger.
Then from inequality~\eqref{eq:spectral} it follows that 
$$\lambda(\theta) = 1 + m \tau \left( -\cfrac{2.2 K_\Phi^2 \epsilon_0}{\delta^{5 / 3}} - \cfrac{2 \Gamma}{h^2} \sin^2 \cfrac{\theta}{2} \right) \tpoint$$
Condition~$|\lambda(\theta)| \leqslant 1$ is satisfied for an arbitrary~$\theta$, if and only if
\begin{equation}
  \tau \leqslant \cfrac{1}{m} \left( \cfrac{1.1 K_\Phi^2 \epsilon_0}{\delta^{5 / 3}} + \cfrac{\Gamma}{h^2} \right)^{-1} \tpoint
  \label{cond:spectral_better_theoretical}
\end{equation}

Numerical experiments described in the next sections indicates that
more strong version of the estimate~\eqref{cond:spectral_better_theoretical} is also valid
(note the doubled denominator):
%
\begin{equation}
  \tau \leqslant \cfrac{1}{2m} \left( \cfrac{K_\Phi^2 \epsilon_0}{\delta^{5 / 3}} + \cfrac{\Gamma}{h^2} \right)^{-1} \tpoint
  \label{cond:spectral_better}
\end{equation}

Finally, more simple estimate not weaker then~\eqref{cond:spectral_better} is:
\begin{equation}
  \tau \leqslant \cfrac{1}{4m} \min \left(\cfrac{\delta^{5 / 3}}{K_\Phi^2 \epsilon_0}, \; \cfrac{h^2}{\Gamma} \right) \tpoint
  \label{cond:spectral_better_simpler}
\end{equation}

Note that the derived stability estimate~\eqref{cond:spectral_better}
for finite-difference scheme~\eqref{sch:transition},\eqref{sch:borders}
includes all the parameters of the equation~\eqref{eq:one_dim}, except~$l$.
Notably, this is the only parameter of the model which has somehow artificial nature and can not be
related directly to the underlying physics.

\endinput
% EOF

%!TEX root = ../main.tex

\section{Вычислительный эксперимент}

Была написана программа, реализующая разностную схему \eqref{sch:all}.

Все предыдущие рассуждения были универсальны для плоского, цилиндрического и сферического случаев, насколько это возможно. Все три случая моделируются одной и той же программой, принимающей $k = \overline{0, 2}$ как параметр.

Зададим параметры модели:
$$\epsilon_0 = 0.2, \; \delta = 0.04, \; l = 1.0, \; \Gamma = 1.0, \; m = 0.5 \tpoint$$

Пусть $R = 5$. По виду графиков будет понятно, что для использованного набора параметров такое $R$ достаточно велико.

Выберем число ячеек $n = 300$ при $\alpha = 0$ и $n = 150$ иначе. Шаг по пространству~$h$, соответственно, равен $1/60 \approx 0.0167$ либо $1/30 \approx 0.0333$. При этом придется брать шаг по времени $\tau = 5 \cdot 10^{-7}$. Если шаг по времени кратно больше, то схема оказывается неустойчива и программа завершается с ошибкой переполнения (в данных возникает значение NaN). Как и ожидалось, $p$-лапласиан и билапласиан (особенно) требуют относительно малых шагов по времени.

Выбор начальных условий неважен. Пусть, например, значения $0$ и $1$ гладко соединяются ветвью синусоиды: $\avphi_i^0 = \sin[(h/2 + ih) \pi / 2], \; h/2 + ih < 1$; далее все значения равны $1$.

Будем останавливать расчет, когда вектор <<воздействия>> на систему достаточно мал, а именно:
$$\max\limits_{i = 0}^n \cfrac{\avphi_i^{j + 1} - \avphi_i^j}{m} < 10^{-9} \tpoint$$
Все конфигурации модели достигли названного условия не более чем через $7.4$ единицы времени.

\begin{figure}[!tp]
	\centering
	\includegraphics[width=0.83\textwidth]{figures/result_volumes.png}
	\vspace{-0.4cm}
	\caption{Решение $\phi$ в плоском случае при $\alpha = 0, \; \beta = 0$}
	\label{fig:result_volumes}
	\vspace{0.5cm}

	\includegraphics[width=0.83\textwidth]{figures/result_volumes_p.png}
	\vspace{-0.4cm}
	\caption{Решение $\phi$ в плоском случае при $\alpha = 0, \; \beta = 1$}
	\label{fig:result_volumes_p}
	\vspace{0.5cm}
	
	\includegraphics[width=0.83\textwidth]{figures/result_volumes_bi.png}
	\vspace{-0.4cm}
	\caption{Решение $\phi$ в плоском случае при $\alpha = 1, \; \beta = 0$}
	\label{fig:result_volumes_bi}
\end{figure}

\begin{figure}[!tp]
	\centering
	\includegraphics[width=0.83\textwidth]{figures/result_volumes_cyl_p.png}
	\vspace{-0.4cm}
	\caption{Решение $\phi$ в цилиндрическом случае при $\alpha = 0, \; \beta = 1$}
	\label{fig:result_volumes_cyl_p}
	\vspace{0.5cm}

	\includegraphics[width=0.83\textwidth]{figures/result_volumes_cyl_bi.png}
	\vspace{-0.4cm}
	\caption{Решение $\phi$ в цилиндрическом случае при $\alpha = 1, \; \beta = 0$}
	\label{fig:result_volumes_cyl_bi}
	\vspace{0.5cm}
	
	\includegraphics[width=0.83\textwidth]{figures/result_volumes_sph_p.png}
	\vspace{-0.4cm}
	\caption{Решение $\phi$ в сферическом случае при $\alpha = 0, \; \beta = 1$}
	\label{fig:result_volumes_sph_p}
\end{figure}

Результаты вычислений изображены на рис. \ref{fig:result_volumes} и \ref{fig:result_volumes_p} (случай 1 из описания разностной схемы), рис. \ref{fig:result_volumes_bi} (случай 2), рис. \ref{fig:result_volumes_cyl_p} (случай 3), рис. \ref{fig:result_volumes_cyl_bi} (случай 4), рис.~\ref{fig:result_volumes_sph_p} (случай~5). Графики функций показаны до зримого момента выхода на примерно постоянное значение $1$, после чего они в действительности продолжаются до $R = 5$. Графики состоят из соединенных значений средних~$\avphi_i^j$, размещенных в серединах ячеек; вблизи~$0$ к ним добавлено несколько значений приближающей функции $g_{1/2}$, имеющей особенность в $0$, если того требует случай задачи.

Отметим, что если в уравнения входит билапласиан ($\alpha \neq 0$), то функция~$\phi$ может быть немонотонной и в некоторых точках превышать значение~$1$ (см.~рис.~\ref{fig:result_volumes_bi},~\ref{fig:result_volumes_cyl_bi}). В работе \cite{zipunova_higher_codimension} это было отмечено и указано, что монотонность~$\phi$ следует ожидать при достаточно малых $\alpha$.

Итак, эксперимент подтверждает, что, несмотря на некоторую громоздкость формулировок, предложенная модификация метода конечных объемов позволяет эффективно моделировать решение $\phi$, даже если на границе области оно имеет особенность.

%!TEX root = ../main.tex

\section{Conclusions}

In this paper we study stability properties of the phase-field model
for electrical breakdown channel evolution.
The central result is a classification of the
equilibrium solutions of the model and their stability.
From practical point of view, these results allows to
make meaningful conclusions regarding qualitative and quantitative
properties of the model. Particularly it was shown under which
conditions small perturbations of the equilibrium solutions
develop into channel-like structure typical for of electrical breakdown
process.

Besides this, a simple explicit finite-difference scheme
for solution of the model in spatially one-dimensional setting is considered.
The main question addressed here are stability conditions which guaranties
correctness of the simulations. Deep connections between
stability conditions of the model and the one of the
finite-difference scheme are shown.
The presented results of the numerical simulations confirms
predictions of the theoretical analysis of the model.

% EOF
\endinput

\clearpage
\printbibliography[
	heading=bibintoc
]

\clearpage
\tableofcontents

\end{document}

%%%%%%%%%%%%%%%%%%%%%%%%%%%%%%%%%%%%%%%%%%%%%%%%%%%%%%%%%%%%%%%%%%%%%%%%%%%%%%%%